\chapter{Hồi 74, 75, 76, và 77} % (fold)
\label{cha:hoi_74_75}

Hồi 74:

\begin{itshape}
``Trường Canh truyền báo nhiều ma dữ.

Hành giả ra tay lắm phép tài.''
\end{itshape}

Hồi 75:

\begin{itshape}
``Bình Âm Dương Ngộ Không khoan thủng.

Đạo chân như ma chúa theo về.''
\end{itshape}

Hồi 76:

\begin{itshape}
``Hành giả về nhà, ma trở mặt.

Ngộ Năng cùng đánh, quái hoàn nguyên.''
\end{itshape}

Hồi 77:

\begin{itshape}
``Lũ ma lừa bản tính.

Tất cả lạy chân như.''
\end{itshape}

\section{Tư tưởng Phật Học} % (fold)
\label{sec:74_75_phat_hoc}

-- Hồi 74, 75, 76, và 77 là các hồi thử thách về công phu giải thoát sâu xa của phái đoàn Tây Du. Tại núi Sư Đà này có ba đại quỷ vương vốn là sư tử của Bồ Tát Văn Thù, bạch tượng của Bồ Tát Phổ Hiền và Đại Bàng thuộc thân thích với Phật cảnh. Đường Tăng, Ngộ Năng và Ngộ Tịnh thì thần thông thua xa ba đại quỷ vương, không phải là đối thủ. Chỉ trừ Ngộ Không (đại trí) mới chiến thắng lão ma (biểu tượng của chấp thủ kiến), vượt qua được sự chấp thủ các tri kiến; Ngộ Không cũng chiến thắng được quỷ vương thứ hai (biểu tượng của đại định). Tôn Ngộ Không vốn tánh năng động, phóng khoáng nên nhẹ về công phu trì giới nên bị kẹt vào cạm bẩy của quỷ vương thứ ba. Tuy nhiên, Ngộ Không biết cách khắc phục nhược điểm của mình bằng cách cầu cứu Như Lai và được Như Lai khai ngộ.

Đường Tăng thì tại đây còn chấp tướng Giới, tướng Định và chấp Kiến nên bị cả ba quỷ vương đánh bại suýt tiêu mạng. Nạn này là một trắc nghiệm về công phu Giới, Định, Tuệ của phái đoàn Tây Du. Sau nạn này, phái đoàn càng hành sâu Giới, Định, Tuệ mới có thể đến Lôi Âm Tự.

-- Tai nạn này, các quỷ vương đều không có yêu khí tỏa ra nên Ngộ Không không thể nào nhận ra được. Chúng lại khôn ngoan tương kế, tựu kế đánh lừa Tôn Hành Giả. Vì thiếu định lực sâu nên Ngộ Không thiếu tâm cảnh giác mà rơi vào bẫy của quỷ vương. Quỷ Đại Bàng vốn có tài đằng vân nhanh như Ngộ Không nên kẹp bắt được Tôn Ngộ Không (ma lực cũng bàng chân lực; hay định ngoại đạo cũng tương tự chánh định, chỉ khác về trí tuệ). Sau nạn này, Ngộ Không dễ thức tỉnh để đoạn trừ các mạn, trạo cử và vô minh (các thượng phần kiết sử) của tự thân.

-- Đối với trí tuệ toàn giác thì Ngộ Không còn cách xa, nhưng đối với các pháp hữu vi như cái bình Âm Dương của lão ma thì Tôn Ngộ Không dễ dàng phá hỏng nhờ vào bữu bối mà Như Lai đã ban. Dù vậy, Ngộ Không cần tu tập nhiều về Định uẩn và Tuệ uẩn mới có thể tiến gần toàn giác.

-- Tại đây Ngô Thừa Ân giới thiệu thêm một điểm tế nhị khác thuộc giáo lý Phật Giáo: ở trong chiếc bình Âm Dương của lão ma, nếu hành giả lặng lẽ đừng phát âm ra tiếng Người (nghĩa là đường tư duy; cần tịnh lự; đừng khởi lên các ngã tưởng) thì ở mãi trong bình vẫn cảm thấy mát mẻ; nếu nói ra tiếng người thì bình sẽ nung nóng với ba con rồng phun lửa (hồi 75).

Nói ra tiếng người là khởi lên các ngã tưởng; mà khởi lên các ngã tưởng thì tham, sân, si dấy khởi. Đây là hình ảnh biểu tượng ba con rồng phun lửa thiêu cháy người.

Không nói ra tiếng người là không khởi lên các ngã tưởng; không khởi lên các ngã tưởng thì tham, sân, si chìm lắng nên tâm mát lạnh.

Tôn Ngộ Không nhờ có ba sợi lông mà Như Lai đã ban (biểu tượng của Vô Ngã, Vô Thường, Khổ đau --- 3 pháp ấn.) mà hành thiền quán thấy rõ thực tướng của các hiện hữu, chọc vỡ được chiếc bình hữu vi, đoạn diệt tham, sân, si. Đây là nội dung của bài kệ cuối trong kinh Kim Cang: \emph{``Nhất thiết hữu vi pháp, như mộng huyễn, bào ảnh, như lộ diệt như điển, ưng tác như thị quán''}.
% section tư_tưởng_phật_học (end)

\section{Quan niệm về con Người} % (fold)
\label{sec:74_75_con_nguoi}

-- Qua bốn hồi truyện này (và các nhiều hồi trước đó), tác giả luôn phơi bày bản chất của phái tà, ma, quỷ, mị là tham lam, sân hận, ích kỷ, cá nhân, hưởng thụ, ngã mạn và hung ác. Bản chất đó trái ngược với các đức tính giải thoát nên gây cản trở bước chân của phái đoàn Tây Du.
% section quan_niệm_về_con_người (end)

\section{Quan niệm về xã hội} % (fold)
\label{sec:74_75_xa_hoi}

-- Một nền văn hóa mới của vị tha, nhân ái, trí tuệ, \ldots ~thì bản chất khác hẳn bản chất của nền văn hóa cũ. Nếu xã hội ghét bỏ các tà tâm và những gì dung dưỡng các tà tâm thì xã hội cần tách khỏi các giá trị, tư duy của nền văn hóa cũ và dấn thân xây dựng nền văn hóa mới của công bằng, nhân ái, vị tha và an lạc.

-- Cái bình Âm Dương của Lão ma đã nói rõ vấn đề chính là cải tạo nhận thức, tâm thức con người trước vấn đề cải tạo xã hội, điều kiện sống của con người, hệt như Tôn Ngộ Không biết giữ mình an lạc, thanh lương trong chiếc bình Âm Dương.

-- Cũng thế, khổ đau dấy khởi từ tâm thức con người hơn là từ ngoại cảnh. Con người làm chủ quyết định khổ đau hay hạnh phúc của chính mình. Giáo dục cái tâm trở thành trọng điểm giáo dục của nền văn hóa giáo dục mới của nhân bản và trí tuệ.
% section quan_niệm_về_xã_hội (end)
% chapter Hồi 74_75 (end)