\chapter{Hồi 96 và 97} % (fold)
\label{cha:hoi_96_97}

Hồi 96:

\begin{itshape}
``Khấu Viên ngoại mừng đãi cao Tăng.

Đường Trưởng lão không ham phú quý.''
\end{itshape}

Hồi 97:

\begin{itshape}
``Trả của cải gây thêm ách họa.

Hiện U hồn cứu giúp chân tu.''
\end{itshape}

\section{Tư tưởng Phật Học} % (fold)
\label{sec:96_97_phat_hoc}

— Theo Phật Giáo, vụ Đường Tăng bị hàm oan ở nhà Viên ngoại phải bị ở tù mấy hôm chỉ là phần Nhân Quả của các kiếp quá khứ để lại. Đây không phải là nạn của vọng thức. Vì thế, Tôn Hành Giả đã chủ động giải tỏa nhẹ nhàng.

Có nhiều người khi được giải thoát thì chấm dứt luôn Nhân Quả ngoại giới; có trường hợp như trường hợp của Đường Tăng còn chịu vướng vài hệ lụy bởi thân sắc này, chỉ có thân sắc hệ lụy, còn tâm thì không.
% section tư_tưởng_phật_học (end)

\section{Quan niệm về con Người} % (fold)
\label{sec:96_97_con_nguoi}

— Giá trị thực của một hành động đạo đức, như đã được bàn, được quyết định do tâm hành động (bao gồm nhận thức, lòng chân thật, vị tha và ý chí hành động). Viên ngoại bố thí cúng dường mà nặng lòng vị kỷ và hiếu danh hiếu lợi nên hậu quả là chiêu họa vào tự thân và liên lụy đến Đường Tăng.

Với người thực hiện nền văn hóa mới mà thực hiện với tấm lòng của nền văn hóa cũ thì kết quả cũng sẽ tương tự việc làm của Viên ngoại.

— Giá trị của hành động đạo đức theo giá trị của nền văn hóa mới không thể đơn thuần căn cứ vào hậu quả của hành động để đánh giá, như không thể căn cứ vào hậu quả của Viên ngoại và Đường Tăng nhận chịu để đánh giá việc bố thí, cúng dường, bởi vì hậu quả đó một phần do nghiệp quá khứ của Viên ngoại, một phần do Viên ngoại chưa thực hành đúng pháp.
% section quan_niệm_về_con_người (end)

\section{Quan niệm về xã hội} % (fold)
\label{sec:96_97_xa_hoi}

— Về giá trị của nền văn hóa mới cũng thế. Thành quả của xã hội mới gồm một phần do hậu quả của xã hội cũ để lại, một phần do xã hội chưa thể hiện đúng chân giá trị của nền văn hóa mới. Đây là bài học kinh nghiệm dành cho những người làm văn hóa lưu tâm.

— Trong xã hội, các thành viên xã hội không có sự giác ngộ đồng đều về nền văn hóa mới, và không có sự nhất trí trong các công tác xã hội nên không thể chờ đợi mọi kết quả tốt đẹp, như sự việc xẩy ra trong gia đình Viên ngoại. Thành quả của văn hóa bao giờ cũng đến chậm hơn và kém hơn khả năng mà nền văn hóa ấy có thể đem lại.

Đây cũng là một kinh nghiệm khác đáng được lưu tâm để ổn định tâm lý mạnh dạn xây dựng các giá trị mới.
% section quan_niệm_về_xã_hội (end)
% chapter Hồi 96_97 (end)