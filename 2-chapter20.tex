\chapter{Hồi 50, 51 và 52} % (fold)
\label{cha:hoi_50_51}

Hồi 50:

\begin{itshape}
``Tình rối, tính theo vì ái dục.

Thần mờ, tâm động gặp yêu ma.''
\end{itshape}

Hồi 51:

\begin{itshape}
``Nghìn mưu Đại Thánh thành vô dụng.

Nước, lửa không công khó gặp ma.''
\end{itshape}

Hồi 52:

\begin{itshape}
``Ngộ Không làm nhộn Kim Đầu động.

Như Lai mách kín vị chủ nhân.''
\end{itshape}

\section{Tư tưởng Phật Học} % (fold)
\label{sec:50_51_phat_hoc}

-- Lúc phái đoàn Tây Du đến vùng núi Kim Đầu, Tôn Ngộ Không đã biết là có yêu tà, đã can ngăn Đường Tăng đặt chân đến một tòa nhà lớn dưới chân núi mà nghi là do yêu tà biến hóa ra, và đã vạch một vòng an toàn ngăn chặn yêu tà cho Đường Tăng, Ngộ Năng, Ngộ tịnh và Tiểu Long Mã. Tôn Ngộ Không phải đi xa xin cơm cho Đường Tăng. Ngộ Năng lòng vốn chộn rộn không yên; Đường Tăng cũng thế; Ngộ Năng lại khởi lên niệm đố kỵ Ngộ Không, dục Đường Tăng đến thẳng ngôi nhà quỷ mà khất thực. Độc Giốc quỷ vương đã bắt giam Đường Tăng, Ngộ Năng, Ngộ Tịnh và Tiểu Long Mã. Đây là nạn nguy hiểm nhất kể từ ngày đầu lên đường thỉnh kinh.

Hồi 50 nói lên cái tâm động của Đường Tăng và của Ngộ Năng: vừa dấy niệm phàm phu, vừa thiếu tuệ, thiếu tình, không biết nghe lời dặn của Ngộ Không (người đệ tử đã rất mực trung thành và rất trí tuệ). Tại đây Đường Tăng và Ngộ Năng bộc lộ rõ ràng thái độ kỷ luật thiếu nghiêm túc, định lực yếu nghiêm túc, định lực yếu và thiếu lòng tin vào tiếng nói trí tuệ (Ngộ Không). Chính khuyết điểm này của tâm thức giải thoát đã mở ra ma nạn ở động Kim Đầu. Với trí tuệ đó, với định lực đó, và với tâm tình đó, hành giả sẽ không thể bước thêm được một bước giải thoát nào, nếu không nói có thể đánh chìm lý tưởng giải thoát. Vì thế Ngô Thừa Ân đã đưa phái đoàn Tây Du (trừ Ngộ Không) vào lao ngục của ác ma lúc bấy giờ.

-- Tại động Kim Đầu này Tôn Ngộ Không đã chiến đấu vất vả hơn bao giờ, đã bị cướp mất cả khí giới. Ngộ Không đã cầu viện Lý Thiên Vương, Na Tra thái tử và cả thần sấm sét; tất cả đều bị ác ma tước mất khí giới. Cuối cùng Tôn Ngộ Không phải cầu cứu Đức Như Lai mách nước. Tất cả đều thất bại chỉ vì chiếc vòng Kim Cương mài của Giốc quỷ vương (ngoài chư Bồ Tát và Phật).

Cái vòng Kim Cương mài (hay Độc Giốc quỷ vương) là biễu tượng của Thái cực của đạo Nho, hay Đạo của Lão giáo. Với đạo Nho, Thái cực sinh ra Lưỡng nghi (Âm Dương); với đạo Lão, Đạo sinh nhất, nhất sinh nhị, \ldots. Thái cực hay Đạo ấy mới thực sự là đầu nguồn của vũ trụ vạn hữu, đầu nguồn của thế giới nhị nguyên, của nền văn hóa nhị nguyên.

Đến hồi 49, phái đoàn Tây Du nhờ đại định, đại tuệ và lòng đại bi của Bồ Tát Quán Thế Âm cứu giúp mới vượt qua được ma nạn Âm Dương và Ngũ Hành, nên hẳn tại đây, núi Kim Đầu, phái đoàn không thể vượt qua được Thái cực hay Đạo (nhất nguyên). Trí tuệ Vô Ngã của Tôn Ngộ Không, dù đã được Bồ Tát Quán Thế Âm chỉ dạy cái dụng vô lượng của nó, cũng chưa thể thắng được Độc Giốc quỷ vương, dù không bị độc giốc hãm hại. Điều này nói lên rõ ràng trí tuệ Vô Ngã của Ngộ Không chưa phát triển được diệu dụng vô biên của nó, vì thiếu đại định, thiếu tâm giải thoát. Vì thiếu tâm giải thoát nên phiền não, lậu hoặc còn ngăn che. Còn bị lậu hoặc ngăn che là còn vướng vào tập khí chấp thủ các ngã tướng --- Ở đây thì chấp vào tướng một, nhất nguyên, một sự chấp thủ tế nhị vào quả chứng đắc, nên phải bị bại trước Độc Giốc quỷ vương là còn nhân duyên rơi vào vòng đối đãi của sinh tử. Cái vòng Kim Cương mài thu hết, khử hết các khí giới, các vật sinh từ Âm Dương, Ngũ Hành là biểu tượng của ý nghĩa đó.

Muốn thoát ly khỏi nạn Kim Đầu (Kim có nghĩa gốc của kim, mộc, thủy, hỏa, thổ; bởi đầu có nghĩa là cái mũ) thì hành giả phải nhờ định sâu vào thiền quán cho đến khi thấy rõ cái gốc sinh khởi của Âm Dương --- Ngũ Hành. Khi thấy rõ cái gốc ấy rồi thì liền thấy rõ con đường vĩnh viễn thoát ly sự trói buộc của Âm Dương, Ngũ Hành. Công phu giải thoát này Ngộ Không chưa chứng đạt, vì thế phải cầu cứu Như Lai.

Đức Phật cho 18 vị A-La-Hán liệng 18 hạt ``kim sa'' để trắc nghiệm dò tìm gốc của chiếc Kim Cương mài và Độc Giốc quỷ vương. Mười tám hạt ``kim đơn sa'' là tượng trưng cho 18 tâm giác tỉnh Vô Ngã để hàng phục 18 tâm ái (\emph{sáu căn, sáu trần và sáu thức}). Khi 18 hạt kim đơn sa bị chiếc vòng Kim Cương mài thu hết thì sự kiện đó có nghĩa là chiếc vòng ấy là của Thái Thượng Lão Quân (bởi chỉ riêng Thái Thượng Lão Quân có chiếc vòng này). Chiếc vòng kim cương mài là biểu tượng của trí tuệ vượt ra khỏi nhị nguyên tính, có thể thoát ly, đoạn trừ các chấp thủ ``nhị thủ'' như Kim Cương có thể chặt đứt các vật thể khác. Tôn Ngộ Không chỉ thất bại trước chiếc vòng này, vì chưa trừ hết tập khí chấp thủ.

Độc Giốc là kẻ phàm phu nên bị chiếc quạt Bát Phong của Thái Thượng Lão Quân thu phục và thu lại chiếc vòng Kim Cương mài (còn phàm phu là còn vướng vào khen, chê, được, mất, thị, phi, danh vọng và lợi dưỡng).

-- Khi biết nguồn gốc của chiếc vòng thì liền biết Độc Giốc là gia nhân của Thái Thượng Lão Quân và việc hàng phục Độc Giốc trở nên dễ dàng.

-- Bài học giải thoát của phái đoàn Tây Du ở đây là an trú vào chánh niệm Vô Ngã, an trú vào Vô tướng tâm định mới thoát khỏi ma nạn; mới hàng phục được dục ái, hữu ái và vô hữu ái. An trú vào Vô tướng tâm định là ý nghĩa chiếc vòng mà Ngộ Không đã vạch cho Đường Tăng, Ngộ Năng, Ngộ Tịnh và Tiểu Long Mã ẩn trú. Phương chi Tôn Ngộ Không nói: \emph{``Chỉ vì không tin cái vòng của con, thầy mới lọt vào cái vòng của người khác, biết bao khổ sở! Than ôi!''} (hồi 53).
% section tư_tưởng_phật_học (end)

\section{Quan niệm về con Người} % (fold)
\label{sec:50_51_con_nguoi}

-- Khác với con người đi vào giải thoát, con người xã hội mãi sống với những khát vọng hạnh phúc trần gian. Khát vọng này đã nuôi dưỡng các ngã tưởng và nhìn thấy thế giới giá trị của nhị nguyên tính thiết thực hơn và gần gũi với mình hơn. Do vậy, ngay cả khi sống trong một xã hội dân chủ, hợp tình hợp lý nhất, thì cái ngã tưởng, ngã niệm ấy cũng tự động xây dựng thế giới giá trị của nhị nguyên tính trong tâm thức. Sự kiện này làm dấy khởi các tâm lý dao động khiến tâm lý đi ra khỏi thế giới giá trị của nền văn hóa giáo dục nhân bản và trí tuệ. Đi ra khỏi thế giới giá trị của vị tha, nhân bản và trí tuệ là nghĩa được Ngô Thừa Ân biểu tượng bằng sự đi ra khỏi cái vòng vẽ an toàn của Tôn Ngộ Không. Đi ra khỏi cái vòng vẽ an toàn ấy là đi vào cái vòng Kim Cương mài đầy khổ lụy của Độc Giốc quỷ vương.

-- Khi tâm thức tự dẫn thân vào giá trị nhị nguyên của các ngã tưởng (như Đường Tăng tự bước vào ngôi nhà của Độc Giốc) thì cái nền văn hóa giáo dục nhân bản vả Vô Ngã trở nên mất tác dụng ở bên ngoài xã hội, như Tôn Ngộ Không hai lần bị Độc Giốc cướp mất khí giới. Vấn đề chủ yếu của nền văn hóa mới là tạo nên một xã hội công bằng, nhân ái, hòa hợp và đoàn kết để con người có điều kiện sống thuận lợi thực hiện an lạc, hạnh phúc của tâm thức. Nếu bên trong của tâm thức đã nắm giữ cái nhân của phiền não, rối ren (nhị nguyên tính) thì an lạc, giải thoát của cá nhân sẽ bị đánh mất. Rất khó đánh thức tâm thức cá nhân trong trường hợp này, bởi khi cá nhân thiếu tỉnh giác Vô Ngã thì thế giới bên trong và bên ngoài trở thành của nhị nguyên, của vô thường và khổ đau. Bấy giờ trí tuệ của cá nhân khó trỗi dậy, nếu công phu huấn luyện tâm yếu. Chỉ có một lối thoát duy nhất là hành sâu thiền quán Vô Ngã để thấy rõ cái thật tướng Vô Ngã của phiền não. Thấy như vậy là thấy tận đầu nguồn của phiền não, hữu vi. Như bài học Đức Phật đã chỉ dạy Tôn Ngộ Không và phái đoàn Tây Du qua sự mách nước chỉ rõ gốc gác của Độc Giốc và chiếc vòng Kim Cương mài.

Tại đây, Ngô Thừa Ân nhấn mạnh đến vai trò giáo dục cá nhân và trách nhiệm cá nhân trong nền văn hóa giáo dục mới, Mỗi người cần tự giác, tự nguyện nuôi dưỡng các giá trị nhân bản, hiện thực và trí tuệ cho mình, cho người và cho xã hội.
% section quan_niệm_về_con_người (end)

\section{Quan niệm về xã hội} % (fold)
\label{sec:50_51_xa_hoi}

-- Cái trở lực nguy hiểm hơn cả là ở chính lực lượng cải tổ văn hóa. Nếu lực lượng này chấp thủ các giá trị mới của nền văn hóa giáo dục mới, hay nghi ngờ sự thật Vô Ngã thì liền biến ngay lực lượng cải tổ ấy thành công bộc của nền văn hóa cũ, một xã hội phong kiến không có vua. Vị vua thực sự thống trị xã hội bấy giờ là tư duy hữu ngã mà Ngô Thừa Ân đã biểu trưng bằng Độc Giốc quỷ vương thần thông quảng đại với chiếc vòng phép mầu vô lượng, còn có sức mạnh tàn phá hơn cả Tề Thiên Đại Thánh thời kỳ đại náo Thiên cung. Đây là điểm tế nhị của con đường văn hóa giáo dục mới đặt trọng tâm vào con người và hạnh phúc của con người sống trong một xã hội hợp lý của nhân bản và trí tuệ.

Trong nền văn hóa mới này, các hành động cá nhân và tập thể lệch lạc là do nhận thức và tư duy sai lầm. Điểm quan trọng là cá nhân và tập thể luôn luôn trở về với nhận thức và tư duy đúng (ý nghĩa của sự việc Như Lai mách nước cho biết rõ cái gốc của quỷ vương Độc Giốc). Đi ra khỏi nhận thức và tư duy ấy là đi ra khỏ quỹ đạo giá trị của nền văn hóa mới, và sẽ rơi vào quỹ đạo giá trị khác, như Đường Tăng, Ngộ Năng và Ngộ Tịnh đi ra khỏi cái vòng của Tôn Hành Giả để rơi vào sào huyệt của Độc Giốc.

-- Những cá nhân và tập thể không có nhận thức và tư duy đúng, hay có tà ý lợi dụng giá trị của nền văn hóa mới để thỏa mãn các ý đồ, dục vọng riêng đều đi vào các hành động sai lầm, làm trở ngại bước phát triển của nền văn hóa mới. Hệt như sai lầm của Đường Tăng và sự lạm dụng chiếc vòng Kim Cương mài của Độc Giốc, tạo nên cảnh can qua ở núi Kim Đầu. Những nhân, nghĩa, lễ, trí, tín, trung, hiếu, tình được quần chúng thể hiện, nhưng là thể hiện qua một nhận thức mới và một thái độ sống mới, cởi mở, sinh động, đầy tính người, như nhận thức và thái độ sống của Tôn Ngộ Không (mà thường bị ngộ nhận bởi Đường Tăng và Bát Giới do bị ràng buộc với các quan niệm giá trị cũ).

Mãi mãi con người xã hội vẫn cần đến thái độ sống nghiêm chỉnh (Giới), kiên trì (Định) và sáng suốt (Tuệ) để hành xử.
% section quan_niệm_về_xã_hội (end)
% chapter Hồi 50_51 (end)