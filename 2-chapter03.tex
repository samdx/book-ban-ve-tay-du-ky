\chapter{Hồi 3} % (fold)
\label{cha:hoi_3}

\begin{itshape}
``Bốn bể nghìn non đều sợ nép.

Chín u, mười loại xóa tên rồi.''
\end{itshape}

\section{Tư tưởng Phật Học} % (fold)
\label{sec:3_phat_hoc}

--- Sau khi dẹp xong giặc Hỗn Thế Ma Vương, Tôn Ngộ Không tiến hành việc tổ chức xã hội khỉ để giữ yên bờ cõi núi Hoa Quả. Bấy giờ có bốn trưởng lão khỉ hỏi về thần thông của Tôn Ngộ Không, Tôn Ngộ Không bảo:

\begin{itshape}
``Ta từ sau khi đắc đạo, có công luyện tập bảy mươi hai phép địa sát và biến hóa, được phép thần thông cân đẩu vân không gì sánh bằng; ẩn mình, tránh mình, cất mình lên, thu hình lại, lên trời cũng có đường, xuống đất cũng có lối, bước vào mặt trời mặt trăng không có bóng, đi vào vàng đá không vướng mắc, nước không thể làm chìm, lửa không thể đốt cháy, chỗ nào mà chẳng đi được''.
\end{itshape}

Như đã đề cập ở hồi hai, tại Tà Nguyệt Tam Tinh động, Tôn Ngộ Không đã đắc pháp nhãn thấy rõ sự vật Vô Ngã, vì sự thật của mọi hiện hữu là Vô Ngã nên sự kiện một người có thể theo ý muốn hiện ra nhiều ngã tướng khác nhau. Tất cả hiện hữu đều Vô Ngã nên đều nhiếp nhau, không ngăn ngại, nên lửa không thể đốt cháy, nước không thể cuốn trôi, gió không lay, đất không cản, vv\ldots

``Lý'' của sự thật là như vậy nên ``Sự'' của sự thật cũng như vậy. Ở cảnh giới Tứ Sắc Định, hành giả có thể có thần không du hành, biến hóa tự tại. Tâm thức có giải thoát Vô Ngã cũng thế, tùy duyên mà thị hiện. Đây là sự thật như thật được ghi chép trong kinh điển Phật Giáo.

Về nghĩa bóng, thì với một người ngộ Vô Ngã mà không có thần thông biến hóa, tâm thức họ cũng được tự tại trước mọi sắc tướng như Tôn Ngộ Không tự tại ra vào các cảnh giới mà tâm lý không ái trước, không bị vướng mắc --- sẽ thấy rõ hơn về sự thật này ở các hồi sau --- Nhưng, để tâm thức an ổn có điều kiện đi sâu hơn vào giải thoát, người ngộ Vô Ngã cũng cần tránh các sự quấy rối từ bên ngoài, như giang sơn Thủy Liêm động cần được giữ gìn an ninh. Đây cũng là một sự thật của hành giả đang còn có nhiều việc phải làm trên đường tới giải thoát tối hậu.

--- Về sự kiện trong giấc mơ Tôn Ngộ Không xuống Âm phủ đại náo và xóa tên sinh tử của bản thân và của loài khỉ, Âm Ty phải cử sứ thần lên tấu với Ngọc Hoàng xử trị mà Ngọc Hoàng lại hòa giải với Tôn Ngộ Không, là sự kiện hoàn toàn biểu tượng.

Thực sự, mức độ tâm linh giải thoát của Tôn Ngộ Không bấy giờ tự nó đã xóa sạch nhân sinh tử đi vào các cảnh giới súc sinh, địa ngục và ngạ quỷ, theo giáo lý nhà Phật. Giải thoát của Tôn Ngộ Không bấy giờ có phước báo còn lớn hơn cả vua Trời, dù vua Trời có muốn hại cũng không được. Sự kiện giải thoát ấy là một thành quả lớn đáng kể của công phu tu tập làm kinh động các cảnh giới sinh tử đã khiến cho triều thần các cảnh giới ấy ganh ghét, đố kỵ khi mà tâm thức họ đang đầy ắp chấp thủ các tự ngã.

Quả thật Ngô Thừa Ân đã viết đầu hồi ba ngày:

\begin{itshape}
``Bốn bể nghìn non đều sợ nép.

Chín u, mười loại xóa tên rồi.''
\end{itshape}

% section tư_tưởng_phật_học_ (end)

\section{Quan niệm về con Người} % (fold)
\label{sec:3_quan_niem_con_nguoi}

--- Cái nhân cách giáo dục mà Ngô Thừa Ân quan niệm, đã được biểu hiện qua Tôn Ngộ Không; sau khi được đào tạo giáo dục từ hệ thống giáo dục Vô Ngã của Tôn giả Tu Bồ Đề thì liền đi vào xã hội để xây dựng xã hội và con người xã hội.

Với trí tuệ Vô Ngã, với nhận thức mọi hiện hữu đều Vô Ngã nên soi thấy mọi giá trị trong cuộc sống cũng Vô Ngã, cũng đều không có một giá trị nhất định (hay chỉ có giá trị tương đối). Các giá trị ước lệ của xã hội về con người và xã hội chẳng những đã không thể đứng vững mà còn gây xáo trộn cho tâm lý con người và gây xáo trộn xã hội. Tất cả các giá trị ước lệ ấy trở thành trói buộc con người và xã hội, kìm hãm khả năng giải phóng của con người và xã hội. Chúng cần được nhận thức mới xét lại, thay thế hay đập vỡ, dù giá trị ấy ở mỗi cá nhân hay tập thể.

Đó là những gì mà Ngô Thừa Ân đã mượn chiếc thiết bổng nặng hơn vạn cân của Tôn Ngộ Không đập nát loạn Hỗn Thế Ma Vương. Cần phải dọn sạch các giá trị \emph{``hỗn thế''} ấy, con Người rất Người mới có được một không khí trong lành để sống. Từ đây, con người sống với tâm lý tự do hơn, thoải mái hơn, có nhiều điều kiện sáng tạo hơn và làm được nhiều việc lành, lợi ích hơn cho bản thân, gia đình và xã hội. Đây là ý nghĩa mà Ngô Thừa Ân đã biểu tượng hóa thành các thần thông tự tại đi vào nước, vào lửa, vào đất, vào không gian của Tôn Ngộ Không.

Từ đây, mỗi người chịu trách nhiệm về hành động của chính mình đối với bản thân, gia đình và xã hội, mà không ai khác quyết định hay bắt ép. Giá trị của con người, và hành động của con người là do tự giác, tự nguyện và tự do chọn lựa của họ, mà không liên hệ gì đến việc thưởng phạt tại Âm phủ, Long phủ hay Thiên đình. Đây là ý nghĩa mà Ngô Thừa Ân đã lồng vào hình ảnh Tôn Ngộ Không xóa sạch sổ ghi thiện ác, sinh tử của loài khỉ (xứ sở của Tôn Ngộ Không) ở Âm ty.

--- Về cá nhân, mỗi người được giáo dục theo tinh thần Vô Ngã sẽ luôn luôn giữ hành động của mình được dẫn dắt bởi tâm ngay chính, nhân ái, trí tuệ và không vướng mắc, luôn luôn kiểm soát tâm mình và giữ tâm lý ở ngoài các \emph{niệm thị phi, đố kỵ, sầu, bi, khổ, ưu não}. Khi tâm lý được huấn luyện đến cấp độ không vướng mắc (hay ít bị vướng mắc) vào các tâm niệm đó thì tâm lý sẽ ở vào trạng thái tự do. Bấy giờ là thời điểm sáng suốt, sáng tạo và hạnh phúc cá nhân. Ở mặt giải thoát tương đối mà nhìn, thì đây là cảnh giới tâm lý tốt đẹp. Cảnh giới tâm lý tốt đẹp này thường xuất hiện trên đôi mắt của Tôn Ngộ Không mà chúng ta sẽ tiếp tục theo dõi.

% section quan_niệm_về_con_người (end)

\section{Quan niệm về xã hội} % (fold)
\label{sec:3_xa_hoi}

Các nhà lãnh đạo nếu sống thể hiện tinh thần Vô Ngã thì sẽ có sức mạnh vô úy, lòng vị tha, khoan dung, thái độ phóng khoáng, không câu nệ cố chấp, có tinh thần trách nhiệm cao, \ldots ~sẽ thu hút được quần chúng đoàn kết sau lưng mình trong mọi công tác giữ nước và dựng nước, như đàn khỉ quây quần chung quanh Tôn Ngộ Không vậy.

--- Nếu mọi người dân được giáo dục Vô Ngã, thâm nhập sự thật Vô Ngã, chấp nhận sự thật Vô Ngã thì sẽ rời xa các thái độ sống chấp thủ, sẽ dễ dàng đoàn kết thành một khối tạo nên sức mạnh lớn lao cho dân tộc.

--- Giáo dục Vô Ngã sẽ tạo nên một nền văn hóa Vô Ngã. Nền văn hóa này sẽ xóa đi các tư duy chấp thủ gây chia rẽ khối đại đoàn kết của dân tộc. Đây là một mục tiêu lớn và lâu dài của xã hội. Tinh thần Vô Ngã sẽ thống trị các dị biệt, gom tất cả về một mối phục vụ cho cùng một mục tiêu: xứ sở.

--- Khi mà đoàn kết dân tộc được thực hiện tốt, và khi mà nhân dân sống với thái độ không cố chấp thì mọi việc nước sẽ được thực hiện dễ dàng, như Tôn Ngộ Không tự tại đi vào nước, vào lửa, \ldots. Đất nước sẽ hùng cường khiến các nước lân bang không thể dòm ngó, không thể quyết định vận mệnh của dân tộc ấy. Vấn đề xâm lăng, bảo hộ sẽ không có đất đứng; sẽ bị xóa sạch khỏi lịch sử như Tôn Ngộ Không xóa sổ sinh tử ở Âm Ty cho loài khỉ ở Thủy Liêm Động.

% section quan_niệm_về_xã_hội (end)
% chapter hồi_3 (end)