\chapter{Hồi 36, 37, 38 và 39} % (fold)
\label{cha:hoi_36_37}

Hồi 36:

\begin{itshape}
``Tâm Viên đứng đắn thắng cơ duyên.

Trừ bỏ đạo tà lòe ánh sáng.''
\end{itshape}

Hồi 37:

\begin{itshape}
``Vua quỷ đương đêm cầu trưởng lão.

Ngộ Không hóa phép dẫn Hài nhi.''
\end{itshape}

Hồi 38:

\begin{itshape}
``Trẻ thơ hỏi mẹ hay tà chính.

Kim Mộc thăm dò rõ thực, hư.''
\end{itshape}

Hồi 39:

\begin{itshape}
``Một hạt kim đơn xin thượng giới.

Ba năm vua cũ về dương gian.''
\end{itshape}

\section{Tư tưởng Phật Học} % (fold)
\label{sec:36_37_phat_hoc}

--- Ở hồi 39, trong đoạn đối thoại giữa Bồ Tát Văn Thù và Tôn Hành Giả về thuyết Nhân Quả của Phật Giáo, tác giả Ngô Thừa Ân đã hơi gượng ép trình bày ra thuyết ``Tiền định'' của Nho Giáo không phù hợp với Nhân Quả Phật Giáo --- không biết có do sự xen kẽ của các người sao chép đời sau sao chép không?! --- Tuy thế, cũng không hẳn là hoàn toàn xa Nhân Quả Phật Giáo. Thuyết ``Tiền định'' nói: \emph{``Nhất ẩm nhất trác giai do tiền định''} vẫn có nét na ná với Nhân Quả Phật Giáo, loại định nghiệp.

--- Về điểm tác giả trình bày cung Đâu Suất ở tầng trời Tam Thập Tam, thật ra trời Tam Thập Tam là cõi trời thứ hai của Dục Giới thiên, còn Đâu Suất là cõi trời thứ tư của Dục Giới thiên. Đây là một điểm khuyết khuyết nhỏ khác.

--- Cũng ở hồi 36, tác giả cho biết rõ thêm rằng: người tu tập giải thoát, hành Bồ Tát hạnh không phải ở đâu trong bất cứ trường hợp nào, việc biểu lộ lòng nhân ái, ôn hòa cũng đúng. Đối với hạng căn cơ như vị Tăng quan chỉ sợ uy quyền mà xem thường tu sĩ thì phải vận dụng tinh thần \emph{``tùy duyên nhi bất biến''}, phải thị uy như Tôn Ngộ Không đã ra uy mới nhiếp phục được.

--- Câu chuyện con sư tử của Bồ Tát Văn Thù xuống trần gian theo chỉ thị của Đức Phật là để khai sáng thêm cho phái đoàn Tây Du. Điểm trí tuệ được khai sáng ấy nhắc nhở Tôn Ngộ Không biết cần phải truyền đạt rõ ràng giáo lý nhà Phật hầu quần chúng có một \emph{Chánh Kiến, Chánh Tư Duy dẫn đến Chánh Ngữ, Chánh Nghiệp, Chánh Mạng, Chánh Tinh Tấn, Chánh Niệm và Chánh Định}. Đây là trí tuệ về Nhân Quả minh bạch đề cao các hành động thiện và trừng phạt các hành động ác (như câu chuyện của vua nước Ô Kê, hồi 37 và 38); và trí tuệ nhìn thấy rõ giá trị thực của mỗi hành động của mỗi người; giá trị ấy nằm ở cái thật tâm tác động lên hành động, mà không phải là cái tướng trạng biểu hiện ở bên ngoài của hành động, như hình ảnh vua Ô-Kê thật và vua Ô-Kê giả, Đường Tăng thật và Đường Tăng giả. Về tâm đại từ, đại bi cũng thế, luôn luôn ở về phía Giới, Định, Tuệ (có mặt Giới, Định, Tuệ hay an trú trong Giới, Định, Tuệ), như Đường Tăng thật chỉ có Ngộ Không, Ngộ Năng và Ngộ Tịnh kết hợp mới nhận ra. Vua Ô-Kê giả dù có có pháp thuật biến hóa ra một Đường Tăng giống hệt về thân hành, khẩu hành thì chân tướng của tâm (ý hành) cũng không thể nào giấu được cái hư giả của nó (vì không biết niệm chân ngôn ``khẩn cô nhi''). Mọi thiện nghiệp, ác nghiệp ở đời đều căn cứ vào ý hành mà thẩm định giá trị, như chỉ có hoàng hậu thật của vua Ô-Kê mới có thể nhận ra vua Ô-Kê nào thật hay Ô-Kê nào là giả.
% section tư_tưởng_phật_học (end)

\section{Quan niệm về con Người} % (fold)
\label{sec:36_37_con_nguoi}

--- Trong vấn đề giáo dục con người, cần có kỹ thuật giáo dục cho phù hợp với từng loại tâm lý, tính tình. Kỹ thuật giáo dục này áp dụng linh động cho từng trường hợp và từng cá nhân khác nhau. Với vị Tăng quan (hồi 36) chỉ sợ quyền lực thì cần ra uy để nhiếp. Với vị này, ngôn ngữ và thái độ lịch nhã của Đường Tăng không đem lại hiệu quả; có khi còn phản tác dụng.

--- Giá trị của hành động đạo đức của con người xã hội cần dựa vào tâm chủ động tác động trên hành động để đánh giá, như phải dựa vào cái tâm cái tình của vua Ô-Kê thật để phân biệt vua Ô-Kê nào là thật, vua Ô-Kê nào là giả, Đường Tăng nào là thật, Đường Tăng nào là giả. Nếu tiêu chuẩn giá trị của giải thoát là lòng đại bi và Giới, Định, Tuệ thì tiêu chuẩn giá trị đạo đức của xã hội phải dựa vào cái tâm thiện, chân thật vì lợi ích, an lạc, hạnh phúc của số đông, như phải dựa vào tấm chân tình của hoàng hậu Ô-Kê để phân biệt vua Ô-Kê thật, giả.

--- Cơ sở của đạo đức hay thiện, ác của con người cần dựa vào luật tắc tự nhiên của Nhân Quả, nghiệp báo: điều thiện là điều phải làm, cần làm cho mình và người vì nó sẽ đem lại an lạc, hạnh phúc cho mình và người trong hiện tại và cũng như tương lai. Điều ác thì ngược lại. Hành động đạo đức phải là hành động phát xuất từ tâm tự nguyện của mỗi người và từ chính nhận thức và ý chí muốn làm của người ấy, mà không phải vì bị áp đặt hay vì nhân danh một thế lực nào từ bên ngoài. Đây là hướng đạo đức nhân bản, theo đó, các quy ước về Hiếu, Trung, Tình, Nghĩa, Đễ, \ldots ~của văn hóa Trung Hoa đương thời cần được xét lại giá trị.
% section quan_niệm_về_con_người (end)

\section{Quan niệm về xã hội} % (fold)
\label{sec:36_37_xa_hoi}

--- Các giá trị đạo đức, luật pháp của một xã hội cần được thiết lập lại theo hướng văn hóa giáo dục nhân bản và trí tuệ. Theo đó tinh thần trách nhiệm cá nhân được đề cao làm cơ sở để thi hành luật pháp nghiêm minh. Thiếu tinh thần trách nhiệm cá nhân thì luật pháp sẽ không có cơ sở vững chắc để thi hành, xã hội sẽ đại loạn. Cần cai trị bằng luật pháp mà không phải bằng uy quyền của nhà vua, hay bằng hình phạt do nhà vua tùy tiện đặt ra để củng cố ngai vàng. Tinh thần trách nhiệm này nằm gọn trong giáo lý Nhân Quả, nghiệp báo của Phật Giáo.

--- Giáo lý Nhân Quả nghiệp báo được soi sáng bằng trí tuệ Vô Ngã sẽ là cơ sở đạo đức cho con người và xã hội. Con người có thể trốn chạy pháp luật để đi vào các hành động sai lầm gây xáo trộn các tổ chức xã hội, nhưng con người không thể chạy trốn lương tâm mình, không thể chạy trốn khỏi sự thật Nhân Quả. Giáo lý Nhân Quả, Nghiệp báo vì vậy sẽ hữu hiệu hướng dẫn con người vào đời sống lương thiện vì lợi ích của bản thân, và sẽ giúp xã hội gìn giữ công bằng, thi hành luật pháp nghiêm minh.

Biểu hiện của các hành động và kết quả việc làm của con người trong xã hội là quan trọng, nhưng thực chất giá trị của hành động vẫn nằm ở cái tâm thiện, đạo đức (được trí tuệ soi sáng) chủ động hành động. Giá trị sống này sẽ là nhân tố dựng nên một nền văn hóa giáo dục nhân bản và thiết thực --- như vừa đề cập ở phần quan niệm về con người.

Nhà lãnh đạo xứ sở cần được chính danh: có đủ tài, đức và tinh thần trách nhiệm, như là Tôn Ngộ Không đã đưa vua Ô-Kê thật trở về lại ngai báu. Nhà lãnh đạo có xứng đáng thì các cộng sự mới xứng đáng tương xứng, nhân dân mới sống hiền thiện và có nơi nương tựa tốt đẹp và vững chắc. Nhân dân có an cư lạc nghiệp thì xã hội mới hưng thịnh. Phải chăng đây là tâm sự, thao thức về một xã hội Trung Hoa tốt đẹp của Ngô Thừa Ân đã được lồng vào các hồi truyện 36, 37, 38 và 39?
% section quan_niệm_về_xã_hội (end)
% chapter Hồi 36 37 38 39 (end)