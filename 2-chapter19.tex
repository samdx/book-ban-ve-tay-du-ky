\chapter{Hồi 47, 48 và 49} % (fold)
\label{cha:hoi_47_48}

Hồi 47:

\begin{itshape}
``Thánh Tăng đêm vướng sông thiên thủy.

Hành giả làm ơn cứu Tiểu đồng.''
\end{itshape}

Hồi 48:

\begin{itshape}
``Ma nổi hàn phong sa tuyết lớn.

Sư cần bái Phật giẫm băng dày.''
\end{itshape}

Hồi 49:

\begin{itshape}
``Chìm đáy sông, Đường Tăng gặp nạn.

Hiện làm cá Bồ Tát trừ tai.''
\end{itshape}

\section{Tư tưởng Phật Học} % (fold)
\label{sec:47_48_phat_hoc}

-- Ở hồi 49, Ngô Thừa Ân viết:

\emph{``Thổ là mẹ Ngũ Hành, thủy là nguồn của Ngũ Hành; không có thổ, không sinh; không có thủy, không lớn''}.

Tòa nhà của quỷ Linh Cảm Đại vương ở Thông Thiên Hà có tên là ``Thủy Nguyên'' là biểu tượng cho gốc của Ngũ Hành, nuôi dưỡng Ngũ Hành. Ngũ Hành phối hợp với Âm Dương mà hình thành vũ trụ, thế giới hữu vi, nên cảnh giới của Ngũ Hành thông Tam giới và được gọi là Thông Thiên Hà.

Muốn đoạn trừ được gốc trưởng dường của Tam giới (hay Ngũ Hành) thì trí tuệ không thì chưa đủ, mà cần có Định sâu hỗ trợ. Tuệ cần nương Định để lặn sâu vào tâm thức, hành giả mới hoàn toàn chế ngự Ngũ Hành. Tại đây, Tôn Hành Giả thiếu Định lực sâu nên khó được tòa nhà của Linh Cảm Đại Vương, phải nhờ đến Đại Tuệ và Đại Định của Bồ Tát Quán Thế Âm. Qua khỏi nạn này là phái đoàn Tây Du chế ngự được Ngũ Hành. Ngũ Hành nay chỉ ở trạng thái tùy miên, đang chờ đợi đi đến đoạn diệt.

-- Bồ Tát Quán Thế Âm thu về con cá vàng Linh Cảm Đại Vương cho tiếp tục tu tập trong ao sen ở Nam Hải là biểu tượng nói lên tinh thần chuyển hóa của Phật Giáo; chuyển hóa cuộc đời đang là thành cảnh giới giải thoát, chuyển hóa cái nhìn hữu ngã thành cái nhìn vô ngã, chuyển hóa các tâm hữu lậu thành vô lậu, chuyển hóa sự giác tỉnh (được biểu trưng bằng con cá vàng, vì nó luôn luôn mở mắt dù thức hay ngủ) phàm phu thành sự giác tỉnh của bậc Thánh như việc con cá vàng đến ao Dao Trì từ Thông Thiên Hà.
% section tư_tưởng_phật_học (end)

\section{Quan niệm về con Người} % (fold)
\label{sec:47_48_con_nguoi}

-- Công cuộc cải tổ về văn hóa giáo dục cần được thực hiện song hành hai mặt: mặt nhận thức chủ quan của con người và mặt xã hội khách quan. Giá trị nhân bản, nhân ái, trí tuệ và giá trị liên hệ phát sinh từ nhận thức trí tuệ và nhân bản, khi đã được đông đảo quần chúng nhận rõ và chấp nhận, cần được bảo vệ. Các sức mạnh võ lực, vật chất có thể cướp đi giá trị ấy ở quần chúng, nếu sức mạnh kia đe dọa sinh mệnh của quần chúng, như Trần Lão là người tốt, biết cúng dường, bố thí, thương yêu con cháu mà phải hiến đồng nữ Nhất Xứng Kim và đồng nam Trần Quang Bảo cho quỷ Linh Cảm Đại Vương. Lực lượng cải tổ còn cần phải tạo một môi trường sống của xã hội thế nào để các giá trị nhân bản được nở hoa kết trái trong quần chúng.

-- Thế hệ tiếp thu nền văn hóa mới tốt nhất là thế hệ các đồng nam, đồng nữ chưa bị hằn sâu vào tâm các giá trị của nền văn hóa cũ. Đây là thế hệ lý tưởng để truyền đạt văn hóa mới. Tác giả Ngô Thừa Ân đã rất quan tâm đến thế hệ này nên đã để Ngộ Không và Ngộ Năng đánh dẹp quỷ Thông Thiên Hà, cũng như ở một hồi ma nạn về sau phái đoàn Tây Du còn trừ ma để cứu mấy trăm trẻ em trong hoàng thành.

-- Về nhận thức mới: văn hóa giáo dục mới có thể giới thiệu đên quần chúng một nhận thức đúng đắn về quan niệm sống mới, nhưng ngay cả khi nhận thức này được hình thành thì chỉ mới hiện diện ở bề mặt ý thức, chưa thấm sâu vào tâm thức, rất dễ bị các tập quán cũ làm đổ vỡ, như lớp băng hà mới chỉ đông ở ngoài. Cần biến các nhận thức mới chuyển hiện hành nếp sống mới, và nếp sống mới cần được xây dựng vững chắc, bền bỉ hầu tránh nạn ``Thông Thiên Hà'', những thế lực tà vạy cũ và mới sẽ cấu kết nhau với nhân danh bảo vệ truyền thống để làm sụp đổ các giá trị mới, như quỷ Linh Cảm Đại Vương và quỷ Bà-bà cấu kết hại Đường Tăng.
% section quan_niệm_về_con_người (end)

\section{Quan niệm về xã hội} % (fold)
\label{sec:47_48_xa_hoi}

-- Nền văn hóa Nho Giáo là nền văn hóa cổ của Trung Hoa cho đến thời đại Ngô Thừa Ân ở thế kỷ XVI đã trở thành một nếp văn hóa có truyền thống lâu dài. Các giá trị của nền văn hóa đó đối với nhân dân Trung Hoa đã khoác thêm tính chất thiêng liêng của hồn sông núi. Thật rất khó khăn cho các nhà cải cách văn hóa trong việc giới thiệu các giá trị của nền văn hóa giáo dục mới. Làm thế nào để có thể làm băng lạnh vùng văn hóa cũ mênh mông ấy dù là ở mùa đông của lịch sử? Ngô Thừa Ân đã ý thức được điều đó, nhưng tại sao lại còn mong muốn dựng nên bộ tiểu thuyết Tây Du Ký? Đây là vấn đề mà ngày nay chúng ta cần phân tích kỹ. Có lẽ Ngô Thừa Ân đã dựa vào các cơ sở sau đây để giới thiệu một cuộc cải tổ văn hóa cũ:

\begin{enumerate}[label=\itshape\arabic*\upshape/]
    \item Tất cả mọi hiện hữu ở trần gian đều có tuổi thọ của nó, nhất là các hiện hữu được sinh từ Âm Dương, Ngũ Hành. Nền văn hóa cũ của Trung Hoa cũng vậy; khi lịch sử đã cho thấy nền văn hóa cũ không còn đảm trách được vai trò lịch sử nữa, thì nó phải được gạn đục khơi trong.

    \item Nền văn hóa Vô Ngã, nhân bản và trí tuệ một mặt đáp ứng được các yêu cầu mới của lịch sử, một mặt thật sự có khả năng loại bỏ những mặt bảo thủ lỗi thời.

    \item Nền văn hóa mới này lại vừa có khả năng giúp con người phát triển mạnh hơn tiềm năng của cá nhân, xứ sở và các tương giao xã hội nên tạo được một sức thu hút mạnh quần chúng, vừa có khả năng dung hợp, sống chung với những nhân tố tích cực của nền văn hóa cũ. Nền văn hóa mới của Phật Giáo không gây xáo trộn hay đảo lộn nếp sống cũ, mà chỉ nhẹ nhàng đi vào hòa hợp với nếp sống cũ.

    \item Cuộc cải tổ văn hóa giáo dục này sẽ được thực hiện từng bước lâu dài trong nhận thức cá nhân và trong các cơ chế tổ chức xã hội, nên có thể bắt rễ sâu hơn và bền hơn vào lịch sử, và có thể tồn tại lâu dài song hành với trời, đất, bởi nó có cùng một sự thật với luật tắc tự nhiên của vạn hữu.
\end{enumerate}

Tuy vậy, Ngô Thừa Ân vẫn dựng nên ma nạn Thông Thiên Hà để cảnh giác công cuộc cải tổ cần phải hết cẩn trọng trong từng bước đi cải tổ, sức đề phòng mọi sức đề kháng, nhất là sức đề kháng của tập tục cũ mà Ngô Thừa Ân gọi là động Thủy Nguyên.

Nếu lực lượng cải tổ thực hiện song hành hai sức mạnh đại trí và đại bi thì sẽ thành công, hệt như nhờ đến bàn tay của Bồ Tát Quán Thế Âm mới thu phục được con cá vàng Linh Cảm Đại Vương (con cá vàng này cũng có gốc từ ao Dao Trì ở Phật cảnh mà chỉ vì chưa đổi thay dục vọng, chưa rửa được lòng trần chấp ngã). Đây là một cảnh ma nạn rất quan trọng, rất nghiêm trọng, có thể nói là tối ư thiêng liêng của con người. Đối với quần chúng, vì thế Ngô Thừa Ân đã xây dựng hình ảnh Bồ Tát rất vội vàng, khẩn cấp cứu nguy Đường Tăng đến nỗi không kịp choàng áo khoác ngoài như mọi khi --- một việc rất hiếm xẩy ra với Bồ Tát.
% section quan_niệm_về_xã_hội (end)
% chapter Hồi 47_48 (end)