\chapter{Hồi 20, 21 và 22} % (fold)
\label{cha:hoi_20_21_22}

Hồi 20:

\begin{itshape}
``Núi Hoàng Phong, Đường Tăng gặp nạn

Đón giữa núi, Bát giới lập công''
\end{itshape}

Hồi 21:

\begin{itshape}
``Hộ pháp làm nhà lưu Đại Thánh

Tu Di Linh Cát bắt Phong Ma''
\end{itshape}

Hồi 22:

\begin{itshape}
``Bát Giới đánh ở sông Lưu Sa

Mộc Xoa vâng lệnh bắt Ngộ Tịnh''
\end{itshape}

\section{Tư tưởng Phật Học} % (fold)
\label{sec:20_21_22_phat_hoc}

— Hành giả tu tập giải thoát nếu chỉ hành trì giới thì khó nhiếp phục được cái động và cái vọng tâm. Tâm động chưa bị hàng phục thì khó đi vào tuệ giải thoát. Công phu tu tập giới là công phu chuẩn bị căn bản giải thoát để tu tập định. Công phu tu tập giới là công phu chuẩn bị căn bản giải thoát để tu tập định. Công phu đến hồi 20 hành giả hẳn là vượt qua ma nạn Hoàng Phong. Ma nạn này là do pháp hữu vi ``tam muội thần phong'' chắn lối. Tôn Ngộ Không và Trư Ngộ Năng còn non định lực nên bị gió định khắc chế làm tổn hại đến đôi mắt tuệ của Ngộ Không. Phải nhờ đến định lực vững như núi Tu—di (gọi là Tiểu tu—di) của Bồ Tát Linh Cát mới đối trị được quái Hoàng Phong (hồi 20). Sau nạn này, hành giả (phái đoàn Tây Du) phải chuyên tâm hành thiền định. Đây là ý nghĩa phái đoàn Tây Du kết nạp thêm Sa Ngộ Tịnh (như đã đề cập ở Tổng luận, hồi 22).

— Phương pháp tu của Phật Giáo để chế ngự tâm, điều phục tâm là phương pháp khắc chế. Ngô Thừa Ân giới thiệu phương pháp tu ấy qua các phương cách phái đoàn Tây Du chiến tháng các yêu quái. Với ``tam muội thần phong'' gió độc và mạnh, thì người ta dùng ``định phong đan'' để trừ. Với đối tượng động thì dùng tĩnh để chế; với đối tượng tĩnh, thì dùng động (của tuệ) để khắc.
% section tư_tưởng_phật_học (end)

\section{Quan niệm về con Người} % (fold)
\label{sec:20_21_22_con_nguoi}

— Mức độ cao hơn về giáo dục con người là mức độ ``huấn luyện tâm lý'', hay thực hành thiền định. Giáo dục có mục đích đem lại cho con người sự ổn định tâm lý và cảm nhận an lạc, hạnh phúc. Giáo dục thiền định vì thế là yêu cầu của học đường, của gia đình và xã hội.

— Trạng thái tâm lý ổn định là trạng thái tự do và sáng tạo, loại bỏ các tâm lý mệt mỏi, chán nản, lười biếng và lo âu, và có khả năng nâng cao hiệu năng làm việc, học hỏi hay giảng dạy. Đây là vai trò chính của ngành Tâm lý giáo dục hiện đại.
% section quan_niệm_về_con_người (end)

\section{Quan niệm về xã hội} % (fold)
\label{sec:20_21_22_xa_hoi}

— Hoàng Phong quái là loại quái vật sinh sống ở cảnh giới Vô Ngã, nhưng do tu luyện chưa hết tà khí, chưa an trú vào Giới uẩn, Định uẩn, nên gặp duyên thì tà niệm phát tán gây nên tai họa lớn về ``tam muội thần phong''. Đây là hình ảnh biểu tượng về hậu quả của sự cải tổ văn hóa giáo dục thiếu triệt để và nghiêm túc, và là biểu tượng về hậu quả tai hại của sự áp dụng sai lạc do không thấu triệt chân giá trị của nền văn hóa, giáo dục Vô Ngã.

— Để tránh hậu quả trên, lực lượng cải tổ xã hội cần được giáo dục kỹ nội dung của văn hóa Vô Ngã và sự trung thành quyết tâm thể hiện nên văn hóa này, một nền văn hóa của tinh thần ``tùy duyên nhi bất biến''. Đấy là ý nghĩa của hai khí giới mà Như Lai trao cho Linh cát Bồ Tát để hàng phục quái Hoàng Phong gồm có:

— ``Định phong đơn'' biểu trưng của tinh thần bất biến.

— ``Phi long trượng'' (con rồng vàng chín móng) biểu tượng cho tinh thần tùy duyên, vận dụng uyển chuyển và sinh động.

Nếu tinh thần Văn hóa Vô Ngã được hiểu một cách khẳng định như văn hóa hữu ngã, thì Vô Ngã biến thành ngã vô (tà pháp), và hậu quả của nền văn hóa này sẽ gây nên cảnh ``hỗn thế'' không khác gì với nền văn hóa hữu ngã của xã hội cũ. Nền văn hóa mới này không đối lập với bất cứ một nền văn hóa nào khác cả, vì thế nó có thể cùng tồn tại bên cạnh nền văn hóa cũ. Giá trị đích thực của nền văn hóa này là ở cái tâm hành động được điều động bởi tâm lý vị tha, nhân ái và bởi trí tuệ không chấp thủ, không tham ái, sân hận. Giá trị của cuộc sống là ở chữ Tâm, như những gì mà Tôn Ngộ Không đã được giáo dục ở lò luyện tâm (Tà Nguyệt Tam Tinh động). vì giá trị nằm ở cái tâm, nên cái tâm con người cần được làm chủ trước, cải thiện trước.
% section quan_niệm_về_xã_hội (end)
% chapter Hồi 20 21 22 (end)