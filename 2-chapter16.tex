\chapter{Hồi 40, 41, 42} % (fold)
\label{cha:hoi_40_41}

Hồi 40:

\begin{itshape}
``Trẻ thơ bởn cợt lòng thuyền rối

Vượn múa đao về, mộc mẫu trở''
\end{itshape}

Hồi 41:

\begin{itshape}
``Đại Thánh lửa đốt trại

Bát Giới ma bắt đi''
\end{itshape}

Hồi 42:

\begin{itshape}
``Đại Thánh ân cần cầu Bồ Tát 

Quan Âm từ thiện trói Hồng Hài''
\end{itshape}

\section{Tư tưởng Phật Học} % (fold)
\label{sec:40_41_phat_hoc}

— Trên đường giải thoát, sau khi qua được dục ái và chế ngự được cái thân hành và khẩu hành, hành giả lại trải qua các ải sắc ái, các trói buộc của các thiền tâm trong các cảnh giới thiền định. càng đi xa, các trở ngại càng lớn và càng nguy hiểm, nếu thiếu chánh niệm tỉnh giác và thiếu định lực. Tại hồi 40, 41 và 42, phái đoàn Tây Du lại mắc nạn Hồng Hài Nhi với pháp thuật ``tam muội hỏa'' là lửa thiền định, phát sinh từ thiền định của ngoại đạo (ma giới). Lửa tam muội của Hồng Hài Nhi là biểu tượng cho lửa tham ái, chấp thủ mạnh mẽ, thật mạnh mẽ.

Chiến thắng được lửa này phải có định sâu và tuệ mạnh. Bấy giờ Đường Tăng và Bát Giới bị hãm đã đành, mà Tôn Ngộ Không cũng bị lửa đốt suýt bị hại do vì định chưa sâu và tuệ chưa thâm. Do vậy phải nhờ Bồ Tát Quán Thế Âm để cầu cứu nạn là mỗi lần Tôn Ngộ Không học hỏi thêm được một bài học giải thoát, tăng trưởng định và tuệ giải thoát.

Tại hồi 42 này, Bồ Tát đã rất kín đáo dạy cho Tôn Ngộ Không cái diệu dụng của thế giới Vô Ngã ấy qua hình ảnh ném chiếc tịnh bình xuống bể để thu vào nước của bốn bể và các sông, hồ, vv\ldots. Hình ảnh biểu tượng ấy nói lên sự thật của thật tướng vô tướng: một nhiếp tất cả, tất cả nhiếp vào một; một nhiếp cả pháp giới, pháp giới nhiếp vào một; tất cả là tương dung tương nhiếp và vô ngại. Chỉ có nước của thật tướng đó mới có năng lực dập tắt ``tam muội hỏa'' kia.

— Hồng Hài Nhi thì có định sâu, có trí linh lợi, nhưng lại sa vào tà kiến và dục ý mạnh. Phải có trí tuệ Vô Ngã thâm huyền của Bồ Tát Quán Thế Âm mới chuyển hóa được Hồng Hài thành Phật tử Thiện Tài để vào sâu giải thoát, hầu cạnh Bồ Tát. Cũng thế, hành giả hành thiền định có định lực cao thì cần hành thiền quán Duyên sinh Vô Ngã mới có thể đi vào tâm giải thoát và tuệ giải thoát (toàn phần giải thoát).
% section tư_tưởng_phật_học (end)

\section{Quan niệm về con Người} % (fold)
\label{sec:40_41_con_nguoi}

— Hình ảnh biểu tượng quỷ Hồng Hài Nhi hóa hiện một trẻ em bị treo trên cành để đánh lừa Đường Tăng hầu hãm Đường Tăng đã che mắt được Đường Tăng, Ngộ Năng và Ngộ Tịnh. Chỉ có Ngộ Không là thấy rõ quỷ tướng của Hồng Hài Nhi, nhưng lại không thuyết phục được phái đoàn Tây Du. Kết cục Hồng Hài Nhi đã bắt được Đường Tăng vào động Hỏa Vân. Tại đây, Ngô Thừa Ân đã đặt vấn đề giá trị đạo đức của nền văn hóa giáo dục nhân bản với xã hội phong kiến Trung Hoa rằng:

a/ Cái giá trị nhân ái đạo đức có tính cách công thức rất ước lệ và hình thức của nền văn hóa cũ (tiêu biểu là hình thức nhân ái, đạo đức của Đường Tăng) cần được xét lại giữa đời sống xã hội có lắm biến động và đầy dẫy những âm mưu ác hại của tà phái. Nhân ái, đạo đức cần được biểu hiện đúng chỗ, đúng thời và đúng người mới thực sự có ý nghĩa là nhân ái, đạo đức. Nhân ái, đạo đức đòi hỏi có mặt trí tuệ, được chỉ đạo, hướng dẫn bởi trí tuệ. Nếu thiếu trí tuệ thì hành động nhân ái, đạo đức chỉ gây tổn hại mình và người sẽ không thể được chấp nhận là nhân ái, đạo đức đúng nghĩa là hành động đem lại an lạc, hạnh phúc cho mình và người.

b/ Tương tự, sự dối gạt cũng có hai giá trị khác nhau: sự dối gạt để hại người và để thỏa mãn dục vọng, sân hận của mình là bất thiện, tà đạo, như sự dối gạt của Hồng Hài Nhi. Sự dối gạt để cứu người, cứu đời là thiện là chánh giáo, như sự dối gật của Tôn Ngộ Không được vận dụng vì mục tiêu cứu sư phụ phái đoàn Tây Du để cứu vớt dân Đông Độ.

Không thể khẳng định mọi hành vi dối gạt là phi đạo đức theo giá trị ước lệ của xã hội cũ, nếu sự dối gạt là cần thiết, đầy trí tuệ, là đạo đức. Đây là quan niệm mới mẽ về giá trị, đạo đức đáng được thay thế các quan niệm về đạo đức của nền văn hóa Nho Giáo cũ.

Nếu bạn bực bội về hành vi dối gạt của Hồng Hài Nhi, hay hành vi cứu người một cách mù quáng của Đường Tăng, thì bạn sẽ bực bội như thế về các giá trị ước lệ về đạo đức của nền văn hóa Nho gia. Nếu bạn hoan hỷ trước hành vi tản lờ trong việc cứu Hồng Hài, thì bạn sẽ hoan hỷ như thế về nền văn hóa nhân bản của trí tuệ.
% section quan_niệm_về_con_người (end)

\section{Quan niệm về xã hội} % (fold)
\label{sec:40_41_xa_hoi}

— Sự kiện Hồng Hài Nhi khéo dối gạt Đường Tăng, dối gạt lòng tốt, gợi ý về một bài học về việc tìm hiểu các hiện tượng xã hội. Bài học ấy là nhà xã hội học, hay người có trách nhiệm đối với xã hội, chỉ khi nào hiểu rõ nguyên nhân chính của một hiện tượng thì mới hiểu rõ hiện tượng ấy và thấy rõ con đường phát triển (nếu là hiện tượng tốt) hay dập tắt (nếu là hiện tượng xấu). Các cuộc điều nghiên vội vàng, hời hợt về các hiện tượng xã hội sẽ dẫn đến các sai lầm đáng tiếc. Đường Tăng là một trường hợp điển hình về cái nhìn xã hội thiếu trí tuệ, rơi vào bẫy gạt của trẻ Hồng Hài.

— Hiện tượng Hồng Hài Nhi nói rõ rằng: một xã hội có đường hướng, chính sách tốt cần được thực hiện bởi những trái tim và khối óc tốt như Tôn Ngộ Không — sáng suốt, vị tha — mà không phải bởi những con người đầy dục vọng, vị kỷ như Hồng Hài Nhi.

— Những người có khả năng mà thiếu tấm lòng tốt thì hầu như có nhiều trong xã hội con người, các nhà lãnh đạo xứ sở của nền văn hóa mới cần quan tâm đối phó hay thu phục.

— Các mục tiêu tốt cần được thực hiện bằng các phương tiện tốt tương xứng, như mục tiêu thỉnh kinh để cứu dân Đông Độ cần được phái đoàn Tây Du thực hiện qua một cuộc viễn trình đầy gian khổ, kiên trì, quyết tâm và hiền thiện, mà không phải qua dục vọng, vị kỷ như Hồng Hài Nhi muốn ăn thịt Đường Tăng để được trường sinh.

— Công cuộc xây dựng nền văn hóa giáo dục mới cần được tính toán kỹ, dự phòng những khó khăn trong từng bước thực hiện để khắc phục, như Thế Tôn đã trước cho Bồ Tát Quán Thế Âm năm bửu bối mà hai trong số bửu bối ấy là để hàng phục Hồng Hài Nhi. Nếu chờ xúc sự mới ứng phó thì khó vượt qua các bất trắc.

Ngô Thừa Ân đã rất tinh tế trình bày tư duy của tác giả qua từng ma nạn. Chúng ta hãy tiếp tục theo dõi những gì Ngô Thừa Ân muốn nói vè việc xây dựng con người và xã hội.
% section quan_niệm_về_xã_hội (end)
% chapter Hồi 40_41 (end)