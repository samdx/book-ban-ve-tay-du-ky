\chapter{Hồi 4, 5, 6, 7} % (fold)
\label{cha:hoi_4_5_6_7}

Hồi 4:

\begin{itshape}
``Quan phong Bật Mã lòng sao xứng.

Tên gọi Tề Thiên dạ chẳng yên.''
\end{itshape}

Hồi 5:

\begin{itshape}
``Loạn vườn đào Đại Thánh trộm thuốc tốt.

Về Thiên cung các thần bắt yêu quái.''
\end{itshape}

Hồi 6:

\begin{itshape}
``Quan Âm đến hội hỏi nguyên do.

Tiểu thánh ra oai thu Đại thánh.''
\end{itshape}

Hồi 7:

\begin{itshape}
``Đại Thánh trốn khỏi lò bát quái.

Hầu Vương bị giam núi Ngũ Hành.''
\end{itshape}

{\bf Lược truyện hồi 4, 5, 6 và 7:}

Từ chuyện Tôn Ngộ Không xuống Long Vương lấy thiếc bổng để thêm khí giới giữ bờ cõi Hoa Quả Sơn, đại náo Long cung, vua Trời muốn xử hòa với Tôn Ngộ Không đã không thành thật phong chức giữ ngựa Trời, rồi Tề Thiên Đại Thánh, rồi chức giữ vườn đào. Các chức ấy đều là hư vị đặt ra do sự dối trá (không thật ý của Thiên đình). Hậu quả của sự dối trá ấy dẫn đến cuộc đại náo Thiên đình của Tề Thiên Đại Thánh khiến Tề Thiên Đại Thánh bị giam vào núi Ngũ Hành năm trăm năm.

\section{Tư tưởng Phật Học} % (fold)
\label{sec:4_phat_hoc}

— Như đã được đề cập ở phần {\bf Tổng Luận}, các hồi 4, 5, 6 và 7 là các hồi chiến cuộc xẩy ra giữa Tôn Ngộ Không và Thiên đình. Tất cả chỉ là những hình ảnh biểu tượng nói về Phật Học và quan niệm về nhân sinh xã hội của tác giả. Không phải mọi hình ảnh đều chuyên chở ý nghĩa về Phật Học được hiểu một cách máy móc.

Chúng ta cần để ý đến các nét chính nổi bật nhất của bối cảnh, rồi tìm hiểu các nét chính ấy có mang biểu tượng nào của Phật Học.

— {\bf Để phát hiện các biểu tượng ấy, chúng ta cần phân biệt rõ ràng:}

\begin{enumerate}[label=\itshape\alph*\upshape/]

    \item Các cõi Trời được đề cập đến trong Tây Du Ký này đều là các cõi trời Dục giới. Các cõi ấy là các cõi thiện: nhờ phước báo thiện hạnh của tiền kiếp, tùy theo phước báo lớn nhỏ mà được làm Ngọc Hoàng, Thiên tướng, Thiên quan hay Thiên dân. Các phước báo kia đều thuộc hữu lậu, và các vị Trời kia cũng còn đầy \emph{ham muốn, bỉ thử, thị, phi}, tương tự ở cõi Người (chỉ ở cấp độ thanh cao hơn). Các cảnh giới thuộc trời Dục giới vẫn còn nạn chiến tranh với A-tu-la do bởi các cuộc tranh chấp về quyền hạn, hay đố kỵ về hạnh phúc.

    \item Quả vị giải thoát của Pháp nhã (Dự Lưu, hay Thất Lai, hoặc Tu-đà-hoàn) như ở cấp độ giải thoát của Tôn Ngộ Không là quả vị ở ngoài các tâm đố kỵ hay tranh chấp.
\end{enumerate}

— Mỹ Hầu Vương, tánh vốn phóng khoáng, lại sống nhập thế tự tại, muốn làm chủ mọi hành động của bản thân mà không câu nệ sự tướng — bởi thấy tất cả đều rỗng không tự ngã — nhưng phía chúng sinh ở cõi Trời, Người thì khác, do vậy xẩy ra các xung đột.

— Tác giả đã vận dụng cuộc đại náo Thiên cung và sự chiến thắng Thiên đình của Tôn Ngộ Không để nói lên rằng tâm thức của Tôn Ngộ Không bấy giờ đã vượt ra khỏi chấp thủ {\bf Lục Đạo} (\emph{Thiên, Nhân, A-tu-la, Địa ngục, Ngạ quỷ, Súc sinh}). Tên gọi Tề Thiên Đại Thánh vì thế được khoa trương.

— Với đôi mắt thánh thiện của Đại Thánh, cõi Trời hiện ra cái bản tướng lẩm cẩm của nó; tất cả chúng Trời đang ngủ yên trong dòng nghiệp của Trời, thiếu nhân duyên để thấy đạo, thấy Vô Ngã, Vô Thường của cảnh trời. Cuộc đại náo có tác dụng đánh thức các cung viện và Thiên chúng, giúp Thiên chúng mở sáng đôi mắt tuệ. Cuộc đại náo, như đã đề cập ở phần \textbf{Tổng Luận}, còn có nghĩa là đại náo trong tâm thức của Ngộ Không và của các hành giả.

— Tôn Ngộ Không là biểu tượng của trí tuệ Vô Ngã, là thuộc thế giới của thực tướng vô tướng, nên các ngã tướng lửa, nước, gió, đất, không gian, thức đều không hại được. Vì thế, đao chém không đứt (đứt đầu rồi lại liền), lò bát quái (lò hữu vi) đốt không cháy.

Nhưng khi đối diện với Ba La Mật toàn giác của Đức Phật thì trí tuệ của Tôn Ngộ Không trở thành trí tuệ cỏn con bị thu phục. Tại đây, Đức Phật dạy cho Tôn Ngộ Không rõ trí tuệ của Tôn Ngộ Không chưa hoàn toàn thoát ly sự trói buộc của Năm Uẩn (\emph{sắc, thọ, tưởng, hành, thức}) qua sự kiện Tôn Ngộ Không không thể nhảy ra khỏi bàn tay của Như Lai, và qua sự kiện bị Ngũ Hành Sơn chụp phủ 500 năm, như Tôn Ngộ Không đang vướng mắc vào tánh tháo động của tâm, tuệ, vướng mắc vào ngã mạn, tự kiêu, hiếu thắng và vô minh (bị vướng mắc mà tự mình không biết).

% section tư_tưởng_phật_học (end)

\section{Quan niệm về con Người} % (fold)
\label{sec:4_con_nguoi}

— {\bf Mở ra cuộc đại náo thiên cung là tác giả muốn phơi bày một sự thật rằng:}

\begin{enumerate}[label=\itshape\alph*\upshape/]

    \item[+] Cái tổ chức trật tự của Thiên cung là sản phẩm của trí hữu ngã (trí của sinh, diệt, mộng ảo). Trí hữu ngã ban cho các hiện hữu một ngã tướng, ngã tính để dễ suy luận, nói năng, truyền đạt, để phân biệt rõ ràng giữa hiện hữu này và hiện hữu khác; đó là tính thiết lập trật tự cho các hiện hữu; chính ý niệm trật tự đã tạo ra xã hội có vua, tôi, quan, dân, \ldots, và chính ý niệm ngã đã đặt để việc làm vua suốt đời và làm vua theo dòng họ — quan dân cũng thế — như ngạn ngữ nhân gian đã nói: \emph{``Con vua thì lại làm vua, \ldots''}.

    Mọi giá trị đặt ra của xã hội hữu ngã là sản phẩm của tư duy hữu ngã, là những gì mộng ảo, không thật, bởi vì tư duy thì khác với thực tại. Một xã hội mà sống trong thế giới giá trị mộng mị như thế làm sao có thể sống có hồn nhân bản? Chính các giá trị hữu ngã ấy đã tạo ra cuộc \emph{``hỗn thế''} mà con người cần phải xây dựng, tổ chức lại theo giá trị Vô Ngã và nhân bản.
\end{enumerate}

— Chính vì giá trị ước lệ của hữu ngã đang chế ngự cõi Trời và cõi Người — đây quả là thế lực của đại tà — không đặt để con người tài đức thật đúng vị trí xã hội của nó mà sinh ra đại loạn. Để thức tỉnh vua Trời, tác giả đã để Tôn Ngộ Không phát biểu: \emph{``Nên thay nhau làm vua Trời''} hay \emph{``Như Lai bảo Ngọc Hoàng nhường ngôi cho lão Tôn thì loạn đại náo sẽ yên''}.

Như thế, mở ra cuộc đại náo Thiên cung làm kinh động cả Trời, Người, là Ngô Thừa Ân muốn giới thiệu một giá trị nhận thức mới: giá trị của Vô Ngã, muốn giáo dục con người nhận thức rõ giá trị này. Khi mọi người đều nhận thức như vậy, thì sẽ hành động xây dựng bản thân, gia đình, và xã hội theo nhận thức ấy, và sẽ tạo nên một xã hội vắng bóng hết thảy các bất công, áp bức, tham quyền cố vị, tham nhũng, khủng bố, sa đọa, \ldots. Ở đó, con người sẽ sống đúng giá trị của con người, không còn đánh mất mình nữa (không còn hiện tượng tha hóa).


% section quan_niệm_về_con_người (end)
\section{Quan niệm về xã hội} % (fold)
\label{sec:4_xa_hoi}

— Mở ra cuộc ``đại náo'' là tác giả đã quá rõ ràng muốn nói rằng: Cơ chế tổ chức xã hội phong kiến là không hợp lý, đã lỗi thời, cần tổ chức một cơ chế xã hội mới thể hiện dân chủ, công bằng và nhân bản.

— Tác giả cũng ý thức rõ rằng cuộc cải cách xã hội ấy khó thực hiện, bởi nó xáo trộn xã hội cũ, điều mà thói quen của cuộc sống khó hưởng ứng, và bởi vì đụng chạm quá mạnh đến giai cấp lãnh đạo và quyền lợi của họ và của quần chúng ``ăn theo''. Vì ý thức như vậy nên tác giả đã để Tôn giả Tu Bồ Đề báo trước cho Tôn Ngộ Không cẩn thận đề phòng ba tai nạn lớn mà Trời sẽ giáng xuống, và đã để Tôn Ngộ Không làm cuộc \emph{``đánh thức''} mà không phải \emph{``cách mạng''}.

Tác giả muốn làm một cuộc thay đổi về văn hóa và giáo dục trước, thực hiện ôn hòa hơn. Nói rõ là tác giả sau khi lường trước tính sau thì chỉ muốn có một cải tổ với quy mô khá lớn để giáo dục cho quần chúng giác ngộ dần và cấp lãnh đạo ý thức được về những đổi thay cần thiết: một xã hội mới tổ chức dưới một giá trị mới và trật tự mới (nếu cần nói như thế), chứ không phải là vô trật tự. Những lý vì sao sẽ được nói rõ vào các hồi tiếp, chúng ta sẽ tìm hiểu.

% section quan_niệm_về_xã_hội (end)
% chapter hồi_4_5_6_7 (end)