\chapter{Hồi 62 và 63} % (fold)
\label{cha:hoi_62_63}

Hồi 62:

\begin{itshape}
``Giặc bẩn, rửa lòng, nên quét tháp.

Bắt ma trả chủ thính tu thân.''
\end{itshape}

Hồi 63:

\begin{itshape}
``Náo Long cung, hai sư diệt quái.

Lấy bảo bối, các Thánh trừ tà.''
\end{itshape}

\section{Tư tưởng Phật Học} % (fold)
\label{sec:62_63_phat_hoc}

--- Giáo lý Phật Giáo dành cho các tu sĩ và cư sĩ tu tập giải thoát thường bị ngộ nhận là giáo lý tự độ. Thật sự cả hệ thống giáo lý Phật Giáo Nguyên Thủy và Phát triển (Đại thừa) đều chủ trương tu tập tự độ và độ tha. Độ tha là giúp đời thấy rõ con đường tự mình đi ra khỏi mọi nỗi khổ. Giáo lý từ bi cũng dạy cứu nạn cứu khổ đời. Vì thế, trên đường thỉnh kinh, Tôn Ngộ Không đã trừ rất nhiều tà ma, quỷ mị để cứu đời. Tại hồi 62, 63 phái đoàn Tây Du giúp vua Tế Trại đi ra khỏi quyết định sai lầm, cứu chư Tăng chùa Kim Quang khỏi hàm oan, trừ tà lấy lại quốc bảo cho nước Tế Trại. Ngộ Không và Ngộ Năng tại đây đã không quản lao nhọc hành Phật sự cứu độ, giúp dân ra khỏi khổ nạn. Độ tha là một phần công phu giải thoát của chính phái đoàn Tây Du.

--- Dục vọng khi được chuyển đổi thì thành dục giải thoát, sức mạnh tinh thần để tự độ và độ tha, như Trư Ngộ Năng đã trở nên năng nổ làm việc thiện mà không cầu danh, cầu lợi.
% section tư_tưởng_phật_học (end)

\section{Quan niệm về con Người} % (fold)
\label{sec:62_63_con_nguoi}

--- Trộm cướp, tham nhũng đều là các hành động gây bất an, rối loạn xã hội; chúng là bất thiện và phi đạo đức. Giáo dục, văn hóa nhân bản kết án nặng nề các hành động ấy nên tác giả Tây Du Ký đã để Ngộ Không và Ngộ Năng đánh chết hết những người chủ mưu trộm cắp quốc bảo tại chùa Kim Quang. Đánh chết chúng là ý nghĩa dẹp sạch các mầm mống rối loạn.

--- Cảnh vua Tế Trại bất minh và bất công, ỷ quyền uy hành hạ chư Tăng chùa Kim Quang gây căm phẫn trong nhân dân, căm phẫn cho độc giả, và tạo ra một tâm lý oán ghét phong kiến, điều mà Ngô Thừa Ân mong muốn, tác giả mong xã hội phong kiến Trung Hoa được thay thế bởi một chế độ pháp trị công bằng và dân chủ hơn.

--- Tôn Ngộ Không đã lấy về quốc bảo như là biểu tượng nói lên trí tuệ Vô Ngã sẽ xây dựng nên nền văn hóa mới tốt đẹp, sẽ định quốc, an dân. Kết quả này mới thật sự là quốc bảo.
% section quan_niệm_về_con_người (end)

\section{Quan niệm về xã hội} % (fold)
\label{sec:62_63_xa_hoi}

--- Trường hợp chùa Kim Quang bị hàm oan là do các thế lực xấu ở trong bóng tối gây ra. Vua quan Tế Trại vội vã gia hình chư Tăng giữa khi không có bằng cớ buộc tội. Càng vô lý hơn khi kết tội người có trách nhiệm giữ quốc bảo lại đi trộm quốc bảo! Trước nỗi hàm oan lớn ấy, người dân không có quyền kêu oan và không có nơi để bày tỏ nỗi hàm oan. Một xã hội được tổ chức phi lý như vậy thì người dân sẽ không có chỗ dựa, xã hội sẽ không có chỗ dựa, xã hội sẽ đại loạn. Hình ảnh xã hội ấy đã gợi nên trong dân chúng bao day dứt, bao ưu tư về một cơ cấu tổ chức bảo vệ an ninh cho dân, bảo vệ công bằng và đầy tình người, \ldots.

--- Sự kiện Ngộ Không và Ngộ Năng trừ tà, lấy lại quốc bảo nói lên rằng việc thiết lập an ninh và công bằng xã hội cũng là công tác thỉnh kinh của Đường Tăng. Kinh là để giáo dục con người sống hiền thiện, nhân ái, công bằng, tôn trọng sự thật. Nếu được hấp thụ giáo lý nhà Phật thì vua Tế Trại đã không hành xử độc đoán và thiếu nhân tình như thế! Nếu người dân thấm nhuần Phật Giáo thì đã không có trộm cắp và các ác hành, nếu nhận thức và giá trị của hành động theo tinh thần Phật Giáo thì bản án quốc bảo đã được xét xử cách khác, nỗi hàm oan của chùa Kim Quang đã được giải, \ldots. Nói tóm, vụ án quốc bảo đã đặt thẳng vấn đề với nhân dân Trung Hoa rằng: Liệu triều đình phong kiến có phải là đối tượng mà nhân dân mong chờ?
% section quan_niệm_về_xã_hội (end)
% chapter Hồi 62_63 (end)