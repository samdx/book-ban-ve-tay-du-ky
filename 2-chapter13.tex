\chapter{Hồi 27, 28, 29, 30 và 31} % (fold)
\label{cha:hoi_27_28}

Hồi 27:

\begin{itshape}
``Thây ma ba lượt trên Tam Tạng

Đường Tăng giận đuổi Mỹ Hầu Vương''
\end{itshape}

Hồi 28:

\begin{itshape}
``Núi Hoa Quả lũ yêu hợp nghĩa

Rừng Hắc Tùng Tam Tạng gặp ma''
\end{itshape}

Hồi 29:

\begin{itshape}
``Thoát nạn Giang Lưu về đất nước 

Đội ơn Bát Giới chuyển sơn lâm''
\end{itshape}

Hồi 30:

\begin{itshape}
``Tà ma phạm chánh đạo

Tiểu Long nhớ Ngộ Không''
\end{itshape}

Hồi 31:

\begin{itshape}
``Bát Giới lấy nghĩa khích Hành Giả

Ngộ Không dùng trí phục Ma Vương''
\end{itshape}

\section{Tư tưởng Phật Học} % (fold)
\label{sec:27_28_phat_hoc}

— Hồi 27 đã được bàn đến ở phần Tổng Luận: mục \nameref{sec:bieu_tuong_cua_hoi_thu_27}, chương \nameref{cha:cac_hinh_anh_bieu_tuong_khac_nhau_gioi_thieu_phat_hoc_trong_tay_du_ky} trang \pageref{sec:bieu_tuong_cua_hoi_thu_26}.

Tại đây, một số điểm sẽ được bàn thêm.

Đường Tăng suốt cuộc hành trình thì chỉ một chiều trang nghiêm giữ gìn giới tướng, một lòng giữ chặt điều nhân, chỉ tu giới và tu phước mà thiếu hẳn đôi mắt trí tuệ để phân biệt thật, giả, thiếu hẳn đôi mắt của Thánh hữu học để nhận ra các tướng ma quỷ trá hình. Do vì thiếu tuệ nhãn mà Đường Tăng không biết nghe theo Tôn Ngộ Không, lại nghe tiếng nói đố kỵ, hiềm khích và thiểu trí của Trư Bát Giới, nên vướng ngay vào các thảm họa của các ma nạn:

$\star$ Ba lần Ngộ Không đánh chết Bạch cốt tinh là ba lần bị Đường Tăng quở la và niệm chú ``khẩn cô nhi'' để hành hạ. Lần cuối thì Đường Tăng viết giấy dứt tình thầy trò dù Tôn Ngộ Không đã cạn lời phân tỏ. Ngô Thừa Ân đã rất thiện xảo dựng nên cảnh ``Bạch cốt phu nhân'' để giới thiệu sâu sắc giáo lý Phật Giáo: Từ bi, Trí tuệ và Bồ Tát hạnh.

— Từ bi của Phật Giáo, như đã bàn, đi đôi với trí tuệ và cần được trí tuệ soi tỏ. Từ bi mà tách rời khỏi trí tuệ thì không còn ý nghĩa từ bi; bấy giờ nó trở thành tương phản với từ bi của Phật Giáo.

— Hạnh Bồ Tát, theo Phật Giáo, là hạnh biểu hiện ba sức mạnh chính: trí tuệ là linh hồn; thiền định là sức mạnh; và từ bi là sắc thái đặc thù, là sứ mệnh độ sinh. Đường Tăng thì thiếu mất sức mạnh của định lực và của trí tuệ Vô Ngã. Vì thế, thảm kịch trong công phu giải thoát xẩy ra. Đây là hiểm họa trên đường về giải thoát mà Tôn giả Tu Bồ Đề gọi là họa ``bi phong'' thổi mạnh làm tan rã công phu giải thoát của hành giả, và đã căn dặn Tôn Ngộ Không cẩn thận đề phòng! Biến cố cắt đứt thâm tình sư đệ này đã làm tan rã phái đoàn Tây Du và làm lạnh buốt cơ thể giải thoát của phái đoàn. Sau khi Tôn Ngộ Không rời phái đoàn, phái đoàn đã kéo lê những bước đi thất thểu, mệt mỏi, thê lương! Tại đây, Ngô Thừa Ân đã giới thiệu mạnh mẽ với người đời rằng: không có một hạnh tu nào của Phật Giáo, hay không có một công phu giải thoát nào của Phật Giáo mà vắng mặt trí tuệ cả. Hễ ánh sáng đi thì bóng tối đến. Nếu vắng mặt Tôn Ngộ Không thì ma ``Bạch cốt phu nhân'' xuất hiện hãm hại. Ma bạch cốt là biểu tượng của cõi vô minh, âm u của tham, sân, si và quyến thuộc của tham, sân, si.

— Qua hồi 27 đầy bi kịch này, Ngô Thừa Ân cũng nói lên sự thật theo giáo lý nhà Phật rằng: chân lý chỉ có mặt trong trí tuệ giải thoát mà không có mặt trong thiền định, giới luật và lòng nhân đạo, như phái đoàn Tây Du chỉ có Tôn Ngộ Không thấy rõ chánh, tà, chân, ngụy, còn Đường Tăng, Sa Ngộ Tịnh và Trư Ngộ Năng thì không; vì thế mà trí tuệ Ba La Mật là giải thoát tối hậu, và chỉ có trí tuệ này mới thể nhập thực tướng của các pháp theo chân lý Nam, Bắc truyền.

— Ở hồi tiếp theo, hồi 28, tác giả đưa Đường Tăng, Ngộ Năng và Ngộ Tịnh sa vào nạn thiên ma ở rừng Hắc Tùng để cảnh giác Đường Tăng, Ngộ Năng và Ngộ Tịnh, vừa như để trừng phạt, cảnh cáo kẻ tu hành thiếu trí, và trừng phạt cách xử sự ``hồ đồ'' của Đường Tăng và Bát Giới.

— Tác giả để cho ma vương ở rừng Hắc Tùng biến Đường Tăng thành con hổ cũng với dụng ý ấy. Đến cuối hồi 28 này thì Tôn Ngộ Không chiếm lại đạo tình đã mất trong phái đoàn.

— Theo kinh tạng Phật Giáo thì chỉ có người trí mới biết ca ngợi ngôi Tam Bảo; chỉ có người trí mới phê phán đúng các việc làm của người trí, do vậy việc hàng phục tà của Tôn Ngộ Không không được Đường Tăng, Trư Ngộ Năng và Sa Ngộ Tịnh hiểu đúng để rồi xẩy ra các hiềm khích, ngộ nhận.
% section tư_tưởng_phật_học (end)

\section{Quan niệm về con Người} % (fold)
\label{sec:27_28_con_nguoi}

— Có một vấn đề giáo dục hiện đại mà Ngô Thừa Ân đã đặt ra suốt tập truyện, đặc biệt là trong các hồi từ 27 đến 31, là giáo dục con người thể hiện sự hòa điệu trong tự thân mỗi người: thế nào để điều hòa giữa tim và óc (từ bi và trí tuệ), giữa lý trí, tình cảm, dục vọng và ý chí của con người, như sự hòa điệu giữa các thành viên của phái đoàn Tây Du. Khi nào có mặt sự hòa điệu thì cá nhân ổn định, sáng suốt, an lạc và hạnh phúc. Khi nào tâm thức bất hòa điệu thì tâm lý rối loạn, thiếu sáng suốt và phiền não, khổ đau.

Cũng thế, cần có sự thể hiện hòa điệu giữa cá nhân mình và các cá nhân khác trong một tập thể. Khi nào sự hòa điệu trong tập thể được thể hiện, thì tương giao của tập thể trở nên xấu đi, mất đoàn kết, và các cá thể trong tập thể cảm thấy bất an. Đây là điểm giáo dục tâm lý rất người và rất thiết thực.

Lý trí và tình cảm, đạo đức và dục vọng thường hay chống trái nhau. Giáo dục là giúp con người nhận thức sự thật chống trái ấy và thể hiện quần[dx2] bình tâm thức theo sự soi sáng của lý trí. Cái nhìn và thái độ sống của Đường Tăng thì khác với cái nhìn và thái độ sống của Tôn Ngộ Không; cái nhìn và thái độ sống của Bát Giới cũng khác như thế. Nghệ thuật sống phải là nghệ thuật thể hiện hòa điệu tâm lý. Đây là nghĩa sống lớn và thực của con người.

— Nghĩa sống của tương giao cũng quan trọng. Nghĩa sống ấy là tình người chân thật, sự hiểu biết nhau, sự thông cảm nhau, sự chấp nhận, tin tưởng nhau, và sự tôn trọng nhau. Nếu đầy đủ các yếu tố vừa kể thì đã không xẩy ra cảnh ngộ nhận đầy bi kịch giữa Đường Tăng và Tôn Ngộ Không, giữa Bát Giới và Tôn Ngộ Không như ở hồi Tôn Ngộ Không đả vong [dx3]Bạch cốt tinh.
% section quan_niệm_về_con_người (end)

\section{Quan niệm về xã hội} % (fold)
\label{sec:27_28_xa_hoi}

— Sức mạnh của tập thể là sự hợp quần. Sức mạnh ấy được phát triển nếu được hướng dẫn bởi tiếng nói trí tuệ. Nếu thiếu sự hợp quần thì tập thể sẽ tan rã, và nếu sự hợp quần không được thực hiện bởi trí tuệ thì tập thể cũng lâm nguy. Đây là những gì Ngô Thừa Ân muốn nói ở hồi 27 và 28 (Bạch cốt tinh và rừng Hắc Tùng).

— Tiếng nói trí tuệ thiện xảo của tập thể cần được truyền đạt một cách thiện xảo thì mới có tác dụng sâu rộng trong quần chúng. Nếu tiếng nói trí tuệ đó thiếu sự hỗ trợ của kỹ thuật truyền đạt thì tập thể vẫn có thể rơi vào những khó khăn đáng kể làm suy yếu sức mạnh hợp quần như thái độ nói năng và ngôn ngữ diễn đạt của Tôn Ngộ Không thiếu tế nhị, thiếu khế cơ đã gây ra những cú ``sốc'' tâm lý ở Đường Tăng và Bát Giới dẫn đến hậu quả tai hại là Đường Tăng cắt đứt nghĩa thầy trò với Tôn Ngộ Không. Tôn Ngộ Không cần tỉnh ngộ về hậu quả này mới có thể diễn xuất thiện xảo vai trò của mình.

— Ngoài yếu tố trí tuệ, sức mạnh hợp quần còn cần đến tình người như hình ảnh biểu tượng Bát Giới lấy nghĩa sư đệ để khích Tôn Ngộ Không lên đường cứu thầy, và cần đến lý tưởng như chất keo gắn bó, như hình ảnh Bát Giới nhắc đến Bồ Tát Quán Thế Âm để khích Tôn Ngộ Không ``tái xuất giang hồ''.

Tại đây, Ngô Thừa Ân muốn nói rằng: con đường văn hóa giáo dục mới là cần thiết mà những kỹ thuật vận dụng để thực hiện vào đời còn cần thiết hơn.
% section quan_niệm_về_xã_hội (end)
% chapter Hồi 27_28 (end)