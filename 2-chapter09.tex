\chapter{Hồi 18 và 19} % (fold)
\label{cha:hoi_18_19}

Hồi 18:

\begin{itshape}
``Chùa Quan âm, Đường Tăng thoát nạn.

Nhà Cao Lão, Đại Thánh trừ ma.''

\end{itshape}

Hồi 19:

\begin{itshape}
``Động Vân Sạn, Ngộ Không thu Bát Giới.

Núi Phù Đồ, Tam Tạng nhận chân kinh.''
\end{itshape}

\section{Tư tưởng Phật Học} % (fold)
\label{sec:18_19_phat_hoc}

— Bước đầu tiên chế ngự tâm là chế ngự lục căn (hồi 15). Công phu chế ngự này cần được thực hành lâu dài để nhiếp phục cái động của thân và khẩu như đã được đề cập ở phần {\bf Tổng Luận}. Ở hồi 18 và 19, việc thu phục Trư Bát Giới làm đệ tử xuất gia là ý nghĩa của việc hành trì giới luật và chế ngự dục ái, chế ngự các tánh buông lung, lười biếng để thiết lập cơ sở tâm lý vững chắc cho sự thành tựu tâm giải thoát (đại định) và tuệ giải thoát (đại tuệ).

— Ở nguyên bản Tây Du Ký, và ở bản dịch Việt ngữ ấn hành năm 1982, nhà xuất bản Văn học Hà Nội, ghi đủ toàn văn bài Bát Nhã Tâm Kinh. Bài tâm kinh này là trái tim của trí tuệ toàn giác mà thiền sư Ô Sào đã trao cho Đường Tăng để giữ an lòng trên suốt cuộc viễn trình Tây phương. Điểm tinh yếu của bản kinh này là: trên cơ sở tâm thức an trú đại định (hay tâm giải thoát) và tâm nguyện độ sinh rộng lớn, hành giả hành sâu thiền quán Vô Ngã, thấy rõ hết thảy vạn hữu là rỗng không tự ngã mà vượt qua hết thảy mọi khổ đau trần thế.

Do vì hành giả an trú vào chánh niệm, chánh tưởng Vô Ngã mà tâm thức không vướng ngại bất cứ điều gì; vì tâm không vướng ngại, hành giả rời xa các thứ sợ hãi, rời xa các tưởng sai lầm và các tình cảm sai lầm; vì rời xa các ngã tưởng, các điên đảo tưởng, hành giả chứng đắc được sự giải thoát hoàn toàn mọi khổ đau. Trên cơ sở thành tựu này, hành giả tự tại ra vào khổ đau để cứu độ đời.

— Dưới con mắt của trí tuệ Vô Ngã, các ngã tướng trần gian là hư huyễn, mộng mị, tạm bợ. Ngô Thừa Ân vì thế gọi cõi đời là ``phù đồ'' (tương tự như phù thế, phù vân).
% section tư_tưởng_phật_học (end)

\section{Quan niệm về con Người} % (fold)
\label{sec:18_19_con_nguoi}

— Một đường hướng giáo dục con người tốt thì không lúc nào rời khỏi mục tiêu đem lại an lạc, hạnh phúc cho con người. Cảm nhận an lạc, hạnh phúc là phần việc của tâm thức mỗi người. Vì thế, công phu huấn luyện tâm lý là công phu chính, bên cạnh công phu mở mang hiểu biết đúng đắn, mở mang trí tuệ.

Con đường giáo dục đích thực và nhân bản không phải là con đường lý thuyết, mà là con đường sống. Con đường đó chỉ rõ rằng con người cần trở về chính mình và làm chủ chính mình ở ba lãnh vực thân, khẩu, ý. Hồi 19 và 20 thì giới thiệu đến phần giáo dục làm chủ các hành động của thân và khẩu (lời nói), chuyển các hành động đó hướng về an lạc, giải thoát (đã được đề cập ở phần Tổng Luận: mục \nameref{sec:qua_cac_nhan_vat_chinh} của chương \nameref{cha:hinh_anh_giao_ly_phat_giao_bac_truyen}, trang \pageref{sec:qua_cac_nhan_vat_chinh}).
% section quan_niệm_về_con_người (end)

\section{Quan niệm về xã hội} % (fold)
\label{sec:18_19_xa_hoi}

— Một trong những điều kiện cho quần chúng sống với nền văn hóa nhân bản và Vô Ngã là tạo nên nếp sống đạo đức cá nhân và đạo đức xã hội. Cá nhân và tập thể cần phải giữ gìn và tôn trọng một số khuôn phép kỷ luật nào đó của xã hội do xã hội đặt ra phù hợp với văn hóa Vô Ngã. Nếp sống văn hóa xã hội này được biểu tượng hóa qua nhân vật Trư Bát Giới.

— Về mặt tâm lý quần chúng thuộc hàng ngũ đổi mới được phân ra nhiều thành phần: có thành phần mang dạng tâm lý của Đường Tăng, có thành phần mang mẫu tâm lý của Tôn Hành Giả, có thành phần thì tâm lý thiếu ổn định dễ trở lui về nếp sống vói các tập tính cũ, như Trư Bát Giới nhiều lần đòi hỏi lại nhà Cao Lão, \ldots

Xã hội cần tổ chức thành một nếp văn hóa mới, thiết thực và tích cực. Vai trò tích cực trong việc hình thành nếp văn hóa mới là vai trò giáo dục thể hiện qua nhiều phương tiện truyền thống. Giáo dục mở hướng đi tới vào thế giới giá trị mới, xa dần các giá trị cũ, như Tôn Ngộ Không kiểm soát, kiềm chế Trư Ngộ Năng vậy.
% section quan_niệm_về_xã_hội (end)
% chapter Hồi 13 và 14 (end)