\chapter{Mẫu người giáo dục của nền giáo dục mới ``hậu hiện đại''} % (fold)
\label{cha:mau_nguoi_giao_duc_cua_nen_giao_duc_moi}

Khi nhà giáo dục thấy rõ quy luật sinh diệt của các hiện hữu, thấy rõ duyên khởi trong tất cả hiện hữu, thấy rõ con người ngũ uẩn của tương quan mật thiết với gia đình, tập thể, xã hội và môi sinh, thấy rõ các hiện tượng xã hội hiện hữu tương duyên với nhau, thấy rõ vai trò của tư duy, dục vọng đối với con người và hạnh phúc của con người, thì nhà giáo dục sẽ hình dung ra được một nội dung giáo dục con người như thế nào.

Một nền giáo dục hiện thực và nhân bản phải là một nền giáo dục tạo ra một môi trường và một nội dung giáo dục thế nào để mọi người tiếp thu đều có điều kiện phát triển kiến thức, tình cảm và trí tuệ tốt nhất nhằm đáp ứng hai yêu cầu chính: \emph{cá nhân và xã hội}.

Dòng sống là trôi chảy. Con người cũng phát triển không ngừng về tâm thức.

Bên cạnh nội dung giáo dục con người cá nhân như thế, học đường còn hoạch định nội dung giáo dục con người xã hội để đáp ứng các yêu cầu của xã hội và lịch sử. Phần này cần được điều hợp với phần giáo dục ở trên thế nào để thể hiện tinh thần giáo dục rằng: \emph{``cá nhân là cá nhân của xã hội, và xã hội là xã hội của cá nhân''}.

Về văn hóa truyền thống của dân tộc cũng là một vấn đề quan trọng mà nhà giáo dục nhân bản và hiện đại cần quan tâm xét đến.

Giáo dục của một xứ sở cần đáp ứng các yêu cầu lịch sử của xứ sở ấy (yêu cầu phát triển kinh tế, an ninh, quốc phòng, \ldots) Cần bảo vệ văn hóa truyền thống và đề cao truyền thống. Nhưng văn hóa truyền thống cũng chỉ là sản phẩm của con người nhằm phục vụ hạnh phúc của con người xứ sở, nên không thể đặt con người sau văn hóa truyền thống, mà cần xét định những gì của văn hóa truyền thống tốt đẹp và phù hợp với hướng phát triển của lịch sử thì bảo trì và phát huy, những gì thuộc văn hóa truyền thống mà không còn phù hợp với hướng phát triển của con người và lịch sử thì thay đổi. Đây cũng là một đặc điểm của nền giáo dục nhân bản và hiện đại.

Có một yêu cầu giáo dục khác, vừa thuộc yêu cầu của cá nhân vừa thuộc yêu cầu xã hội, đó là yêu cầu tín ngưỡng. Con người có quyền tự do chọn lựa thức ăn và màu áo cho mình thì cũng có quyền chọn lựa tín ngưỡng như là màu áo của tình cảm, tâm hồn. Xã hội và học đường cần đáp ứng tốt yêu cầu tín ngưỡng này. Nhưng các tín ngưỡng dị biệt dễ dàng tách rời các cá nhân xa nhau, đây là vấn đề của xã hội: làm thế nào để các cá nhân có tín ngưỡng dị biệt có thể gần gũi nhau và cảm thông nhau thì yêu cầu giáo dục về đoàn kết dân tộc mới thực hiện được.

Học đường hiện đại hay ``hậu hiện đại'', vì thế cần có một triết lý giáo dục chung vượt lên trên (hay vượt ra ngoài) các dị biệt ấy để có thể thống nhất các dị biệt. Ngô Thừa Ân, tại Ngũ Trang Quán đã mở ra một bữa tiệc Nhân Sâm để Phật Giáo (Bồ Tát Quán Thế Âm và phái đoàn Tây Du) và Lão Trang, Nho Giáo (các thiên tiên, địa tiên) thân mật, đề huề; và để Tôn Ngộ Không cùng vị chủ nhân Ngũ Trang Quán kết nghĩa huynh đệ, sau một hồi tranh chấp xung đột. Đây cũng là một vấn đề lớn của giáo dục. Tất cả đang gặp gỡ nhau, và sẽ mãi mãi gặp gỡ nhau, ở địa bàn con người rất người, hạnh phúc con người và cuộc đời (hay môi sinh và xứ sở). Phải chăng đây là cơ sở gặp gỡ của giáo dục để thể hiện mục tiêu giáo dục cảm thông, thương yêu và đoàn kết?

Tất cả đang trông chờ vào vai trò giáo dục ở học đường, tập thể, gia đình và xã hội. Tất cả đang trông chờ vào các phương tiện truyền thông: đài phát thanh, truyền hình, sách và báo chí, văn nghệ, hội họa, \ldots ~chuẩn bị cho con người của thời đại một nhận thức mới thiết thực, nhân bản và trí tuệ trước hết.


\hrulefill

\hfill
\begin{minipage}{0.6\textwidth}
    \begin{center}
Viết xong ngày 18 tháng 9 năm 1991

{\bf Thiền viện Vạn Hạnh.

Tỳ Kheo Thích Chơn Thiện.}
    \end{center}
\end{minipage}

% chapter mẫu_người_giáo_dục_của_nền_giáo_dục_mới_hậu_hiện_đại_ (end)