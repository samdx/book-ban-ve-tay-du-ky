\chapter{Hồi 78 và 79} % (fold)
\label{cha:hoi_78_79}

Hồi 78:

\begin{itshape}
``Nước Tỳ Kheo, thương trẻ khiến âm thầm.

Điện Kim Loan, yêu ma bàn đạo đức.''
\end{itshape}

Hồi 79:

\begin{itshape}
``Tìm động bắt yêu ma gặp sao Thọ.

Giữa triều dạy vua, thấy trẻ con.''
\end{itshape}

\section{Tư tưởng Phật Học} % (fold)
\label{sec:78_79_phat_hoc}

--- Giáo lý Phật Giáo là trí tuệ, nhân ái và thiết thực, nhưng phần truyền vào dân gian thì thường là phần tín ngưỡng nặng hình thức tôn giáo, lại pha lẫn với Nho, Lão và các tín ngưỡng dân gian khác. Các nhà chân tu Phật Giáo lại ở ẩn trong những cánh rừng u tịch, xa quần chúng. Chánh pháp vì thế vắng mặt ở đời. Ngô Thừa Ân quan niệm Phật Giáo cần đi vào xã hội thể hiện đúng chân tinh thần cứu khổ để độ đời, như phái đoàn Tây Du tích cực cứu triều đình và các trẻ em nước Tỳ Kheo.

Vì giữ lòng vô tham, vô dục, người Phật tử mới có thể lấy cái dục của thiên hạ làm cái dục của mình, lấy cái tâm thiên hạ làm cái tâm của mình. Vì sống vị tha, thấy hạnh phúc từ cái tâm của mình, vì sống vị tha, thấy hạnh phúc từ cái tâm vị tha, nên mới có thể vui với cái vui thiên hạ. Vì giác tỉnh Vô Ngã mới có thể khởi tâm vô úy sống vì hạnh phúc của nhân quần xã hội. Vì thấy rõ Vô Thường nên tích cực làm nhiều điều thiện. Những yếu tố tích cực ấy cần được đưa vào nội dung giáo dục con người.

--- Đối với nhà Vua, Tôn Ngộ Không khuyên: \emph{``Tâu bệ hạ, từ đây chớ tham sắc dục, tích nhiều ân đức, \ldots''} (tập 3 tr. 180 sđd)
% section tư_tưởng_phật_học (end)

\section{Quan niệm về con Người} % (fold)
\label{sec:78_79_con_nguoi}

--- Vấn đề giáo dục con người, giáo dục tuổi trẻ là một vấn đề lớn của quốc gia. Thế hệ trẻ là tài sản lớn của quốc gia, vô cùng quý giá, cần được giáo dục một cách trân quý. Nếu nhà vua (hay cấp lãnh đạo) nghe theo lời sàm tấu xem thường thế hệ trẻ thì sẽ là họa lớn cho quốc gia. Về mặt này, theo Ngô Thừa Ân, Đạo gia và Nho gia không quan tâm đúng mức bằng Phật Giáo (Ngô Thừa Ân đã giới thiệu Tôn Ngộ Không cứu trẻ trong khi các Đạo gia, Nho gia thì mưu hại trẻ).
% section quan_niệm_về_con_người (end)

\section{Quan niệm về xã hội} % (fold)
\label{sec:78_79_xa_hoi}

--- Song song với việc giáo dục con người về ý sống, xã hội cần xác định rõ rằng: Sắc đẹp và tình yêu là lẽ sống của con người, nhưng không phải là lẽ sống duy nhất. Con người còn có giá trị sống khác cao quý hơn như sống vì tập thể, xã hội, quê hương, vì an lạc và hạnh phúc của số đông. Trường hợp sa đọa của vua nước Tỳ Kheo là một trường hợp điển hình cần được nêu gương cho người đời để tránh.

--- Vấn đề hưng suy của một xã hội có mối liên quan mật thiết với vấn đề giáo dục đạo đức cá nhân và xã hội. Nhiều cá nhân tốt mới xây dựng nên một xã hội tốt. Vấn đề đạo đức của cấp lãnh đạo cũng cần được đặc biệt quan tâm. Nhà lãnh đạo cũng cần sống vì hạnh phúc của trăm họ, là nơi nương tựa của trăm họ.
% section quan_niệm_về_xã_hội (end)
% chapter Hồi 78_79 (end)