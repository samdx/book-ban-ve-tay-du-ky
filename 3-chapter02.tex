\chapter{Sự thật của con người và cuộc đời} % (fold)
\label{cha:su_that_cua_con_nguoi_va_cuoc_doi}

Sự thật và hạnh phúc là vấn đề lớn mà nhân loại mãi tìm kiếm. Còn tìm kiếm có nghĩa là còn chưa thấy. Vì chưa thấy rõ sự thật của cuộc đời nên con người tiếp tục nói về, bàn về cuộc đời. Vì sự thật của con người là khổ đau nên con người khát vọng tìm kiếm hạnh phúc. Tìm kiếm hạnh phúc là vai trò của văn hóa và giáo dục.

Cách đây 26 thế kỷ, thái tử Siddhattha (người xứ Ấn) đã tự mình tìm kiếm lời giải đáp cho hai vấn đề lớn nói trên. Sau một thời gian hành sâu thiền định và sáu năm hành khổ hạnh, thái tử vẫn mịt mờ trước sự thật. Nhưng sau bốn (hay bảy) tuần lễ thiền quán dưới cội cây bồ đề, thái tử đã chứng ngộ sự thật Duyên khởi, thấy rõ gốc của khổ đau và con đường dẫn đến đoạn diệt khổ đau, và thái tử bấy giờ trở thành bậc Toàn Giác, Phật, Thế Tôn.

\hrulefill

{\bf Sự thật mà Đức Phật đã chứng ngộ có thể được tóm tắt như sau:}


{\bf Do cái này có mặt nên cái kia có.

Do cái này sinh, nên cái kia sinh.

Do cái này không có mặt, nên cái kia không có.

Do cái này diệt, nên cái kia diệt.}

— \emph{``Do vô minh mà có hành, do hành mà có thức, do thức mà có danh sắc, do danh sắc mà có lục nhập, do lục nhập mà có xúc, do xúc mà có thọ, do thọ mà có ái, do ái mà có thủ, do thủ mà có hữu, do hữu mà có sinh, do sinh mà có lão tử, sầu, bi, khổ, ưu, não. Đây là sự tập khởi của toàn bộ khổ uẩn''}. (Tương Ưng II, bản dịch của HT Thích Minh Châu, 1982, tr. 4)

\hrulefill

Vô minh là chỉ phần đầu của 12 nhân duyên, nhưng không phải là nguyên nhân đầu tiên; nó cũng do duyên mà sinh. Vô minh đã được Đức Phật định nghĩa:

\emph{``Và này các Tỳ kheo, thế nào là vô minh? Này các Tỳ kheo, không biết rõ về khổ, không biết rõ về khổ tập, không biết rõ về khổ diệt, không biết rõ về con đường đưa đến khổ diệt. Này các Tỳ kheo, đây gọi là vô minh''}. (sđd. tr.4)

Từ định nghĩa trên, vô minh có thể được hiểu là do không hiểu Tứ Đế, không hiểu Duyên Khởi, hay do chấp thủ các ngã tướng. Giờ chúng ta thử quan sát sự vận hành của vô minh trong mỗi cá nhân hiện tại trong nền văn hóa hiện đại.

\section{Nhóm Vô minh và Hành} % (fold)
\label{sec:vo_minh_va_hanh}

Nền văn hóa phổ biến hiện nay của nhân loại là sản phẩm của tư duy hữu ngã. Tư duy này đặt để các ngã tính cho mọi hiện hữu để thiết lập trật tự cho tư duy và cuộc sống. Như vậy ở đó có mặt Vô minh và Hành. Nói khác đi, nền văn hóa ấy đang biểu thị sự vận hành của Vô minh.

Con người chịu ảnh hưởng của văn hóa, tư duy được định hình bởi văn hóa, nên mặc nhiên xem tự ngã như là có thật.

Thế là Vô minh xâm nhập và ngự trị các hành động thân, lời và ý của con người.

% section nhóm_vô_minh_và_hành (end)

\section{Nhóm Thức và Danh sắc} % (fold)
\label{sec:thuc_va_danh_sac}

Giáo dục làm nên văn hóa. Văn hóa đã là hữu ngã thì giáo dục cũng mang tính chất hữu ngã. Giáo dục thì truyền đạt các kiến thức, rèn luyện tư duy, nên kiến thức và tư duy ấy là hữu ngã và là Vô minh. Thức là hữu ngã là là vô minh. Nếp sống của con người Danh sắc hình thành nghiệp thức và mở ra dòng sống là chuyên chở Vô minh và khổ đa.

Thế là, cái gọi là sống của mỗi cá nhân chỉ là sự biểu hiện của những ý niệm hữu ngã, đã đánh mất sự sống.

% section nhóm_thức_và_danh_sắc (end)

\section{Nhóm Lục nhập, Xúc, Thọ} % (fold)
\label{sec:luc_nhap_xuc_tho}

Lục nhập, Xúc, Thọ chỉ là sự biểu hiện của Thức và Danh sắc là hiện hành của Vô minh của nền văn hóa hữu ngã. Cái gọi là cá nhân xúc, thọ thật sự chỉ là Vô minh xúc, thọ. Và, vì thế gọi là ``tôi khổ đau'' chỉ là một vọng tưởng ám ảnh cá nhân.

% section nhóm_lục_nhập_xúc_thọ (end)

\section{Nhóm Ái, Thủ} % (fold)
\label{sec:ai_thu}

Lòng khát sống của cá nhân trước cuộc sống bốc cháy các khát khao ngũ dục lạc (hay dục ái) các khát khao tồn tại (hữu ái) và các khao khát vĩnh cửu (vô hữu ái). Khát khao ấy thúc dục cá nhân nắm chặt đối tượng khát khao. Đây là Chấp thủ. Hậu quả của Chấp thủ là các cuộc xung đột, chiến tranh, kỳ thị tôn giáo, màu da và phái tính, \ldots ~mở ra một cuộc diện bất an hiện nay.

Lòng khát ái thì dẫn đến các hiện tượng phát triển ``sex'', kinh doanh ``sex'', đẩy con người chìm sâu vào các thụ hưởng thấp kém làm rã dần nền đạo đức cá nhân, gia đình và xã hội.

Khi mà Tham ái và Chấp thủ phát triển mạnh, thì các tâm lý vị tha, nhân ái, công bằng, khoan dung phải co mình lại, xã hội lâm vào các nguy kịch của các tệ trạng, của sự tàn phá rừng, biển và ô nhiễm môi sinh, của sự khai thác vô độ lòng dục ái và tinh thần cạnh tranh, tách xa dần hướng giáo dục nhân bản.

Trước sự thật cuộc sống đang đi vào băng hoại ấy, các nhà giáo dục cần cấp thiết mở ra các hướng giáo dục thoát khổ, hướng giáo dục Duyên Khởi vận hành để đoạn trừ Vô minh, như Đức Phật đã dạy về Duyên Khởi, và đã mở đường rằng:

\emph{``Như vậy này các Tỳ kheo, Vô minh duyên Hành, Hành duyên Thức, Thức duyên Danh sắc, Danh sắc duyên Lục nhập, Lục nhập duyên Xúc, Xúc duyên Thọ, Thọ duyên Ái, Ái duyên Thủ, Thủ duyên Hữu, Hữu duyên Sinh, Sinh duyên Khổ, Khổ duyên tín, tín duyên hân hoan, hân hoan duyên hỷ, hỷ duyên khinh an, khinh an duyên lạc, lạc duyên định, định duyên tri kiến như chân, tri kiến như chân duyên yếm ly, yếm ly duyên ly tham, ly tham duyên giải thoát, giải thoát duyên trí về đoạn diệt.''} (Tương Ưng II, sđd, tr. 37).

Giáo dục cần chỉ cho con người thấy rõ các nguy hiểm của nền văn hóa có chiều hướng đang bị vận hành bởi Vô minh, bởi những ý niệm chấp ngã, chấp \emph{``Tôi''} để mở lòng tin về một nền giáo dục duyên khởi loại bỏ Vô minh, mở hướng vào an lạc và hạnh phúc trong hiện tại.

% section nhóm_ái_thủ (end)

\section{Nhóm Hữu, Sinh, Lão, Tử, Sầu, Bi, Khổ, Ưu, Não} % (fold)
\label{sec:huu_sinh_lao_tu_sau_bi}

Do Vô minh, nền văn hóa hiện tại đang hiện hành với những hiện tượng xã hội báo động. Đây là Hữu, Sinh, Lão tử, Sầu, Bi, Khổ, Ưu, Não, là sự sinh khởi của Vô minh và khổ đau.

Con đường xây dựng hạnh phúc cho đời vì vậy cần được vận hành bởi Minh hay Chánh Kiến, trí tuệ. Đấy là sự vận hành của nhận thức Vô Ngã sẽ mở ra hướng sống nhân ái, vị tha, vì hạnh phúc an lạc của số đông, và sẽ mở ra một nếp sống đạo đức nhân bản, hiện thực và trí tuệ cho cá nhân, gia đình và xã hội.

Suốt hơn bốn mươi năm giáo hóa, Đức Phật chỉ nói khổ và con đường diệt khổ cho đủ mọi căn cơ xuất gia và tại gia, vì thế toàn tạng Kinh, Luật, Luận của Phật Giáo cũng chỉ kiết tập về những lời dạy tập chú về việc trình bày khổ và con đường diệt khổ và nói đến nếp sống, nếp nghĩ thế nào để đạt hạnh phúc trong hiện tại và trong tương lai.

Giáo lý của Phật đã giới thiệu rất nhiều nếp sống phù hợp với mọi căn cơ ở đời đi đến hạnh phúc, nên đạo Phật cần được giới thiệu, phổ biến rộng rãi cho đời, trong thế giới học đường. Đây là một nội dung mà các hệ thống văn hóa, giáo dục hiện đại đang tìm kiếm.

% section nhóm_hữu_sinh_lão_tử_sầu_bi_khổ_ưu_não (end)
% chapter sự_thật_của_con_người_và_cuộc_đời (end)