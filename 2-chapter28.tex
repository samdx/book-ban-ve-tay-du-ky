\chapter{Hồi 68, 69, 70 và 71} % (fold)
\label{cha:hoi_68_69}

Hồi 68:

\begin{itshape}
``Nước Chu Tử, Đường Tăng bàn lịch sử.

Chữa Quốc Vương, Hành giả làm lương y.''
\end{itshape}

Hồi 69:

\begin{itshape}
``Tâm chủ thâu đêm hoàn xong thuốc tể.

Quân vương tên tiệc thuật chuyện yêu tà.''
\end{itshape}

Hồi 70:

\begin{itshape}
``Bảo bối yêu ma tung khói lửa.

Mưu mô Đại Thánh trộm nhạc vàng.''
\end{itshape}

Hồi 71:

\begin{itshape}
``Đại Thánh giả danh thu quái vật.

Quan Âm hiển tượng bắt tà yêu.''
\end{itshape}

\section{Tư tưởng Phật Học} % (fold)
\label{sec:68_69_phat_hoc}

-- Về mặt sự thật của Nhân Quả, giá trị của một hành động phụ thuộc vào nhiều nhân duyên, không có một công thức thiện ác ước lệ nào. Cùng một hành động ác mà kết quả khác nhau ở hai người làm khác nhau. Về hành động thiện cũng thế. Vua nước Chu Tử phạm tội sát nhưng vì tâm ác tác động lên hành động không mạnh; lại sau khi đã làm sám hối nên kết quả của nghiệp báo nhẹ đi: chỉ khổ tương tư ba năm rồi cùng chánh hậu đoàn tụ lại trong hạnh phúc. Có người khác, như loài quỷ, yêu, thì bị đền mạng và đọa lạc ngay.

-- Về sự giữ giới trong giáo lý Phật Giáo cũng được hành với Chánh Kiến, Chánh Tư Duy và Chánh Niệm. Có trường hợp lường gạt, nói dối mà không phạm, như trường hợp nói dối để trừ yêu quỷ, cứu người. Ngô Thừa Ân nêu lên ý nghĩa đó qua nhân vật Ngộ Không thường dối gạt tía lịa. Thực tế Tôn Ngộ Không đã làm chủ tâm thức luôn hành động để trừ tà, hành động đã vượt lên thiện, ác, trở nên duy tác.

-- Tôn Ngộ Không đến đây dù không tạo thêm nghiệp mới, nhưng tập khí tham, sân, si (\emph{thượng phần kiết sử}) vẫn còn nên không dám đương đầu với cái Lục Lạc của quỷ vương núi Kỳ Lân, động Giải Trãi. Lục lạc này phóng ra ba loại: lửa, khói và cát vàng là biểu tượng của sân, tham và si -- cội gốc của sinh tử. Bồ Tát Quán Thế Âm tại đây dạy cho Tôn Ngộ Không bài học rằng: muốn vào trí tuệ Ba La Mật, Tôn Ngộ Không phải vận dụng thiền quán thấy rõ tham, sân, si và sự tồn tại, sự hoại diệt và sự sinh khởi của chúng, chỉ khi ấy mới có thể đoạn diệt chúng lấy được ``lục lạc'' và thắng được yêu tà.

-- Chiếc áo tiên mà Trương Tử Dương trao cho chánh hậu để bảo vệ trinh tiết là biểu tượng của Chánh Niệm và Thánh Giới Uẩn. An trú vào Chánh Niệm hay an trú Thánh Giới thì ngoại tà không xâm nhập được. Đây là đặc tính của pháp. Pháp vốn như vậy.
% section tư_tưởng_phật_học (end)

\section{Quan niệm về con Người} % (fold)
\label{sec:68_69_con_nguoi}

-- Nếu sự chấp thủ, ái trước các nội thọ, ngoại thọ của người tu giải thoát ngăn trở sự phát triển trí tuệ giải thoát như thế nào, thì sự đam mê, ái trước của con người về danh, lợi, tình, quyền lực cũng trói buộc con người làm suy yếu tinh thần và trí óc của con người như thế ấy.

Vua nước Chu Tử quá yêu vị Kim Hậu nên khi Kim Hậu bị quỷ giam giữ ba năm đã khiến nhà vua lâm bệnh tương tư nặng, lòng khổ sầu khôn nguôi, rất trầm trọng. Đây là bài học về giáo dục tình cảm con người. Con người cần biết làm chủ tình cảm trước dòng sống vô thường, nếu không thì sẽ đau khổ nhiều.

-- Nỗi khổ của vua Chu Tử là do quỷ vô thường, tham, sân, si gây ra mà phương thức đối trị là trí tuệ Vô Ngã (trí tuệ thấy rõ Vô Ngã, Vô Thường, Khổ đau) như nhà vua đã được Tôn Hành Giả chữa trị.

-- Vì vậy, nền văn hóa giáo dục nhân bản và trí tuệ phải là nền văn hóa đem lại an lạc, hạnh phúc cho con người cả hai mặt vật chất và tâm lý. Quan tâm đến an lạc của tha nhân, tập thể cũng là gia tăng hạnh phúc cho bản thân. Sự thật này cần được các cá nhân chứng nghiệm.

-- Bốn hồi truyện 68, 69, 70 và 71 xác định rằng danh vọng và đời sống vật chất sung mãn chưa đủ đáp ứng yêu cầu hạnh phúc của con người. Cả khi con người có đủ các yêu cầu của đời sống, bao gồm tình yêu, thì vô thường và nỗi ám ảnh của vô thường cũng gây đau nhức tim, óc con người. Vấn đề căn bản để con người sống an lạc là biết nhìn và biết sống. Đây là một yêu cầu quan trọng trong việc giáo dục con người.
% section quan_niệm_về_con_người (end)

\section{Quan niệm về xã hội} % (fold)
\label{sec:68_69_xa_hoi}

-- Xã hội hưng thịnh hay suy yếu đều do yếu tố con người. Mọi hành động gây xáo trộn tâm lý cá nhân, xáo trộn gia đình và xã hội chung quy đều do các tâm lý tham, sân, si và sợ hãi gây ra. Do vậy, giáo dục con người để chế ngự tham, sân, si, sợ hãi là yêu cầu giáo dục thiết yếu. Nền văn hóa mới cần được xây dựng để phục vụ mục tiêu giáo dục này.

Đây là bài học kinh nghiệm lịch sử Trung Hoa từ thời đại Tam Hoàng, Ngũ Đế, và Nghiêu Thuấn hưng thịnh, thái bình cho đến thời suy thoái từ Thành, Chu, Tống, Tề, Lương, Trần. Đường Thế Dân, vị vua anh minh, đức độ, có nhiều trung thần tài giỏi, bắt đầu nghĩ đến công cuộc canh tân, muốn đưa văn hóa Phật Giáo vào văn hóa Trung Quốc nên đã cử phái đoàn Đường Tăng đi Tây Trúc thỉnh kinh (sđd. tr 396, tập 3, Hà Nội 1988). Đây là câu chuyện lịch sử được bàn bạc giữa Đường Tăng và vua nước Chu Tử; đây cũng là tâm sự của Ngô Thừa Ân muốn có một cuộc cải tổ, canh tân văn hóa cổ Trung Quốc.

-- Căn bệnh tương tư của vua nước Chu Tử là biểu tượng của căn bệnh trầm kha của nền văn hóa Nho học của Trung Quốc. Nền văn hóa ấy như đã đánh mất những gì của chất liệu sống như vua Chu Tử bị mất Kim Hậu. Phải chăng chất liệu sống làm hồi sinh được văn hóa Trung Hoa là Phật Giáo? Hệt như Tôn Ngộ Không đã bốc thuốc giúp vua Chu Tử hồi sinh, chất liệu sống làm hồi sinh văn hóa này cần được các nhà lãnh đạo đánh giá cao như thể vua Chu Tử đánh đổi Kim Hậu bằng ngai vàng. Tại đây, xã hội cần chú ý bảo vệ giá trị nền văn hóa mới hơn là bảo vệ ngai vàng.
% section quan_niệm_về_xã_hội (end)
% chapter Hồi 68_69 (end)