\chapter{Hồi 87} % (fold)
\label{cha:hoi_87}

Hồi 87:

\begin{verse}
\begin{itshape}
Quận Phượng Tiên dối trời bị nạn.\\
Tôn Đại Thánh khuyến thiện làm mưa.
\end{itshape}
\end{verse}

\section{Tư tưởng Phật Học} % (fold)
\label{sec:87_phat_hoc}

-- {\bf Ngô Thừa Ân đã vạch ra sai lầm của lối hành xử của xã hội cũ ở hồi 87 này rằng:}

\begin{enumerate}[label=\itshape\arabic*\upshape/]
    \item Thương Quan quận Phượng Tiên giận việc nhà mà làm hỏng việc dân là một sai lầm lớn.

    \item Ngọc Hoàng lẽ ra phải trừng phạt Thương Quan lại gây hạn hán ba năm làm khổ dân lành. Hình phạt như vậy không đúng đối tượng mắc vào một sai lầm lớn khác.

    Theo quy luật Nhân Quả, và theo luật pháp công minh, thì đúng lẽ là ai làm người ấy chịu trách nhiệm. Phái đoàn Tây Du, tiêu biểu cho tiếng nói của nền văn hóa mới sửa lại sai lầm đó bằng cách đem mưa về cứu dân.
\end{enumerate}

Nền văn hóa mới như thế được xây dựng phù hợp với các vận chuyển của trời, đất, với quy luật Nhân Quả.

-- Tại quận Phượng Tiên, phái đoàn Tây Du đang hành đạo Bồ Tát, quan tâm sâu xa đến đời sống nhân dân trước mắt.
% section tư_tưởng_phật_học (end)

\section{Quan niệm về con Người} % (fold)
\label{sec:87_con_nguoi}

-- Vấn đề thiện, ác và giúp nhân dân sống thiện tránh ác là vấn đề đạo đức. Nền đạo đức thiết thực, nhân bản và trí tuệ không phải là một hệ thống các mệnh lệnh đến từ các đấng uy quyền, thiêng liêng, mà là tiếng nói đích thực của quy luật tự nhiên giới (quy luật Nhân Quả), của an vui và hạnh phúc của các cá nhân và xã hội. Ngô Thừa Ân đã dựng lên cảnh các thiên thần khích lệ con người hành thiện, là ca ngợi điều thiện chính là đạo đức và chính là hạnh phúc. Đây là cơ sở vững chắc của đạo đức, bởi con người có thể bằng các tướng trạng giả dối, trá hình để chạy trốn dư luận, chạy trốn pháp luật, nhưng con người không thể trốn chạy lương tâm và luật Nhân Quả.

Các cá nhân đạo đức sẽ hình thành một xã hội đạo đức. Một xã hội đạo đức thì sẽ hưng thịnh, và sẽ là xã hội mà muôn thuở con người mong ước.

-- Giáo lý Nhân Quả còn giúp con người thường xuyên giác tỉnh sống với tinh thần trách nhiệm cá nhân, cơ sở trưởng thành của một nhân cách, và là cơ sở phát huy tinh thần trách nhiệm đối với tập thể xã hội.
% section quan_niệm_về_con_người (end)

\section{Quan niệm về xã hội} % (fold)
\label{sec:87_xa_hoi}

-- Xã hội cần được tổ chức trật tự và ổn định, xã hội trật tự nhờ vào luật pháp công minh. Luật pháp sẽ được thi hành hữu hiệu nếu quần chúng sống chân thật và hướng thiện, biết sợ quả báo của các hành động bất thiện, biết giữ gìn các giới cấm và biết chế ngự tâm lý. Giáo lý Nhân Quả sẽ giúp quần chúng ham hành thiện, tránh ác; Giới (giữ ngũ giới) sẽ giúp con người chế ngự các hành động của thân và lời; Thiền Định sẽ giúp chế ngự ý.

Thế là Phật Giáo quả cần thiết để giáo dục quần chúng xây dựng một xã hội ổn định và hưng vượng. Điều này là điều mà Ngô Thừa Ân muốn đưa vào văn hóa Trung Hoa và gọi là để cứu khổ dân Đông Độ (tức Trung Hoa).
% section quan_niệm_về_xã_hội (end)
% chapter Hồi 87 (end)