\chapter{Kết Luận} % (fold)
\label{cha:ket_luan}

Phần bàn trên đây chỉ là vài nét chấm phá, điểm xuyết. Chúng ta có thể tìm thấy ở tiểu thuyết Tây Du Ký của Ngô Thừa Ân có nhiều phản ảnh tư tưởng Phật Học có giá trị, nếu hiểu năm nhân vật của phái đoàn Tây Du là biểu hiện của chính một con người trên đường về giải thoát sinh tử. Con người đó có thể là Đường Tăng, có thể là Tôn Ngộ Không, mà cũng có thể là bất cứ một người nào trong quá khứ, trong hiện tại hay trong tương lai. Tất cả các ma nạn đều là cảnh giới của tâm thức, xẩy ra trong tâm thức của hành giả, do tác động từ nội tâm hoặc từ ngoại giới.

Tôn Ngộ Không sau khi học xong đạo vô sinh, từ giã Tôn giả Tu Bồ Đề trở về động Thủy Liêm là đắc đạo giải thoát ít nhất là quả thứ nhất của dòng Thánh (quả Tu Đà Hoàn). Quả vị giải thoát này mới đoạn trừ Thân Kiến, Nghi và Giới Cấm Thủ, còn phải tiếp tục tu hành để đoạn trừ bảy kiết sử còn lại: \emph{Dục, Sân, Hữu Ái, Vô Hữu Ái, Mạn, Trạo Cử và Vô Minh}, như đã được biểu hiện qua các ma nạn trên đường phò Đường Tăng đến Lôi Âm Tự. Với tâm thức này, Ngộ Không đã vượt qua chuyện tái sinh về địa ngục, ngạ quỷ, súc sinh, và không rơi vào dục ái của cõi Người và cõi Trời (trời dục giới).

Vì thế tác giả Ngô Thừa Ân diễn đạt quả chứng ấy bằng các cuộc tung hoành hiên ngang của Tề Thiên Đại Thánh ở Địa phủ, Long cung và Thiên đình. Ý nghĩa đại náo là thế chứ không phải là ý nghĩa đập phá mà người đời thường hiểu. Theo phò Đường Tăng là ý nghĩa Ngộ Không tiếp tục tu tập giải thoát để chứng đắc ba quả thánh còn lại và hành Bồ Tát hạnh theo giáo lý Bắc Truyền.

Các hồi còn lại của bộ truyện mà người viết bài phiếm luận này chưa bàn đến cần được ghi nhận là chưa bàn đến. Những hồi ấy có thể lập lại những hồi cũ bằng các sự kiện mới, theo tinh thần huấn luyện tâm lý của Phật Giáo: \emph{mỗi quả vị chứng đắc cần được tu tập nhiều lần cho thuần thục trước khi đi đến một chứng đắc mới}.

Lại nữa, dù bất cứ nội dung giải thoát nào mà Ngô Thừa Ân bàn đến đều được bao gồm vào con đường tu tập Giới, Định, Tuệ và bi nguyện độ sinh của hàng Bồ Tát, tất cả không thể đi ra khỏi con đường truyền thống giải thoát ấy, như Tề Thiên Đại Thánh không thể nhảy ra khỏi bàn tay của Đức Phật.

Để kết thúc phần phiếm luận này, người viết xin ghi lại cảm nhận của mình khi đọc truyện Tây Du và khi xem phim Tây Du rằng: Cả hai thời ấy tôi đều bị chụp phủ lên tâm thức nét buồn bã của Mỹ Hầu Vương trước cuộc sống vô thường, tôi đều bị ám ảnh bởi các ma nạn mạnh đến độ tôi cứ ngỡ rằng có cái gì thật hư ảo chung quanh các dung sắc và chung quanh cuộc sống của mình, và có một ảnh hưởng nào đó thu hút tôi về phía ngay chính, nhân ái, hiện thực và trí tuệ. Tôi tự nghĩ: cuộc sống sẽ bắt đầu hạnh phúc nơi nào cụm ``Ngũ Hành Sơn'' tung vỡ.
% chapter kết_luận (end)
