\chapter{Hồi 91, 92, 93, 94 và 95} % (fold)
\label{cha:hoi_91_92_95}

Hồi 91:

\begin{itshape}
``Phủ Kim Bình, vui xem đèn Tết

Động Huyền Anh phải khai khẩu cung''
\end{itshape}

Hồi 92:

\begin{itshape}
``Ba nhà sư đánh núi Thanh Long

Bốn vì sao bắt ma ma Tê Giác''
\end{itshape}

Hồi 93:

\begin{itshape}
``Vườn Cấp Cô, hỏi cũ bàn nguồn

Nước Thiên Trúc chầu vua được vỡ nợ''
\end{itshape}

Hồi 94:

\begin{itshape}
``Sư dự yến vườn hoa

Quán mong vui tình dục''
\end{itshape}

Hồi 95:

\begin{itshape}
``Xem hình biết quả bắt ngọc thố

Giúp chính về chân rõ nguồn gốc''
\end{itshape}

\section{Tư tưởng Phật Học} % (fold)
\label{sec:91_92_95_phat_hoc}

— Sư tử chín đầu là nạn tập khí tâm thức của chín cảnh giới trước khi đến cảnh giới Phật. Tập khí đó bốc cháy lên lần sau cùng trước sức mạnh của thiền chỉ và thiền quán Vô Ngã của hành giả. Tập khí đó là sự huân tập qua nhiều kiếp sinh tử phát sinh từ tâm ái. Phải chờ đến lúc ngọn lửa Ái sau cùng bị dập tắt thì các nhân tố của sinh tử mới bị đoạn tận, khổ đau mới được dứt trừ. Tại đây, hành giả mới làm xong những gì cần làm.

— Vì thế, tại đây Đường Tăng lại tiếp mắc vào nạn Ngọc Thố ở cung Hằng trá làm công chúa để ép duyên người về giải thoát, vừa để trả thù Tố Nga (công chúa thật).

Nạn ái dục này là thuộc tâm khát ái có từ thuở tâm vọng động dấy khởi lên các ngã tưởng, như từ cái tát tai của Tố Nga đánh con Ngọc Thố. Nó là hư vọng.

Khi Tôn Ngộ Không tìm ra gốc gác của Ngọc Thố, thì Ngọc Thố liền bị thu về cung Hằng. Đây là biểu tượng nói lên rằng khi hành giả bằng thiền quán thấy rõ Ái là do duyên sinh, là giả tướng, không thật thì Ái liền tiêu mất. Tâm thức hiện rõ tuệ Vô Ngã và tâm sạch phiền não lậu hoặc. Màn ảo thuật của sinh tử chấm dứt ngang đây.

— Công phu giải thoát của Đường Tăng ngang đây đã rất thiện xảo, đã thoát khỏi các chấp thủ ngã tướng, nên đã linh động thỏa hiệp tạm thời gá nghĩa với công chúa giả (Ngọc Thố) để chờ Ngộ Không mở đường thoát thân, đi đến Lôi Âm tự.

Đường Tăng vốn là hiện thân của chấp thủ ngã thế mà ở đây Đường Tăng đã tự tại với các ngã tướng, nói lên rõ ràng đây là trở ngại tâm thức sau cùng.
% section tư_tưởng_phật_học (end)

\section{Quan niệm về con Người} % (fold)
\label{sec:91_92_95_con_nguoi}

— Vấn đề giáo dục truyền đạt các giá trị của nền văn hóa nhân bản và Vô Ngã là vấn đề rất tinh tế. Khi hình thành nền văn hóa này, thì tâm lý con người có khuynh hướng chấp thủ nó và biến nó thành một giá trị đối lập với các giá trị cũ. Vô hình trung con người rơi trở lại tình trạng tâm thức của nền văn hóa hữu ngã và ước lệ cũ, như là Tố Nga ở cung Hằng bổng động lòng trần giáng thế tạo nên một cuộc can qua mới. Do vậy, ở đây giáo dục mới cần xác định rõ rằng giá trị của nền văn hóa Vô Ngã không phải đố lập với giá trị hữu ngã. Thế giới giá trị Vô Ngã là bất định; vì bất định nên không thể đóng khung nó trong một phạm trù khu biệt nào; vì bất định nên nó tùy duyên mà hiện ra các giá trị như giá trị hữu ngã. Vì nó là Vô Ngã, nên nó hiện ra tất cả ngã tướng nhằm đem lại an lạc, hạnh phúc cho mọi người, cho số đông. Nó là giá trị sống.
% section quan_niệm_về_con_người (end)

\section{Quan niệm về xã hội} % (fold)
\label{sec:91_92_95_xa_hoi}

— Một xã hội của nền văn hóa Vô Ngã là một xã hội mở, dung nhiếp các dị biệt thế nào để đoàn kết, thống nhất để xây dựng và phát triển hưng thịnh xã hội, và để mỗi người sống hạnh phúc rất là Người.

— Vì là xã hội Dân chủ mở nên nó không chấp nhận các hình thức chuyên chế, độc tài.

— Vì là nhân bản, nên không chấp nhận mệnh lệnh từ các nhân danh thần quyền.

Con người hành động theo ý thức tự giác, tự nguyện vì an lạc, hạnh phúc của số đông. Lý tưởng cao cả để phục vụ là con người và hạnh phúc của con người ở đời này và là của xứ sở, mà không phải sống phục vụ cho các mục tiêu trừu tượng, tưởng tượng của suy lý. Mọi giá trị xã hội đều xoay quanh giá trị cơ bản là nhân bản, thiết thực Vô Ngã và trí tuệ nhằm loại trừ càng nhiều càng tốt các khổ đau trần thế.

Đích đến là Lôi Âm tự, nguồn văn hóa chỉ đạo cho văn hóa ở Đông độ.
% section quan_niệm_về_xã_hội (end)
% chapter Hồi 91_92_95 (end)