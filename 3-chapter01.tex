\chapter{Nhìn Chung} % (fold)
\label{cha:nhin_chung}

Tình người, an lạc và hạnh phúc cho con người trong đời sống hiện tại là mơ ước nghìn thu của cuộc đời. Các văn hào, thi hào, nghệ sĩ, đạo sĩ, các nhà xã hội, văn hóa, giáo dục tư tưởng qua các thời đại vẫn tự nhận cái sứ mệnh nói lên niềm mơ ước rất Người ấy.

Đức Phật Thích Ca đã dạy: {\bf ``Ta ra đời vì lợi ích, an lạc của Chư Thiên và loài Người''}. Có lẽ Ngô Thừa Ân nghe được lời dạy ấy vọng lên trên từng dòng kinh Phật nên mới sáng tác bộ tiểu thuyết Tây Du Ký để thể hiện sứ mệnh của một tư tưởng lớn giới thiệu một hướng văn hóa đầy tình người, an lạc và hạnh phúc. Tác giả hầu như đặt trọn niềm tin vào đấng Toàn Giác như một nhà đại giải thoát và đại văn hóa của nhân loại, vừa giới thiệu con đường thoát khổ vừa đề bạt một hướng văn hóa mới cho nhân dân Đông Độ.

Ngày nay, xã hội đã ý thức rằng giáo dục làm nên văn hóa và văn minh, công việc xây dựng một nền văn hóa mới là công việc xây dựng một hệ thống giáo dục mới và toàn bị. Một hệ thống giáo dục mới đòi hỏi có sự hoàn bị về lý thuyết nhân tính (\emph{Theory of Personnality}) xác định rõ con người là gì, để xác định rõ giáo dục con người như thế nào. Giáo dục con người như thế nào là câu hỏi về nội dung giáo dục và kỹ thuật giáo dục. Tại đây cần xây dựng một triết lý về con người, đề cập đến sự thật như thật của con người và cuộc đời. Đề cập đến vai trò của nhận thức, tư duy hữu ngã và vấn đề giá trị, đạo đức của đời sống.

Về con người, nền văn hóa cổ Hy Lạp và La Mã chưa có tiếng nói thuyết phục; nền văn hóa Trung Hoa cũng thế, chưa có một cái nhìn ổn định. Ngô Thừa Ân nhận thấy cần có cái nhìn mới về con người, giá trị. Cái nhìn mới ấy đang ngự trị Lôi Âm Tự, vùng văn hóa Tây Trúc. Nền văn hóa này giới thiệu con người là một tập hợp của Năm Uẩn: \emph{sắc, thọ, tưởng, hành thức uẩn}. Sắc uẩn là phần vật lý; thọ, tưởng, hành, và thức thuộc tâm lý. Bên trong con người Năm Uẩn ấy có mặt trong nguồn năng lực và một trí tuệ vô hạn có thể giúp con người chuyển khát vọng hạnh phúc và chân lý thành hiện thực ngay tại đời sống này.

Sắc uẩn gồm có nội sắc (thân vật lý của mình) và ngoại sắc (thân tha nhân và thế giới vật lý). Thọ uẩn gồm có nội thọ (những cảm nhận hỷ, lạc trong thiền định) và ngoại thọ (những cảm nhận hỷ, lạc qua các giác quan bên ngoài). Tưởng uẩn gồm có tưởng về sắc, thanh, hương, vị, xúc và về pháp. Hành uẩn gồm có tư duy (tác ý) về sắc, thanh, hương, vị, xúc, pháp. Thức uẩn gồm có nhãn thức, tỉ, thiệt, thân và ý thức.

Như vậy, con người theo giáo lý nhà Phật và theo quan điểm của nền văn hóa mới, không phải là một ngã thể độc lập với tha nhân, gia đình, xã hội, thiên nhiên, mà gắn liền với các hiện hữu ấy. Thế giới này, xã hội này là chính cơ thể và tâm thức con người cần được bảo vệ như bảo vệ tự thân và hạnh phúc của tự thân.

Với nhận thức Duyên Khởi, Vô Ngã của nền văn hóa mới, tư duy mới (Chánh Tư Duy) con người, gia đình, xã hội và môi trường là một khối nhất thể, không thể tách ly: con người là con người của gia đình, xã hội và môi sinh; môi sinh là môi sinh của con người và xã hội, vv\ldots.

Cái nhìn mới này sẽ đưa ra các giá trị mới, thái độ sống mới và một nền văn hóa, giáo dục mới nhằm đem lại hòa bình, an lạc và hạnh phúc lâu dài cho xã hội.

Hạnh phúc không phải là của quan niệm, khái niệm, mà là của sự sống giác tỉnh trong hòa điệu của Năm Uẩn. Hạnh phúc vì thế không phải của các ngã tưởng, ngã niệm hay của dục vọng vị kỷ, mà là của sự chế ngự các ngã tưởng, chế ngự các dục vọng. Hạnh phúc và con người là một, và là cái một của tương quan Duyên Khởi. Con đường sống của nền văn hóa mới vì vậy là con đường sống của vị tha, của hòa điệu, của sự dập tắt các ngã tưởng (với đời sống xã hội thì chế ngự các ngã tưởng).

Ngày nay, nền văn minh hiện đại đã mạnh mẽ nói lên rằng văn minh vật chất càng phát triển thì cần phải tổ chức tốt nền văn minh tinh thần. Cuộc khủng hoảng về dân số, đạo đức và ô nhiễm môi sinh hiện tại đang đặt ra những thử thách lớn. Làn sóng ung thư và SIDA đang ám ảnh con người. Nếu tất cả (mọi người) đều thấy rõ sự thật của Duyên sinh hay Năm Uẩn thì sẽ ý thức rằng khổ đau, rối ren ở chỗ này là khổ đau rối ren của chỗ kia, của toàn trái đất, và sẽ cùng nhau mở đường đi ra khỏi cuộc khủng hoảng, cùng nhau bảo vệ trái đất như bảo vệ chính mình. Cái thấy biết sự thật ấy đang trông cậy vào vai trò của văn hóa giáo dục \emph{``hậu hiện đại''}.

Các xã hội nếu được xây dựng trên cơ sở Duyên sinh Vô Ngã thì chiến tranh sẽ dần dần biến mất, các tệ nạn xã hội (cướp bóc, tham nhũng, vv\ldots) sẽ dần dần biết mất. Con người nếu rõ Duyên sinh, Vô Ngã thì sẽ nhận ra, qua thể nghiệm, dục vọng là khổ đau, vị kỷ là khổ đau, si mê là khổ đau, và sẽ tự động tách rời khỏi chúng. Tại đây con người sẽ sung sướng nhận ra rằng: cuộc sống đang còn nguyên ở đó tràn đầy những âm thanh và màu sắc; lẽ sống của mỗi cá nhân không phải là ham muốn; và hạnh phúc phải là thỏa mãn các ham muốn, mà là sự biết sống, biết nhìn, biết đủ và biết rằng cuộc sống chính là mình. Mọi sự vẫn tiếp tục trôi chảy; con người tiếp tục sống như đang sống, nhưng sống với sự giác tỉnh nói trên.

Từ đây, từ sự thật Duyên khởi, con người sẽ không bàn đến hệ tư tưởng cho rằng có một đấng toàn năng sáng tạo ra vũ trụ, và giữ quyền thưởng phạt con Người. Từ đây, con người trở về chính mình sống nương tựa mình và nương tựa sự thật Duyên khởi; trở về với sự hòa điệu của tim và óc, của cá nhân và xã hội, của thiên nhiên và con người, trở về với sự tách rời khỏi các nhân tố gây ra khổ đau cho mình và người (như chấp ngã, dục vọng, ích kỷ, đố kỵ, vv\ldots).

Đấy là một số ý niệm tổng quát, Ngô Thừa Ân đã mở ra một cuộc Tây Du vượt qua 81 ma nạn trước khi đặt chân tới Lôi Âm Tự, thì nền văn hóa giáo dục hậu hiện đại cũng cần thời gian để giáo dục con người và tổ chức xã hội theo nhận thức, tư duy mới (nhân bản và trí tuệ) để hình thành nếp sống văn hóa mới, phải vượt qua nhiều khó khăn trước khi nhìn thấy thành quả. Có nhiều việc làm mà chúng ta sẽ lần lượt bàn đến.

% chapter nhìn_chung (end)