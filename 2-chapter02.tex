\chapter{Hồi 2} % (fold)
\label{cha:hoi_2}

\begin{verse}
\begin{itshape}
Thấu lẽ bồ đề, vào chính quả.\\
Dứt căn ma quỷ, rõ nguyên nhân.
\end{itshape}
\end{verse}

\section{Về tư tưởng Phật Học} % (fold)
\label{sec:2_tu_tuong_phat_hoc}

-- Hồi thứ hai là hồi học đạo và hành đạo của Tôn Ngộ Không. Tôn Ngộ Không đã học, bàn luận, và hành {\bf ``Tam Thừa Giáo''} và {\bf ``Nhất Thừa Giáo''} suốt bảy năm liền, như Ngô Thừa Ân giới thiệu:

\begin{verse}
\begin{itshape}
\ldots\\
Diễn đủ Tam Thừa Giáo,\\
Lúc diễn thiền môn, khi giảng đạo,\\
Ba nhà hợp lại nghĩa thêm càng.
\end{itshape}
\end{verse}

Trong thời gian này Tôn Ngộ Không làm công tác của một người xuất gia, chấp tác và thực hành \emph{văn tuệ, tư tuệ và tu tuệ} để nuôi dưỡng trí tuệ giác ngộ Không tánh (Vô Ngã tánh) của các pháp. Ngô Thừa Ân đã vẽ nên thời gian tu tập này rành rẽ hệt như một nhà Phật Học chính gốc. Động cơ và nguyện vọng xuất gia của Tôn Ngộ Không là để giải thoát hết thảy phiền não, khổ đau của sinh tử rất tương ứng với giáo lý của Phật Giáo. Ngô Thừa Ân đã dẫn dắt Tôn Ngộ Không đi đúng tiến trình Giới, Định, Tuệ giải thoát khiến cho người đọc Tây Du Ký không còn nghi ngờ gì giá trị của bộ tiểu thuyết này mà trọng tâm là giới thiệu con đường (hay lộ trình tu tập giải thoát của Đạo Phật). Chúng ta hãy tiếp tục theo dõi.

-- Sau bảy năm hành đạo, khi Tôn Ngộ Không hiểu rõ đường đi chỗ đến, Tôn giả Tu Bồ Đề dạy tiếp tục công phu Thiền Chỉ và Thiền Quán để chế ngự thân và ý, để điều hòa thân và điều hòa tâm. Tôn Ngộ Không thực hành công phu này trong ba năm thì sáng tỏ tâm và tuệ.

Một hôm Tôn giả hỏi Tôn Ngộ Không:

\begin{quote}
--- Ngươi tu đã đắc quả gì rồi?

--- Dạ, đệ tử gần đây pháp tánh hơi thông, căn nguyên cũng dần kiên cố.

--- Đã thông pháp tánh, đã rõ căn nguyên, còn phải đề phòng ba tai họa nữa.
\end{quote}
Nghĩa là cần tiếp tục công phu chỉ, quán để vượt qua ba tai họa ấy, điều mà kinh nghiệm giải thoát của Tôn Ngộ Không chưa đặt chân đến được.

Do vì thành quả tu tập của Tôn Ngộ Không ngang đây là sắp cướp được quyền sinh diệt của tạo hóa, nên tạo hóa sẽ giáng xuống ba tai họa. Đó là họa \emph{``sét đánh''} có thể làm ngưng trệ sinh mệnh giải thoát; họa \emph{``âm hỏa''} (không phải là lửa trời, cũng không phải là lửa của người), và họa \emph{``bi phong''} (không phải là gió trời) rất mạnh cũng gây ảnh hưởng khốc hại đến sự nghiệp giải thoát.

Ngôn ngữ của Ngô Thừa Ân tại đây rất là tiểu thuyết và rất là biểu tượng, nhưng đồng thời cũng diễn đạt các bước đi tu tập giải thoát rất là thiện xảo.

Trong Phật Giáo, ba tai nạn đó là gì? Trở về với giáo lý truyền thống Phật Giáo, khi hành giả thông rõ pháp tánh là khi đắc pháp nhãn thanh tịnh, bước vào dòng Thánh; bấy giờ còn lại tập khí sinh tử phải trừ, bao gồm bảy kiết sử: \emph{dục, sân, hữu ái, vô hữu ái, mạn, trạo cử và vô minh}.

Nạn ``sét đánh'' là biểu tượng của vô minh (hay \emph{si}). Vô minh ở đây là nội dung của sự việc tâm thức bị rơi vào chấp trước quả đắc (gọi là \emph{mạn}), lúng túng không thấy lối ra (gọi là \emph{trạo cử}) mà tự mình không biết (gọi là \emph{vô minh}).

Sự chấp trước này thì rất tế nhị nhưng rất quyết định trong việc làm tắc nghẽn trí tuệ giải thoát sau cùng, tựa như sấm sét nhanh chóng kết liễu sinh mệnh con người. Để vượt qua, hành giả cần hành định sâu và thiền định sâu và thiền quán mạnh về Vô Ngã để cũng nhanh chóng như tiếng sét cắt đứt ngay sinh mệnh của chấp trước vô minh.

Nạn ``âm hỏa'' là biểu tượng cho lửa sân, và lửa tham dục, ở cấp độ tế nhị. Đó là \emph{dục, sân, kiết sử, hữu ái và vô hữu ái}; lửa này cũng có khả năng thiêu cháy trí tuệ toàn giác làm héo úa thân huệ mạng.

Nạn ``bi phong'' là biểu tượng của tâm lay động chao đảo của hành giả do còn chấp pháp, do còn non định lực giải thoát trước bi nguyện độ sinh vô lượng. Định và tuệ Vô Ngã còn hạn lượng của hành giả sẽ không thể đứng vững được trước bi nguyện độ sinh vô hạn lượng. Bi nguyện vô hạn lượng này, sẽ như một cơn gió lốc vô cùng mạnh, làm choáng ngợp hành giả, khiến hành giả không thể hiện được Bồ Tát hạnh để chứng đắc trí tuệ toàn giác.

Chính ba tai nạn này sẽ tái hiện nhiều lần dưới nhiều thể trạng khác nhau đối với hành giả mà các hồi sau chúng ta sẽ có dịp tiếp tục bàn đến.

-- Để giúp đệ tử Tôn Ngộ Không có điều kiện thoát khỏi ba tai nạn trên, Tôn giả Tu Bồ Đề đã chỉ dạy sâu thêm công phu tu tập để phát triển mạnh thiền quán Vô Ngã, củng cố thêm nhiều cho trí tuệ giải thoát, bằng pháp môn ``địa sát'' gọi là 72 pháp biến hóa thần thông (thất thập nhị huyền công) và ``cân đẩu vân'', hầu có đủ mọi cách làm bật lên gốc rễ của vô minh và quyến thuộc vô minh.

Chuẩn bị hành trang giải thoát cho Tôn Ngộ Không kỹ càng như thế quả Ngô Thừa Ân đã nắm vững lộ trình tâm thức giải thoát của Phật Giáo, ít nhất là về mặt giáo lý.

Tại đây, Tôn Ngộ Không hẳn đã có trí tuệ giải thoát ở mức độ ít nhất là giác tỉnh năm uẩn là Vô Ngã, không còn tự ý giam hãm tâm thức giải thoát của mình vào năm uẩn. Và hẳn đã có tâm giải thoát ở mức độ Tứ Sắc Định để có thể biến hóa theo ý muốn, hệ lụy do tập khí thì vẫn còn. Tại đây, sau khi từ giã bậc đạo sư Đại Trí Tuệ, đệ nhất ly thủ, đệ nhất giải Không, trở về Hoa Quả Sơn, Tôn Ngộ Không đã dễ dàng dẹp Hỗn Thế Ma Vương, loạn của các ngã tưởng, để bình định lại giang sơn trí tuệ của mình tại Thủy Liêm động.
% section về_tư_tưởng_phật_học (end)

\section{Quan niệm về con Người} % (fold)
\label{sec:2_quan_niem_ve_con_nguoi}

-- Ở mặt giải thoát thì Tôn Ngộ Không là biểu tượng của trí tuệ cần được tu tập để phát triển giải thoát và để chế ngự các cảm thọ khổ đau. Xây dựng một nhân cách giải thoát là xây dựng Giới, Định, Tuệ và lòng đại bi cho nhân cách ấy. Ở mặt xã hội, con người cũng cần được giáo dục, huấn luyện một nhân cách xã hội gồm tri, hành và lòng nhân ái thế nào để thành công, lợi lạc cho mình và người.

Tại hồi hai, vai trò lý trí, được biểu trưng bằng nhân vật Tôn Ngộ Không, là vai trò chỉ đạo mọi hành động của con người cũng cần được huấn luyện, giáo dục. Thường thì mẫu người giáo dục của Trung Hoa được đào tạo theo khuôn mẫu đạo Khổng Nho và được thêm vào một ít Lão, Trang. Ngô Thừa Ân đã rất sáng tạo và rất can đảm giới thiệu một khuôn mẫu giáo dục theo giáo lý nhà Phật.

Có lẽ sự chọn lựa này đã đến với Ngô Thừa Ân sau khi Ngô Thừa Ân chứng kiến sự đổ vỡ của xã hội phong kiến Trung Hoa với đầy rẫy những bất công áp bức, chủ nghĩa hình thức, kém nhân bản, vv\ldots ~và vai trò của Nho Giáo không đáp ứng nổi nhiều yêu cầu mới của con người và lịch sử -- Như về sau Kim Dung, trong tiểu thuyết kiếm hiệp Tiếu Ngạo Giang Hồ, đã xây dựng mẫu người Lệnh Hồ đại hiệp thay thế quân tử kiếm Nhạc Bất Quần -- Tây Du Ký ra đời như là tiếng mời gọi nhân dân Trung Hoa chú ý đến tiếng nói rất trí tuệ, hiện thực và nhân bản vọng về từ Tây Trúc. Không phải là chú ý đến sự nghiệp Tây Du thỉnh kinh của Đường Tăng, mà là chú ý đến nhân cách được xây dựng từ giáo lý này: một nhân cách sống vì hạnh phúc an lạc của số đông, sống hiền thiện vì công bằng, bình đẳng, tôn trọng sự thật, trách nhiệm cá nhân và đặc biệt là sống tùy duyên rất là trí tuệ.

-- Nhân cách ấy cần được huấn luyện, giáo dục rất thực, mà không phải của từ chương và hình thức -- rất vững chắc cả ba mặt hành động của \emph{thân, lời và ý}, như là Tôn Ngộ Không đã được huấn luyện \emph{văn, tư, tu} trong bảy năm và khả năng làm chủ chủ tư duy và tâm lý (thiền định) trong ba năm trước khi vào đời để tiếp tục thực hành và học hỏi thực tế của trường đời.

-- Triết lý chỉ đạo cho nhân cách ấy là giáo lý Vô Ngã, Vô Thường, Khổ, Không của Phật Giáo (mà nói tắt là giáo lý Vô Ngã). Do thấm nhuần sự thật Vô Ngã, người học viên dần dần làm bật lên được gốc rễ chấp ngã, gốc rễ của các tâm lý vị kỷ, bảo thủ, chật hẹp, tham lam, sân hận, lừa dối và các sầu bi, khổ, ưu, não; và tại đây, một nhân cách mẫu mực xuất hiện để xây dựng an lạc, hạnh phúc cho bản thân, gia đình và xã hội.

Triết lý hành động của nhân cách ấy là trí tuệ Vô Ngã (Chánh Kiến và Chánh Tư Duy) phân biệt chánh, tà, hư, thực, tốt, xấu; là lòng vị tha và sức mạnh vô úy. Tất cả ấy là con người, vì con người, cho con người hiện thực, và do con người trách nhiệm. Đây là mẫu nhân cách trí tuệ, nhân bản và sinh động.

Khi mà cá nhân có tự do tâm lý, làm chủ được tâm lý thì cá nhân sáng suốt hơn và có nhiều sáng tạo hơn. Chính yếu tố tâm lý sáng tạo này sẽ đóng góp lớn cho xã hội.

Chúng ta sẽ tiếp tục bàn đến nhân cách này và giá trị của nhân cách này trong các hồi kế tiếp.

Ngô Thừa Ân quan niệm nhân cách giáo dục ấy tốt hơn nhân cách xây dựng từ đạo Nho nhiều, nhưng đồng thời cũng thấy khó khăn trong việc thực hiện trong xã hội phong kiến Trung Hoa, bởi giáo lý Phật Giáo nói lên tính bình đẳng, dân chủ, tự chủ, độc lập sẽ lay đổ ngai vàng và quyền lợi của ngai vàng. Để bảo vệ quyền lợi của phong kiến, triều đại vua chúa (Trời) sẽ gieo rắc ngay tai họa chết người mà được gọi là nạn ``sét đánh'', nạn ``âm hỏa'' và nạn ``bi phong''.

Vì thế Ngô Thừa Ân đã chuẩn bị khá kỹ lưỡng cho Tôn Ngộ Không với một căn bản định, tuệ và ``thất thập nhị huyền công'' để vận dụng đối phó với các kháng lực đến từ chính quyền phong kiến và từ phía các tà giáo trên đường thực hiện lý tưởng xây dựng một \emph{xã hội công bằng, dân chủ, bình đẳng và nhân bản, và xây dựng một nếp tư duy mới về giá trị nhân sinh và xã hội}.

-- Bước đầu tiên của việc thể hiện lý tưởng mà Tôn Ngộ Không đã được giáo dục là dẹp giặc loạn xâm lược của Hỗn Thế Ma Vương và tiêu diệt toàn bộ thế lực xâm lược này. Hỗn Thế Ma Vương là biểu tượng của dục vọng, ác hại mà Tôn Ngộ Không cần loại bỏ trước khi bắt tay vào việc giáo dục kiến thiết giang sơn Thủy Liêm Động.

% section quan_niệm_về_con_người (end).

\section{Quan niệm về xã hội} % (fold)
\label{sec:2_quan_niem_ve_xa_hoi}

-- Về mặt xã hội, từ đây toàn thể đoàn khỉ đều có họ Tôn của Tôn Ngộ Không và chịu sự lãnh đạo tổ chức của Ngộ Không. Tất cả đều bình đẳng về quyền sống trong động Thủy Liêm, núi Hoa Quả. Tất cả đều đoàn kết thành một khối ở đằng sau Tôn Ngộ Không và cùng có trách nhiệm trong việc xây dựng và bảo vệ xã hội loài khỉ theo hướng phát triển hưng thịnh.

-- Tôn Ngộ Không liền tổ chức chặt chẽ an ninh, quốc phòng để bảo vệ thành quả thanh bình của xứ sở. Thành quả xã hội ấy sẽ đến từ nền giáo dục tinh thần trách nhiệm cá nhân, tinh thần độc lập tự cường, tinh thần Vô Ngã, vị tha, tinh thần dân chủ và nhân bản. Các tinh thần giáo dục ấy sẽ tạo nên tình đoàn kết keo sơn của xã hội, sẽ tạo nên những công dân tốt, loại trừ được các tệ trạng xã hội như dối trá, cướp bóc, tà hạnh, tham nhũng, áp bức, lười biếng, thủ lợi, sa đọa, \ldots

Tại đây, xã hội loài khỉ, hiện ra như một xã hội thí điểm mà Ngô Thừa Ân muốn giới thiệu, và mẫu thí điểm này là dành cho xã hội con Người, đặc biệt là xã hội Trung Hoa với nhiều thành kiến. Hẳn nhiên Ngô Thừa Ân đã dự phòng trước các sức mạnh đề kháng từ bên trong và bên ngoài trong việc xây dựng xã hội mới hợp lý hợp tình hơn. Chúng ta hãy chờ xem trong các hồi tới Ngô Thừa Ân sẽ làm gì tiếp cho xã hội mới này?
% section quan_niệm_về_xã_hội (end)
% chapter hoi_2 (end)