\chapter{Hồi 43} % (fold)
\label{cha:hoi_43}

Hồi 43:

\begin{verse}
\begin{itshape}
Ma sông Hắc Thủy lừa sư bắt.\\
Rồng bể Tây Dương tróc quái về.
\end{itshape}
\end{verse}

\section{Tư tưởng Phật Học} % (fold)
\label{sec:43_phat_hoc}

-- Các nạn gây trở ngại giải thoát cho người tu hành đều phát sinh từ \emph{dục ái, hữu ái và vô hữu ái}. Có thể nói các nạn ấy phát sinh từ Vô Minh, chấp thủ các ngã tướng. Theo Phật Giáo, tất cả các ma nạn đều xẩy ra từ thế giới \emph{hữu vi, hữu tướng, từ sự chấp thủ năng và sở}.

Nói theo ngôn ngữ đạo Nho, các khổ nạn đều sinh ra từ Âm Dương, Ngũ Hành biểu hiện qua nhiều cấp độ và tướng trạng khác nhau. Như:

\begin{enumerate}[label=\itshape\arabic*\upshape/]
    \item nạn Bạch Cốt tinh thì sinh ra từ Thổ (xương thuộc thổ), hồi 27;

    \item nạn Hoàng Bào tại rừng Hắc Tùng do Khuê Mộc Lang tinh gây ra, đó là nạn sinh ra từ Mộc, hồi 28, 29, 30;

    \item nạn ở động Liên Hoa và Ngân Giốc và Kim Giốc tạo là nạn sinh từ kim, hồi 32 -- 36;

    \item nạn Hồng Hài Nhi tại động Hỏa Vân, là nạn sinh từ Hỏa, hồi 40 -- 42;

    \item nạn tại sông Hắc Thủy là nạn sinh từ Thủy, hồi 49.
\end{enumerate}

Riêng hai nạn sinh ra từ Hỏa, Thủy thì có thể gọi là nạn sinh từ Âm, Dương.

Tại đây, chúng ta hiểu rằng nền văn hóa Âm Dương, Ngũ Hành đã là các ngăn ngại sự phát triển nền văn hóa Vô Ngã. Ngô Thừa Ân đã đưa phái đoàn Tây Du vượt qua các nạn kể trên là muốn xác định rằng tư tưởng Phật Học có giá trị nhân sinh và xã hội vượt lên trên giá trị của đạo Nho, có phải chăng là như thế?

-- Con sông, trong giáo lý nhà Phật thường được biểu tượng cho \emph{sông ái, sông mê}. Với công phu giải thoát đến hồi 43 này thì sông ái chỉ có thể hiểu là chỉ cho sắc ái, lòng ham muốn tồn tại ở ngoài Dục giới của Ngũ Dục Lạc. Chính lòng ham muốn hiện hữu này khiến hành giả kẹt vào các thiền cảnh (từ sơ thiền đến tứ thiền) như bị kẹt vào Hỏa Vân động và Hắc Thủy hà.

Để tự mình ra khỏi nạn đó, hành giả cần hành thiền quán Vô Ngã, phát khởi mạnh trí tuệ Vô Ngã như là phái đoàn Tây Du bấy giờ cần đến Tôn Ngộ Không mới hàng phục được quái ở Hắc Thủy hà (Hắc Thủy hà còn là hình ảnh biểu tượng cho các ái tùy miên đang chìm sâu trong tâm thức; nó sẽ trỗi dậy khi có duyên và sẽ nhấn chìm hành giả, hệt như quái Hắc Thủy hà đã nhấn chìm Đường Tăng và Trư Bát Giới. Nếu nạn Hắc Thủy là nạn do tập khí nghiệp, do các tùy miên, thì nạn Hỏa Vân là do dục vọng bốc cháy nơi các căn và các trần).
% section tư_tưởng_phật_học (end)

\section{Quan niệm về con Người} % (fold)
\label{sec:43_con_nguoi}

-- Khi nền văn hóa, giáo dục mới được giới thiệu sâu rộng đến quần chúng thì quần chúng sẽ phân ra làm hai nhóm:

\begin{enumerate}[label=\itshape\arabic*\upshape)]

    \item {\bf Nhóm ủng hộ: nhóm này bao gồm nhiều nhân cách:}

    \begin{enumerate}[label=\itshape\alph*\upshape/]

        \item Nhân cách lý tưởng có đủ trí tuệ, ý chí và tấm lòng như nhân vật Tôn Hành Giả.

        \item Nhân cách tốt có quyết tâm và có lòng tốt nhưng non trí tuệ như nhân vật Sa Ngộ Tịnh.

        \item Nhân cách có tấm lòng, nhưng thiếu trí và thiếu quyết tâm như nhân vật Trư Ngộ Năng.

        \item Nhân cách nhân ái, có lòng và có quyết tâm cao nhưng thiếu trí tuệ như Đường Tăng.
    \end{enumerate}

    \item {\bf Nhóm phản kháng:}

    \begin{enumerate}[label=\itshape\alph*\upshape/]

        \item Thành phần có danh vọng lớn (triều đình) có quyền lợi gắn liền với triều đình và thành phần ``ăn theo''.

        \item Thành phần bảo thủ nền văn hóa cũ do có định kiến xã hội như Thiên Triều, Thiên Tiên và Địa Tiên.

        \item Thành phần vị kỷ, sống dựa vào các thế lực lãnh đạo cũ, như trường hợp quái Hắc Thủy.

        \item Thành phần hung ác như các ma, quỷ.
    \end{enumerate}
\end{enumerate}

Người cán bộ của nền văn hóa mới cần nhận rõ các nhóm quần chúng để vận dụng thành công công cuộc xây dựng và phát triển nền văn hóa mới.

-- Trong trách nhiệm giáo dục, xây dựng nhân cách mới, người hướng dẫn cần vạch rõ cho quần chúng biết các vướng ngại của tự thân, gồm:

\begin{enumerate}[label=\itshape\alph*\upshape/]
    \item Các tập quán nhận thức và hành động cũ biểu hiện ra bên ngoài đời sống xã hội như là các kháng lực đối với đời sống xã hội và tương giao xã hội.

    \item Các tập quán nhận thức, hành động và tình cảm đã ăn sâu vào tâm thức. Các tập quán này có thể ảnh hưởng tới quá trình xây dựng nền văn hóa mới. Hình ảnh quỷ sông Hắc Thủy với cung điện nằm sâu dưới lòng nước đen đã nhấn chìm một nửa phái đoàn Tây Du là biểu tượng của các nghĩa vừa được trình bày.
\end{enumerate}

Hơn ai hết, cán bộ nòng cốt của nền văn hóa mới như Tôn Ngộ Không và đồng sự của Tôn Ngộ Không cần ý thức sâu sắc nghĩa biểu tượng của hồi 43 này để hành động cẩn trọng và hiệu quả.
% section quan_niệm_về_con_người (end)

\section{Quan niệm về xã hội} % (fold)
\label{sec:43_xa_hoi}

Về mặt xã hội, tương tự về mặt nhân sinh, đối với nền văn hóa giáo dục mới cũng có thể có những phản kháng bốc cháy ở bên ngoài (như nạn Hồng Hài Nhi) hay những kháng lực ngấm ngầm khó thấy đầy sức mạnh như nạn sông Hắc Thủy. Những người làm văn hóa mới cần đối phó với cả hai sức mạnh phản kháng này, mỗi sức mạnh gây ra một nguy hiểm.

Ngô Thừa Ân để lại tập truyện Tây Du Ký như bản di chúc văn hóa cho hậu lai thực hiện qua nhiều thế hệ tiếp nối. Công tác văn hóa bao giờ cũng đòi hỏi thời gian dài như thời gian thực hiện giải thoát của mỗi cá nhân vậy.
% section quan_niệm_về_xã_hội (end)
% chapter Hồi 43 (end)