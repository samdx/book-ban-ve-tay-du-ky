\chapter{Một nền giáo dục Duyên Khởi} % (fold)
\label{cha:mot_nen_giao_duc_duyen_khoi}

Hơn ba phần tư nhân loại hiện nay không biết đến đạo Phật, hay không nghiên cứu đạo Phật. Số còn lại có sống với, có tìm hiểu, có nghiên cứu hay quan sát sinh hoạt của Phật Giáo.

Các nhà xã hội học, các nhà văn hóa, tư tưởng hay giáo dục hiện đại vẫn chưa có nhân duyên tìm hiểu cặn kẽ, sâu xa giáo lý Phật Giáo. Với những vị có nghiên cứu sâu thì chỉ nghiên cứu đạo Phật như như một hệ thống tư tưởng triết lý. Với những vị này chỉ thiếu điều kiện thực hành nên cũng khó nhận ra chân giá trị của đạo Phật. Hầu hết cứ đinh ninh rằng Phật Giáo là một tôn giáo, vì thế Phật Giáo chỉ dành cho lãnh vực tín ngưỡng, mà xa lạ với các nếp sống con người khát vọng tạo dựng hạnh phúc trần gian. Hoặc giả đinh ninh rằng Phật Giáo cứ nói mãi về khổ đau thì là tiếng nói bi quan; hay Phật Giáo chủ trương ly dục là một chủ trương không thiết thực, xa lạ với con người, vv\ldots. Thật là đầy ngộ nhận! Làm sao người ta có thể tin rằng Phật Giáo có thể giới thiệu với đời một đường hướng giáo dục rất thiết thực, nhân bản và trí tuệ?!

Vào thế kỷ XVI, Ngô Thừa Ân, qua tiểu thuyết Tây Du Ký, tin tưởng rằng Phật Giáo sẽ giúp cải thiện đường hướng văn hóa, giáo dục của xã hội Trung Hoa, ông đã viết những suy nghĩ của ông thành bộ tiểu thuyết Tây Du Ký bất hủ, nhưng những tâm sự, suy nghĩ, hoài bão của ông vẫn được gói kín trong 100 hồi truyện cho đến ngày nay. Những tư duy phát hiện giá trị đạo Phật đầy trân quý của ông chỉ để giúp đời giải trí sau những ngày giờ làm việc mệt mỏi, bởi vì đấy là tiểu thuyết, mà không phải là giáo dục, tư duy và tâm lý. Kinh Phật thì chỉ để dành cho tu sĩ Phật Giáo và để tôn quý, bởi vì đó là tôn giáo mà không phải là văn hóa giáo dục. Thật là đáng tiếc!

Ngày nay, chúng ta thử nhìn xem có thể xây dựng một đường hướng giáo dục trên căn bản những lời dạy của Đức Phật không?

{\bf Giáo dục có hai mục tiêu chính:}

\begin{enumerate}[label=\itshape\arabic*\upshape/]

    \item Giáo dục con người tự thân hiểu mình và biết hướng dẫn đời mình đi đến hạnh phúc ngay trong hiện tại.

    \item Giáo dục con người xã hội hiểu biết xã hội mình đang sống và đáp ứng các yêu cầu của xã hội và lịch sử với các mục tiêu tức thời và lâu dài.
\end{enumerate}

Về mục tiêu thứ nhất của giáo dục, thì đạo Phật đáp ứng trọn vẹn, nếu không nói là rất lý tưởng. Về mục tiêu thứ hai của giáo dục là giảng dạy cung cấp cho học viên, sinh viên các kiến thức chuyên môn để xây dựng và phát triển xã hội. Trong mục tiêu này, xã hội và lịch sử còn yêu cầu học đường dạy một tinh thần đoàn kết dân tộc keo sơn thể hiện ngay trong học đường và ngoài xã hội. Về yêu cầu này, học đường cần có một triết lý giáo dục như thế nào vừa nhân bản vừa dân tộc, vừa có thể vượt qua các dị biệt. Hẳn là giáo lý Duyên Khởi, Vô Ngã đáp ứng yêu cầu này. Duyên Khởi là sự thật của vạn hữu, nên không thuộc riêng của ai hay của dân tộc nào, vì vậy nó là của mọi người. Vì Vô Ngã nên khi một người chấp nhận Duyên Khởi, Vô Ngã thì dễ dàng hòa hợp với các ngã, với các tư duy tín ngưỡng hữu ngã; do vậy dễ dàng đến với mọi người, dễ dàng hòa hợp đoàn kết. Vì có nhận thức Vô Ngã nên có thái độ sống vô chấp, không bảo thủ, khoan dung, nên cũng dễ dàng đoàn kết, thống nhất với mọi người, mọi khuynh hướng vì đại nghĩa xã hội, dân tộc.

Ngoài hai đóng góp quan trọng trên, Duyên khởi và toàn bộ giáo lý Giới, Định, Tuệ của Phật Giáo có thể có rất nhiều đóng góp rất hữu ích khác nhau cho giáo dục, như giáo dục tinh thần tự tri, tự giác tự trách nhiệm, độc lập, tùy duyên, tự trọng, tinh thần phê phán, tinh thần thiết thực hiện tại, nhân bản, thiền định và sáng tạo, tinh thần khích lệ, vv\ldots

Về triết lý giáo dục, giáo lý Duyên Khởi và Ngũ uẩn có thể giúp học đường xây dựng một lý thuyết về nhân tính ổn định và giá trị. Hai giáo lý này có thể là cơ sở để hình thành nhận thức luận và giá trị luận sáng giá.

Về tâm lý, đạo Phật là con đường đoạn diệt phiền não, lo âu, vốn là vai trò chính của ngành tâm lý giáo dục hiện đại.

Với một số ưu điểm tiêu biểu vừa đề cập, chúng ta có thể đi đến kết luận rằng văn hóa Phật Giáo có nhiều điểm thể hiện các giá trị nhân bản phù hợp với các chuẩn mực đạo đức và văn hóa mà chúng ta đang xây dựng.

Kinh nghiệm lịch sử Việt Nam dưới triều đại Lý, Trần (từ thế kỷ XI đến cuối thế kỷ XIV) cho thấy văn hóa Phật Giáo làm chủ đạo cho văn hóa Việt Nam thời bấy giờ và đã tạo nên một sức mạnh dân tộc phi thường; Bắc thắng Tống, Nam bình Chiêm dưới triều Lý; ba lần đại phá quân Nguyên Mông dưới đời Trần -- đạo quân xâm lược rất hùng mạnh đã đánh bại Trung Hoa và nhiều nước Châu Âu đương thời.

Trong thời kỳ lịch sử cận đại và hiện đại, bên cạnh sự phát triển khoa học, tinh thần giáo lý từ bi và thiền định Phật Giáo đã đóng góp rất nhiều cho sự hùng mạnh của nước Nhật Bản.

Vẫn còn rất nhiều bài học hữu ích, giá trị cần thiết cho con người và cho lịch sử dân tộc rút ra từ Phật Giáo, chúng ta cần công phu nghiên cứu, vận dụng trong sự nghiệp giáo dục văn hóa của nước nhà.

% chapter một_nền_giáo_dục_duyên_khởi (end)