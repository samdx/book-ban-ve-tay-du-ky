\chapter{Hồi 32, 33, 34 và 35} % (fold)
\label{cha:hoi_33_34}

Hồi 32:

\begin{verse}
\begin{itshape}
Núi Bình Đính, cây tào truyền tín.\\
Động Liên Hoa, Bát Giới gặp tai.
\end{itshape}
\end{verse}

Hồi 33:

\begin{verse}
\begin{itshape}
Ngoại đạo làm mê tính thực.\\
Nguyên thần đến giúp lòng ngay.
\end{itshape}
\end{verse}

Hồi 34:

\begin{verse}
\begin{itshape}
Ma Vương mưu giỏi khốn Hầu Vương.\\
Đại Thánh khéo lừa lấy bửu bối.
\end{itshape}
\end{verse}

Hồi 35:

\begin{verse}
\begin{itshape}
Ngoại đạo ra oai lừa kẻ thẳng.\\
Ngộ Không được báu phục tà ma.
\end{itshape}
\end{verse}

\section{Tư tưởng Phật Học} % (fold)
\label{sec:33_34_phat_hoc}

-- Sau phần tu tập củng cố Từ Bi kết hợp với Trí Tuệ là phần đề cập đến vấn đề Giới tu tập với Tuệ. Giới mà tách rời hẳn Tuệ thì giải thoát bị tù hãm ngay, như Bát Giới đi tuần một mình ở núi Bình Đính liền bị thiên ma Ngân Giốc và Kim Giốc cầm tù tại động Liên Hoa.

Ngân Giốc và Kim Giốc vốn là hai tiểu đồng của Thái Thượng Lão Quân do Bồ Tát Quán Thế Âm chủ ý mượn để dựng lên cảnh nạn ở động Liên Hoa để củng cố công phu tu tập của hành giả, của phái đoàn thỉnh kinh.

Các bửu bối đều là các sức mạnh rút ra từ đầu nguồn của thế giới nhị nguyên. Qua được cảnh nạn này là hành giả có kinh nghiệm tu tập để thắng vượt một số trói buộc của Tam Giới. Tại đây, đúng như giáo lý Phật Giáo, tác giả Ngô Thừa Ân xác định vai trò của Giới chỉ để chế ngự các hành động của thân và khẩu, và chỉ chế ngự được dục ái mà khó qua được ý nghiệp với nhiều sức mạnh cản trở thuộc thế giới của \emph{hữu ái và vô hữu ái} (của các cõi Trời Sắc và Vô Sắc). Định cũng thế. Với các tâm sắc giới thanh cao, thì khả năng của định chưa sâu cũng khó tránh sự đắm trước chấp thủ chúng, như Ngộ Tịnh dễ dàng bị Ngân Giốc và Kim Giốc cầm giữ. Ở đây chỉ có sức mạnh của trí tuệ Vô Ngã là có thể chiến thắng, nếu trí tuệ đó đi đôi với Đại Định và Thanh Tịnh Giới. Nếu Giới yếu và Định yếu thì hành giả sẽ rơi ngay vào cảnh nạn trên đường về giải thoát, như trường hợp của phái đoàn Tây Du tại núi Bình Đính này.

-- Nếu trí tuệ thiền quán Vô Ngã chưa sâu thì sẽ không quán sát rõ khía cạnh nguy hiểm, trói buộc của các tâm Sắc Giới và Vô Sắc Giới, như Tôn Hành Giả tại núi Bình Đính không hiểu cách sử dụng của sợi dây vàng và đã bị Ngân Giốc niệm chú dùng dây vàng trói Hành Giả, và dùng hồ lô thu Hành Giả. Nếu người có tu tập thiền định thấy rõ được khía cạnh nguy hiểm của các thiền tâm, thì công phu thiền quán Vô Ngã sẽ thấy rõ con đường xuất ly khỏi trói buộc của các thiền tâm, như Tôn Hành Giả sau khi bị trói và bị nhốt kín vào hồ lô đã tự kiếm được lối ra và đi đến chiến thắng Kim Giốc và Ngân Giốc.
% section tư_tưởng_phật_học (end)

\section{Quan niệm về con Người} % (fold)
\label{sec:33_34_con_nguoi}

-- Trong nội dung để giáo dục con người phát triển toàn diện có phần quan trọng về giáo dục đạo đức. Nếu giáo dục chỉ chú trọng đến đạo đức hình thức và ước lệ mà nhẹ về tri thức và trí tuệ, hay nếu giáo dục đạo đức mà không dựa vào cơ sở lý trí và trí tuệ -- nghĩa là nội dung đạo đức không phù hợp với thực tế con người và thực tế của xã hội, lịch sử -- thì đó là một nền giáo dục bất toàn, khập khễnh cần được cứu nguy, hệt như Bát Giới (biểu tượng cho giáo dục, kỷ luật, luân lý, đạo đức) đi một mình đến động Liên Hoa thì bị lâm nguy cần được Tôn Hành Giả cứu.

-- Có một số vấn đề nội dung giáo dục về đức dục và trí dục cũng tạo ra một bộ mặt con người nhất thời có giá trị, mà thực ra là kìm hãm sự phát triển nhân cách toàn diện của con người, tù hãm con người. Đó là nội dung giáo dục sản phẩm của tư duy hữu ngã (Nhị Nguyên) đầy ước lệ như các khuôn mẫu Trung, Hiếu, Tiết, Nghĩa, \ldots ~và cá giá trị học thuật rất từ chương.

Nền văn hóa giáo dục ấy vẫn có khả năng thu hút đông đảo quần chúng, mê hoặc lòng người, nhưng thiếu hẳn giá trị sự thật khách quan của cuộc đời và giá trị nhân bản đích thực. Giá trị của nền văn hóa, giáo dục ấy cần được xét lại. Nói rõ ra, nền văn hóa, giáo dục của xã hội phong kiến Trung Hoa được biểu tượng qua hình ảnh Kim Giốc và Ngân Giốc (biểu trưng cho Lưỡng Nghi) và năm bửu bối (biểu trưng cho Ngũ Hành) đã bị đánh bại bởi nền văn hóa, giáo dục nhân bản và trí tuệ của Phật Giáo (được biểu tượng bằng hình ảnh Tôn Hành Giả).

Bồ Tát Quán Thế Âm đã mượn năm bửu bối của Thái Thượng Lão Quân và hai tiểu đồng để giàn cảnh vây hãm Tôn Ngộ Không ở động Liên Hoa. Cuộc đọ sức này đã xác định ở đây rằng Tôn Ngộ Không đã thắng điểm tuyệt đối (thật sự là đã hạ Knock-out) Ngân Giốc và Kim Giốc. Không phải Bồ Tát Quán Thế Âm thiếu người và bửu bối mà phải mượn người và bửu bối của Thái Thượng Lão Quân, ở đây Bồ Tát muốn để cho người đời, và cả Tôn Ngộ Không thấy rõ giá trị của nền văn hóa, giáo dục của Phật Giáo (đại diện là Tôn Ngộ Không do Tôn giả Tu Bồ Đề chỉ dạy) ưu việt hơn nền văn hóa giáo dục của Nho Giáo (đại diện là 2 đệ tử và năm bửu bối của Thái Thượng Lão Quân).

-- Điểm ưu việt thứ nhất là về giá trị nhân bản: con người có khả năng bên trong mỗi người có thể tự vận dụng để giải quyết các vấn đề khó khăn trong cuộc sống. Tự tín và tự lực mà không phải đánh mất mình bằng sự dựa vào sức mạnh bên ngoài (ngược lại Kim Giốc và Ngân Giốc dựa vào bửu bối), không để bất cứ sức mạnh nào ở ngoại giới quyết định sinh mệnh mình hay sinh mệnh xã hội mình.

-- Điểm ưu việt thứ hai là giá trị về Vô Ngã tùy duyên rất thiện xảo, thoát ngoài sự trói buộc của các giá trị có tính cố định của nền văn hóa, giáo dục ước lệ đương thời của Trung Hoa.

Ngô Thừa Ân đã khéo léo lồng tâm sự và hoài bảo của mình vào trong bốn hồi tiểu thuyết này, và muốn xã hội Trung Hoa chú ý đến mẫu người văn hóa và giáo dục Tôn Ngộ Không được đào tạo từ núi Linh Đài Phương Thốn (thuộc Linh Thứu sơn, Lôi Âm Tự), động Tà Nguyệt Tam Tinh ( chữ Tâm -- \begin{CJK*}{UTF8}{bkai}心 \end{CJK*}) và có kinh nghiệm về Giới và Định ngang mức độ bấy giờ tại động Liên Hoa.
% section quan_niệm_về_con_người (end)

\section{Quan niệm về xã hội} % (fold)
\label{sec:33_34_xa_hoi}

-- Song song với công tác văn hóa và giáo dục để giáo dục con người là công tác xây dựng và phát triển xã hội.

-- Từ hồi 27 đến 31, Tôn Ngộ Không do thiếu tinh tế về mặt vận động tâm lý (trí phương tiện) đã tạo ra sự chia rẽ làm tan rã hàng ngũ. Tại đây, Tôn Ngộ Không đã thức tỉnh, không quá ỷ lại vào sức mạnh của trí tuệ Vô Ngã của mình, nên đã khéo vận dụng tâm lý Đường Tăng và Trư Ngộ Năng tạo nên sức mạnh hợp quần trước khi tiến vào mặt trận xây dựng và phát triển xã hội.

-- Tại đây, sau khi có sức mạnh hợp quần, và sau khi sức mạnh của kỷ luật và trí tuệ của tập thể được nâng cao, tập thể còn cần đi từng bước tổ chức khéo léo và hợp lý hợp tình như Tôn Ngộ Không đã khéo léo khắc chế được năm bửu bối, Ngân Giốc, Kim Giốc và các yêu động Liên Hoa.

Đây là vài điểm cần nắm vững trong toàn bộ kế hoạch xây dựng một nền văn hóa giáo dục mới tại một xã hội cũ đầy thành kiến, định kiến sai lầm về con người và xã hội.
% section quan_niệm_về_xã_hội (end)
% chapter Hồi 33_34 (end)