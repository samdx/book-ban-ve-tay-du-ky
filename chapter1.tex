\chapter{Lời Trần Thuyết} % (fold)
\label{cha:loi_tran_thuyet}

Gần đây Đài truyên hình Thành phố Hồ Chí Minh chiếu bộ phim Tây Du Ký, Dương Khiết đạo diễn, đã đem lại nhiều cảm giác sinh thú cho người xem. Nhiều bài báo, nhiều lời bình phẩm và dư luận của quần chúng về bộ phim sôi nổi, đặc biệt của luồng dư luận là sự lẫn lộn giữa giá trị của pháp sư Huyền Trang, thiền sư vừa là nhà Phật học, nhà dịch thuật, với giá trị của Đường Huyền Trang trong bộ phim Tây Du Ký và Đường Huyền Trang trong tiểu thuyết Tây Du Ký của Ngô Thừa Ân. Dư luận cũng lẫn lộn giữa giáo lý Phật Giáo đích thực với giáo lý Phật Giáo được phản ảnh qua bộ phim hay qua tiểu thuyết Tây Du Ký.

Người viết bài này thiển nghĩ người Phật tử cần xác định nhận thức rằng:

Pháp sư Huyền Trang của lịch sử Phật Giáo Trung Hoa là có thực, là một thiền sư thông rõ Kinh, Luật, Luận Phật Giáo, là một nhà Phật học lỗi lạc, là một nhà dịch thuật tài danh, đã du học ở Ấn Độ và chiêm bái Phật tích tại đó suốt 17 năm, sau đó trở về Trung Quốc dịch các Kinh, Luận trong suốt 18 năm thì mất. Cuộc đời và sự nghiệp xuất thế của Pháp sư có giá trị độc lập với Trần Huyền Trang trong phim ảnh, và độc lập với Trần Huyền Trang trong tiểu thuyết Tây Du Ký.

Không thể căn cứ vào Trần Huyền Trang trong tiểu thuyết hay phim ảnh với nhiều tình tiết hư câu để đánh giá Pháp sư Trần Huyền Trang có thực trong lịch sử. Ổn định nhận thức như thế để người Phật tử khỏi phải phiền não trước những diễn xuất hay diễn đạt kém phần giải thoát của nhân vật Trần Huyền Trang trong phim ảnh hay trong tiểu thuyết.

— Tương tự, người Phật tử cần ổn định nhận thức của mình về sự khác biệt giữa giáo lý Phật Giáo đích thực với giáo lý Phật Giáo được phản ánh có chỗ thiếu trung thực qua phim ảnh hay tiểu thuyết để khỏi phải băn khoăn trong việc tìm lời lẽ biện minh thế này thế khác.

— Người viết cũng cảm thấy rằng ngay cả khi người Phật tử đã ổn định nhận thức của mình về Pháp sư Huyền Trang đích thực thì vẫn không tránh được cảm nhận khó chịu trước hình ảnh Đức Phật (diễn xuất trong phim) chụp Ngũ Hành sơn xuống mình Tôn Ngộ Không (bấy giờ đã là Tôn Ngộ Không sau ngày thụ giáo vớiTôn giả Tu Bồ Đề) hơi nặng nề (thiếu nét từ bi) rồi liền xoay lại vui vẻ nhìn Hằng Nga hát múa, và trước hình ảnh Tôn giả Đại Ca Diếp, Tôn giả A Nan, đại đệ tử của Đức Phật, lôi thôi độ trái cây và lôi thôi đòi ``hối lộ'' đầy nét phàm phu. Hai hình ảnh ấy thật là xa lạ đối với Phật Giáo và thật khó hiểu đối với người Phật tử hiểu đạo! Hình ảnh hối lộ ấy diễn ra hệt như trong tiểu thuyết của Ngô Thừa Ân, điều mà người viết bài này nghi ngờ không phải là đoạn sáng tác của Ngô Thừa Ân, một tiểu thuyết gia nổi danh có một kiến thức Phật học sâu sắc đã được biểu hiện qua bản truyện Tây Du Ký.

— Trở về tiểu thuyết Tây Du Ký của Ngô Thừa Ân. Đọc xong Tây Du Ký, người viết liền khởi tưởng:

\begin{itemize}
   \item[•] Sự nghiệp du học, phiên dịch và tu hành của pháp sư Huyền Trang là vĩ đại, đã để lại nhiều sự ngưỡng mộ trong quần chúng Phật tử hậu lai, và nhiều hứng khởi trong các nhà nghiên cứu, dịch thuật và sáng tác văn học về sau. Quần chúng ngưỡng mộ pháp sư qua văn học truyền khẩu NGÔ THỪA ÂN đã hứng khởi đến độ dựng thành bộ tiểu thuyết Tây Du Ký bất hủ.

   \item Ngô Thừa Ân hẳn và viết về những gì trong giáo lý Phật Giáo đã tạo nên pháp sự Trần Huyền Trang và sự nghiệp vĩ đại của người. Đó là con đường tu tập thoát ly mọi nỗi khổ đau trần thế, cái nỗi khổ đau đang đè nặng cuộc đời của Ngô Thừa Ân và xã hội Trung Hoa phong kiến đương thời.

   \item Tác giả đã biểu tượng hóa lộ trình tu tập giải thoát, theo Phật Giáo, bằng hình ảnh bốn thầy trò Đường Tăng và con ngựa trắng đi Tây Trúc thỉnh kinh với tám mươi mốt khổ nạn.
\end{itemize}

Trong bài phiếm bàn này, người viết chỉ bàn đến các biểu hiện giáo lý Phật Giáo tản mạn qua các nhân vật và các cảnh nạn, mà không đi vào các quan niệm nhân sinh, xã hội và lại càng không bàn đến văn phong bút pháp của tác giả.

% chapter lời_trần_thuyết (end)