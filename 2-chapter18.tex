\chapter{Hồi 44, 45 và 46} % (fold)
\label{cha:hoi_44_45}

Hồi 44:

\begin{itshape}
``Thần thông phép vận đun xe nặng.

Yêu quỷ tâm trừ vượt cổng cao.''
\end{itshape}

Hồi 45:

\begin{itshape}
``Quán Tam Thanh Đại Thánh lưu danh.

Nước xa trì Hầu Vương hóa phép.''
\end{itshape}

Hồi 46:

\begin{itshape}
``Ngoại đạo cậy tài lừa phép chính.

Ngộ Không hiển thánh giết yêu ma.''
\end{itshape}

\section{Tư tưởng Phật Học} % (fold)
\label{sec:44_45_phat_hoc}

— Đến hồi 44, 45 và 46 thì Tôn Hành Giả tương đối đã đủ Định lực và Tuệ lực, tiêu biểu cho Định và Tuệ của phái đoàn Tây Du vượt qua các pháp thuật của Đạo Gia và Nho Gia, vượt qua các cạm bẫy của Âm Dương và Ngũ Hành, nên đã dễ dàng thắng ba con quỷ Hổ lực, Lộc lực và Dương lực với các thần thuật của chúng.

— Tác giả Ngô Thừa Ân đã đánh giá rất cao giá trị giải thoát sinh tử khổ đau của Giới, Định, Tuệ (trong khi các pháp thuật của Đạo Gia và Nho Gia thì không thể đi vào cõi sống chân thật của giải thoát) nên đã dựng nên cảnh Quán Tam Thanh để cho ba đạo gia Hổ, Lộc và Dương lạy cầu uống nước tiểu của Ngộ Không, Ngộ Năng và Ngộ Tịnh. Lại dựng nên cảnh thi thố tài năng giữa Tôn Ngộ Không và ba đạo gia để làm lộ rõ cái bản tướng đầy dục vọng, dối trá và vị kỷ của họ. Tác giả, qua toàn truyện, hầu như đã lập lại việc đối chiếu nhiều giá trị này dưới nhiều dạng thức khác nhau.

— Giữa thần thông của Ngộ Không và pháp thuật của các đạo gia cũng khác nhau từ cơ bản:

\begin{enumerate}[label=\itshape\alph*\upshape/]
    \item Thần thông của Tôn Ngộ Không thì phát sinh từ ý lực, từ công phu tu tập loại trừ vọng niệm; còn pháp thuật của các đạo gia thì sinh từ dục vọng.

    \item Thần thông của Tôn Ngộ Không thì từ Định và Tuệ Vô Ngã mà khởi; còn pháp thuật của các đạo gia thì từ tâm chấp thủ ngã tướng mà nên.

    \item Thần thông của Tôn Ngộ Không thì thị hiện để hàng phục tà để cứu nhân, độ thế; còn pháp thuật của đạo gia thì phô trương để nuôi tà ý hại chúng sinh.

    \item Thần thông của Tôn Ngộ Không thì biến hóa vô cùng; còn pháp thuật của các đạo gia thì thi triển giới hạn. Đây cũng là sắc thái khác nhau của hai nền văn hóa Tây Trúc và Đông Độ.
\end{enumerate}

— Các thần thông mà Tôn Ngộ Không biểu hiện là biểu tượng của tâm thức giải thoát, tự tại (dù chưa toàn phần) của người tu đã giác ngộ sâu sắc sự thật Vô Ngã, Vô Dục.

% section tư_tưởng_phật_học (end)

\section{Quan niệm về con Người} % (fold)
\label{sec:44_45_con_nguoi}

— Con người là hiện thân của tà và chánh, của tự chủ và tha hóa, của vị kỷ và vị tha, của thiện và ác, vv\ldots. Hai thế lực của tâm thức ấy luôn dằng co trong mỗi người. Chúng cũng cùng có mặt và tranh chấp nhau ở ngoài xã hội. Ở đâu có mặt chánh đạo, ở đó có mặt tà giáo, ở đâu có chính nhân quân tử, ở đó cũng có mặt dua nịnh, tiểu nhân, và ngược lại. Giáo dục của con người hoàn thiện là giúp con người phát triển nhóm thiện và nhân bản, và loại trừ nhóm ác và ly thể.

Hướng thiện và nhân bản là hướng phát triển các tâm lý vị tha, nhân ái, hành động vì lợi ích của tha nhân, tập thể; hướng chấp ngã, vị kỷ, hành động vì tự lợi hệt như khuôn mẫu hành động của Tôn Ngộ Không (thiện) và ba đạo sĩ bất nhân (ác).

— Giá trị của con người và nghĩa sống của con người luôn luôn xây dựng trên sự thật và lợi ích an lạc của tự thân và của số đông, hệt như phái đoàn Tây Du luôn luôn vì sự giải thoát của tự thân và vì lợi ích, an lạc của nhân dân ở Đông Độ.

— Kết quả tốt của một hành động gọi là đạo đức chưa đủ để đánh giá đúng hành động ấy. Giá trị đạo đức của hành động chủ yếu căn cứ vào động cơ của hành động hay gọi là căn cứ vào sự dụng tâm hành động. Ngô Thừa Ân đã phơi bày rõ quan niệm mình về hành động đạo đức qua ba nhân vật Hổ lực, Lộc lực và Dương lực: ba đạo sĩ này đã dùng pháp thuật làm mưa cứu mùa màng cho nhân dân ở địa phương, nhưng không phải làm vì thương dân, giúp nước, mà vì thủ đoạn muốn chiếm lòng vua để thoán đoạt ngôi báu. Các đạo sĩ đã tàn ác nhúng tay vào việc hành hạ hai ngàn tu sĩ Phật Giáo hiền đức đến nỗi có đến hơn nửa số tu sĩ đó đã mạng vong vì đói, vì rét.

— Ngược lại, sau khi trừ xong ba đạo sĩ, Tôn Ngộ Không đã khuyên nhà vua đối xử tốt cả tam giáo (Thích, Lão, Nho), mà không đối xử kỳ thị.

— Trí tuệ phải hướng dẫn hành động đạo đức, nếu thiếu trí tuệ thì dù có lòng nhân, hành động cũng có thể gieo rắt khổ đau, tai họa. Hành động như vậy không thể gọi là đạo đức đúng nghĩa nếu thiếu trí tuệ.

Ngô Thừa Ân nói rõ ý này qua hình ảnh nhà vua trong hồi 44, 45 và 46, đã vô trí tin dùng ba đạo sĩ hung ác suýt làm khổ dân và đánh mất ngôi báu, nếu không có Tôn Ngộ Không cứu nguy.

% section quan_niệm_về_con_người (end)

\section{Quan niệm về xã hội} % (fold)
\label{sec:44_45_xa_hoi}

Bên cạnh các lực lượng phản kháng công khai và ngấm ngầm còn có các cá nhân và tổ chức vì tham vọng cá nhân riêng tư làm trở ngại công cuộc xây dựng nền văn hóa mới. Thành phần trở ngại này ẩn giấu dưới nhiều bộ mặt khác nhau gán vào các thế lực lớn của xã hội cũ như trường hợp ba đạo sĩ nuôi dưỡng tham vọng cá nhân dưới trướng nhà vua. Lực lượng phục vụ nền văn hóa mới cần quan tâm để đối phó, làm thế nào để thực hiện đoàn kết dân tộc đi đôi với cải tổ, như Tôn Ngộ Không đã khuyên nhà vua đối xử tốt và bình đẳng đối với các tu sĩ và các đạo sĩ.

— Ở ba hồi tiểu thuyết này, Ngô Thừa Ân đã gián tiếp phơi bày các yếu điểm của chế độ phong kiến, nguy hiểm cho xứ sở, nhất là khi quyền quyết định quốc gia nằm trong tay một người thiếu sáng suốt và thiếu lòng nhân như nhà vua hôn mê ở đây đã để cho ba tay đạo sĩ ác ôn thao túng, lộng hành phá hư phép nước.
% section quan_niệm_về_xã_hội (end)
% chapter Hồi 44_45 (end)