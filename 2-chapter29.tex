\chapter{Hồi 72 và 73} % (fold)
\label{cha:hoi_72_73}

Hồi 72:

\begin{itshape}
``Động Bàn Ty bảy tình làm mê gốc.

Suối Trạc Cấu, Bát Giới suýt quên mình.''
\end{itshape}

Hồi 73:

\begin{itshape}
``Giận cũ vì tình gây độc ác.

Tâm viên diệt quỷ phá Kim Quang.''
\end{itshape}

\section{Tư tưởng Phật Học} % (fold)
\label{sec:72_73_phat_hoc}

--- Phần tư tưởng Phật Học của hai hồi 72 và 73 này đã được đề cập ở phần {\bf Tổng Luận} (trang \pageref{sec:bieu_tuong_hoi_54_64_va_72}). Tại đây, một vài điểm nhận xét khác sẽ được bàn đến.

--- Bảy yêu nhền nhện là biểu tượng của \emph{``thất tình''} (\emph{hỉ, nộ, ai, lạc, ái, ố, dục}). Chúng nhả ra các đường tơ mịn bắt trói Đường Tăng, Bát Giới và vây hãm Ngộ Không. Tại đây, đối với phái đoàn Tây Du thì thất tình chỉ biểu hiện ở cấp độ \emph{``vô hữu ái''}, như bảy yêu nhền nhện không ham dục ái hay thiên ái mà chỉ mong nhờ suối trời ``Trạc Cấu'' rửa sạch cáu bẩn của tâm, và chỉ mong ăn thịt Đường Tăng để được thoát ly cõi hữu trong sinh tử. Thế là chúng rơi vào vô hữu ái.

Ái vô hữu cũng là một loại tâm ái tinh tế trói buộc tâm giải thoát và tuệ giải thoát, vì thế Ngô Thừa Ân đã để cho các đường tơ nhền nhện trói Đường Tăng và Bát Giới.

Chỉ có trí tuệ Vô Ngã sắc bén của Tôn Ngộ Không mới cắt đứt các đường tơ ấy, dẹp tan bảy yêu nữ nhền nhện.

--- Nhưng gốc của vô hữu ái vẫn là vô minh chấp cái ``ngã vô'', như bảy yêu nhền nhện dựa vào sức mạnh của Bách Nhỡn ma quân mà quậy phá. Bách Nhỡn ma quân là biểu tượng của si mê -- Vô minh (Bách nhỡn ma quân có thần lực lớn, thấy biết rất nhiều ngã tướng mà không thấy thật tướng).

Với gốc vô minh tinh tế này thì với trí tuệ và định lực của Tôn Hành Giả chưa đủ khả năng hàng phục. Cần có trí tuệ Bát Nhã thâm sâu, nhuần nhuyễn mới trừ được vô minh vi tế ấy, như cần đến công phu tu luyện, luyện từ con mắt Mặt Trời thành cây kim trí tuệ của Mão Nhật Tinh Quân mới có thể chọc mù mắt vô minh của Bách Nhỡn. Đây là công phu thiền quán thâm hậu mà hành giả cần thực hiện để đi đến Lôi Âm Tự.
% section tư_tưởng_phật_học (end)

\section{Quan niệm về con Người} % (fold)
\label{sec:72_73_con_nguoi}

--- {\bf Có hai nhóm tình cảm làm khuất mờ nhân trí là:}

\begin{description}[leftmargin=!,labelwidth=\widthof{\bfseries Thất tình:}]
    \item[Thất tình:] \emph{hỉ, nộ, ái, lạc, ái, ố và dục}.

    \item[Lục dục:] \emph{ham muốn sắc, thinh, hương, vị, xúc và pháp}.
\end{description}

Các thứ tình cảm ấy che khuất trí tuệ như bảy yêu nhền nhện trói chặt Bát Giới và Đường Tăng.

Tình người vị tha và sự sáng suốt của tâm sẽ sáng chói, nếu các thứ tình cảm trên được khắc phục. Giáo dục để chế ngự các tình cảm ấy là giáo dục tâm lý, huấn luyện tâm lý và khai mở nhận thức Vô Ngã. Khi mà con người đã hấp thụ sâu sắc nền văn hóa mới thì tự khắc có khả năng chế ngự \emph{thất tình lục dục}, như chiếc áo của Trương Tử Dương (hồi 70) bảo vệ Kim Hậu khỏi thất tiết vì quỷ vương tại núi Kỳ Lân.
% section quan_niệm_về_con_người (end)

\section{Quan niệm về xã hội} % (fold)
\label{sec:72_73_xa_hoi}

--- Cái tình cảm của quần chúng đối với các giá trị của nền văn hóa cũ, tập quán sống và tư duy cũ, là một thứ tình cảm tinh tế được hình thành trong tâm thức người dân, như những đường tơ nhền nhện mịn màng. Cần vận dụng trí tuệ Vô Ngã, thiết thực để cắt đứt thứ tình cảm \emph{``dẫu lìa ngó ý còn vương tơ lòng''} ấy, và cắt đứt đến tận gốc rễ thì nền văn hóa mới, mới có nhân duyên phát triển mạnh.

--- Các đạo sĩ thường được các vua chúa trọng dụng và sử dụng để giữ vững ngai báu, Ngô Thừa Ân quan niệm văn hóa của các Đạo gia không có chân đứng trong nền văn hóa nhân bản và trí tuệ nên cho các đạo sĩ về giữ vườn cho Bồ Tát Tỳ Lam, nghĩa là chỉ giữ một vai trò hỗ trợ nhỏ.
% section quan_niệm_về_xã_hội (end)
% chapter Hồi 72_73 (end)