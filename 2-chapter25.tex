\chapter{Hồi 64} % (fold)
\label{cha:hoi_64}

Hồi 64:

\begin{itshape}
``Núi Kinh Cúc Ngộ Năng gắng sức.

Am Mộc Tiên Tam Tạng làm thơ.''
\end{itshape}

\section{Tư tưởng Phật Học} % (fold)
\label{sec:64_phat_hoc}

— Phái đoàn Tây Du trải qua bao nhiêu công phu tu Giới, tu Định, tu Tuệ, đã cắt lìa dục ái, dục vọng đã được chuyển thành dục giải thoát. Tâm hồn của mỗi thành viên giờ đây thanh thản, đầy hỷ lạc. Cả một vùng núi Kinh Cúc 800 dặm rộng với cây cỏ chằn chịt không làm chùn bước Ngộ Năng, Ngộ Năng đêm ngày tinh cần phát quang cho phái đoàn tiến tới. Ý nghĩa phát quang ở đây là cắt đứt các ngoại duyên cho tâm dễ định. Đây là một giai đoạn thành tựu tâm giải thoát.

Khi tâm hành giả đi vào bốn cõi Định Sắc Thiền, lòng tuy đã rời xa Dục giới nhưng lại đầy khát vọng tồn tại trong cảnh an tịnh, thanh nhàn. Tâm hành giả vì thế bị ngưng trệ trói buộc tại đây. Ngô Thừa Ân đã dựng tại đây cảnh Am Mộc Tiên gồm toàn yêu tinh thảo mộc ưa ngâm vịnh, thưởng thức tiên trà và tiên tửu của cảnh giới sắc ái (hay hữu ái). Cảnh giới này trói buộc bước chân giải thoát đầy nguy hiểm không kém các ma nạn vừa trải qua. Cái lòng sắc ái tế nhị ấy đã được biểu hiện qua hình dáng Hạnh Tiên thanh nhã của thiên giới. Tại đây, Đường Tăng kịp giác tỉnh không để bị chìm sâu vào cảm thọ hỷ lạc, bèn nhất mực cự tuyệt tình cảm của Hạnh Tiên. Nhưng Đường Tăng không dễ gì thoát ly được cảnh trói buộc này nếu Tôn Ngộ Không không xuất hiện kịp thời cứu giá.

Đối với hữu ái thì chỉ có trí tuệ Vô Ngã, thiền quán Vô Ngã mới cắt đứt được nó, như chiếc thiết bổng của con người trí tuệ Ngộ Không đã đập nát cảnh nạn Am Mộc Tiên. Nếu thiếu trí tuệ Vô Ngã thì hành giả sẽ dễ bị đốt cháy toàn thân huệ mạng trong các cảm thọ. Nguy hiểm thay!
% section tư_tưởng_phật_học (end)

\section{Quan niệm về con Người} % (fold)
\label{sec:64_con_nguoi}

— Giáo dục con người toàn diện cần chú trọng đến mặt giáo dục về rèn luyện ý chí và quyết tâm. Ý chí và quyết tâm là sức mạnh của tâm lý thể hiện ra hành dộng để đạt đến mục tiêu của cuộc đời.

Con người có khuynh hướng thích nhàn, phóng túng và đầy ham muốn như hiện thân của Trư Bát Giới vào những ngày đầu Tây Du. Được Đường Tăng thu phục và được Ngộ Không kiềm chế, Trư Bát Giới đã trở thành Trư Ngộ Năng có sức mạnh mở đường vào giải thoát. Ý chí cũng thế, nếu được rèn luyện tốt cũng có khả năng mở đường đi đến các mục tiêu xã hội.

Thật sự, ý chí không phải là một phần tố tâm lý được hình thành sẵn trong tâm thức mỗi người, mà là sự hình thành do sự tự nhắc nhở, lặp đi lặp lại nhiều lần ý muốn nào đó, hướng đến một thành tựu nào đó. Sự thức tỉnh tâm thức được làm sinh khởi nhiều lần, và được nuôi dưỡng, sẽ trở thành ý chí giải thoát.

Phải thể hiện đủ ba mặt, mới hình thành một hành động đạo đức, đó là:

\begin{enumerate}[label=\itshape\arabic*\upshape/]
    \item Lòng Từ Bi, Vị Tha.

    \item Trí Tuệ Vô Ngã.

    \item Ý chí thực hiện.
\end{enumerate}

Cảnh sống của Am Mộc Tiên là cảnh sống của Lão Trang, xa lánh cả thân và tâm, trở nên yếm thế, thiếu lòng từ, độ khổ cuộc đời. Ngộ Không, Ngộ Năng đánh nát cảnh Am Mộc Tiên trước khi tiếp tục lên đường là xóa bỏ các ý tưởng hưởng nhàn, yếm thế trong tâm tức mình.

Đối với Phật Giáo, vấn đề chủ yếu là loại bỏ lòng dục trong tâm, mà không phải là loại bỏ cuộc đời, hay xa lánh cuộc đời. Đây là hình ảnh tích cực của Phật Giáo, nói lên ý nghĩa Phật Giáo gắn liền với cuộc sống, rằng Phật Pháp có mặt trên thế gian, và giác ngộ có mặt trong mỗi người trên thế gian, điều mà các hành giả Phật Giáo thường nói: \emph{Phật Pháp bất ly thế gian giác}.
% section quan_niệm_về_con_người (end)

\section{Quan niệm về xã hội} % (fold)
\label{sec:64_xa_hoi}

— Xã hội Trung Hoa, qua lịch sử nhiều nghìn năm thể nghiệm triết lý sống của Đạo gia và Nho gia, đã nhận ra rằng:

\begin{enumerate}[label=\itshape\alph*\upshape/]
    \item Triết lý Nho gia thì tích cực dấn thân vào xã hội, nhưng tư duy về con người và cuộc đời thì thiên lệch, đẻ ra một nguồn máy phong kiến nặng nề, và đẻ ra các giá trị ước lệ đầy hình thức và câu thức, trói buộc con người; tất cả đó tạo ra một cuộc ``hỗn thế'' (theo Ngô Thừa Ân) thường xuyên bất ổn.

    \item Triết lý Lão gia thì lánh đời, các người tài đức thì ẩn sâu vào các nơi cô tịch, rừng núi, chỉ chú tâm vào cái nhàn nhã của tự thân, vui thú với tự nhiên giới.
\end{enumerate}

Giữa hai thái độ sống đi về hai thái cực ấy, Ngô Thừa Ân giới thiệu đạo Phật, con đường sống trung đạo vừa thể hiện giải thoát tự thân vừa cứu đời, thoát ly các trói buộc của cuộc đời ngay giữa lòng đời. Tác giả Tây Du Ký tin tưởng con đường văn hóa giáo dục của Phật Giáo sẽ đưa xã hội Trung Hoa đến thành công, an hưởng thái bình hạnh phúc, thịnh trị, như phái đoàn Tây Du sẽ về đến Tây Trúc yết kiến Phật Tổ.

Các hồi trước thì Ngô Thừa Ân trình bày các cảnh loạn lạc, chướng ngại của xã hội phong kiến Nho gia; hình ảnh Am Mộc Tiên thì biểu hiện nếp sống của các Đạo gia, cái nếp sống mà tác giả không muốn nó tồn tại trong xã hội Trung Hoa, đã mượn chiếc Đinh Ba và Thiếc Bổng của Ngộ Năng, Ngộ Không để phá hủy.

Các quan niệm sống trên thường xuất hiện trong tâm thức con người. Mỗi người cần được soi sáng nhận thức để có những chọn lựa đúng thái độ sống vị tha và tích cực.
% section quan_niệm_về_xã_hội (end)
% chapter Hồi 64 (end)