\chapter{Hình Ảnh Giáo Lý Phật Giáo Bắc Truyền Được Phản Ảnh Qua Tây Du Ký} % (fold)
\label{cha:hinh_anh_giao_ly_phat_giao_bac_truyen}

\section{Qua các nhân vật chính (tổng quan)} % (fold)
\label{sec:qua_cac_nhan_vat_chinh}

Bảng liệt kê các Kinh, Luật, Luận mà Đường tăng thỉnh về Trung Quốc, theo Ngô Thừa Ân, là thuộc giáo lý Bắc truyền. Với các nhà nghiên cứu Phật học vững vàng như Ngô Thừa Ân mới nhận ra tư tưởng Bát Nhã là cột sống của giáo lý Bắc truyền. Các bản kinh tiêu biểu và phổ biến nhất của giáo lý Bát Nhã là Bát Nhã Tâm Kinh và Kim Cương Bát Nhã. Vì thế Ngô Thừa Ân đã chọn lựa và giới thiệu các nét giáo lý tinh yếu của hai bản Kinh ấy vào cuốn tiểu thuyết thời danh Tây Du Ký.

$\star$ Tạng Kinh Bát Nhã (ngót 700 cuốn) do pháp sư Huyền Trang dịch. Bản Bát Nhã Tâm Kinh là bản Kinh mà pháp sư thường đọc tụng, ngay cả những lúc cấp nạn như tại sa mạc Gobi — theo đúng sử liệu — Bản kinh này là tinh yếu của tư tưởng Bát Nhã đã được tác giả đưa vào tiểu thuyết làm triết lý cho cuộc Tây Du.

Hồi thứ 19 (của Tây Du Ký) viết rằng:

\begin{itshape}
``Đông Vân San, Ngộ Không thu Bát Giới,

Núi Phù Đồ, Tam Tạng nhận Tâm Kinh''.
\end{itshape}

Đường Tăng tại đây được thiền sư Ô Sào (hóa thân của Bồ Tát) truyền dạy bài Bát Nhã Tâm Kinh (Nhà xuất bản Văn học, Hà Nội. 1982) như là vũ khí để chiến thắng vượt qua các ách nạn.

$\star$ Đọc cuộc hành trình Tây Du, Đường tăng thường tụng niệm bản Bát Nhã Tâm Kinh — và thường được Tôn Ngộ Không nhắc nhở — để thắng vượt các sợ hãi, âu lo.

$\star$ Cuối đường Tây Du, Đường Tăng được hóa thân Phật dùng chiếc thuyền Bát Nhã (không đáy) chở qua sông mê đến bờ bên kia của Phật cảnh. Bước vào thuyền, Đường Tăng liền thoát xác: được pháp thân thanh tịnh vô tướng, và để lại chiếc sắc thân, như một xác chết, nổi bềnh bồng giữa sóng nước sinh tử.

$\star$ Suốt đường thỉnh kinh, phái đoàn Tây Du luôn luôn được Bồ Tát Quán Thế Âm theo sát cứu độ. Quán Thế Âm theo là tên của vị Bồ Tát mở đầu bản Bát Nhã Tâm Kinh, vị Bồ Tát do vì thấy rõ Năm uẩn (sắc, thọ, tưởng, hành, thức — nghĩa là con người và vũ trụ) là không có tự ngã mà vượt qua hết thảy khổ ách, đi vào sinh tử tự tại cứu độ chúng sinh.

Như thế, tại đây đã có thể kết luận rằng: Ngô Thừa Ân đã diễn lại nội dung Bát Nhã Tâm Kinh (hay tư tưởng tinh yếu của Bát Nhã) và hành trình để chứng đắc trí tuệ Bát Nhã qua toàn bộ tiểu thuyết Tây Du Ký mà mỗi bước đi của phái đoàn thầy trò Đường Tăng là mỗi bước đi tiến gần giải thoát.
% section qua_các_nhân_vật_chính_tổng_quan_ (end)

\section{Các nhân vật chính của phái đoàn thỉnh kinh.} % (fold)
\label{sec:cac_nhan_vat_chinh_cua_phai_doan_thinh_kinh}

Nếu hiểu mỗi nhân vật của phái đoàn thỉnh Kinh là một biểu tượng độc lập, riêng lẽ thì khó mà nắm được toàn mạch tư tưởng nhất quán của Ngô Thừa Ân, chúng ta sẽ lúng túng trong việc tìm hiểu các ảnh tượng giáo lý Phật Giáo trải khắp toàn truyện.

Thực sự cuộc hành trình Tây Du là một hành trình giải thoát của mỗi người muốn thoát ra khỏi mọi nỗi khổ đau của sinh tử. Nhân vật thầy trò Đường Tăng là biểu tượng các phần tố tâm thức của một tâm hồn Đường Tăng.

\subsection{Về Đường Tăng} % (fold)
\label{sub:ve_duong_tang}


Đường Tăng là tiếng nói của hạnh nguyện giải thoát, của bi nguyện độ sinh, và sau hết là tiếng nói của cho tim trần thế.

Tiếng nói ấy là linh hồn của nghĩa sống, như Đường Tăng là người dẫn đầu đoàn thỉnh kinh. Đây là không là tiếng nói của trí tuệ, của khối óc, nên thường thiếu khả năng phân biệt chánh tà, hư thực, thường bị mắc vào cạm bẩy của ác ma. Tiếng nói của con tim ấy cần được soi sáng bởi tiếng nói thiền định và giới đức như là Đường Tăng cần đến ba môn đồ phò tá (sẽ bàn tiếp)...
% subsection về_đường_tăng (end)

\subsection{Về Tôn Ngộ Không (Tôn Hành Giả)} % (fold)
\label{sub:ve_ton_ngo_khong}

$\star$ Tôn Ngộ Không là biểu trưng cho Chánh kiến và Chánh tư duy (Thánh tuệ uẩn của mỗi hành giả, là trí tuệ Vô Ngã thấy rõ mọi hiện hữu là Vô Ngã, vô thường và dẫn đến tan rã, khổ đau. Trí tuệ ấy khác với trí tuệ sinh diệt. Nó là vô sinh nên tác giả Ngô Thừa Ân giới thiệu Mỹ hầu vương được sinh ra từ trứng đá, kết tinh của tú khí trời đất. Trí tuệ ấy tự biết tìm đường đi ra khỏi sinh tử như Mỹ hầu vương biết tìm đường đến với đại đệ tử của Đức Phật (Tôn giả Tu Bồ Đề) để học đạo bất sinh bất diệt.

$\star$ Tôn giả Tu Bồ Đề là đệ nhất ly dục, ly ái (còn có nghĩa là đệ nhất rời chấp thủ hết thảy các ngã tướng) trong hàng đệ tử của Đức Phật — theo Kinh Kim Cang Bát Nhã — Đạo và Mỹ hầu vương được truyền dạy thấy rõ Vô Ngã tướng (hay không tướng) của vạn hữu và tự tâm rời xa mọi tham ái. Nắm được sở đắc ấy thì liền tự tại, ở ngoài mọi khổ đau. Sự kiện tự tại này đã được Ngô Thừa Ân biểu hiện qua 72 phép thần thông biến hóa của pháp môn Địa—sát.

Trí tuệ này là cao nhất để đi đến trí tuệ giải thoát sau cùng, không còn trí tuệ nào khác cao hơn, nên được gọi là Vô sư trí. Vì thế Tôn giả Tu Bồ Đề cấm Tôn Ngộ Không tiết lộ danh tánh của Thầy dạy đạo cho Tôn Ngộ Không.

Đạt được trí tuệ xa lìa khổ đau ấy, Mỹ hầu vương nhận được pháp danh Tôn Ngộ Không. chữ Tôn theo lời cắt nghĩa của Tôn giả Tu Bồ Đề, nếu xóa bộ khuyển bên cạnh thì thành chử Tử (con) và chữ Hệ (trẻ con). Như thế trí tuệ giải thoát sau cùng, mà chưa là trí tuệ giải thoát sau cùng, trí tuệ này cần được tu tập thêm Giới và Định.

$\star$ Trí tuệ, tự thân nó là động, tháo động, vì thế Tôn Ngộ Không mang thân tướng giống khỉ. Cái động của trí tuệ cần được thuần hóa và nuôi dưỡng bằng định tâm và sự thực hành giới hạnh. Định tâm sẽ rửa sạch cái động của ý, giới đức sẽ rửa sạch cái động của trí tuệ cần được thuần hóa và nuôi dưỡng bằng định tâm và sự thực hành giới hạnh. Định tâm sẽ rửa sạch cái động của ý, giới đức sẽ rửa sạch cái động của thân, khẩu. Chưa đủ, có những thời điểm manh động của trí cần phải nhờ đến đại định để chế ngự như là Tôn Ngộ Không cần phải đội trên đầu chiếc vòng ``Khẩn cô nhi'' (còn gọi là vòng ``Kim cô'' hay vòng ``định tâm'') và cần được chế ngự bởi ``định tâm chú'' (hay chú Khẩn cô nhi) của Bồ Tát Quán Thế Âm.

$\star$ Khi mà trí tuệ ấy chưa được Giới, Định chế ngự và nuôi dưỡng thì nó sẽ bị Năm uẩn (hay vũ trụ, cuộc đời) khống chế với vô lượng phiền não. Đây là hình ảnh mà Tôn Ngộ Không bị Ngũ Hành sơn chụp lên mình năm trăm năm mà không trăn trở được. Đó là cái họa đại náo Thiên cung của Tề Thiên Đại Thánh, do vì Đại Thánh thấy rõ cái hư, cái rởm (và cả dỏm nữa) của trên trời và dưới thế, không chịu được mà đại náo, đập phá, đạp đổ.

$\star$ Đường giải thoát chưa dừng lại ở đây. Ngộ Không (hay trí tuệ) cần tiếp tục vào đại định và lòng đại bi, cần phải tu tập nhiều lần nữa. Nghĩa là Ngộ Không phải tinh tấn lên đường thực hành giải thoát. Bấy giờ Ngộ Không có thêm một pháp hiệu nữa là Hành Giả.

$\star$ Trí tuệ của Tôn Hành Giả (nặng phần tự độ) cần phải được tu tập cùng với bi tâm độ sinh (phần độ tha của Đường Tăng) thì mới thiện xảo, mới tiến gần giải thoát tối hậu. Cũng thế, bi tâm cần được trí tuệ Vô Ngã dẫn đường, nếu không thì dễ lạc đạo.

Tác giả Ngô Thừa Ân diễn đạt điểm giáo lý này qua sự xây dựng hai nhân vật Đường Tăng và Tôn Hành Giả. Khi nào mà Đường Tăng không nghe Tôn Hành Giả thì phái đoàn Tây Du trở nên buồn bã ảm đạm như một phái đoàn đưa ma (như cảnh quỷ Hoàng Bào hãm hại Đường Tăng sau khi Ngộ Không bị đuổi về núi Hoa Quả).

$\star$ Người tu giải thoát rời xa trí tuệ một bước thì bị họa liền một bước. Cần phải thường xuyên giữ chánh niệm hay ``như lý tác ý'' để tránh các nạn ở am Mộc Tiên (hồi 64): ``Đường Tăng mắc vào cảnh mê thơ, rượu và tình. Bấy giờ, khi Tôn Hành Giả xuất hiện kịp thời thì ma cảnh liền tan biến, Đường Tăng ra khỏi sự đắm trước nội thọ và ngoại thọ.''
% subsection về_tôn_ngộ_không_tôn_hành_giả_ (end)
\subsection{Về Trư bát Giới (Trư Ngộ Năng)} % (fold)
\label{sub:ve_tru_bat_gioi}

$\star$ Theo Phật Giáo Bắc và Nam truyền, thiền định để chế ngự cái động của tuệ và của tâm cần được tu tập trên căn bản phạm hạnh. Vì thế hành trình giải thoát cần có công phu thực hành Thánh giới uẩn, như phái đoàn Tây Du cần có mặt Trư Bát Giới (bát giới là tám giới căn bản của người xuất gia).

$\star$ Trư Bát Giới vốn là Thiên Bồng Nguyên soái, do vì say men tiên tửu, dục ái trỗi dậy quấy rồi Hằng Nga mà bị đày xuống hạ giới với thân mình thô lậu. Trư Bát Giới quả là hiện thân của dục vọng, của sự buông lung thân, khẩu hành. Khi Bát Giới trở thành môn đệ của Đường Tăng là khi đoàn Tây Du thể hiện công phu hành trì giới uẩn để chế ngự dục vọng và tẩy trừ thân, khẩu nghiệp. Tác giả Tây Du Ký đã khéo xây dựng nhân vật Trư Bát Giới tham ăn, tham ngủ nghĩ và nói năng thô lậu như là tập khí sinh tử còn lại để trên con người xuất thế, tập khí này cần được tẩy rửa khỏi Tâm giải thoát và Tuệ giải thoát.

$\star$ Tánh của Trư Bát Giới thường không hợp với Tôn Hành Giả là cũng do vì sự có mặt của tập khí sinh tử ấy. Lúc nào Giới được tu tập cùng với Tuệ thì công phu giải thoát ổn định, lúc nào giới rời khỏi tuệ thì công phu giải thoát rối loạn. Hệt như những lúc Trư Bát Giới hòa thuận với Tôn Hành Giả thì phái đoàn Tây Du êm ả, những lúc Trư Bát Giới nghịch ý Tôn Hành Giả thì phái đoàn Tây Du lâm nạn lớn.

$\star$ Nhưng Trư Bát Giới cũng lập được nhiều công trên đường thỉnh Kinh. Đây là ý nghĩa Bồ Tát Quán Thế Âm đặt pháp danh Ngộ Năng cho Bát Giới. Khi mà dục vọng hướng về giải thoát, thì dục vọng trở thành một sức mạnh giải thoát cần thiết cho hành giả.

$\star$ Những đoạn đường tu để chiến thắng dục vọng là những đoạn đường dục ái, hữu ái và vô hữu ái mà hành giả phải vượt qua. Khi Trư Bát Giới theo phò Đường Tăng là khi dục vọng chuyển thành giải thoát mà khởi đầu là bỏ dục ái (Trư Bát Giới rời khỏi nhà nhạc gia Cao Lão) để tiến đến cắt lìa dục ái, hữu ái và vô hữu ái. Tại đây, hình ảnh của Trư Bát Giới là hình ảnh ái nghiệp còn tồn đọng (tùy miên) trong tâm thức Đường Tăng hay trong tâm thức của một hành giả trên đường về giải thoát.

Ngô Thừa Ân xây dựng vai trò Trư Bát Giới là nói lên rằng, theo Phật Giáo, khi có trí tuệ Vô Ngã (chánh kiến và chánh tư duy) và bi nguyện giải thoát độ sinh, hành giả cần phải tu tập Giới uẩn, Định uẩn và Tuệ uẩn cho đến thời điểm giải thoát sau cùng, tan hết tập khí sinh tử (trừ hết các sanh y).

% subsection về_trư_bát_giới_trư_ngộ_năng_ (end)
\subsection{Về Sa Ngộ Tịnh} % (fold)
\label{sub:ve_sa_ngo_tinh}

$\star$ Vai trò của giới là chế ngự cái loạn động của thân hành và khẩu hành; riêng cái loạn động của tâm, trí thì phải cần đến công phu thiền định. Vì thế, phái đoàn Tây Du cần kết nạp thêm nhân vật Sa Ngộ Tịnh (Tịnh có nghĩa là định tâm). Ngộ Tịnh là biểu tượng của công phu tu tập Thánh định uẩn. Vì thế, nhân vật Ngộ Tịnh trầm lặng, chuyên chú, cần mẫn và ổn định suốt cuộc hành trình Tây Du.

$\star$ Nhân dịp tập Giới, hành giả mới có tâm định nên tác giả Tây Du Ký đã kết nạp Ngộ Tịnh sau Ngộ Năng và làm sư đệ Ngộ Năng.

$\star$ Sa Ngộ Tịnh vố là Quyển Liêm đại tướng ở nhà trời, do vì chếch choáng rượu trời, đánh mất chánh niệm, làm đổ ly ngọc mà bị đày xuống trần gian (dục giới). Khi đi theo Đường Tăng là khi Ngộ Tịnh thiết lập lại chánh niệm tỉnh giác, hành trì định uẩn.

$\star$ Định uẩn gắn liền với Tuệ uẩn, theo Phật Giáo, không có định thì không có tuệ, không có có tuệ thì không có định. Do vậy, Ngô Thừa Ân đã vẽ nên một tình huynh đệ thắm thiết giữa Ngộ Không và Ngộ Tịnh. Sa Ngộ Tịnh là hình ảnh biểu hiện trạng thái tâm lý ổn định của hành giả trên đường về giải thoát, tạo thêm sự sáng suốt của trí tuệ và tâm thức như có lần Ngộ Tịnh đã mách nước cho Ngộ Không loại trừ Đường Tăng giả.

% subsection về_sa_ngộ_tịnh (end) 

\subsection{Về Tiểu Long Mã} % (fold)
\label{sub:ve_tieu_long_ma}

$\star$ Tiểu Long Mã là con Tây Hải Ngao Nhuận (giữa Trời và Đất), do vì khởi niệm nghịch ý cha, châm lửa đốt ngọc minh châu trên điện, bị thiên đình xử tội chết, sau nhờ Bồ Tát Quán Thế Âm cứu vớt khỏi tội chờ ngày phò tá Đường Tăng sang Tây Trúc.

$\star$ Đây là một nhân vật ẩn kín của phái đoàn Tây Du mà nếu thiếu sự chú tâm, chúng ta sản xuất không phát hiện ra. Khi Tiểu Long Mã lên đường đi Tây Du là khi hiếu tâm bắt đầu được tu tập cho đến khi thành bi tâm và giải thoát tâm. Cấu trúc thêm nhân vật Tiểu Long Mã vào phái đoàn Tây Du là Ngô Thừa Ân muốn giới thiệu công phu tu tập Giới, Định, Tuệ được thực hiện trên căn bản hiếu tâm, như giáo lý Phật Giáo từng dạy: ``Tâm hiếu là tâm tu, hạnh hiếu là hạnh tu'' và Điều thiện lớn nhất là chí hiếu, điều ác cùng cực là bất hiếu''.

Với Phật Giáo, đại bi tâm chỉ là sự phát triển rộng, sâu của hiếu tâm, vì thế Tiểu Long Mã là con ngựa cưỡi của Đường Tăng cho đến Lôi Âm tự.

$\star$ Qua nhân vật thứ năm này, Ngô Thừa Ân đã nói lên được sự đề cao hiếu đạo của Phật Giáo, và sự quan tâm của Phật Giáo đến việc xây dựng một nhân cách đạo đức cho một xã hội tốt đẹp cho con người. Tại đây, chúng ta cũng thấy rằng cuộc đời đang cần đến Phật Giáo như Tiểu Long Mã cần đến bàn tay cứu độ của Bồ Tát Quán Thế Âm.

% subsection về_tiểu_long_mã (end)
\subsection{Về Lôi Âm Tự và tám mươi mốt ách nạn.} % (fold)
\label{sub:ve_loi_am_tu_va_tam_muoi_mot_ach_nan}

$\star$ Lôi Âm tự là một danh từ rất biểu tượng. Nó biểu tượng háo một thế giới có ngôn ngữ chân thật nói lên được chân tướng của vạn hữu, như là tiếng sám sét siêu vượt lên trên các âm thanh trần thế — thứ âm thanh của hữu niệm, hữu ngã. Vì thế Tôn giả Đại Ca Diếp và Tôn giả A Nan thoạt đầu đã trao cho Đường Tăng các bản Kinh vô tự, bởi vì vô tự mới thực là Chân Kinh. Nhưng khó khăn thay, trần gian không đọc được Kinh vô tự: Như Lai đã phải dạy trao Kinh hữu tự do vậy không nói lên được thực tướng. Hành giả phải hành Giới, Định, Tuệ và tâm đại bi mới thể nhận được thực tướng. Đây là công phu đánh thức dậy giác tính vốn có trong mỗi hành giả như sấm sét đánh thức mình ra khỏi cơn mê của 81 cảnh giới tâm.

\subsection{81 cảnh giới tâm ấy là gì?} % (fold)
\label{sub:81_canh_gioi_tam}

Ngô Thừa Ân đã khéo dựng lên tám mươi mốt cảnh nạn phải vượt qua trên đường Tây Du. Theo Phật Giáo, có mười cảnh giới hiện hữu, đó là Phật, Bồ Tát, Duyên giác, Thinh văn, Trời, Người, A—tu—la, Địa ngục, Ngạ quỷ và Súc sinh. Muốn đến cảnh giới Phật, hành giả phải đi qua chín cảnh giới còn lại. Mỗi cảnh giới có đủ chín cảnh giới tâm còn lại ấy, thành thử có tất cả 81 cảnh giới tâm sai biệt còn lại (9x9). Mỗi cảnh giới tâm này đều che mờ Phật trí, như con suối nước đổ xuống bao che động Thủy Liêm ở Hoa Quả sơn, mà hành giả cần tẩy rửa. Công việc tẩy rửa 81 cảnh giới tâm ấy chính là nỗ lực của phái đoàn Tây Du vượt qua 81 ách nạn vậy.
% subsection 81_cảnh_giới_tâm_ấy_là_gì_ (end)
% subsection về_lôi_âm_tự_và_tám_mươi_mốt_ách_nạn_ (end)
% section các_nhân_vật_chính_của_phái_đoàn_thỉnh_kinh_ (end)
% chapter hình_ảnh_giáo_lý_phật_giáo_bắc_truyền_được_phản_ảnh_qua_tây_du_ký (end)