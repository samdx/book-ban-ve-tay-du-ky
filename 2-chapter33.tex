\chapter{Hồi 84, 85 và 86} % (fold)
\label{cha:hoi_84_85}

Hồi 84:

\begin{itshape}
``Khó diệt nhà sư tu chính quả

Pháp vương thành đạo thể theo giời''
\end{itshape}

Hồi 85:

\begin{itshape}
``Hành Giả trêu Bát Giới

Chúa ma mưu bắt sư''
\end{itshape}

Hồi 85:

\begin{itshape}
``Bát Giới giúp oai trừ quái vật

Ngộ Không hóa phép diệt yêu tà''
\end{itshape}

\section{Tư tưởng Phật Học} % (fold)
\label{sec:84_85_phat_hoc}

— Theo Phật Giáo, tâm con người vốn có đủ chủng tử của tất cả cảnh giới thiện và ác, tùy theo nhân duyên và sự vận dụng của mỗi người mà các cảnh giới tâm hiện hành. Có thể nói rằng tâm lý của con người vốn Vô Ngã nên bất định tính, tùy theo sự huấn luyện giáo dục mà tâm hình thành. Tâm vốn không phải là \emph{``tánh bổn ác''} hay \emph{``tánh bổn thiện''}, cũng không phải do thiên mệnh hay định mệnh. Con người có khả năng làm chủ tâm mình và thuần hóa tâm mình như người huấn luyện ngựa thiện xảo có thể huấn luyện con ngựa chứng thành ngựa thuần.

Ngô Thừa Ân trình bày các ý nghĩa trên qua sự việc vua nước Diệt Pháp do căm tức các sãi mà truyền lệnh giết 10.000 vị sư (đã giết 9.996 vị). Với ông vua độc ác như thế mà chỉ qua một đêm Tôn Hành Giả với kỹ thuật giáo dục thiện xảo đã chuyển hóa tâm ông ta. Sáng hôm sau, vua và cả triều thần đều trở thành các Phật tử mộ đạo.

Tất cả sự việc thiện, ác ấy đều do nhân duyên mà sinh, và do con người chủ động. Vấn đề hưng suy của một xã hội cũng thế, do con đường văn hóa, giáo dục quyết định.

— Vạn hữu, dù tâm hay vật, đều do duyên mà sinh. Trí tuệ thấy thẳng sự thật Duyên khởi này gọi là Tự nhiên trí. Tôn Ngộ Không có trí tuệ về thực tướng của các hiện hữu, nhưng chưa chứng đắc được Tự nhiên trí. Vì thiếu Tự nhiên trí nên Ngộ Không còn bị lầm lẫn các tưởng, rơi vào kế \emph{``chia hoa mai''} của quỷ Nam Sơn Đại Vương, và rơi vào kế trá đầu người (trá đầu người khác làm đầu Đường Tăng). Mãi cho đến khi Tôn Ngộ Không, ngưng tâm vận dụng hết sức mạnh của trí tuệ mình dò xét mới biết Đường Tăng vẫn bình yên, rồi nỗ lực phá tan ma nạn này.

Từ Tự nhiên trí, hành giả tiếp tục an trú vào thiền quán Vô Ngã để đi đến trí tuệ giải thoát tối hậu. Ma nạn này nhắc nhở Ngộ Không tiếp tục công phu thiền quán để mở đường vào Tam Minh.
% section tư_tưởng_phật_học (end)

\section{Quan niệm về con Người} % (fold)
\label{sec:84_85_con_nguoi}

— Trí tuệ Vô Ngã và sự thật Duyên khởi là cơ sở giá trị của nền văn hóa nhân bản, thiết thực và trí tuệ. Trí tuệ ấy là linh hồn của nền văn hóa giáo dục mới. Con người cần thường xuyên giác tỉnh an trú vào cái nhìn trí tuệ ấy. Thiếu nó thì con người sẽ rơi ngay vào vùng giá trị phi nhân bản, phi thực tại, hệt như Tôn Ngộ Không thiếu định lực và thiền quán sâu xa thì phải rơi vào bẫy của ác ma.

— Một tinh thần giáo dục khác của nền văn hóa mới là tinh thần \emph{``tùy duyên nhi bất biến''} hay \emph{``Dĩ bất biến ứng vạn biến''}. Tinh thần này không cố chấp vào các hình thức cứng nhắc của văn hóa cũ (chẳng hạn sự việc Quan Vân Trường từ khước \emph{``về tiểu lộ''} để phải vong mạng). Tất cả giá trị là vì con người và hạnh phúc của con người.

Đến đây thì Đường Tăng đã tỉnh ngộ về sự trói buộc của các hình thức giá trị cũ, đã chịu nghe theo Tôn Ngộ Không cải trang thành lái buôn để vượt qua kinh thành nước Diệt Pháp (chỉ thay đổi chiếc áo, mà không phải thay đổi tâm hồn người tu sĩ). Con người sẽ bị chết đuối trong các giá trị hình thức!
% section quan_niệm_về_con_người (end)

\section{Quan niệm về xã hội} % (fold)
\label{sec:84_85_xa_hoi}

— Ngô Thừa Ân, tại các hồi truyện này, phơi bày một sự thật phi lý và phi nhân: Vua truyền lệnh chém giết một vạn nhà sư vô cớ và vô tội vạ.

Nếu nhà vua và các triều thần hiểu rõ Nhân Quả trong giáo lý Phật Giáo thì sẽ nghĩ đến hậu quả mỗi khi quyết định một công việc, và sẽ không hành động vì sân hận, ganh ghét hay đố kỵ.

Giáo lý Nhân Quả sẽ là một triết lý làm cơ sở cho tinh thần trách nhiệm tự thân và cho các hành vi đạo đức.

— Chính Tôn Ngộ Không đã cảnh tỉnh được vua Diệt Pháp nên tên nước đã được đổi thành Hưng Pháp (làm cho nước thêm hưng thịnh). Thế nên có nghĩa là, theo Ngô Thừa Ân Phật Giáo sẽ giúp xã hội phong kiến trung hoa phát triển hưng vượng.
% section quan_niệm_về_xã_hội (end)
% chapter Hồi 84_85 (end)