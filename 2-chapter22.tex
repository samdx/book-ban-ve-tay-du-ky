\chapter{Hồi 56, 57 và 58} % (fold)
\label{cha:hoi_56_57}

Hồi 56:

\begin{verse}
\begin{itshape}
Điên hồn trừ giặc cỏ.\\
Mê dạo đuổi Hầu Vương.
\end{itshape}
\end{verse}

Hồi 57:

\begin{verse}
\begin{itshape}
Hành Giả thực, kể khổ núi Lạc Già.\\
Hầu Vương giả, đọc văn Thủy Liêm động.
\end{itshape}
\end{verse}

Hồi 58:

\begin{verse}
\begin{itshape}
Hai lòng xáo trộn cả càn khôn.\\
Một thế khó tu hành tịch diệt.
\end{itshape}
\end{verse}

\section{Tư tưởng Phật Học} % (fold)
\label{sec:56_57_phat_hoc}

-- Như đã đề cập nhiều lần ở các phần trước, tập khí nghiệp của chúng sinh đã huân qua nhiều kiếp nên khi đoạn xả cũng cần xả nhiều lần. Tu tập là công phu huấn luyện tâm lý lập đi lập lại nhiều lần ở mỗi thành quả.

Tập khí nghi ngờ, chấp thủ các tướng (tướng từ, tướng thiện, tướng oai nghi, \ldots) và thỉnh thoảng sân hận, đố kỵ, hiềm khích còn nhiều trong Đường Tăng và Trư Ngộ Năng; tập khí bộc trực, ngang bướng, nóng vội thì còn trong Tôn Ngộ Không. Các tập khí ấy tạo nên những xung khắc bất ổn trong phái đoàn Tây Du. Bất ổn lớn nhất là vụ Đường Tăng đuổi Tôn Ngộ Không về núi Hoa Quả sau khi Tôn Ngộ Không ba lần đánh chết ``Bạch cốt phu nhân'' (ma nữ), và nay, ở hồi 56, Đường Tăng lại quá giận Tôn Ngộ Không, niệm chú ``khẩn cô nhi'' nhiều lần và xua đuổi Tôn Ngộ Không khỏi phái đoàn Tây Du. Lần này Tôn Ngộ Không sầu não không biết đi đâu, đã phải qua Nam Hải bộc bạch sự tình oan khuất với Bồ Tát Quán Thế Âm. Mỗi lần xung khắc là mỗi lần phái đoàn cảm nhận sự nguy hiểm của xung khắc và sự nguy hiểm thiếu vắng Tôn Ngộ Không. Tình cảnh ấy có lẽ cần được lập lại nhiều lần mới giúp được phái đoàn Tây Du giác tỉnh ra khỏi các tập khí kia. Hy vọng lần xung đột tối nguy hiểm này là xung đột cuối cùng trên đường đến Lôi Âm Tự .

Khi xung đột chìm lắng là khi biểu tượng sự điều hòa Giới, Định, Tuệ, điều hòa Đại Bi và Đại Trí, hành giả sẽ tiến nhanh đến giải thoát.

-- {\bf Tại lần xung đột này, phái đoàn Tây Du mắc vào các sai lầm sau đây:}

\begin{enumerate}[label=\itshape\alph*\upshape/]
    \item Đường Tăng mắc sai lầm nặng nhất như lời Tôn Ngộ Không nhận xét, trình Bồ Tát Quán Thế Âm: \emph{``\ldots ~vị sư trưởng bội nghĩa vong ơn, chỉ mê một tấm thiện duyên không xét đến cái khổ đen trắng''} (tập 3, tr.202, sđd).

    \item Với sai lầm của Tôn Ngộ Không, Bồ Tát Quán Thế Âm bảo: \emph{``Như nhà người đã có thần thông vô biên, cần gì phải giết nhiều giặc cỏ! Bọn chúng tuy là bất lương, kết cục vẫn thân người, không nên đánh chết, \ldots ~theo ta công bình mà luận, nhà ngươi là người bất thiện''} (sđd. tr. 203, tập 3).

    \item Về Trư Ngộ Năng, thì như chúng ta đã biết, tâm tư nhỏ hẹp, trí tuệ nông cạn, hay đố kỵ, ganh ghét.

    \item Về Sa Ngộ Tịnh, thì định lực chưa sâu nên tuệ cũng chưa tỏa, chưa có khả năng dàn hòa các xung đột; chưa thấy rõ lòng của Ngộ Không.
\end{enumerate}

Các sai lầm trên còn mắc phải là do hành giả chưa thực hành nhuần nhuyễn Giới, Định, Tuệ. Rơi vào ma nạn ở hồi 56, 57, và 59 là vì thế.

-- Trong mỗi người tu hành luôn luôn xuất hiện hai con người trong từng bước đi giải thoát: một con người của thanh tịnh, vô trước vươn thẳng đến giải thoát; một con người của tập khí sinh tử làm trí tuệ giải thoát.

\begin{enumerate}[label=\itshape\alph*\upshape/]
    \item Với Trư Ngộ Năng, khi thì chân chất dễ cảm, khi thì càn dở khó coi.

    \item Với Sa Ngộ Tịnh, khi thì cảnh giác, khi thì nghi ngờ, bất định.

    \item Với Đường Tăng, khi thì minh mẫn đầy đức cảm, khi thì u tối, cố chấp.

    \item Với Tôn Ngộ Không, khi thì xử sự tế nhị, khế cơ, khi thì hành động vội vàng, nóng nảy.
\end{enumerate}

Hai con người ấy trong ấy trong mỗi thành viên của phái đoàn Tây Du cứ thay đổi chỗ nhau trên cuộc hành trình tâm thức khi tâm thức chưa đạt đến tâm giải thoát và tuệ giải thoát.

Cái hư tâm của tâm giữ giới thì dễ thấy; cái vọng niệm của thiền giả cũng dễ nhận; cái chấp của tâm Từ cũng dễ soi; nhưng cái thủ trước tế nhị của trí tuệ thì khó nhận, như sự kiện xuất hiện một Tôn Hành Giả giả đã làm náo loạn cả trời đất, chỉ có trí tuệ Toàn Giác của Như Lai mới biện rõ.

Tại đây Ngô Thừa Ân đã khéo léo dựng lên hai phái đoàn Tây Du (một thật, một giả). Nếu người tu không phát hiện và tiêu trừ phái đoàn Tây Du giả thì việc thỉnh kinh sẽ không thành tựu. Nếu vọng tâm chưa trừ sạch, thì chân tâm khó hiện tiền. Có thể nói rằng hành trình là hành trình phá đổ hết thảy vọng tâm để chứng đạt chân tâm, điều này mà mở đầu kinh Kim Cang Bát Nhã Ba La Mật Tôn giả Tu Bồ Đề đã bạch hỏi Đức Phật: \emph{``làm thế nào để an trú chân tâm và hàng phục vọng tâm? -- vân hà ưng trụ, vân hà hàng phục kỳ tâm?''}

Câu trả lời cho câu hỏi ách yếu đó là cả một công phu tu tập lâu dài an trú tưởng Vô Ngã để loại trừ hết thảy ngã tưởng (\emph{ngã tưởng, nhơn tưởng, chúng sinh tưởng, thọ giả tưởng, pháp tưởng, phi pháp tưởng, tưởng và phi tưởng}). Vì thế Ngô Thừa Ân đã để cho chính tay Tôn Ngộ Không thật đánh chết Tôn Ngộ Không giả (Lục Nhĩ Hầu) sau khi được Như Lai chỉ rõ gốc gác của Lục Nhĩ Hầu. Lục Nhĩ Hầu chết là nhân của sinh tử, phiền não diệt.

Từ đây, dù chưa đọc các hồi truyện kế tiếp, độc giả cũng có thể hình dung ra rằng: các xung đột trong phái đoàn Tây Du sẽ lắng dịu, Đường Tăng càng ngày càng tin hiểu Tôn Ngộ Không; Trư Ngộ Năng càng ngày càng tuân phục Tôn Hành Giả. Độc giả cũng có thể hình dung ra các hồi truyện tiếp theo sẽ là những hồi đoạn trừ các lậu hoặc, rửa sạch các trần tâm, cắt đứt các kiết sử, khi phái đoàn trở thành một thể thống nhất.

{\bf Ba hồi truyện 56, 57 và 58 vừa giới thiệu cái nhìn của Phật Giáo:}

\begin{enumerate}[label=\itshape\alph*\upshape/]
    \item Dưới cái nhìn không chấp thủ các tự ngã, các tướng thật sẽ hiện rõ.

    \item Dưới cái nhìn chấp thủ các tự ngã, cuộc đời hiện ra là hư vọng.
\end{enumerate}

Cái thật và cái hư quyện chặt vào nhau, chúng là sản phẩm của tâm thức mà không phải của cuộc đời. Vấn đề chuyển đổi cái nhìn, chuyển đổi tâm thức hay tu tập tâm và giải thoát tâm trở thành vấn đề trọng tâm của cuộc hành trình Tây Du.
% section tư_tưởng_phật_học (end)

\section{Quan niệm về con Người} % (fold)
\label{sec:56_57_con_nguoi}

-- Con người là hiện hữu rất phức tạp, đầy biến động, hiểu con người đã khó, giáo dục con người càng khó hơn. Nhưng phức tạp không có nghĩa là không thể hiểu được; khó giáo dục không có nghĩa là không thể giáo dục. Tìm hiểu đúng con người và giáo dục tốt con người là sứ mệnh của nền văn hóa, giáo dục mới như là sứ mệnh thỉnh kinh của Đường Tăng cứu độ nhân dân Đông Độ.

-- Nói đến con người là nói đến các biểu hiện trí tuệ, tình cảm, tâm lý, ý chí và hành động của con người. Các biểu biện đó dù dưới hình thức, tướng trạng nào (thiện hay ác) cũng khoác vào một trong hai giá trị: \emph{chánh hoặc tà, thực hay hư}. Thực chánh hay thực tà chỉ có trí tuệ Vô Ngã mới rọi ra bản tướng, hoặc chỉ có tự thân mới nhận ra động cơ của hành động của mình. Vì thế, nền văn hóa giáo dục nhân bản về trí tuệ sẽ cung cấp cho người đời một cái nhìn mới về giá trị và một sự đánh giá mới. Cái nhìn mới là cái nhìn Vô Ngã, vị tha và nhân bản. Sự đánh giá mới là sự đánh giá thật khách quan, sự thành thật tự đánh giá mình, nền giáo dục mới này giúp con người tự hiểu mình, tự trách nhiệm, tự tín, tự giác và tự cải thiện mình, và nó được gọi là nền giáo dục nhân bản và trí tuệ.

{\bf Ngô Thừa Ân đã gián tiếp đặt vấn đề với thời đại, rằng:}

\begin{enumerate}[label=\itshape\alph*\upshape/]
    \item Chỉ có Tôn Hành Giả thật và Tôn Hành Giả giả mới thật sự biết mình là ai (ngoại trừ Như Lai).

    \item Chỉ có Đường Tăng, Ngộ Năng, Ngộ Tịnh và Ngộ Không mới xác định được Đường Tăng giả.
\end{enumerate}

-- Như đã được đề cập khi Đường Tăng giả xuất hiện, giá trị đạo đức của một hành động không thể được đánh giá chỉ qua sự biểu hiện của hành động, mà còn cần căn cứ vào động cơ của tâm hành động và hiệu quả của hành động. Hai hành động biểu hiện giống nhau (như hai Tôn Ngộ Không) mà một là thật, một là giả. Điểm này là một tâm sự mà Ngô Thừa Ân muốn đối thoại với hệ thống giá trị rất hình thức của đạo Nho đã gây ra nhiều rối rắm cho đời (\emph{xáo trộn cả càn khôn}).

-- Vấn đề sẽ trở nên tế nhị hơn, khó nhận hơn nếu từ mặt trí huệ mà nhìn. Hai hành động có chủ tâm hành động giống nhau, có biểu hiện hệt nhau và đi đến kết quả ở mặt hiện tượng giới giống nhau nhằm đem lại lợi ích cho xã hội, nhưng vẫn chuyên chở hai giá trị khác nhau: một hành động chân thật, chánh kiến, còn hành động kia thì hư dối, tà kiến, như hình ảnh xuất hiện hai Tôn Hành Giả. Với Phật Giáo, với nền văn hóa giáo dục mới, giá trị của hành động phụ thuộc vào Chánh Kiến, Chánh Tư Duy và Chánh Niệm hay không phụ thuộc vào.

Nếu độc giả thiết tha với Tôn Hành Giả thật và phái đoàn Tây Du thật, thì sẽ thiết tha với nền văn hóa giáo dục mới. Nếu độc giả không chấp nhận Lục Nhĩ Hầu với phái đoàn Tây Du trá hình bằng pháp thuật của Lục Nhĩ Hầu, thì sẽ khó chấp nhận nền văn hóa pháp thuật của Đạo gia, và nền văn hóa hình thức ước lệ của Nho gia.
% section quan_niệm_về_con_người (end)

\section{Quan niệm về xã hội} % (fold)
\label{sec:56_57_xa_hoi}

Tương tự như những gì vừa trình bày về quan niệm con người, quan niệm về xã hội cũng thế.

Một xã hội ổn định, an cư lạc nghiệp là một xã hội tốt, nhưng có thể chưa đáp ứng được yêu cầu nhân bản và trí tuệ của nền văn hóa mới.

Nếu xã hội ổn định dưới sự thống trị của một quyền lực phi dân chủ và phi nhân bản, thì không phải là xã hội của nền văn hóa mới.

Nếu xã hội ổn định trong guồng máy dân chủ và nhân bản thì vẫn chưa hẳn là xã hội của nền văn hóa mới, nếu xã hội đó vẫn được điều động bởi các tư duy hữu ngã, bởi sớm hay muộn xã hội đó sẽ rơi vào khủng hoảng tâm lý và xã hội. Một xã hội mới phải là dân chủ, nhân bản, nhân ái và trí tuệ (đặc biệt là trí tuệ) mới thật sự là xã hội tốt đẹp và ổn định lâu dài.

Để đi đến xã hội mong ước đó, con người cần phải trải nhiều công phu xây dựng, như phái đoàn Tây Du phải trải qua nhiều cuộc phấn đấu.
% section quan_niệm_về_xã_hội (end)
% chapter Hồi 56_57 (end)