\chapter{Hồi 80, 81, 82 và 83} % (fold)
\label{cha:hoi_80_81}

Hồi 80:

\begin{verse}
\begin{itshape}
Gái trẻ thèm chồng tìm bạn lứa.\\
Ngộ Không giữ chú biết yêu tà.
\end{itshape}
\end{verse}

Hồi 81:

\begin{verse}
\begin{itshape}
Chùa Trấn Hải, Ngộ Không biết quái.\\
Rừng Hắc Tùng, đồ đệ tìm thầy.
\end{itshape}
\end{verse}

Hồi 82:

\begin{verse}
\begin{itshape}
Gái trẻ gạ trai chay.\\
Nguyên thần giữ đạo lớn.
\end{itshape}
\end{verse}

Hồi 83:

\begin{verse}
\begin{itshape}
Hành giả biết được đầu mối.\\
Gái tơ lại về kiếp xưa.
\end{itshape}
\end{verse}

\section{Tư tưởng Phật Học} % (fold)
\label{sec:80_81_phat_hoc}

Người tu tập giải thoát theo giáo lý nhà Phật, khi chưa trừ sạch các lậu hoặc thì các kiết sử vẫn còn, ít nhất là còn năm hay ba thượng phần kiết sử. Kiết sử còn thì vô minh còn. Vô minh còn thì vẫn chưa phân biệt rõ thực, hư. Vì thế, Đường Tăng trên suốt đường hành trình mãi rơi vào cạm bẫy của ác ma, Sa Tăng và Bát Giới cũng thế. Chỉ có Ngộ Không là thấy rõ chánh, tà; nhưng lại khó thuyết phục Đường Tăng tin theo. Dù đã kinh qua nhiều thất bại chua xót, Đường Tăng vẫn đắm mê vào các tướng thiện, tướng từ, tướng khái niệm, và do vậy rơi vào vọng tưởng bị ác ma chế ngự, Đường Tăng cần kinh qua nhiều thể nghiệm cái nguy hiểm của chấp thủ mới nghe theo Ngộ Không.

Tại núi Hảm Không, động Vô Để Đường Tăng đang rơi vào mê lầm, rơi vào một mặt chấp ngã khác, cho rằng thực tướng Không tướng như là một tướng khác với các ngã tướng nên Ngô Thừa Ân gọi là rơi vào động Hảm Không. Chấp Hữu đã là khổ nạn, mà chấp Không lại là đại khổ nạn, còn nguy hiểm hơn. Rơi vào đó thì khó thấy lối ra, như là rơi vào động Vô Để (không đáy) thăm thẳm mịt mù.

Đây là một biểu hiện của tâm ái gọi là \emph{Vô sắc ái} hay \emph{Vô hữu ái}. Đối trị với vô sắc ái thì chỉ có trí tuệ thiền quán Vô Ngã thâm sâu. Ngộ Không chưa vào sâu định nên thật vất vả trong việc đối phó với yêu nữ ở động Vô Để này.

Nếu với thiền quán Vô Ngã thấy được cái gốc của vô sắc ái là vô minh (thượng phần kiết sử) thì liền nhổ được gốc của nó, như khi Ngộ Không tìm ra manh mối nữ yêu Vô Để là con nuôi của Thác Tháp Thiên Vương Lý Tịnh thì liên có ngay giải pháp đối phó, giải ách nạn cho Đường Tăng.

-- Hình ảnh gái, trai, nam, nữ ân ái ở cảnh nạn này chỉ là biểu tượng cho tướng nhị thủ (năng và sở). Nỗ lực thoát ly động Vô Để là nỗ lực đoạn trừ hết thảy ngã tưởng (dù là tưởng về Vô hữu), đoạn trừ tướng nhị thủ ấy. Trừ sạch tướng nhị thủ là trừ sạch chấp thủ, trừ sạch vô minh (gốc của vô sắc ái).
% section tư_tưởng_phật_học (end)

\section{Quan niệm về con Người} % (fold)
\label{sec:80_81_con_nguoi}

-- Giáo dục về nhân ái là giáo dục về tâm nhân ái, hành động nhân ái và ý chí biểu hiện hành động để cứu giúp tha nhân.

Nói đến hành động nhân ái là nói đến đối tượng đón nhận lòng nhân ái cần phải xét lòng nhân ái thì chỉ có một mà sự biểu hiện thì qua nhiều hình thái khác nhau tùy theo chủ thể, từng đối tượng, từng hoàn cảnh. Tác giả nêu ra trường hợp điển hình về nữ yêu quái động Vô Để đã giả vờ mắc nạn để đánh lừa Đường Tăng. Vì thiếu trí tuệ xét đoán chính xác đối tượng cần được cứu giúp, Đường Tăng đã để lại tai họa lớn cho phái đoàn Tây Du. Lòng nhân ái cần được phát khởi từ trí tuệ Vô Ngã, mà không phải thể hiện một cách hình thức đầy ước lệ. Đây là nội dung nhân ái của nền văn hóa, giáo dục mới luôn đem lại an lạc, hạnh phúc cho đời; đem lại an vui, Chánh Kiến cho đối tượng đón nhận lòng nhân ái.
% section quan_niệm_về_con_người (end)

\section{Quan niệm về xã hội} % (fold)
\label{sec:80_81_xa_hoi}

-- Nếp sống văn hóa nhân bản và trí tuệ của một xã hội mới cần tạo cho quần chúng một nếp tư duy về giá trị nhân ái, đạo đức đúng, thực, mà không mang tính chất hình thức ước lệ (về nhân, nghĩa, hiếu, tình, \ldots ~cũng thế) của xã hội phong kiến cũ.

Đường Tăng thật sự có lòng nhân ái và thật lòng yết Phật thỉnh kinh hay không là do chính lòng của Đường Tăng, và Đường Tăng tự biết (căn bản là Đường Tăng tự biết), mà không phụ thuộc vào câu nói đầy vẻ hình thức của yêu nữ rằng: \emph{``gặp người mắc nạn mà không cứu thì làm sao mà đi về Tây Trúc thỉnh kinh được''}. Đường Tăng đã bỏ trống tâm mình và vướng mắc vào vòng luận lý hình thức đầy gian dối của nữ yêu đầu! Đường Tăng thiếu hẳn chánh niệm tỉnh giác và trí tuệ soi sáng vấn đề giá trị nên bị rơi vào ma nạn. Người của nền văn hóa mới cần phân biệt rõ sự thật của tâm biểu hiện trong cuộc sống thì khác với ý tưởng trong ngôn ngữ.

-- Nếu bạn đọc cảm thấy đau đớn tức tưởi trước thái độ nhân ái của Đường Tăng đối với yêu nữ thì hẳn bạn đọc cũng cảm thấy đau đớn tức tưởi như vậy đối với nền văn hóa phong kiến với các giá trị ước lệ.

-- Sống là sống với cuộc sống, chứ không phải nói về cuộc sống. Nói về cuộc sống là ngôn ngữ, là lý luận hay triết lý. Sự sống là triết lý siêu đẳng và là ngôn ngữ siêu đẳng. Nền văn hóa nhân bản và trí tuệ là nói và nghĩ bằng chính cuộc sống!
% section quan_niệm_về_xã_hội (end)
% chapter Hồi 80_81 (end)