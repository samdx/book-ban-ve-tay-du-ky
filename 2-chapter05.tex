\chapter{Hồi 8} % (fold)
\label{cha:hoi_8}

\section{Tư tưởng Phật Học} % (fold)
\label{sec:8_phat_hoc}

-- Sau khi hàng phục Tôn Ngộ Không và giữ Tôn Ngộ Không tại Ngũ Hành Sơn (có chú ếm và có thần giữ an toàn cho Tôn Ngộ Không), Đức Phật trở về Lôi Âm Tự. Vừa đặt chân về Lôi Âm Tự, Đức Phật liền dạy:

\begin{itshape}
``Ta đem: Cặp mắt nhìn xa,

Nhìn khắp Tam giới

Cỗi rễ tính xưa,

Thấy đều tịch diệt

Hình thể hư không

Chẳng còn gì hết

Hàng phục khỉ già

Không mấy ai biết

Sống chết rành rành

Đấy là pháp tướng''.
\end{itshape}

Thế rồi, Đức Phật liền truyền mở hội Vu Lan ở xứ Phật và tính việc truyền bá ba Tạng kinh Phật đến Nam Thiệm Bộ Châu (từ Tây Trúc đến Đông Độ --- Trung Hoa).

-- Như đã bàn ở phần {\bf Tổng Luận}, Tôn Ngộ Không có tuệ Vô Ngã nhưng vì Giới và Định chưa tu tập vững nên còn bị trói buộc bởi Năm Uẩn (hình ảnh Ngũ Hành Sơn chụp phủ lên người 500 năm). Còn bị trói buộc bởi Năm Uẩn là còn bị trói buộc bởi các pháp hữu vi, chưa thật sự đi ra khỏi sinh tử, luân hồi. Sự kiện Tôn Ngộ Không không thể nhảy ra khỏi bàn tay của Đức Phật là bài học Đức Phật dạy tiếp cho Tôn Ngộ Không rằng cần nhiếp tâm tu mạnh hơn nữa, cần hành trì Giới uẩn, Định uẩn và Tuệ uẩn sâu xa hơn nữa. Đây là công phu mới tiếp tục công phu mà Tôn giả Tu Bồ Đề đã dạy tại núi Linh Đài Phương Thốn ở động Tà Nguyệt Tam Tinh. Công phu này sẽ bắt đầu tu tập từ khi Đường Tăng lên đường thỉnh kinh, điều mà Đức Phật đang sắp đặt tại hội Vu Lan Bồn tại Lôi Âm Tự.

-- Tại Thiên cung thì thường có hội Bàn Đào để hưởng lạc và để tăng tuổi thọ cho một số cấp lãnh đạo cao tại đó. Sau vụ ``đại náo'' thì Ngọc Hoàng bày hội ``Yên Trời''. Tại Phật xứ thì tương phản, bày hội \textbf{``Vu Lan''}, còn được gọi là hội \emph{``Hiếu Hạnh''}, để báo đáp ân phụ mẫu, sư trưởng. Cuộc thỉnh kinh về Đông Độ mà Đức Phật sẽ sắp đặt được mở đầu bằng hội ``Hiếu Hạnh'' này.

-- Đức Phật hẳn đã dự tính phái đoàn đi Tây Trúc thỉnh kinh từ khi có vụ đại náo Thiên cung và từ khi hàng phục Tôn Ngộ Không. Phật sự thỉnh kinh chính là con đường công phu giải thoát đi đến trí tuệ Ba La Mật của Tôn Ngộ Không mà Đức Phật bày vẽ; giáo dục, dẫn dắt Tôn Ngộ Không vượt qua ba tai nạn khốc liệt (\emph{sét đánh, âm hỏa và bi phong}) mà Tôn giả Tu Bồ Đề đã báo trước.

-- Đức Phật đã chuẩn bị sẵn năm bửu bối để Bồ Tát Quán Thế Âm trao cho Đường Tăng sau này: một áo cà sa, một tích trượng và ba chiếc vòng (trong đó có một vòng để khống chế tính cứng đầu, loạn động của Tôn Ngộ Không). Cà sa và tích trượng là hành trang giải thoát của một tu sĩ Phật Giáo (chỉ thiếu bình bát). Y cà sa là biểu tượng cho phạm hạnh nên có thể giúp hành giả thoát khỏi luân hồi; tích trượng thường được người xưa dùng để chế ngự thú dữ, rắn rít và những kẻ cướp đường; ở đây tích trượng của Phật tự nó khắc chế các ác vật, ác thú.

Bồ Tát Quán Thế Âm là vị Bồ Tát đã đắc trí tuệ Ba La Mật và có đại bi nguyện độ sinh (đệ tử của Bồ Tát là Huệ Ngạn đã là ở bên bờ của trí tuệ giải thoát rồi) được Đức Phật giao phó sứ mệnh theo dõi phái đoàn thỉnh kinh. Đây là biểu tượng của đích trí tuệ mà các hành giả nhắm đến.

-- Bốn đệ tử mà Bồ Tát Quán Thế Âm thu nạp để phò Đường Tăng thỉnh kinh là: Tôn Ngộ Không, Tiểu Long Mã, Sa Ngộ Tịnh, và Trư Ngộ Năng (xem \nameref{prt:tong_luan} trang \pageref{sec:qua_cac_nhan_vat_chinh}).

% section tư_tưởng_phật_học (end)
\section{Quan niệm về con Người} % (fold)
\label{sec:8_con_nguoi}

Đức Phật biết rõ tâm tư của Tôn Ngộ Không, Ngài đến Thiên cung hàng phục Tôn Ngộ Không theo lời thỉnh cầu của Ngọc Hoàng, nhưng thực sự là đến cứu Tôn Ngộ Không và giữ yên Tôn Ngộ Không tại một chỗ bình yên như là tạm thời an trí để chờ đến thời điểm tốt hầu tiếp tục hành Phật sự.

Hẳn chúng ta còn nhớ lại Tôn giả Tu Bồ Đề, vị đại đệ tử của Đức Phật, đã giáo dục Tôn Ngộ Không thành người có trí tuệ chân chính để phục vụ đạo và đời, và đã cho hạ sơn trở về Nam Thiệm Bộ Châu. Đây không phải là việc làm tình cờ, mà đã có chủ ý của Phật. Phương Chi, núi nơi trú xứ của Tôn giả Tu Bồ Đề có tên gọi là Linh Đài Phương Thốn (có nghĩa là một tấc đất của Linh Sơn, Lôi Âm Tự), phải chăng đây là phân viện giáo dục các Bồ Tát để độ sinh? Đây là phân viện truyền trao lý tưởng giáo dục độ sinh?

-- Giai đoạn đại náo Thiên cung của Tôn Ngộ Không chỉ là giai đoạn đầu của việc truyền bá con đường văn hóa giáo dục Vô Ngã (hay Từ Bi và trí tuệ Vô Ngã), chỉ để gây một tiếng vang đánh thức cuộc đời. Bấy giờ chưa phải là thời điểm cho Tôn Ngộ Không thay thế Ngọc Hoàng. Dù Tôn Ngộ Không có chiếm được Thiên cung thì lý tưởng của Tôn Ngộ Không không dễ gì truyền bá được liền, và xã hội Thiên cung không hẳn sẽ được yên ổn hơn, mà có thể sẽ có chiến tranh kéo dài để tranh lại ngai vàng đã mất khi Thiên chúng chưa nhận ra lý tưởng cải thiện xã hội tốt đẹp của Tôn Ngộ Không.

Lại nữa, khi thu phục Tôn Ngộ Không tại Ngũ Hành Sơn, Bồ Tát Quán Thế Âm đã dạy: \emph{``Nói ra một điều thiện thời ngoài nghìn dặm sẽ ứng theo; nói ra một điều bất thiện thì ngoài nghìn dặm sẽ chống lại''}.

Điều này Bồ Tát cố ý nhắc nhở cho Tôn Ngộ Không biết rằng Tôn Ngộ Không đã có lý tưởng tốt đẹp lo gì thiên hạ không nghe theo. Tôn Ngộ Không cần phải biết trầm tỉnh ẩn nhẫn mà hành động. Phải tri thời và phải tri pháp.

Người xưa đã nói: trồng cây thì cần 10 năm; trồng người thì phải 100 năm, Tôn Ngộ Không \emph{làm văn hóa} còn phải cần một khoảng thời gian dài hơn, trước tiên phải ở yên 500 năm tại Ngũ Hành Sơn, tiếp đến, cần thời gian để kết nạp những người cùng chí hướng xiển dương con đường văn hóa giáo dục mới, thời gian để đối phó với các kháng lực đầy nguy hiểm.

Tưởng chúng ta cần biết rằng sau lưng Tôn Ngộ Không chỉ có đám khỉ nhỏ chưa đủ sức để bảo vệ Hoa Quả Sơn, sau đó nếu có lực lượng quần chúng hậu thuẫn nhiều thêm thì lực lượng này không thể một lòng một dạ và có quyết tâm như Tôn Ngộ Không, mà: hoặc là bất nhất, dễ bỏ cuộc như Trư Bát Giới; hoặc là tình cảm mềm lòng như Đường Tăng; hoặc trung thành và có quyết tâm nhưng thiếu dũng lược như Sa Ngộ Tịnh, thì rất khó mà cải tổ nhanh chóng được xã hội trước kháng lực có mặt khắp các ngõ ngách như một con ``bạch tuộc''.

-- Vấn đề con người luôn luôn liên hệ chặt chẽ với vấn đề trí tuệ (tức trình độ văn hóa), với vấn đề Chánh Kiến (tức các quan điểm chủ trương), vấn đề tâm lý tình cảm, và vấn đề ý chí. Con đường tư duy mới về giá trị của con người và xã hội sẽ mở ra một hướng văn hóa, giáo dục mới. Chấp nhận con đường mới ấy không phải chỉ có sự chấp nhận của tư duy mà đủ, mà cần có ý chí, tình cảm và tâm lý chấp nhận nữa.

Con đường văn hóa giáo dục này vì thế đòi hỏi cần có thời gian lâu dài để chuẩn bị con người. Tại đây chưa nói đến cơ chế tổ chức của xã hội cũ, guồng máy Nhà nước phong kiến cũ rất cũ kỹ và rất cồng kềnh. Con đường văn hóa giáo dục ấy cũng cần có thời gian để cuộc đời trắc nghiệm giá trị.

Đó là các lý do mà Đức Phật đã chuyển Tôn Ngộ Không qua một hướng hành động khác có tính khoa học hơn và thuyết phục hơn. Thời điểm Tôn Ngộ Không an nghỉ ở Ngũ Hành Sơn là mốc điểm thời gian chuyển hướng, mà không phải là đầu hàng. Đó là thời gian cần thiết để Tôn Ngộ Không kiểm điểm công tác và định hướng công tác.

% section quan_niệm_về_con_người (end)

\section{Quan niệm về xã hội} % (fold)
\label{sec:8_xa_hoi}

Xã hội và con người cá thể quyện chặt vào nhau như một thực thể. Hai hiện hữu ấy có tác động ảnh hưởng qua lại: khi xã hội văn hóa thay đổi thì cá thể thay đổi theo; ngược lại, khi nhận thức, tư duy của con người thay đổi thì xã hội cũng dần dần thay đổi.

Cả hai đều có tính chất Vô Ngã hay bất định, có thể được thay đổi tùy theo các điều kiện sống, môi trường sống. Ý thức như vậy, tác giả Ngô Thừa Ân đã chọn lựa thay đổi nhận thức con người trước, bởi chính con người là chủ thể xây dựng xã hội.

Nhưng nhận thức con người thì bị ràng buộc chặt chẽ với tổ chức của xã hội, nhất là xã hội ấy là một xã hội phong kiến lâu đời. Các giá trị phong kiến đã ăn sâu vào tiềm thức con người và cơ cấu xã hội. Bất cứ một thay đổi nào về giá trị thuộc xã hội phong kiến đều vấp phải một kháng lực có quyền uy rất lớn của vua, tôi, triều thần và quần chúng dưới trướng triều đình. Đối mặt với sức mạnh này, tất cả sức mạnh này, tất cả sức mạnh của các cá nhân và của các nhóm quần chúng nhỏ đều phải tan vỡ.

-- Lại nữa, xã hội phong kiến đã nuôi dưỡng các lực lượng tà phái rất mạnh, tồn tại dựa vào các khuyết điểm của xã hội phong kiến; các lực lượng này cũng không muốn thay đổi xã hội này; bởi vì sự thay đổi ấy đe dọa đến sự tồn tại và phát triển của các thế lực ấy. Các lực lượng này trở thành vật cản đáng kể dưới bước đi của Tôn Ngộ Không.

Tôn Ngộ Không và nhóm chính phái của Tôn Ngộ Không, phải thường xuyên lưỡng đầu thọ địch. Đó là chưa nói đến việc có những sức cản nằm trong nhóm quần chúng của Tôn Ngộ Không như vừa đề cập ở mục \nameref{sec:8_con_nguoi}, \nameref{cha:hoi_8} này.

Thực hiện con đường lý tưởng về văn hóa, giáo dục Vô Ngã, vì thế Tôn Ngộ Không cần, rất cần, đến sự soi sáng của trí tuệ toàn giác và lòng đại bi bủa ra của trí tuệ toàn giác (biểu tượng bằng sự dẫn dắt của Đức Phật và Bồ Tát Quán Thế Âm).

Nhìn lại lịch sử Việt Nam dưới thời Lý, Trần, chúng ta có thể có một khái niệm về sự chuẩn bị cho công việc thực hiện lý tưởng trên. Vạn Hạnh thiền sư, Khánh Vân thiền sư, và cả ngài Khuông Việt, Pháp Thuận thiền sư đã dốc bao nhiêu tâm lực để tìm ra Lý Công Uẩn và mở dần con đường cho Lý Công Uẩn lên ngôi báu.

Nhân tâm cũng là một yếu tố ảnh hưởng mạnh đến việc thay đổi chế độ xã hội, có thể tạo điều kiện thuận lợi, mà cũng có thể bất lợi. Vì thế con đường sống vị tha, đầy tình người và trí tuệ, cần được mở đầu bằng hiếu đạo, điều mà Đức Phật đã quan tâm mở ra hội ``Vu Lan Bồn'' khi chuẩn bị lập thành phái đoàn Tây Du.

Chúng ta sẽ dần dần thấy rõ những gì mà Tôn Ngộ Không cần phải vượt qua vào các hồi kế tiếp.

% section quan_niệm_về_xã_hội (end)
% chapter hồi_8 (end)