\chapter{Hồi 13 và 14} % (fold)
\label{cha:hoi_13_14}

Hồi 13:

\begin{verse}
\begin{itshape}
Sa hang cọp Kim Tinh cứu thoát.\\
Núi Long Hoa Bá Khâm mời sư.
\end{itshape}
\end{verse}

Hồi 14:

\begin{verse}
\begin{itshape}
Vượn già theo chính.\\
Sáu giặc mất tăm.
\end{itshape}
\end{verse}

\section{Tư tưởng Phật Học} % (fold)
\label{sec:13_phat_hoc}

Sau khi phát tâm cầu giải thoát và phát đại nguyện độ sinh, hành giả bắt đầu hạ thủ công phu giải thoát. Giải thoát không thể đạt được bằng chỉ một tấm lòng mong muốn giải thoát. Tấm lòng hay khát vọng giải thoát chỉ là sự nhắc nhở đến mục đích giải thoát cần phải đến, tự nó chưa có khả năng hình thành một bước đi vào giải thoát nào. Vì thế Đường Tăng vừa mới lên đường đã sa ngay vào hang cọp, ba con cọp chúa tể sơn lâm ấy là biểu tượng của \emph{tham, sân, si} ở nội tâm và ở ngoại cảnh đang thời sung mãn; chúng dễ dàng ``thịt'' mất chí nguyện giải thoát độ sinh của hành giả. Giai đoạn này nếu không nhờ tha lực của Kim Tinh cứu mạng thì lý tưởng Tây Du đã không còn.

-- Trước khi lên đường, Đường Tăng đã ý thức việc tu dưỡng Tâm, nên \emph{``lẳng lặng chỉ tay vào bụng mấy lượt''} để nói với Tăng chúng tiễn chân rằng: \emph{``Tâm sinh thì ma nghiệp sinh, tâm diệt thì ma nghiệp diệt''}.

Vậy khi ma chướng khởi (ma nạn sinh) thì bấy giờ hành giả và chí nguyện giải thoát chính là ma chướng, hay bị khống chế bởi ma chướng. Ma chướng đó là \emph{tà niệm, vọng niệm}.

Đối với vọng niệm, lòng khát vọng giải thoát không có đôi mắt để nhìn thấy và cũng không có khả năng để nhiếp phục. Nếu thiếu trí tuệ thì hành giả không thể tu tập giải thoát qua hai công phu chính yếu sau đây:

\begin{enumerate}[label=\itshape\arabic*\upshape/]
    \item Nhận ra đâu là chánh niệm để duy trì và phát triển; chánh niệm nào chưa sinh thì khiến sinh khởi, chánh niệm nào đã sinh thì duy trì và phát triển.

    \item Nhận ra đâu là tà niệm; tà niệm đã sinh thì cần được tìm cách đoạn diệt, tà niệm chưa sinh thì không để nó sinh khởi.
\end{enumerate}

Ma nạn ``hang cọp'' này là nạn cảnh giác Đường Tăng phải tu tập có Chánh Kiến và Chánh Tư Duy (hay Trí Tuệ) trước, rồi vận dụng trí tuệ này dẫn đầu các công phu tu tập khác.

Đây mới là chánh đạo. Do vậy, vào hồi 14 Đường Tăng cần thu nhận Tôn Ngộ Không phò tá như một vai trò trí tuệ trên cuộc hành trình Tây Du.

-- Chính nhờ Trí Tuệ (tức vai trò của Tôn Ngộ Không) nhận ra bạn, thù và có khả năng rất lớn có thể đốt cháy, đánh tan các vọng niệm khởi lên do tác động từ trong lẫn ngoài, như Tôn Ngộ Không đã diệt đám giặc cỏ lục tặc ở hồi 14 này, đám giặc thường gây hỗn loạn tâm thức và cuộc đời.
% section tư_tưởng_phật_học (end)

\section{Quan niệm về con Người} % (fold)
\label{sec:13_con_nguoi}

-- Để xây dựng một xã hội tốt đẹp đúng nghĩa nhất cần có con người tốt. Con người tốt là con người được phát triển, giáo dục toàn diện, có khả năng nhận định tốt, xấu, phải, trái, hư, thật và có khả năng để hành xử các việc cá nhân, gia đình và xã hội.

-- Con người cần được hiểu một cách thực tế rằng không phải là được sinh ra như một tờ giấy trắng, mà là có đủ một \emph{mầm tâm lý tốt, xấu lẫn lộn}. Các tâm lý xấu như tham, sân, si, ganh ghét, đố kỵ, kiêu căng, ngã mạn, phóng túng có mặt mạnh hơn là các tâm lý tốt. Điều kiện sống ở xã hội phong kiến (và xã hội nói chung) kích động các tâm lý xấu nhiều hơn là các tâm lý tốt. Do vậy, yêu cầu đầu tiên của giáo dục là cần giáo dục con người có một hiểu biết, tư duy phân biệt rõ, tốt, xấu, thực, hư, vv\ldots ~và có khả năng chỉ đạo tốt các hành vi của con người như Đường Tăng cần đến sự có mặt của Tôn Ngộ Không trong phải đoàn Tây Du.

-- Tư duy đúng và hiểu biết đúng cần được giáo dục tế nhị, cần được khéo léo tiếp thu, cần có sự luyện tập nhiều về \emph{văn và tư} để thắng vượt các thành kiến, định kiến của văn hóa, truyền thống và các tập quán tình cảm, tư duy cũ. Các tập quán này trói buộc con người rất mạnh như biểu tượng Ngũ Hành Sơn giam hãm Tôn Ngộ Không dài đến 500 năm. Phá bỏ được các ràng buộc này con người mới có được cái hiểu biết và tư duy sáng tạo và giá trị áp dụng vào đời sống cá nhân và xã hội.
% section quan_niệm_về_con_người (end)

\section{Quan niệm về xã hội} % (fold)
\label{sec:13_xa_hoi}

-- Bước cải tổ xã hội đầu tiên nếu không được lý trí tính toán kỹ thì khó vượt qua được rào cản lớn của các lực lượng chủ đạo của con đường văn hóa, xã hội và thể chế cũ (được biểu trưng bằng: mãnh thú sơn lâm hãm hại Đường Tăng), lực lượng cải tổ sẽ bị tổn thất nặng, có thể bị lâm nguy.

-- Lý tưởng cải tổ xã hội không thể đánh giá đúng ở mặt thuần lý thuyết mà được đánh giá vào thử thách trong thực tế xã hội để làm sáng tỏ giá trị của nó như là Tôn Ngộ Không đứng dậy khỏi Ngũ Hành Sơn, lên đường thực hiện và mang thêm một tên gọi mới là Tôn Hành Giả (hành giả nghĩa là người thực hiện, hành động).

-- Lý tưởng thực hiện cần được hoạt động uyển chuyển, không quá cứng rắn, không quá nôn nóng, vội vàng, như cần phải có cái thắng ``khẩn cô nhi'' trên đầu Tôn Hành Giả.

-- Đó là những gì mà chúng ta có thể hình dung về các quan niệm hành động để cải tổ xã hội của Ngô Thừa Ân.
% section quan_niệm_về_xã_hội (end)
% chapter Hồi 13 và 14 (end)