\chapter{Hồi 65 và 66} % (fold)
\label{cha:hoi_65_66}

Hồi 65:

\begin{verse}
\begin{itshape}
Yêu tà đặt Tiểu Lôi Âm giả.\\
Bốn vị rơi vào nạn ách to.
\end{itshape}
\end{verse}

Hồi 66:

\begin{verse}
\begin{itshape}
Các thần gặp tay độc ác.\\
Di Lặc tróc nã yêu ma.
\end{itshape}
\end{verse}

\section{Tư tưởng Phật Học} % (fold)
\label{sec:65_66_phat_hoc}

-- Người tu giải thoát, theo Phật Giáo phải đi qua nhiều cảnh giới tâm thức. Tâm lý đi sâu dần vào định, vào các cảm giác hỷ, lạc, thanh lương và khởi lên nhiều cái thấy biết khác nhau ở các cấp độ tâm thức khác nhau.

Các cảm thọ hỷ, lạc, thanh lương phát sinh từ nỗ lực thiền định. Chúng là vô thường và chịu sự tan rã. Sự tan rã nào cũng để lại các khổ đau. Vì thế hành giả cần luôn giữ tâm an trú vào chánh niệm, tỉnh giác, nghĩa là an trú vào xả tâm và vào cái thấy biết trực tiếp về Duyên Sinh Vô Ngã từ dòng tâm lý và vật lý của chính mình. Nuôi dưỡng chánh niệm ấy, thấy biết ấy là nuôi dưỡng Tâm và Tuệ giải thoát. Rời khỏi chánh niệm Vô Ngã trên thì hành giả liền rơi vào vọng niệm mà tác giả Ngô Thừa Ân biểu thị bằng các ma cảnh, ma nạn.

Đường Tăng bị thất niệm trong các cảm thọ tiên cảnh nên sa vào nạn Am Mộc Tiên, rồi lại tiếp tục bị rơi vào cảnh Tiểu Lôi Âm, chỉ vì thiếu giác tỉnh Vô Ngã mà Ngô Thừa Ân diễn đạt bằng thái độ không biết nghe theo ý kiến của Tôn Ngộ Không.

Đường Tăng, Ngộ Năng và Ngộ Tịnh đều thiếu trí tuệ Vô Ngã nên thường bị các huyễn ngã huyễn tướng đánh lừa. Cảnh Tiểu Lôi Âm biểu hiện rất rõ nét về chuyện ngã tướng đánh lừa ấy khiến cả phái đoàn Tây Du thêm một phen lận đận, khốn đốn.

Đường Tăng với tập khí tín ngưỡng Tam Bảo, tôn sùng các sắc tướng của đạo nên dễ dàng bị Tiểu Lôi Âm che khuất trí tuệ. Nói khác đi, trong tâm thức của Đường Tăng, Ngộ Năng và Ngộ Tịnh lúc bấy giờ (tại Tiểu Lôi Âm) chỉ có sự hiện diện của các ngã tướng về thực tướng, mà không có cái tuệ thấy được thực tướng Vô tướng. Do tâm thức còn mơ màng về thực tướng như vậy nên Ngô Thừa Ân bèn dựng nên cảnh Tiểu Lôi Âm để đánh thức phái đoàn Tây Du, đánh thức Đường Tăng.

-- Tôn Ngộ Không thì tự tâm phân biệt rõ thực tướng và giả tướng, nhưng vì còn tập khí sinh tử nên cũng bị hệ lụy với Đường Tăng. Bị khốn đốn vì Tiểu Lôi Âm, bị túm vào ``túi vải'' và bị giam vào cái ``chuông vàng''.

Ý nghĩa tôn kính Tam Bảo đúng nghĩa là tôn kính tu tập Giới, Định, Tuệ, mà không phải là lễ kính các pho tượng, chư Tăng và các bản kinh. Theo giáo nghĩa Kim Cang Bát Nhã, phàm cái gì là ngã tướng thì hư vọng. Đường Tăng bị vướng mắc cái ngã tướng (cái tướng thiện) nên rơi vào nạn Tiểu Lôi Âm.

-- Cái ``túi vải'' và cái ``não vàng'' của Bồ Tát Di Lặc là biểu tượng của cảnh giới vô hạn lượng của thực tướng Vô Ngã tướng. Phái đoàn Tây Du, gồm cả Tôn Ngộ Không, chưa rửa sạch tâm chấp thủ tự ngã nên tất cả không tìm được lối thoát ra khỏi ``túi vải'' và ``não vàng''.
% section tư_tưởng_phật_học (end)

\section{Quan niệm về con Người} % (fold)
\label{sec:65_66_con_nguoi}

-- Giáo dục trí tuệ cho con người là phần giáo dục giúp cho con người thấy rõ sự thật của các hiện hữu, như là:

\begin{enumerate}[label=\itshape\alph*\upshape/]
    \item Các hiện tượng vật lý, gồm thân sắc mình và người, đều do nhân duyên sinh. Vì do các duyên mà sinh nên các hiện hữu không tự có, không có tự ngã, chúng rỗng không tự ngã gọi là Vô Ngã.

    Người đời thì cho rằng các hiện hữu có tự ngã khiến tâm sinh ái luyến hay ghét bỏ. Với các hiện hữu ưa thích thì mong nắm giữ, mong chúng tồn tại không mất; nhưng sự thật thì chúng biến dịch trong từng sát na để đi dần đến hoại diệt, tan rã. Sự kiện tan rã này gây ra các nỗi khổ đau tâm lý.

    \item Các hiện tượng tâm lý cũng thế, cũng do duyên sinh, chúng là vô thường và dẫn đến tan rã, khổ đau.

    Con người cần được giáo dục kỹ, cần quán sát tư duy mãi, quán sát, tư duy nhiều lần sự thật Duyên Khởi, Vô Ngã để vơi bớt khổ đau hay tiêu đi nỗi khổ đau trước dòng sống vô thường. Đó là thái độ sống, thái độ nhìn cuộc sống một cách khôn ngoan, chế ngự được nhiều phiền não khổ đau.

    \item Quan trọng hơn cả là cần luôn cảnh giác rằng tất cả hiện tượng tâm lý và vật lý kia rất dễ dàng đánh lừa cá nhân, tráo đổi cái hư làm thật, nếu cá nhân thiếu cái nhìn trí tuệ. Các ngã tướng của hiện tượng vốn là không thật, vốn là gốc tạo ra khổ đau nếu cá nhân nắm giữ, ái luyến chúng. Sống là sống với cái thật, cái chân thật, bởi chỉ có cái chân thật mới đem lại an lạc, hạnh phúc chân thật, và sống là sống với hiện tại, với hạnh phúc trong hiện tại. Điều này có nghĩa là con người cần sống với sự thật Vô Ngã, với nền văn hóa giáo dục Vô Ngã.
\end{enumerate}

Ngô Thừa Ân phải chăng đã đặt ra vấn đề đó với xã hội Trung Hoa khi dựng nên ma nạn Tiểu Lôi Âm để cảnh giác Đường Tăng và cuộc đời?

Phương chi, đương khi mắc nạn Tiểu Lôi Âm, thấy rõ cái nguy hiểm của các giả tướng, ngã tướng, Đường Tăng tỉnh ngộ biết lỗi mình không nghe lời can của Ngộ Không (trí tuệ Vô Ngã) mà nói rằng: \emph{``Đồ đệ hãy cứu ta một lần này nữa, từ đây về sau, mọi việc ta sẽ nghe theo con, không dám cưỡng nữa.''} (sđd, tr 335, tập 3, 1988)

-- Suốt cả bộ truyện Tây Du Ký nói nhiều đến sự việc người đời thường đánh giá người khác qua cái tướng trạng bên ngoài, đều sợ hãi cái tướng xấu của ba đồ đệ của Đường Tăng. Ngô Thừa Ân đã xác nhận về Ngộ Không, Ngộ Năng và Ngộ Tịnh rằng: \emph{``tướng xấu mà tâm tốt, làm được việc''}. Hình như tác giả Tây Du Ký muốn tạo một tướng trạng tương phản với nền văn hóa Trung Hoa vốn chuộng từ chương và các tướng nhân, nghĩa, trung, hiếu, \ldots ~bên ngoài, vừa đánh thức dậy tâm thức dân tộc (Trung Hoa), vừa muốn gây sự chú ý đến giá trị của cái tâm và cái tuệ. Hạnh phúc và nghĩa sống nằm ở cái tâm, cái tuệ ấy.
% section quan_niệm_về_con_người (end)

\section{Quan niệm về xã hội} % (fold)
\label{sec:65_66_xa_hoi}

-- Xã hội nào, đoàn thể nào cũng cần phải hành động nhân danh con người và hạnh phúc của con người. Có những tổ chức và những hướng hành động sai mà người đời dễ thấy rõ. Có những cá nhân nhân danh tốt mà hành động sai thì người đời cần lượng định để tránh. Có những nhân danh tốt và hành động đúng mà được thực hiện với giả tâm thì người đời cũng cần biết để tránh xa, như trường hợp Lục Nhĩ Hầu, Đường Tăng giả, hay Tiểu Lôi Âm.

-- Một nhân danh tốt được thể hiện bằng hành động đúng, và được thể hiện với tâm tốt, trí tuệ, thì hẳn là tốt. Một xã hội tốt, một tổ chức tốt phải hội đủ các duyên ấy.

-- Con đường văn hóa mới không vụ vào danh nghĩa, chiêu bài, mà vụ vào thực chất trí tuệ đem lại an lạc và hạnh phúc lâu dài cho đời.

Phải chăng đây là hoài bão của tác giả Tây Du Ký?
% section quan_niệm_về_xã_hội (end)
% chapter Hồi 65_66 (end)