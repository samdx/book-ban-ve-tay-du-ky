\chapter{Hồi 23} % (fold)
\label{cha:hoi_23}

Hồi 23:

\begin{itshape}
``Tam Tạng không quên gốc

Bốn thánh thử lòng thiền''
\end{itshape}

\section{Tư tưởng Phật Học} % (fold)
\label{sec:23_phat_hoc}

— Công phu giải thoát càng vào sâu càng đòi hỏi hành giả phát triển sâu Giới, Định, Tuệ; càng vào sâu, các tâm lý trói buộc càng tinh tế. Hành giả càng tinh tấn, nỗ lực và giác tỉnh liên tục mới phát triển được định và tuệ giải thoát. Điều kiện đòi hỏi trước tiên và ít nhất là: hành giả phải dứt trừ dục ái, vì vậy bốn vị Bồ Tát (Lê Sơn, Quán Âm, Phổ Hiền và Văn Thù):

\begin{itemize}
   \item[–] Lê Sơn biểu tượng của đại nguyện;
   \item[–] Phổ Hiền biểu tượng của đại định và đại hạnh;
   \item[–] Quán Thế Âm biểu tượng của đại bi và đại trí;
   \item[–] Văn Thù là biểu tượng đỉnh cao của trí tuệ giải thoát.
\end{itemize}

Thị hiện ra trắc nghiệm phái đoàn Tây Du và hỗ trợ cho phái đoàn trước khi đi vào một giai đoạn mới. Qua cuộc trắc nghiệm này, Trư Bát Giới bộc lộ rõ lòng trần (dục ái) chưa sạch, cần được cảnh giác tu tập nhiều hơn.
% section tư_tưởng_phật_học (end)

\section{Quan niệm về con Người} % (fold)
\label{sec:23_con_nguoi}

— Tâm lý tham dục thường che mờ tâm trí con người, đưa đẩy con người vào các sai lầm, đánh mất lý tưởng nhân sinh và xã hội cao thượng. Đây là điểm tâm lý cần được giáo dục, huấn luyện kỹ để tránh các rối loạn cho cá nhân, gia đình và xã hội. Người mà nặng lòng tham, nhất là tham sắc, thì có khuynh hướng sống vị kỷ, gây tổn hại tha nhân, dễ đánh mất nhân cách tốt đẹp của con người và dễ rơi vào phiền não, khổ đau lâu dài.
% section quan_niệm_về_con_người (end)

\section{Quan niệm về xã hội} % (fold)
\label{sec:23_xa_hoi}

— Do từ tư duy hữu ngã mà chấp thủ ngã tướng sinh và lòng khởi lên tham lam, sân hận, ích kỷ, ganh ghét, đố kỵ, \ldots. Chính xác tâm lý xấu này sẽ bộc hiện ra các hành động hại mình, hại người.

— Để chế ngự tư duy ấy và các tâm lý xấu ấy, con người cần phải thường huấn luyện Văn, Tư, Tu về Vô Ngã, vị tha. Đây là hệ văn hóa giáo dục thực tiễn và thực nghiệm. Xã hội phải có những con người này trước thì công cuộc xây dựng và phát triển đất nước mới thành công tốt đẹp.

Ngô Thừa Ân đã chọn lựa con đường văn hóa này, khảo sát các hồi tiếp chúng ta sẽ rõ.
% section quan_niệm_về_xã_hội (end)
% chapter Hồi 23 (end)