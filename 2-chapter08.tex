\chapter{Hồi 15, 16 và 17} % (fold)
\label{cha:hoi_15_16_17}

Hồi 15:

\begin{itshape}
``Núi Rắn Cuốn, mọi thần ngầm giúp.

Khe Ưng Sầu, ý mã thắng cương.''
\end{itshape}

Hồi 16:

\begin{itshape}
``Viện Quan Âm sư lừa bảo bối.

Núi Hắc Phong quái trộm cà sa.''
\end{itshape}

Hồi 17:

\begin{itshape}
``Tôn Hành Giả đại náo Hắc Phong

Quan Thế Âm thu phục con yêu quái''
\end{itshape}

\section{Tư tưởng Phật Học} % (fold)
\label{sec:15_phat_hoc}

— Tại hồi 15, khi mà công phu tu tập được trí tuệ Vô Ngã soi sáng thì công phu ấy đúng nghĩa là công phu giải thoát. Bấy giờ tâm sẽ khởi lên nhiều niệm lành, giải thoát. Đây là ý nghĩa của hình ảnh biểu tượng các thiện thần tụ họp tại núi Rắn Cuốn để giúp đỡ Đường Tăng.

— Ở hồi 14, sau khi Tôn Ngộ Không hàng phục Lục Tặc — giặc cỏ quấy rối tâm thức — thì tâm thức trở nên an tịnh, vắng lặng. Bấy giờ sơn thần ở Lạc Già Sơn, vâng mệnh Bồ Tát Quán Thế Âm trao yên và cương để thắng Tiểu Long Mã — mà Ngô Thừa Ân gọi là \emph{``ý mã thắng cương''} — Đây là hình ảnh biểu tượng khi tâm thức đã được chế ngự, không còn vọng ngoại (hướng ra sáu trần), đã quay về giải thoát (hướng nội) như Tiểu Long Mã đã từ bỏ long cung và từ bỏ tâm ngỗ ngịch để theo phò Đường Tăng.

Khi tâm thức lắng đọng như bấy giờ thì sáu trần được soi tỏ bản tướng, không còn tác loạn tâm ý được nữa, sầu muộn sẽ ra đi, vĩnh viễn từ giã tâm ý. Vì thế khi nước tâm trong suốt thì các loài chim không thể bay qua, vĩnh viễn bay xa. Con suối bấy giờ được Ngô Thừa Ân gọi là ``Khe Ưng Sầu''.

— Ở cấp độ tâm thức này, hành giả không còn bị lòng dục các trần đốt cháy, nên tại Viện Quan Âm (hồi 16) lửa tham dục các trần chỉ đốt cháy sư trụ trì trộm áo cà sa quý của Đường Tăng, mà không thiêu Đường Tăng và các sư sãi còn lại.

— Tại đây, con ma gấu ở núi Hắc Phong, bạn của sư trụ trì Quan Âm Viện, cũng tham lam đánh cắp chiếc cà sa, nhưng tại sao không bị đốt cháy, hay bị thiếc bổng của Tôn Ngộ Không đánh chết? Sự thật là cùng làm một việc tham lam, đánh cắp áo cà sa, nhưng hai người ấy có tác ý khác nhau ở hai hoàn cảnh khác nhau, nên gặt hái hậu quả khác nhau.

Sư trụ trì là kẻ xuất gia sống ở nhà chùa đã hơn 200 năm mà còn lòng tham ấy thì quả là tâm tu quá xấu; với tâm tu ấy thì chỉ cần một lòng tham áo cà—sa là đã đủ để đốt cháy tâm giải thoát. Sư trú trì bị thiêu cháy là thế.

Ma Gấu ở núi Hắc Phong là loài yêu tinh mà thích giao du với nhà chùa là chứng tỏ có một tâm thức hướng thượng. Ma Gấu trộm áo cà-sa và định mở hội cà sa cũng là hành động với tâm thức hướng về đạo, nhưng hướng về đạo ở cấp độ tâm lý thích nhà chùa và những pháp khí thuộc nhà chùa. Xét kỹ tâm thức này là đang đi lên. Do đó, Bồ Tát Quán Thế Âm không cho hại ma Gấu. Lần đầu gặp ma Gấu, Bồ Tát đã nói: \emph{``Nghiệt súc chiếm cứ sơn động này cũng có chút duyên đạo''}, rồi sau đó Bồ Tát đã quy y cho ma Gấu và thu về cho giữ núi Lạc Già ở Nam Hải.

— Khi Bồ Tát Quán Thế Âm biến hóa ra yêu tinh Lăng Hư (bạn của ma Gấu) thì Tôn Ngộ Không liền nói: \emph{``Đẹp quá! Đẹp quá! Thế Bồ Tát là yêu tinh hay yêu tinh là Bồ Tát?''}

— Bồ Tát Quán Thế Âm dạy: \emph{``Ngộ Không, Bồ Tát, yêu tinh đều là ý niệm, nếu bàn cho đến gốc, đều là không có''}. Lời dạy này giúp Tôn Ngộ Không tỉnh lòng thêm. Tại đây Ngô Thừa Ân đã diễn đạt một tư tưởng khác của Bát Nhã rằng:

\begin{enumerate}[label=\itshape\arabic*\upshape/]
    \item[+] Tướng thật của Bồ Tát là Vô Ngã; tướng thật của yêu tinh cũng là Vô Ngã. Vì thế, với tâm thức giác ngộ thật tướng thì yêu tinh và Bồ Tát không khác biệt, đều đồng thể Vô Ngã (Bồ Tát $\doteq$ yêu tinh $\doteq$ Vô Ngã).

    Nhưng với tâm thức còn vướng mắc tập khí chấp ngã, thì tướng Bồ Tát khác với tướng yêu tinh. Tôn Ngộ Không tuy lòng đã giác tỉnh Vô Ngã từ lâu, nhưng tập khí chấp thủ tướng vẫn còn, nên tâm thức thỉnh thoảng ngỡ ngàng vướng ngại các ngã tướng. Do sự vướng ngại ấy mà Tôn Ngộ Không còn vướng mắc các ma nạn.
\end{enumerate}

Nói tóm lại, ngã tướng là của ý niệm, của ngã tưởng; nó là hư vọng, mà không phải là thực tại.

Ở đây, Ngô Thừa Ân muốn giới thiệu với đời rằng: Bồ Tát hàng phục yêu Gấu là đánh chết cái tính ``yêu tinh'' trong con Gấu, chứ không phải giết chết con gấu. Con Gấu tự nó không mang một ngã tính yêu hay Bồ Tát gì cả.

% section tư_tưởng_phật_học (end)

\section{Quan niệm về con Người} % (fold)
\label{sec:15_con_nguoi}

— Ngô Thừa Ân mong muốn con người quay trở về chính mình để tu tập tâm và giải thoát tâm, như Tôn Ngộ Không tu tập ở Tà Nguyệt Tam Tịnh động — Mọi giá trị thực là ở tâm thức con người. Khi trí tuệ Vô Ngã đã xuất hiện thì tâm thức sẽ dần dần sáng tỏ. Các thiện tâm và giải thoát tâm sẽ sinh khởi. Giáo dục nhân bản là giáo dục giúp con người quay trở về chính minh, chính nguồn sáng bên trong mình. Mọi thứ may mắn, an lạc, tự do, hạnh phúc đều có mặt ở đó, như phái đoàn Tây Du có mặt Tôn Ngộ Không. Đây là ý nghĩa được biểu tượng bởi hình ảnh các thiện thần đến giúp Đường Tăng ở núi Rắn Cuốn. Đây cũng là ý nghĩa mà ngạn ngữ Pháp nói: {\bf ``Hãy tự giúp mình rồi trời sẽ giúp mình''} (Aide—toi, Le ciel t'aidra).

— Vấn đề không phải là buộc người khác làm các điều thiện, mà là giúp người khác nhận thức rõ về điều thiện và giá trị của điều thiện. Khi đã có nhận thức đúng thì tự khắc con người hành động đúng. Tương tự như khi Khe Ưng Sầu trong suốt, thì các loài chim không còn hiện đến. Cũng thế, khi tâm đã lắng, thì các trần không còn xâm nhập quấy phá. Khi một người đã thấy rõ con đường văn hóa Vô Ngã, thì các quan niệm về văn hóa hữu ngã sẽ không còn khả năng thuyết phục.

— Về điều thiện ác, con đường văn hóa giáo dục mới cần nói rõ về giá trị của các hành động, việc làm, yếu tố quyết định giá trị của hành động là ở tâm tác động trên hành động. Vì thế, cùng một hành động có thể có nhiều giá trị khác nhau và chịu các hậu quả khác nhau, điển hình như ma Gấu và sư trụ trì Quan Âm Viện đều đánh cắp tấm cà sa quý mà một người thì bị lửa thiêu, một người thì được xuất gia với Bồ Tát. Hay một người thì bị đốt cháy thân phạm hạnh và thân tuệ mạng, người kia thì phát khởi tuệ mạng, tuệ tâm.

— Theo tinh thần giáo dục Vô Ngã và nhân bản thì con người là vô tự tính và hành động là vô tự tính, bởi cả hai đều do các duyên tập hợp mà sinh. Không thật sự có con người xấu mà chỉ có hành động được ghi nhận là xấu. Cả đến hành động, không thật sự có hành động xấu, mà chỉ có các ý niệm xấu tác động trên hành động. Vì thế nền giáo dục mới đặt trọng tâm ở công phu chuyển đổi các ý niệm, chế ngự các ý niệm (gọi là tu tâm), dù hành giả vẫn quan tâm đến các biểu hiện của hành động và hậu quả của hành động. Khi ý niệm xấu được chế ngự, thì các hành động xấu sẽ dần dần biến mất. Điều này được Ngô Thừa Ân diễn đạt qua hình ảnh ma Gấu được Bồ Tát Quán Thế Âm và Tôn Ngộ Không chế ngự và thu nhận về cảnh giới xuất gia giải thoát ở Nam Hải.

Tại đây, người chủ xướng văn hóa, giáo dục Vô Ngã và nhân bản, Tôn Ngộ Không, cần nhận thức rõ sứ mệnh của mình là chuyển hóa tâm thức con người, tiêu diệt các sai lầm, chứ không phải là tiêu diệt những con người sai lầm (hồi 16, 17).

Dù vậy, người làm giáo dục không được xao lãng đến vai trò tác động của ngoại giới trên tâm thức con người; giáo dục cần vạch rõ khía cạnh nguy hiểm của ngoại giới có thể gieo rắc tai hại đến con người.

Hồi 15, 16, và 17 quả là các hồi đề cập đến đạo đức của Phật Giáo, vấn đề giá trị của các hành vi, hành động, và vấn đề đánh giá các giá trị của các hành vi, hành động.
% section quan_niệm_về_con_người (end)

\section{Quan niệm về xã hội} % (fold)
\label{sec:15_xa_hoi}

— Trở lại vấn đề đã được đề cập, căn bản của việc cải tạo xã hội là vấn đề cải tạo con người, và căn bản của vấn đề cải tạo con người là vấn đề cải tạo nhận thức của con người. Khi mà đa số quần chúng đã giác ngộ con đường sống với những giá trị mới thì hẳn nhiên khung sườn xã hội mới sẽ phải hình thành. Khi đã là Khe Ưng Sầu trong suốt, thì chim chóc đến sẽ lâm nạn. Do vì nguy hiểm ấy, chim sẽ không còn đến. Cũng thế, khi tâm thức con người đã sáng tỏ giá trị của nền văn hóa giáo dục mới, thì các giá trị lỗi thời sẽ ra đi. Ngô Thừa Ân đưa ra hình ảnh Khe Ưng Sầu là xác định sự thật mà người làm công tác cải tổ văn hóa giáo dục phải chọn lựa, rằng: Văn hóa, giáo dục cần được cải tổ trước, cơ chế xã hội cải tổ sau. Công tác văn hóa giáo dục bao giờ cũng cần thời gian lâu dài. Với xã hội phong kiến Trung Hoa vào thế kỷ 16, khó có một chọn lựa nào khác tốt đẹp hơn.

— Cần tạo điều kiện thuận lợi cho nền văn hóa mới phát triển, các điều kiện sống có ảnh hưởng đến việc khơi dậy các nhận thức mới — tương tự như chiếc áo cà sa quý khơi dậy lòng tham nơi sư trụ trì Quan Âm Viện — như hạnh của Bồ Tát Quán Thế Âm thường cứu khổ người đời trước khi nói Pháp cứu khổ. Phương thức này sẽ hỗ trợ sự nghiệp giáo dục thuyết phục quần chúng quay về với các giá trị mới.
% section quan_niệm_về_xã_hội (end)
% chapter Hồi 13 và 14 (end)