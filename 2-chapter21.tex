\chapter{Hồi 53, 54 và 55} % (fold)
\label{cha:hoi_53_54}

Hồi 53:

\begin{itshape}
``Tam Tạng uống lầm mang nghén quỷ

Sa Tăng lấy nước phải thai ma''
\end{itshape}

Hồi 54:

\begin{itshape}
``Pháp tính sang Tây Qua nữ quốc

Tâm viên lập mẹo, thoát vòng tình''
\end{itshape}

Hồi 55:

\begin{itshape}
``Dâm tà bỡn cợt Đường Tam Tạng

Đứng đắn kiên trì chẳng hại thân''
\end{itshape}

\section{Tư tưởng Phật Học} % (fold)
\label{sec:53_54_phat_hoc}

— Về mặt đoạn trừ gốc ái, gốc của sinh tử luân hồi, người tu cần lần lượt đi từng bước chế ngự rồi loại trừ dục ái, hữu ái và vô hữu ái (như đã đề cập trong phần Tổng Luận). Công phu tu tập giải thoát đến hồi 53 này dù với Trư Bát Giới (với lòng đắm mê sắc dục) cũng đã ổn định tâm giải thoát, giác tỉnh trước sắc đẹp tuyệt trần của nữ chúa Tây Lương. Lòng Trư Bát Giới yên là tâm của phái đoàn Tây Du đã hoàn toàn ổn định về dục ái.

— Dù tâm ổn định về dục ái, thì cái gốc chủng tử dục ái chiêu tập qua nhiều đời sinh tử vẫn còn nằm sâu trong tâm thức người tu, nên cần phải trục cho sạch gốc rễ dục ái ấy. Ngô Thừa Ân diễn đạt tư tưởng Phật Học ấy bằng sự nhỡ lầm uống phải nước kết thai trên dòng Tử Mẫu của Đường Tăng và Bát Giới. Gốc của ái ấy chỉ có Định và Tuệ sâu mới trục được, như phải nhờ đến Tôn Ngộ Không và Ngộ Tịnh cùng đến am Tụ Tiên để lấy nước Giải Dương trục thai. Dương đã giải thì âm không còn tồn tại để quấy rối. Từ đây, phái đoàn Tây Du chỉ còn tập trung công phu vào việc cắt đứt hữu ái và vô hữu ái

— Kết quả cắt lìa dục ái đã được trắc nghiệm qua nạn ma nữ tiếp theo ở am Tỳ Bà. Đường Tăng biểu tượng của tâm giải thoát và lý tưởng giải thoát, đã đứng vững trước các thủ đoạn quyến dụ của yêu nữ có sắc nước hương trời với lòng xuân mênh mang như bể.

Bước giải thoát này thật đáng kể đối với hành giả sau khi vượt qua nhiều ma nạn!
% section tư_tưởng_phật_học (end)

\section{Quan niệm về con Người} % (fold)
\label{sec:53_54_con_nguoi}

— Trong nền văn hóa nhân bản, Vô Ngã, con người gắn liền với xã hội. Con người phục vụ cho mục tiêu hạnh phúc của bản thân đi đôi với lý tưởng phục vụ xã hội, mỗi thành viên xã hội cần được tôi luyện ý chí, tình cảm và trí tuệ như là sửa soạn cho một chiến sĩ lên đường vào nhiều mặt trận, nhiều chiến trường. Một trong các chiến trường phải chiến đấu là tình trường. Tình yêu là dung sắc nữ nhi thường làm yếu lòng chiến sĩ. Nước mắt nữ nhi có thể làm mềm gan tráng sĩ. Muốn thành nghiệp lớn, con người cần phải làm chủ tình cảm, nhất là trước các trang tuyệt sắc như nữ chúa Tây Lương hay nữ yêu động Tỳ Bà.

— Cần phân biệt rõ giữa tình cảm cao thượng và dục vọng, đam mê. Tình cảm nào nuôi dưỡng lý tưởng, an lạc, hạnh phúc của cá nhân và tập thể là tình cảm tốt đẹp; tình cảm nào mà nhận chìm lý tưởng phục vụ nhân quần, làm tối tăm trí tuệ là xấu xa. Tình cảm hướng về vị tha là tốt đẹp; tình cảm đi vào vị kỷ, hưởng thụ là thấp hèn. Nghĩa là sống một nhân cách cao thượng không phải đi tìm những cảm giác riêng tư, mà là mở hướng sống an vui cho cuộc đời. Đây cũng là một mặt trận trải dài từ nội tâm đến ngoại giới mãi làm máy động chiếc thiếc bổng của Tôn Hành Giả.
% section quan_niệm_về_con_người (end)

\section{Quan niệm về xã hội} % (fold)
\label{sec:53_54_xa_hoi}

— Nền văn hóa bất ổn cũ có bốn tập quán trói buộc quần chúng, đó là:
\begin{itemize}
    \item[–] tập quán nhận thức sai lầm,
    \item[–] tập quán của tâm hướng thiện mà sai lầm,
    \item[–] tập quán tưởng sai lầm,
    \item[–] tập quán tình cảm sai lầm.
\end{itemize}

Cho đến hồi 52, tác giả để đề cập đến ba tập quán đầu vốn được văn hóa truyền thống nuôi dưỡng. Nay, hồi 53, 54, và 55, tác giả bàn đến tập quán tình cảm. Mạnh nhất của tập quán tình cảm là tình yêu nam nữ. Tình yêu đó có thể mở ra một hướng tình cảm tốt đi vào nhân ái, vị tha, nhưng cũng có thể mở ra một cánh cửa vị kỷ. Nếu là tình cảm vị tha thì cao thượng có thể xây dựng văn hóa nhân bản và trí tuệ được biểu tượng bằng hình ảnh Đường Tăng quên mình vì hạnh phúc của nhân dân Đông Độ. Nếu là tình cảm vị kỷ, giới hạn giữa đôi nam nữ thì sẽ dễ dàng cuốn trôi các giá trị vị tha của nền văn hóa mới, bởi lẽ lòng vị kỷ sẽ nuôi dưỡng tập quán chấp thủ tự ngã vốn là động cơ xây dựng nền văn hóa nhị nguyên với các giá trị đầy ước lệ và đầy nước mắt.

Ngô Thừa Ân tại đây đã xây dựng hình ảnh Đường Tăng đứng vững trong nền văn hóa vị tha, tách khỏi nữ vương và nữ quái.

— Con người nếu biết từ bỏ tình yêu vị kỷ thì sẽ đi vào tình yêu vị tha đẹp đẽ và hạnh phúc hơn. Từ bỏ tình yêu vị kỷ không phải là một sự hy sinh hay một cách sống thiếu thiết thực, mà là chọn lựa cách sống khôn ngoan của trí tuệ Vô Ngã. Lòng vị tha càng lớn thì hạnh phúc càng chắp cánh bay cao và bay xa. Đây là một sự thật (không phải là nói triết lý huyền đàm) Ngô Thừa Ân muốn hậu thế đi vào thực nghiệm. Khát vọng muôn thuở của con người vẫn là tìm kiếm hạnh phúc và lẩn tránh khổ đau, nên nền văn hóa mới của trí tuệ hẳn là nhằm mở rộng phương trời hạnh phúc cho đến Lôi Âm Tự, cảnh giới của chân hạnh phúc.

Phái đoàn thỉnh kinh của Đường Tăng là phái đoàn mở đầu cuộc thử nghiệm ấy cho dân Đông Độ, và đã thành công. Đây quả là nền văn hóa và văn minh được chọn lựa cho Đông Độ. Phải chăng thế giới hiện đại sau bao nhiêu thử nghiêm đang hướng về một nền văn minh ``hậu hiện đại'' của an lạc, hòa bình và thịnh vượng chung của nhân loại, một nền văn minh không chia lìa thiên nhiên và con người, không chia lìa con người và xã hội, không chia lìa tình và ý, không chia lìa tâm và vật, và không chia lìa các màu da chủng tộc trên toàn trái đất? Phải chăng cuộc đại náo của Tôn Ngộ Không đánh thức con người đi vào cuộc sống ``tuy hai mà một'' đã được Đức Phật chỉ dạy và được Bồ Tát Quán Thế Âm hướng dẫn?
% section quan_niệm_về_xã_hội (end)
% chapter Hồi 53_54 (end)