\chapter{Các Hình Ảnh Biểu Tượng Khác Nhau Giới Thiệu Phật Học Trong Tây Du Ký} % (fold)
\label{cha:cac_hinh_anh_bieu_tuong_khac_nhau_gioi_thieu_phat_hoc_trong_tay_du_ky}

Đọc Tây Du Ký chúng ta bắt gặp rất nhiều hình ảnh biểu tượng và ngôn ngữ biểu tượng, biểu tượng hóa giáo lý Phật Giáo. Chúng ta sẽ có dịp thích thú chia sẻ với các hứng khởi sáng tác của tác giả.

Trong giới hạn của bài phiếm luận, người viết không có tham vọng trình bày xuyên suốt từ đầu đến cuối truyện các hình ảnh và ngôn ngữ biểu tượng ấy, mà chỉ điểm xuyết vài nét chấm phá gọi là để đáp lễ dư luận của đại chúng sau khi bộ phim Tây Du Ký được chiếu trên màn ảnh nhỏ của Đài Truyền Hình thành phố vừa qua.

\section{Về đôi mắt vàng của Tề Thiên Đại Thánh} % (fold)
\label{sec:ve_doi_mat_vang_cua_te_thien}

--- Mỹ Hầu Vương đặc biệt có đôi mắt vàng sáng chói, chiếu suốt qua các cung Trời làm Ngọc Hoàng rúng động, kinh ngạc. Đôi mắt vàng ấy phân biệt rõ chính tà, hư thật. Đôi mắt vàng ấy đã ràn rụa nước mắt trước cảnh đời vô thường, khổ đau, đi tìm đường học đạo bất sinh bất diệt từ Tôn giả Tu Bồ Đề tại một trú xứ xa xăm. Đôi mắt vàng ấy đã là linh hồn của cuộc hành trình thỉnh Kinh mà thiếu nó thì tức thời phái đoàn rơi vào ma nạn.

Đôi mắt vàng ấy là gì, nếu không phải là biểu tượng của trí tuệ Bát Nhã, của giáo lý trí tuệ Bát Nhã của Phật Giáo?

--- Như giáo lý Phật Giáo đã \emph{``dựng đứng dậy những gì bị quăng ngã xuống, mở ra những gì bị che kín''}, đôi mắt vàng của Mỹ Hầu Vương đã thấy và đã làm cho chúng ta thấy cái sự thật mộng mị, bất toàn, hư dối từ Âm Phủ đến Thiên Cung. Cuộc đại náo của Tề Thiên Đại Thánh tại Long Cung và Thiên Cung là sự đánh thức dậy sự thật ấy cho chúng sinh tại đó thấy rõ lối ra khỏi vô thường, không thật để đi vào nguồn giải thoát chân thật. Vì vậy chiếc thiết bổng nặng nghìn cân của Đại Thánh Tề Thiên, biểu hiện sức mạnh của đôi mắt vàng, là chiếc gậy đánh thức mà không phải nổi loạn, là xây dựng mà không phải đập phá. Chiếc gậy sắt ấy đập phá các nguyên nhân gây ra đau khổ cho cuộc đời và xây dựng an lạc, hạnh phúc của vô sinh. Chỉ đôi mắt vàng tuyệt với kia xuất hiện trong tiểu thuyết hay truyện phim là đủ để chúng ta đánh giá cao tiểu thuyết ấy, phim ấy.

Hầu như suốt thời gian theo dõi cuộc hành trình thỉnh Kinh, chúng ta đã bị cuốn hút bởi cái nhìn chính xác và bởi thái độ tự chủ trước các hiểm nạn và trước mọi cám dỗ của Tôn Hành Giả. Mỗi cái nhìn, mỗi bước đi của Hành Giả như vang lọng lời Kinh Bát Nhã:

{\bf ``\ldots ~Dĩ vô sở đắc, cố Bồ---đề---tát---đỏa y Bát Nhã Ba La Mật đa, cố tâm vô quái ngại, vô quái ngại cố, vô hữu khủng bố, viễn ly điên đảo mộng tưởng, cứu cánh Niết bàn''}.
% section về_đôi_mắt_vàng_của_tề_thiên_đại_thánh (end)

\section{Biểu tượng của hồi thứ 14} % (fold)
\label{sec:bieu_tuong_cua_hoi_thu_14}

Ở hồi thứ 14 này, sáu tên cướp đường là \emph{``mắt nhìn mừng''}, \emph{``tai nghe giận''}, \emph{``mũi ngửi thích''}, \emph{``lưỡi nếm nghĩ''}. \emph{``thân vốn lo''}, và \emph{``ý thấy muốn''} đón Đường Tăng và Tôn Hành Giả. Tôn Hành Giả đã dễ dàng đánh chết chúng, và đã thẳng thừng thưa sư phụ Đường Tăng: \emph{``Thưa sư phụ, đệ tử không đánh chết chúng thì chúng sẽ đánh chết sư phụ''}. Câu này có nghĩa là:

``Nếu đệ tử không đánh chết các tâm phiền não (mừng, giận, ưa thích, ham muốn, lo buồn, \ldots) khởi lên từ sáu căn thì đệ tử sẽ đánh mất lý tưởng giải thoát độ sinh''.

Sáu tên cướp đường kia là biểu tượng cho \emph{sắc, thanh, hương, vị, xúc và pháp}. Nếu thiếu giác tỉnh chế ngự chúng, thì chúng sẽ đột nhập sáu căn (tai, mắt, mũi, lưỡi, thân và ý) gây ra giặc phiền não, khổ đau. Đánh chết chúng là biểu hiện mạnh mẽ quyết tâm giải thoát và nhiếp được phần thô động của \emph{thân hành, khẩu hành và ý hành}. Đây là công phu giải thoát bước đầu vậy.
% section biểu_tượng_của_hồi_thứ_14 (end)

\section{Biểu tượng của hồi thứ 26} % (fold)
\label{sec:bieu_tuong_cua_hoi_thu_26}

Tại Ngũ Trang Quán, trú xứ của vị đại Địa tiên Trấn Nguyên Tử, Tôn Hành Giả ăn trộm ba quả Nhân Sâm và nổi cơn thịnh nộ đánh bật gốc rễ cây Nhân Sâm vô cùng quý báu của người. Cây Nhân Sâm là loại kinh căn có mặt từ khi thiên địa mới khai tích, quý nhất trong vườn cây của vị đại Địa tiên. Vì thế, Trấn Nguyên Tử, với thần thông vô lượng, đã bắt giữ trọn phái đoàn Tây Du. Tôn Hành Giả phải nhiều phen chiến đấu vất vả, rồi đến Đông Hải cầu Bồ Tát Quán Thế Âm dùng Cam Lồ Thủy hồi sinh cây Nhân Sâm mới thoát nạn.

Tại đây, Ngũ Trang Quán là biểu tượng của Ngũ Hành, lúc thiên đại khai tích, là biểu tượng lúc Âm Dương mới tượng. Âm Dương phối hợp với Ngũ Hành, theo đạo Nho, sinh ra vạn hữu. Cây Nhân Sâm vì thế là biểu tượng cho nguồn gốc của các pháp hữu vi. Thế là, ở đây Tôn Hành Giả, với trí tuệ Vô Ngã của mình, đã có khả năng làm bật gốc các pháp hữu vi (\emph{Tam Giới}), nhưng vì tập khí sinh tử còn nhiều nên thân tâm còn phải chịu hệ lụy trong sinh tử, như phái đoàn Tây Du đang bị giam giữ tại Ngũ Trang Quán.

Nước Cam Lồ của Bồ Tát Quán Thế Âm là biểu tượng cho nước phạm hạnh lấy từ trí tuệ Bát Nhã Ba La Mật của Bồ Tát, là thứ nước dập tắt \emph{lửa ái, lửa thủ, vô minh}, là thứ nước của vô sinh, siêu vượt sinh diệt. Do vì siêu vượt sinh diệt nên nó thiết lập được sinh diệt, làm hồi sinh được cây Nhân Sâm.

Pháp Phật cao là ở chỗ này. Pháp Phật thoát ly sinh tử là ở chỗ này.

Xây dựng cảnh nạn Ngũ Trang Quán là Ngô Thừa Ân muốn giới thiệu nét giáo lý đặc thù vô tỉ của Phật Giáo và vừa chỉ đường cho hành giả phá đổ tâm sinh diệt của các ngã tưởng (\emph{ngã tưởng, nhơn tưởng, chúng sinh tưởng, thọ giả tưởng, pháp tưởng, phi pháp tưởng, tưởng và phi tưởng}).
% section biểu_tượng_của_hồi_thứ_26 (end)

\section{Biểu tượng của hồi thứ 27} % (fold)
\label{sec:bieu_tuong_cua_hoi_thu_27}

Tại đây, nữ yêu tinh Bạch Cốt ba lần quyết hại Đường Tăng, ba lần hóa hiện dân lành để đánh lừa lòng từ bi của Đường Tăng, nhưng cả ba lần đều bị Tôn Hành Giả phát hiện đánh chết. Sự kiện này khiến Đường Tăng phẫn nộ Hành Giả. Bát Giới lại xúc siểm với Đường Tăng, bảo rằng Tôn Hành Giả đã ác ý giết chết ba người lương thiện mà thưa dối với Đường Tăng là ba con quỷ. Đường Tăng quyết định đuổi Tôn Hành Giả về Hoa Quả Sơn. Đọc truyện và xem phim đến hồi này ai cũng xúc động đến chảy nước mắt, cũng thở ngắn than dài: Ôi! Thật là bi thương! Thật là bi kịch!

Hậu quả của bi kịch trên do Đường Tăng mê mờ không thấy rõ hư, thật, và do lòng ganh ghét, đố kỵ, xúc siểm của người sư đệ còn đầy vô minh, Trư Bát Giới.

Đây là chỗ diễn đạt tài tình của Ngô Thừa Ân và diễn xuất tuyệt vời của phim ảnh.

--- Nữ ma Bạch Cốt là biểu tượng của các vọng tâm sinh khởi từ \emph{tham, sân, si} mà giáo lý có khi chỉ nói vắn tắt là ái tâm.

Lần thứ ba, Tôn Hành Giả đánh chết Bạch Cốt Tinh là biểu tượng cho việc tu tập đoạn trừ vọng tâm phải thực hiện nhiều lần (ở Phạn Văn, số ba mới là số nhiều). Lại nữa, trong thiền quán, hành giả chỉ \emph{thấy và đoạn vọng tâm khi nó diệt, chưa đủ; thấy và đoạn vọng tâm khi nó tồn tại, cũng chưa ổn; khi cần phải thấy và đoạn vọng tâm khi nó khởi nữa}, cho đến lúc này vọng tâm mới thật sự được diệt trừ. Đây là ý nghĩa rất chuyên môn về Phật Học (Thiền Định Phật Giáo).

Dựng nên cảnh bi kịch này, Ngô Thừa Ân muốn nói lên một sự thật giáo lý Phật Giáo rằng: nếu hành giả tu tập Giới và tu tập từ tâm mà thiếu trí tuệ giải thoát thì công phu tu tập giải thoát chỉ là cái xác sống không hồn, ảm đạm, vô cùng ảm đạm. Không có trí tuệ Vô Ngã thì sẽ không có giải thoát và không có đạo Phật.

Tại đây, chúng ta đâu dám nghĩ rằng tiểu thuyết Tây Du Ký của Ngô Thừa Ân đầy hư cấu là thần thoại, nhảm nhí?

% section biểu_tượng_của_hồi_thứ_27 (end)

\section{Biểu tượng về Hồng Hài Nhi, La Sát và Ngưu Ma Vương (Hồi 42, 60 và 61)} % (fold)
\label{sec:bieu_tuong_ve_hong_hai_nhi_va_nguu_ma_vuong}

Theo dõi hành trình của phái đoàn Tây Du, chúng ta thấy rằng cứ mỗi lần vượt qua được một ma nạn, là mỗi lần phái đoàn tiến thêm một bước gần giải thoát sinh tử; mỗi lần kẹt vào một ma nạn là mỗi lần phát hiện ra chỗ ngưng trệ của tâm thức giải thoát của phái đoàn, cũng là chỗ ngưng trệ tâm thức của hành giả tu tập.

--- Hồng Hài Nhi đã bắt giam Đường Tăng, Ngộ Năng, Ngộ Tịnh và Tiểu Long Mã. Ngộ Không cũng chiến bại trước vòng xe lửa của Hồng Hài Nhi. Vòng xe lửa ấy biểu tượng của \emph{lửa tham và lửa sân}, thế lực mạnh nhất của ma giáo. Ngộ Không thì không tham, có trí tuệ, nhưng còn cái động của sân nên đã bị vòng lửa của Hồng Hài Nhi đốt sém, mà không thể dập tắt được vòng xe lửa, dù đã vận dụng nhiều thứ thần thông. Ngộ Không đã phải cầu viện Bồ Tát Quán Thế Âm thu phục Hồng Hài Nhi mới thoát nạn.

Ngang đây, Ngô Thừa Ân đã giới thiệu với người học Phật và hành giải thoát rằng: cần phải vận dụng thiền quán sâu xa về trí tuệ Vô Ngã thì hành giả mới dập tắt được tham, sân, si và quyến thuộc của chúng. Cái ngưng trệ của giải thoát tại đây là do hành giả hành Giới chưa thuần (ý nghĩa Bát Giới bị trói), Định chưa vững (ý nghĩa Ngộ Tịnh bị giam) và Tuệ chưa ổn (ý nghĩa Ngộ Không suýt bị đốt). Từ đây hành giả cần tu tập phát triển mạnh thêm Giới, Định, và Tuệ.

--- Ở hồi mượn quạt Ba Tiêu để dập tắt Hỏa Diệm Sơn trên đường Tây Du, Ngộ Không bị bà La-Sát phất một quạt bị đẩy đi xa mười vạn dặm. Sau nhờ Bồ Tát Văn Thù cho uống ``Định Phong Đơn'' mới vô hiệu hóa tác dụng của quạt Ba Tiêu, mới vận dụng được kế sách chế ngự bà La-Sát.

Quạt Ba Tiêu là tượng trưng bát phong ở đời (\emph{Khen, Chê, Mừng, Giận, Danh vọng, Lợi dưỡng, Được, Nất}). Tám thứ ấy thường làm giao động tâm thức người tu, nếu thiếu giác tỉnh và thiếu định lực. Vì vậy khi uống vào ``Định Phong Đơn'' nghĩa là khi tâm định đã vững, thì hành giả thoát được nạn quạt Ba Tiêu, hàng phục được nhiều thứ vọng tâm.

Qua được nạn này là phái đoàn Tây Du đi được nữa đoạn đường giải thoát về Tây Trúc, tự tại được trước các thị phi, bỉ, thử, danh vọng và lợi dưỡng ở đời. Thành quả giải thoát này thật là đáng kể!

--- Cũng ở hồi này (60 và 61), sách lược chiến đấu của Tôn Hành Giả là đánh thẳng vào đầu não của ma quân, chui ngay vào bụng La-Sát mà quậy phá. La-Sát chỉ còn một cách chọn lựa: đầu hàng. Đây là ý nghĩa thiện xảo của các pháp tu của đạo Phật, luôn luôn loại trừ vọng niệm từ gốc rễ, từ đầu nguồn.

--- Đối với nhân vật Ngưu Ma Vương, Tôn Hành Giả đã hóa ra Ngưu Ma Vương giả để gạt La-Sát lấy quạt Ba Tiêu, nhưng chính Ngưu Ma Vương đã tương kế tựu kế biến hóa ra Ngộ Năng giả để lấy lại quạt Ba Tiêu từ tay Hành Giả. Hành Giả đã phải buông lời than: \emph{``mình là người bắt rận thiện xảo lại xui xẻo để bị rắn cắn''}.

Đây là biểu tượng của pháp tu hành {\bf ``Tùy Quán''} của thiền định. Hành pháp này hành giả cần cẩn trọng giữ chánh niệm, hễ thất niệm thì rơi vào vọng tâm và bị trói buộc vào vọng tâm. Cần phải luôn luôn biết rõ tâm mình đang ở đâu và nó là gì (\emph{quyến thuộc của tham, sân, si hay vô tham, vô sân, vô si}).
% section biểu_tượng_về_hồng_hài_nhi_la_sát_và_ngưu_ma_vương_hồi_42_60_và_61_ (end)

\section{Biểu tượng của hồi thứ 54, 64 và 72} % (fold)
\label{sec:bieu_tuong_hoi_54_64_va_72}

Một nhà quân sự, khi đánh một đồn, bốt địch thì thường phải có kế sách: công đồn, chặn viện. Sau khi chặn viện xong thì công hãm đồn, nếu không thì sẽ lưỡng đầu thọ địch. Cũng vậy, hành giả trên đường giải thoát cần phải dẹp giặc trong và đánh giặc ngoài. Khi hàng phục được giặc {\bf ``Lục Tặc''} và {\bf ``Bát Phong''} ở ngoài thì hành giả tập trung vào trong để diệt giặc nội. Đấy là công việc chế phục, loại trừ \emph{dục ái, hữu ái và vô hữu ái}. Về dục ái, thì có thể dùng định tâm để nhiếp, nhưng còn hữu ái và vô hữu ái, thì phải dùng trí tuệ để trừ. Chúng ta hãy tiếp tục theo dõi cuộc hành trình Tây Du.

--- Hồi thứ 54 diễn ra cảnh nữ vương nước Tây Lương thiết tha yêu Đường Tăng. Đường Tăng đã hoàn toàn tự chủ trước mối tình đẹp và lạ này. Đây là biểu tượng Đường Tăng chế phục được dục ái.

--- Hồi thứ 64 xẩy ra cuộc tình thơ mộng giữa thơ, nhạc và tiên tửu của Hạnh Tiên. Mối tình nhẹ nhàng như tình ở Thiên giới, nhưng hiểm họa cũng khốc liệt. Tại đây Đường Tăng vẫn tỉnh táo, nhưng xem ra thế tự vệ đã có chiều nguy hiểm, sinh tử thấy gần kề. May nhờ Tôn Hành Giả cứu giá. Đây là ý nghĩa biểu tượng: chỉ có trí tuệ Vô Ngã ở trong thiền định mới cắt đứt được hữu ái. Am Mộc Tiên (ở hồi 64 này) quả là nơi đáng nhớ của Đường Tăng!

--- Hồi thứ 72 xẩy ra ở động Bà Ty. Tại đây Đường Tăng một mình tự dấn thân vào sào huyệt của bảy tiên cô nhền nhện tuyệt đẹp, những tiên nương thường tắm gội ở suối trời Trạc Cấu (suối rửa sạch các cấu bẩn của dục ái). Các tiên nương này không nghĩ gì đến chuyện mây mưa mà chỉ muốn ăn thịt Đường Tăng.

Bảy yêu nhền nhện là biểu tượng cái tinh tế của ái, đó là vô hữu ái. Vì thế, chúng chiến thắng được Ngộ Năng, nhưng bị tiêu diệt bởi Hành Giả.

Bảy yêu nhền nhện cũng là biểu tượng của \emph{thất tình} (\emph{hỷ, nộ, ai, lạc, ái, ố, dục}). Với thất tình, công phu của Giới không thể chiết phục (nên Bát Giới bị trói) mà chỉ có trí tuệ loại trừ.

Bảy yêu nhền nhện lại dựa vào sức mạnh của người sư huynh Bách Nhỡn ma quân mà tác quái. Bách Nhỡn ma quân có tà pháp phun ra sức nóng và khói mù che kín cả trời đấy khiến Ngộ Không thúc thủ. Ngộ Không phải cầu cứu Bồ Tát Tỳ Lam dùng Độc Kim luyện từ ánh sáng mặt trời để chế trị Bách Nhỡn ma quân.

Bách Nhỡn ma quân vì thế là ý nghĩa biểu tượng của \emph{vô minh} (si mê, phóng ra khói mù chấp ngã và lửa dục) chỉ chịu khuất phục trước trí tuệ thuần phục của Tỳ Lam Bồ Tát (biểu tượng bằng cây kim luyện từ mặt trời).

Đây là hiểm nạn khó qua của Tôn Hành Giả.
% section biểu_tượng_của_hồi_thứ_54_64_và_72 (end)

\section{Biểu tượng của hồi thứ 58} % (fold)
\label{sec:bieu_tuong_hoi_58}

\begin{itshape}
``Hai lòng xáo trộn cả Càn Khôn.

Một thế khó tu thành tịch diệt.''
\end{itshape}

Cái cảnh hai Tôn Hành Giả xuất hiện trong ngôn ngữ và hành động giống hệt nhau gây xáo trộn cả trời đất, thiên đình và Bồ Tát đều không phân biệt được, chỉ có Như Lai mới vạch rõ chân tướng của Tôn Hành Giả giả.

{\bf Đây là một cảnh diễn đạt về một cảnh tâm rất tế nhị.}

Ngô Thừa Ân tại đây đang giới thiệu rằng chân tướng của các pháp chỉ có thể đạt được bằng thật trí, Phật trí và sự thật luôn luôn ở ngoài các ngã tướng, như Kinh Kim Cang dạy: \emph{``Phàm cái gì là ngã tướng thì hư vọng''} (Phàm sở hữu tướng gia thị hư vọng). Hệt như trường hợp trước hai chú tiểu cuốn sáo, một thiền sư kêu lên: \emph{``Một được, một mất''}.

Hồi thứ 58 này cũng nói lên rằng giá trị chân thật của con người cũng ở ngoài các tướng trạng, rời khỏi tướng \emph{``nhị thủ''} (chấp có năng và sở). Chấp vào trí tuệ Vô Ngã cũng là chấp thủ, sẽ ngăn ngại giải thoát, như phái đoàn Tây Du khó xử trước hai Tôn Hành Giả vậy.
% section biểu_tượng_của_hồi_thứ_58 (end)

\section{Kế sách đối trị ác ma} % (fold)
\label{sec:ke_sach_doi_tri_ac_ma}

Dọc cuộc hành trình Tây Du, Tôn Hành Giả phải đương đầu với nhiều loại ác ma, đương đầu với nhiều loại vũ khí tàn độc, Tôn Hành Giả vận dụng đủ 72 phép thần thông dựa vào bốn kế sách chính:

\begin{enumerate}[label=\itshape\arabic*\upshape/]
    \item Kế sách khắc chế: Dùng khắc chế để đối trị như dùng nước để dập tắt lửa, dùng gà để diệt rết.

    \item Vào tận hang cọp để bắt cọp con: Tôn Hành Giả thường chui ngay vào bụng ác ma để quấy phá.

    \item Tận diệt ác ma để trừ hậu hoạn.

    \item Khi sử dụng thần thông thất bại thì cầu viện Như Lai, Bồ Tát (đây là ý nghĩa vận dụng trí tuệ Bát Nhã thâm sâu).
\end{enumerate}

Bốn kế sách trên cũng là bốn phương pháp mà người tu sử dụng để loại trừ các vọng niệm.
% section kế_sách_đối_trị_ác_ma (end)

\section{Về đạo đức Phật Giáo} % (fold)
\label{sec:ve_dao_duc_Phat_Giao}

--- Suốt tập truyện Tây Du Ký, bên cạnh các giáo lý Phật Giáo vừa được đề cập ở trên, Ngô Thừa Ân thường nói đến Nhân Quả, nghiệp báo và khuyến khích hành thiện như trong cảnh vua Đường du địa phủ, trong hồi 71, vua Chu Tử bị nạn, trong hồi 87 nói về nạn hạn hán, trong hồi 94 về chuyện công chúa bị yêu ma đánh tráo, vv\ldots

--- Tác giả, dưới hình thức nhận diện bổn sinh, thường vạch rõ chân tướng của các yêu, ma, quỷ, quái đều là gốc thú vật cả, là những gì gớm tởm cần phải tránh xa.
% section về_đạo_đức_phật_giáo (end)

\section{Thêm vài điểm phiếm luận.} % (fold)
\label{sec:them_vai_diem_phiem_luan}

--- Có ý kiến đánh giá rất thấp tiểu thuyết Tây Du Ký chỉ vì tác giả Ngô Thừa Ân xây dựng các nhân vật hoàn toàn là huyền hoặc hư cấu.

Người viết bài phiếm luận này thì đánh giá ngược lại. Không phải vì tính huyền hoặc và hư cấu mà đánh giá thấp cuốn truyện, bởi vì chính huyền hoặc và hư cấu là chất liệu của sáng tác văn học, tiểu thuyết. Điều quan trọng là tìm xem cuốn tiểu thuyết có nói lên được những giá trị sống nào không? Câu chuyện lịch sử Việt Nam về Lạc Long Quân và Âu Cơ, câu chuyện đức Thánh Gióng phá giặc Ân, vv\ldots ~không phải là đầy hư cấu, thần thoại đó sao? Các tập Việt Điện U Linh Tập và Lĩnh Nam Chích Quái của văn học Việt Nam cũng cùng tính chất hư cấu và thần thoại ấy, ai dám bảo rằng không có giá trị?

--- Có ý kiến cho rằng Tây Du Ký là chuyện hoang tưởng, cấu trúc rối rắm, khó nhận ra những gì tác giả muốn nói.

Như chúng ta điều biết, mỗi loại ngôn ngữ chuyên chở một ý nghĩa riêng và một cách diễn đạt riêng. Ngôn ngữ khoa học thì khác ngôn ngữ ngoại giao, khác ngôn ngữ thi ca, và hiển nhiên là khác với ngôn ngữ tiểu thuyết. Ngôn ngữ tiểu thuyết cũng có nhiều loại khác nhau, ngôn ngữ tiểu tuyết giả tưởng thì khác với ngôn ngữ tiểu thuyết hiện thực; ngôn ngữ tiểu thuyết Quỳnh Giao thì khác với ngôn ngữ tiểu thuyết kiếm hiệp của Kim Dung, và khác với ngôn ngữ biểu tượng của Tây Du Ký.

Chính dựa vào ngôn ngữ và hình ảnh biểu tượng mang nhiều ý nghĩa, Ngô Thừa Ân mới cùng lúc nói lên được nhiều ý nghĩa tôn giáo, xã hội và con người. Đây là thứ ngôn ngữ phong phú ý tưởng làm cho cuốn tiểu thuyết trở nên hấp dẫn hơn. Cũng chính từ thứ ngôn ngữ này mà dư luận đã dấy lên nhiều cuộc trao đổi, tranh cãi sôi nổi về Tây Du Ký trong mấy tháng qua, và có thể kéo dài trong nhiều tháng đến.

Mỗi người có một cái nhìn riêng về Tây Du Ký, và biến hóa bộ truyện Tây Du Ký của Ngô Thừa Ân thành Tây Du Ký của mình, như là Tôn Ngộ Không phù phép thổi lông biến hóa ra vô số Tôn Ngộ Không vậy. Nói theo thuật ngữ của Phật Giáo, hiện nay có đến tám vạn bốn nghìn ý kiến khác nhau về Tây Du Ký.
% section thêm_vài_điểm_phiếm_luận_ (end)
% chapter các_hình_ảnh_biểu_tượng_khác_nhau_giới_thiệu_phật_học_trong_tây_du_ký (end)