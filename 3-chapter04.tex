\chapter{Triết lý Giáo Dục} % (fold)
\label{cha:triet_ly_giao_duc}

Một đường hướng giáo dục, ngoài lý thuyết về nhân tính, còn có triết lý giáo dục của nó để từ đó hình thành một mẫu người giáo dục và nội dung giáo dục.

Có ba vấn đề chủ yếu hình thành một đường hướng triết lý nói chung và triết lý giáo dục nói riêng đó là \emph{vấn đề bản thể, vấn đề nhận thức và vấn đề giá trị} cần được bàn đến đối với giáo lý Phật Giáo.

\section{Vấn đề bản thể (Ontology)} % (fold)
\label{sec:van_de_ban_the}

Hệ tư tưởng Âu, Mỹ chịu ảnh hưởng của tư duy nhị nguyên (tư duy ngã tính) đã nêu ra vấn đề bản thể khi đi tìm hiểu sự thật của con người và cuộc đời. Đặt ra vấn đề bản thể là đặt vấn đề nguồn gốc của sự vật, hay vấn đề bản chất, hoặc bản tính của sự vật, hoặc vấn đề nguồn gốc hay nguyên nhân đầu tiên.

Sự vật được tìm hiểu ở đây trở thành đối tượng của tìm hiểu mà chủ thể tìm hiểu của nó là tư duy của con người hoạt động qua các giác quan giới hạn, sự ghi nhận của các giác quan về các hiện hữu vốn giới hạn, khó phản ảnh được sự thật của các hiện hữu. Tư duy thì hoạt động qua các nguồn tin tức mà các giác quan cung cấp, cộng thêm với khả năng phối kiểm tổng hợp đầy ngã tính. Thế nên, dưới cái nhìn trí tuệ như thật của Phật Giáo, sự thật của các hiện hữu không có mặt trong các câu hỏi và trong các câu trả lời của hệ thống (cơ cấu) tư duy ấy. Nói khác đi, qua sự thật Duyên Khởi đã được Đức Phật chứng ngộ dưới cội bồ đề, thì vấn đề bản thể, bản chất hay nguồn gốc theo nghĩa \emph{``ontology''} ấy không phù hợp với giáo lý nhà Phật. Nó thuộc tư duy nhị nguyên, mà không phải của thực tại. Phật Giáo phủ nhận các câu hỏi về bản thể, bản chất, nguồn gốc hay nguyên nhân đầu tiên ấy. Câu hỏi đã đi xa khỏi thực tại, thì câu trả lời cho câu hỏi đó càng đi xa hơn khỏi thực tại.

Vì phủ nhận nguyên nhân đầu tiên của các hiện hữu, nên giáo lý Duyên Khởi phủ nhận các quan điểm cho rằng có đấng sáng tạo vũ trụ, đấng Thượng đế hay Phạm thiên tạo ra loài người.

Vì là Duyên Khởi, Vô Ngã, nên giáo lý Phật Giáo không chấp nhận, không chủ trương duy tâm, không chủ trương duy vật, duy linh, vv. Vũ trụ và con người là do duyên sinh. Tâm, Vật, Linh, vv\ldots ~là do duyên sinh.

\hrulefill

{\bf Với Phật Giáo, vũ trụ và vạn hữu đang là, là duyên sinh:}


{\bf Do cái này có mặt, nên cái kia có.

Do cái này sinh, nên cái kia sinh.

Do cái này không có mặt, nên cái kia không có,

Do cái này diệt, nên cái kia diệt.}

\hrulefill

Do duyên mà sinh nên mọi hiện hữu dù là ý niệm như lông rùa, sừng thỏ, hay thái dương hệ, đều rỗng không tự ngã và được gọi là Vô Ngã, Vô Tánh hay Không Tánh.

Nhưng, vũ trụ, vạn hữu đến với mỗi người thì không còn như chúng hiện hữu, đã bị méo mó đi nhiều do sự can thiệp của các giác quan con người, tư duy tình cảm và các định kiến khác. Chỉ khi nào loại ra khỏi cái nhìn của con người các nhân tố can thiệp ấy thì vũ trụ vạn hữu mới xuất hiện như thật, như chúng đang là.

Cái thấy biết ngã tính của con người như vậy là sai lầm (sai với thực tại) và được giáo lý nhà Phật gọi là \emph{điên đảo kiến}.

Cái tâm của con người hoạt động, tác ý trên các ngã tướng là không thật và được gọi là \emph{điên đảo tâm}.

Cái niệm tưởng của con người là ngã tính và niệm tưởng về các ngã tướng gọi là ngã tưởng nó sai lầm và được gọi là \emph{điên đảo tưởng}.

Thói quen tình cảm của con người sống với các ngã tưởng, ngã tướng, sống với các dục niệm khiến trở ngại Chánh Kiến, Chánh Niệm nên được gọi là \emph{điên đảo tình}.

Phật Giáo với vai trò giáo dục chỉ rõ cho người đời thấy rõ bốn thứ điên đảo ấy và tu tập tâm để thoát ly chúng. Khi bốn thứ điên đảo ấy được nhiếp tịnh, thì vũ trụ và con người chân thật xuất hiện. Vấn đề giáo dục ở đây không phải giới thiệu vũ trụ và con người chân thật là gì, mà là giới thiệu những gì đang ngăn che tâm con người và huấn luyện tâm để làm sụp đổ các ngăn che ấy. Khi các ngăn che sụp đổ, thì các thật tướng xuất hiện. Con người sống đúng nghĩa là sống với thực tại, sống với sự thật ở ngoài các ngăn che.

% section vấn_đề_bản_thể_ontology_ (end)

\section{Vấn đề nhận thức (Epistemology)} % (fold)
\label{sec:van_de_nhan_thuc}

Tư duy con người hoạt động dựa vào các tin tức được cung cấp từ các giáo quan tai, mắt, mũi, lưỡi, thân và ý căn, dựa vào kinh nghiệm của giác quan và dựa vào kinh nghiệm tư duy gán cho sự vật. Ngã tính ấy là hệ quả của ba nguyên lý căn bản làm nền tảng để tư duy có thể hoạt động.

{\bf Ba nguyên lý đó là:}

\begin{enumerate}[label=\itshape\arabic*\upshape/]
    \item Nguyên lý đồng nhất: Một vật được gọi là A thì mãi là A như thế thì tư duy mới làm việc được.

    \item Nguyên lý cấm mâu thuẫn: Một vật được gọi là A hoặc được gọi là B, chứ không thể khi thì A khi thì B. Ổn định tên gọi như vậy thì tư duy mới có thể vận động.

    \item Nguyên lý triệt tam: Một vật có thể được gọi là A hay B, chứ không thể vừa là A vừa là B. Có vậy tư duy mới có thể hoạt động.
\end{enumerate}

Ba nguyên lý trên hàm ý rằng (hay giả định rằng) mọi hiện hữu đều phải có tên gọi riêng và có ngã tính bất biến. Thế là yêu cầu để tư duy có thể hoạt động là gán cho mỗi hiện hữu một ngã tính. Vậy nên tư duy này được gọi là tư duy hữu ngã, tư duy ngã tính hay tư duy nhị nguyên (chủ thể nhìn và đối tượng nhìn đều có ngã tính độc lập).

Trong khi đó, thực tại thì trôi chảy liên tục, bạn không thể đặt chân hai lần trên một dòng nước. Như thế hoạt động tư duy của con người thì khác với thực tại. Đây là điểm đầu nguồn quyết định thân phận con người, luôn luôn đặt con người ngoài thực tại, mâu thuẫn với thực tại để làm dấy sinh vô số khổ đau, phiền não, tham, sân, si, dục vọng và các phiên não đều dấy khởi từ tư duy hữu ngã ấy.

Các căn thì được cấu tạo có điều kiện, khả năng đón nhận sự vật rất hạn chế, đã cung cấp tin tức cho tư duy rất giới hạn, có khi sai lạc hẳn. Đây là một điểm khác nói lên giới hạn và sai lầm của tư duy.

Nhận thức, hiểu biết của con người là đến từ tư duy và giác quan giới hạn ấy, nên dẫn con người đến các thấy biết hạn hẹp, sai lầm, che mờ sự thật, sự thật của vạn hữu thực tại. Giáo dục có vai trò chỉ cho con người thấy cái giới hạn của nhận thức, và vạch mở con đường giúp con người thấy sự thật và thấy hạnh phúc tại trần gian.

Phật Giáo rất tuyệt vời trong việc đảm nhận vai trò giáo dục này. Đức Phật dạy có năm cấp độ thấy biết thực tại:

\begin{enumerate}[label=\itshape\arabic*\upshape/]
    \item Tưởng tri (\emph{Sannàjànati}): khả năng phân biệt các sự vật hiện hữu, khả năng kinh nghiệm, ghi nhớ và hồi tưởng.

    \item Thức tri (\emph{Vinnàjànati}): khả năng ghi nhận sự hiện diện của các pháp qua mắt, tai, mũi, lưỡi, thân và ý căn.

    \item Tư duy (\emph{Mannàjànati}): khả năng nối kết, phân tích tổng hợp, phối kiểm và suy luận lên các dữ kiện, tin tức do các giác quan cung cấp.

    \item Thắng tri (\emph{Abhijànati}): là cái thấy biết sự vật trực tiếp bằng sự quán sát của tâm thức trong thiền định (không có mặt tư duy). Cái thấy biết của thắng tri tương tự như trực giác triết học của phương Tây. Tại đây, con người thấy sự vật như thật đang trôi chảy.

    \item Liễu tri (\emph{Parijanàti}): Tiếp nối cái thấy biết của thắng tri cho đến khi thật nhuần nhuyễn, khi mà tâm tham, sân, si đều rụng đổ hết, thì cái thấy biết bấy giờ gọi là liễu tri. Liễu tri mới là cái thấy biết sau cùng, như thật, như chân mọi hiện hữu.
\end{enumerate}

Ba cấp độ thấy biết đầu là thuộc nhận thức, tri thức và các tri kiến của con người và của thế giới học đường cũ. Học đường ngày nay, hay nền văn hóa \emph{``hậu hiện đại''} cần giáo dục con người đi vào hiểu biết thắng tri và liễu tri, và thực hiện thắng tri, liễu tri. Sự thật và hạnh phúc chân thật có mặt trong thắng tri và liễu tri này.

Học đường cũng cần vạch rõ: mỗi loại tri thức đem lại cho con người một loại hiểu biết và được vận dụng vào một lãnh vực xã hội tương ứng. Tất cả đều cần thiết nếu con người không chấp thủ chúng.
% section vấn_đề_nhận_thức_epistemology_ (end)

\section{Vấn đề giá trị, đạo đức (Axiology)} % (fold)
\label{sec:van_de_gia_tri_dao_duc}

Nói đến giá trị là nói đến sự đánh giá. Chủ thể của sự đánh giá là nhận thức, tư duy của con người. Như được trình bày ở mục nhận thức trên, nhận thức và tư duy đã là giới hạn và sai lầm nên các giá trị, vấn đề thiện, ác, đạo đức do tư duy ngã tính đặt để cần được xét lại. Giá trị của một hành động là do nhiều nhân duyên kết hợp mà thành gồm sự chủ tâm tác ý, ý chí thực hiện, tình cảm thực hiện và cả kết quả của hành động, nếu chỉ căn cứ vào cái tướng biểu hiện ở bên ngoài mà đánh giá thì rất thiếu sót, dễ sai lầm.

Ví dụ nhưng một bà mẹ diễn đạt sự rất âu yếm người con trẻ mà nghiến răng nói rằng: ``ta ghét mầy quá!'' (nhưng nói với tấm lòng rất đỗi là âu yếm). Nếu căn cứ vào cái tướng và lời diễn đạt kia thì quả khó thấy được giá trị của hành động ấy (ở đây sự âu yếm là động cơ của hành động). Hai người cùng làm một việc giống nhau mà giá trị lại khác nhau. Ví như hai người cùng giúp đỡ một người nghèo với một số tiền bằng nhau, nhưng một người giúp vì tình thương, một người giúp vì có ý đồ lợi dụng; như vậy một người là thiện lương, người kia là không thiện lương.

Cùng một con người làm cùng một hành động ở hai hoàn cảnh hay thời điểm khác nhau thì có giá trị khác nhau. Chẳng hạn ăn cơm đúng bữa là tốt, ăn cơm lúc khỏe là bình thường, chứ ăn cơm lúc đau thương hàn thì có thể dẫn đến cái chết, vv\ldots

Do đó, không thể có một bảng liệt kê giá trị ấn định sẵn các giá trị của các hành động con người. Người ta cũng không dễ dàng phê phán đúng hành động của những người khác khi không biết rõ động cơ hành động và tâm tác động lên hành động của người đó. Vì thế, giá trị nhân bản là mỗi người tự biết mình, đánh giá mình, và rất cẩn trọng khi đánh giá việc làm của những người khác.

Giá trị nhân bản là hành động nhằm đem lại an lạc hạnh phúc cho số đông trong hiện tại và tương lai vì chính con người, mà không phải vì nhân danh một quyền lực hay một đấng tối cao nào khác. Tiêu chuẩn giá trị phải là con người và hạnh phúc của con người. Giá trị chính là đạo đức, và đạo đức chính là hạnh phúc.

Ở đâu có đạo đức, ở đó có mặt hạnh phúc; ở đâu có hạnh phúc ở đó có mặt đạo đức. Vì thế, đạo đức và hạnh phúc còn đồng nghĩa với vị tha và từ bi (hay lòng nhân ái); và đi xa hơn, đạo đức và hạnh phúc còn đồng nghĩa với sự chế ngự chấp thủ bản ngã, với sự chế ngự dục vọng; dục vọng càng ít hạnh phúc càng nhiều. Đây là loại giá trị của thực tại đòi hỏi con người thực nghiệm để nhận chân giá trị.

Vấn đề giá trị của nhận thức, vấn đề tư duy và đánh giá các giá trị, hay vấn đề giá trị nói chung, đã được Ngô Thừa Ân đưa vào nhiều hồi truyện Tây Du Ký và đã được người viết bàn đến nhiều. Hẳn đây là một vấn đề ách yếu của nội dung của nền giáo dục nhân bản và trí tuệ.

Những công thức giá trị mang vẽ ước lệ, hình thức của các nền văn hóa cũ của nhân loại đã gây ra nhiều bi kịch trong cuộc sống mà nhiều văn hào, thi hào nhân bản thế giới đã nỗ lực tháo gỡ (nhưng thiếu tính toàn diện không như Ngô Thừa Ân) như câu chuyện Le Cid (Pháp), các tiểu thuyết kiếm hiệp của Kim Dung, và rất nhiều hình ảnh, phim ảnh hiện đại.

Phải chăng đã đến lúc các nhà văn hóa, giáo dục nhân bản cần hình thành một nền giáo dục \emph{``hậu hiện đại''} đem lại nhiều tình người, trí tuệ và hạnh phúc hơn cho nhân loại?
% section vấn_đề_giá_trị_đạo_đức_axiology_ (end)
% chapter triết_lý_giáo_dục (end)