\chapter{Hồi 88, 89 và 90} % (fold)
\label{cha:hoi_88_89_90}

Hồi 88:

\begin{itshape}
``Sư đến Ngọc Hoa làm phép lạ.

Ngộ Không, Bát Giới dạy con vua.''
\end{itshape}

Hồi 89:

\begin{itshape}
``Tinh sư tử vỡ hội yến Đinh Ba.

Ba đồ đệ đại náo núi Đầu Báo.''
\end{itshape}

Hồi 90:

\begin{itshape}
``Sư tử bắt thầy trò Tam Tạng.

Thiên Tôn thu yêu quái cửu đầu.''
\end{itshape}

\section{Tư tưởng Phật Học} % (fold)
\label{sec:88_89_90_phat_hoc}

--- Tại Bốc Sa Đình của Ngọc Hoa Vương, Ngộ Không, Ngộ Năng và Ngộ Tịnh (biểu tượng của Tuệ, Giới và Định) đã mở tâm cho ba tiểu vương (con vua Ngọc Hoa Vương) thấy rằng các giá trị chân thật không ở nơi các tướng trạng biểu hiện. Bên trong ba thân hình xấu xí của ba đồ đệ Đường Tăng là ba tâm hồn đẹp đẽ, trong sáng và đầy thần lực. Cái tập quán người đời đánh giá con người qua hình tướng là một thiếu sót lớn.

Khi thấy rõ chân giá trị của Ngộ Không, Ngộ Năng và Ngộ Tịnh, ba tiểu vương liền thi lễ và cầu xin thụ giáo để trở thành ba nhà lãnh đạo giỏi của nhân dân.

--- Con sư tử chín đầu hiệu là Cửu linh Nguyên Thánh là con vật cưỡi của Thái Ất Thiên Tôn cứu khổ; nó đã đắc đạo, và là biểu tượng chín cảnh giới của tâm chúng sinh đã được chuyển hóa tâm giải thoát. Tại đây Ngộ Không và cả phái đoàn Tây Du chưa vượt khỏi chín cảnh giới tâm nên không thể chiến thắng con sư tử chín đầu ấy. Phái đoàn (hay hành giả) cần hướng tâm giải thoát đến Phật trí (được biểu tượng bằng Thái Ất Thiên Tôn Cứu Khổ) và tinh tấn tu tập cho đến khi thuần thục giải thoát các lậu hoặc mới có thể vượt khỏi chín cảnh giới tâm ấy. Bởi thế, Tôn Ngộ Không đã phải đến cầu cứu với Thái Ất Thiên Tôn mong nhờ ngài thu phục sư tử chín đầu mới giải thoát được Đường Tăng, Ngọc Hoa Vương và Tam Tiểu Vương.

Ma nạn bấy giờ không phải do yêu tinh trần thế, mà là tinh ở thượng thượng giới. Trở ngại là do chín cảnh giới tâm mà hành giả đang an trú. Sự trói buộc này quả thật là khó. Hành giả mỗi lần đi vào một cảnh giới Định và Tuệ mới thì liền đó cảnh giới ấy trở thành chướng ngại bước chân giải thoát mà hành giả cần vượt qua. Cứ thế, công phu giải thoát cần phải được liên tục phát triển với tuệ Vô Ngã và tâm Vô Trước. Tâm dính trước thì liền bị các chướng ngại. Sự dính trước vào mỗi cảnh giới tâm được biểu hiện bằng hình ảnh người chăm sư tử say rượu mê ngủ để sư tử sổng chuồng xuống hạ giới quấy phá. Ngô Thừa Ân quả thật rất rành Phật lý!

--- Từ đây nhìn lui các ma nạn do con sâu cưỡi của Bồ Tát Quán Thế Âm, chú tiểu đồng của Bồ Tát Di Lặc, hay con trâu xanh của Thái Thượng Lão Quân gây ra đều là các cảnh biểu tượng nói lên sự thiếu chánh niệm tỉnh giác của hành giả trong công phu hành giải thoát đến Tâm và Tuệ giải thoát. Mất giác tỉnh thì hành giả liền rơi vào vọng niệm và trở ngại công phu phát triển trí tuệ mà Ngô Thừa Ân trình bày biểu trưng bằng các ma nạn.
% section tư_tưởng_phật_học (end)

\section{Quan niệm về con Người} % (fold)
\label{sec:88_89_90_con_nguoi}

--- Con người của xã hội mới cần được soi sáng về giá trị chân thật của hành động và giá trị chân thật của con người để nuôi dưỡng giá trị đó. Nuôi dưỡng giá trị đó là nuôi dưỡng các nhân tố tích cực trong việc xây dựng nền văn hóa mới, như trường hợp ba tiểu vương làm lễ bái sư với Ngộ Không, Ngộ Năng và Ngộ Tịnh để học võ thuật và thần thông (của Tuệ, Giới, Định).

--- Như đã đề cập, cái tốt và cái xấu gắn liền nhau trong xã hội và trong mỗi cá nhân. Với xã hội của nền văn hóa mới, các quan niệm cũ về giá trị đã trở thành lực của việc xây dựng xã hội mới; như sư tử vàng và sư tử chín đầu là trở lực của phái đoàn Tây Du, và là trở lực của việc học tập võ thuật của ba tiểu vương vậy.

Sự cọ xát giữa hai quan niệm giá trị hẳn phải có. Nỗ lực để chiến thắng của quan niệm mới cần được kiên trì vận dụng cho đến thời điểm thành công, thời điểm thuyết phục được các quan điểm bảo thủ.

--- Trở lực lớn nhất là sự bảo thủ của hàng ngũ Vua quan. Lực lượng cải tổ văn hóa cần vận động đổi mới tư duy ở hàng ngũ này, như Ngộ Không, Ngộ Năng và Ngộ Tịnh đã thuyết phục ba tiểu vương.
% section quan_niệm_về_con_người (end)

\section{Quan niệm về xã hội} % (fold)
\label{sec:88_89_90_xa_hoi}

--- Thay đổi quan điểm ở một cá nhân đã là khó, thay đổi các tập tục lạc hậu phi văn hóa còn khó hơn. Sức phản kháng và đối với nền văn hóa mới mạnh nhất mà ta thấy trong tác phẩm Tây Du Ký là ở hàng ngũ vua quan và các thế lực được hàng ngũ đó cung dưỡng. Khi phái đoàn Tây Du cảm hóa được Ngọc Hoa Vương và ba Tiểu vương thì liền phải đối đầu với lực lượng sư tử chín đầu nguy hiểm. Lực lượng sư tử này không vì lý tưởng bảo vệ giá trị của nền văn hóa cũ, nhưng tựu trung thì nó cũng thuộc hệ văn hóa hữu ngã, bảo vệ quyền lợi của phe nhóm đầy tham vọng cá nhân.

Chính nghĩa của xã hội, bao giờ cũng vị tha, nhân ái, vì hạnh phúc dài lâu của số đông, như đã được phái đoàn Tây Du biểu hiện. Con đường cải tổ văn hóa không thể không gặp các kháng cự mạnh mẽ. Trở lực sau cùng sẽ là trở lực lớn nhất bùng lên như ngọn đèn trước khi tắt. Bạn đọc hãy tiếp tục theo dõi các hồi còn lại.
% section quan_niệm_về_xã_hội (end)
% chapter Hồi 88_89_90 (end)