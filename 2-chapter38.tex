\chapter{Hồi 98, 99 và 100} % (fold)
\label{cha:hoi_98_99_100}

Hồi 98:

\begin{verse}
\begin{itshape}
Vượn thuộc, ngựa thuần, vừa thoát xác.\\
Công thành, hạnh đủ, gặp chân như.
\end{itshape}
\end{verse}

Hồi 99:

\begin{verse}
\begin{itshape}
Tám mươi mốt nạn yêu ma hết.\\
Công hạnh tu tròn, đạo lớn thành.
\end{itshape}
\end{verse}

Hồi 99:

\begin{verse}
\begin{itshape}
Về thẳng phương Đông.\\
Năm Thánh thành Phật.
\end{itshape}
\end{verse}

\section{Tư tưởng Phật Học} % (fold)
\label{sec:98_99_100_phat_hoc}

-- Ở hồi 98, đến quán Ngọc Chân Đường Tăng mới thực sự hội diện với thực tướng. Ở đây các tập khí sinh tử còn thiếu sót lại (gọi là hơi hám sinh tử) được tẩy sạch (biểu tượng phái đoàn tắm gội nước thơm).

-- Đến Lôi Âm Tự có hai lối đi: một lối đường mây mà Tôn Ngộ Không thường biết. Đây là lối đến bằng trí tuệ giải thoát, trí tuệ Vô Ngã. Trí tuệ Vô Ngã cũng diện kiến được cảnh giới Phật ngay tại đây ở đời này, như Tôn Ngộ Không có thể dùng cân đẩu vân đến bất cứ lúc nào muốn. Một lối đến bằng tâm giải thoát, bằng thiền định, phải vượt qua nhiều cảnh giới tâm như lối mà Đường Tăng đi qua với 81 ma nạn.

-- Lôi Âm Tự là nơi hội diện của hai lối đi ấy, nơi hội diện tâm giải thoát và tuệ giải thoát (gọi là \emph{toàn phần giải thoát}).

-- Thấy cảnh giới Phật bằng tâm và tuệ giải thoát là thấy như thật (thấy chân như) -- \emph{chân đế}.

Nếu thấy chỉ bằng tâm hay tuệ giải thoát thì chưa được gọi là thật thấy. Với các hành giả này, việc chỉ bày cảnh giới Phật chỉ có ý nghĩa \emph{``tục đế''}.

Con đường tuệ là con đường mây; con đường tâm là con đường bộ.

-- Với Phật Giáo, từ khi có tuệ Vô Ngã cho đến khi thể nhập sự thật Vô Ngã là một công phu tu tập dài để trừ sạch tập khí, điều mà Tôn Ngộ Không nói lên rằng: \emph{``Trông thấy rồi còn chạy đổ ngựa nữa kia''}.

-- Cho đến khi Đức Bảo Tràng Quang Vương Phật chèo đò đưa Đường Tăng qua con sông cuối cùng (con sông ngăn đôi bờ ngã tướng và thực tướng, hay nói khác đi con sông nối đôi bờ ngã tướng và thực tướng). Đường Tăng mới thực giải thoát, để lại cái xác ngã tướng huyễn mộng trên dòng sông Ái, Thủ.

Chiếc thuyền lướt trên sông nước Ái, Thủ là thuyền trí Vô Ngã (hay gọi là trí tuệ không chấp thủ, hoặc trí tuệ Ba La Mật).

-- Kinh văn viết về thế giới thực tướng phải là sự kinh vô tự, bởi hữu ái thì chuyên chở khái niệm của ngã tướng. Nhưng nếu là vô tự không chuyên chở khái niệm ngã tính thì tư duy khái niệm của người đời làm sao có thể thể nhận? Do vậy, Đức Như Lai truyền dạy trao kinh hữu tự giới thiệu con đường bộ về Tây Trúc như Đường Tăng đã trải qua vậy. Sự thật mà người đời có thể đón nhận qua kinh chỉ là Tục Đế. Sự thật không có trong những dòng kinh, nhưng qua công phu tu tập Giới, Định, Tuệ mà các dòng kinh chỉ dạy thì thật tướng liền xuất hiện ngay trong tâm hành giả, như đã xuất hiện trong tâm Tôn Ngộ Không tại núi Linh Đài Phương Thốn, động Tà Nguyệt Tam Tinh.

-- Cuộc hành trình của phái đoàn Tây Du thực ra là cuộc hành trình giải thoát cho mỗi người đời mà ba tạng kinh điển đã vạch ra. Ba tạng kinh mà Đường Tăng sẽ thỉnh về Đông Độ từ Tây Trúc. Cuộc hành trình phải tiêu trừ tập khí của 81 cảnh giới tâm để hiển lộ chân tâm bất diệt gọi là Phật Tâm, Phật Cảnh hay Phật Tri Kiến. Khi đã về đến Phật cảnh, hay nói cách khác khi tâm thức bấy giờ thành thanh tịnh, Vô tướng. Toàn bộ phái đoàn Tây Du (gồm năm thầy trò Đường Tăng -- bao gồm Tiểu Long Mã) đều thành Phật, đều an trú cảnh giới Phật. Con đường giải thoát là con đường chuyển hóa tâm thức, thanh tịnh hóa tâm thức hay gọi là thăng hoa tâm thức từ cấu sang tịnh, từ ngã tướng qua Vô Tướng. Vì thế Đường Tăng khi thành tựu có hiệu là Chiên Đàn Công Đức Phật, tỏa ngát hương giải thoát như hương Chiên Đàn; từ hương đạo đức muốn cứu khổ chúng sinh trong dòng sinh tử trở thành hương giải thoát của một bậc giác ngộ đạo vàng.

-- Tôn Ngộ Không, trong cuộc đấu tranh tiễu trừ các ma nghiệp trên hành trình tu tập, đã luôn luôn lập công đi đầu qua các ma nạn, nay thành tựu đạo -- ở đỉnh cao của chiến thắng ma nghiệp, nên có hiệu là Đấu Chiến Thắng Phật.

-- Trư Ngộ Năng suốt lộ trình tu tập giải thoát phải đấu tranh với chính dục vọng, tính lười biếng và tính tham ăn của mình, nên khi đến đích giải thoát, cuối đường phấn đầu trở thành Tịnh Đàn Sứ Giả (làm chủ vật thực, hoa quả).

-- Sa Ngộ Tịnh, suốt đường tu ổn định tâm lý, kham nhẫn khó khăn nên khi thành đạo, tâm ấy hiển lộ ánh sáng tâm giải thoát sáng chói, có tên là A-La-Hán mình vàng (kiên cố, bất hoại).

-- Tiểu Long Mã, suốt cuộc chiến đấu với ma quân luôn luôn là phương tiện đưa đường cho Đường Tăng, luôn luôn phù trì hạnh nguyện độ sinh, nên khi thành chánh quả có hiệu là Thiên Long Bát Bộ, tiếp tục cứu độ giải khổ cho chúng sinh.

-- Sự việc sau cùng khi Tôn Giả Ca Diếp và Tôn giả A Nan đòi phẩm vật dâng cúng khi trao chân kinh mang dáng dấp mà người đời gọi là \emph{``hối lộ''} đã khiến nhiều độc giả ưu tư, là sự kiện biểu hiện pháp tổ chức truyền thống thể hiện tại thế gian vốn đã được Đức Phật sắp đặt để Phật Pháp có thể tồn tại lâu dài ở đời.

Đây là ý nghĩa phân công phân nhiệm rất công bình trong tứ chúng đệ tử của Đức Phật: hàng xuất gia thì hành trì giải thoát và giảng dạy Pháp cho hàng tại gia; đáp lại hàng tại gia lo tứ sự cúng dường (chỗ ở, thực phẩm, thuốc men, y áo).

Chỉ có điều bất toàn là ngôn ngữ diễn đạt và thái độ phản ứng của Ngộ Không và Ngộ Năng đối với yêu cầu quà tặng của Tôn Giả Ca Diếp, A Nan hơi nặng nề. Lẽ đúng, sau khi tắm gội nước thơm sạch sẽ thì các tập khí thô động cũ đã được trừ bỏ. Thái độ tỏ lộ nét ưu tư của Đường Tăng lúc bấy giờ (đòi quà tặng) cũng khá lạ so với tâm và tuệ giải thoát của Đường Tăng. Người viết nghi ngờ đoạn kết ấy là đoạn bị xen kẽ về sau.

Dù sao thì đó vẫn là đoạn kết của Tây Du Ký và không thể thiếu cảnh cuối là cảnh đòi \emph{``hối lộ''}. Tác giả Ngô Thừa Ân đã rất thâm hiểu Phật Giáo và rất muốn có một Đạo Phật đi vào cuộc đời nên mới dựng ra cảnh đòi vật cống trước khi trao kinh. Thực tế, hai Tôn Giả Ca Diếp và A Nan có đại thần thông đâu còn cần đến bình bát vàng.
% section tư_tưởng_phật_học (end)

\section{Quan niệm về con Người} % (fold)
\label{sec:98_99_100_con_nguoi}

-- Việc giáo dục, đào luyện nhân cách con người cũng tương tự việc đào luyện nhân cách giải thoát, nhưng chỉ giới hạn ở cấp độ thông tục. Nhà giáo dục cần nhận thức rõ tâm lý của mỗi người luôn có mặt hai khuynh hướng đối lập: dục và ly dục, vị kỷ và vị tha, thiện và ác, đố kỵ và tùy hỷ, tình thương và hận thù. Để phát triển khuynh hướng vị tha, tình người, thiện, \ldots ~con người cần được sống trong môi trường Giới, Định, Tuệ. Xã hội cần tạo dựng môi trường sống đó cho cá nhân, gia đình và tập thể phát triển. Xây dựng môi trường sống đó đòi hỏi thời gian và nhiều công phu. Nhưng, dù gian khó đến đâu xã hội cũng nỗ lực thực hiện như cuộc hành trình Tây Du vậy. Đó là chọn lựa duy nhất và trí tuệ, và là trí tuệ ở đỉnh cao của văn minh Châu Á, như đỉnh cao của Hi Mã Lạp Sơn (\emph{Hymalaya}). Con người hay thế giới, qua nhiều biến động rồi cũng quay về môi trường sống đó, bởi vì đó là môi trường sống ổn định nhất cho hướng phát triển được mở ra từ một trí tuệ toàn giác ({\bf Đức Phật}). Khó có thể mong chờ một môi trường sống nào khác.

-- Cuộc đời luôn có mặt khổ đau. Cá nhân nào cũng cảm nhận khổ đau này hay cách khác. Đây là sự thật. Vì thế, chỉ còn cách tốt đẹp nhất là truyền đạt cho các cá nhân cái nhìn trí tuệ Vô Ngã, cái nhìn mà có thể hóa giải khổ đau, tạo dựng sức sống của an lạc, hạnh phúc. Sau khi thâm nhập văn hóa nhân bản và Vô Ngã ấy, con người có thể trút bỏ các phiền não khổ đau ra khỏi bản thân và xã hội như Đường Tăng đã để lại cái xác ngã tướng, ngã niệm bềnh bồng trên sóng nước sinh tử.

Nếu sống là phấn đấu và chọn lựa thì giáo dục mới cần giới thiệu với đời phấn đấu và chọn lựa nền văn hóa nhân bản và Vô Ngã, như Ngô Thừa Ân đã chọn lựa cho nền văn hóa Trung Hoa vậy.
% section quan_niệm_về_con_người (end)

\section{Quan niệm về xã hội} % (fold)
\label{sec:98_99_100_xa_hoi}

-- Qua toàn tập Tây Du Ký, Ngô Thừa Ân đã trình bày với độc giả nhiều quan niệm tốt về nhân sinh và xã hội. Bạn đọc có thể nhận ra ngay xã hội ấy là một xã hội dân chủ được tổ chức trật tự, chặt chẽ, đề cao tính người, tính trách nhiệm, tự chủ, đề cao tinh thần hiện thực, lòng nhân ái và trí tuệ.

\hrulefill

Dòng sống và dòng lịch sử tiếp tục chảy trôi, chảy trôi mãi, nhưng tình người và những ước mơ an lạc hạnh phúc mãi mãi vẫn thế. Sống là sống với dòng đời trôi chảy và sống hạnh phúc, mà không phải nói về hay mơ ước về. Để thể hiện nghĩa sống đó, trong đời sống với nhiều giá trị nhận thức, văn hóa giáo dục mới phải nắm giữ vai trò như vai trò của Tôn Hành Giả trong phái đoàn Tây Du.
% section quan_niệm_về_xã_hội (end)
% chapter Hồi 98_99_100 (end)
