\chapter{Hồi 24, 25 và 26} % (fold)
\label{cha:hoi_24_25_26}

Hồi 24:

\begin{itshape}
``Núi Vạn Thọ, Đại Tiên lưu bạn cũ.

Quán Ngũ Trang, Hành Giả trộm Nhân Sâm.''
\end{itshape}

Hồi 25:

\begin{itshape}
``Trấn Nguyên Tử đuổi bắt người lấy kinh.

Tôn Hành Giả làm nhộn Ngũ Trang quán.''
\end{itshape}

Hồi 26:

\begin{itshape}
``Nơi Tam Đảo, Ngộ Không tìm thuốc.

Nước Cam Lồ, Bồ Tát chữa cây.''
\end{itshape}

\section{Tư tưởng Phật Học} % (fold)
\label{sec:24_25_26_phat_hoc}

Tư tưởng Phật Học ở ba hồi này đã được bàn ở phần {\bf Tổng Luận}: mục \nameref{sec:bieu_tuong_cua_hoi_thu_26}, chương \nameref{cha:cac_hinh_anh_bieu_tuong_khac_nhau_gioi_thieu_phat_hoc_trong_tay_du_ky} trang \pageref{sec:bieu_tuong_cua_hoi_thu_26}.
% section tư_tưởng_phật_học (end)

\section{Quan niệm về con Người} % (fold)
\label{sec:24_25_26_con_nguoi}

Tư tưởng Nho học vẫn là tư tưởng chủ đạo của nền văn hóa Trung Hoa vào thế kỷ XVI (thế kỷ mà Ngô Thừa Ân sống) và qua hầu như suốt dòng lịch sử Trung Hoa cho đến thế kỷ XIX. Hệ thống tư tưởng này đã xây dựng nên chế độ phong kiến. Con người và nhân cách con người được giáo dục theo khuôn mẫu Nho Giáo với công thức đầy ước lệ về Hiếu, Trung, Tình, Nghĩa, \ldots, hay Nhân, Lễ, Nghĩa, Trí, Tín, rất máy móc và rất hình thức. Nhân cách đó đã trói buộc sự phát triển con người và tính nhân bản của con người, và hạn chế cả sự phát triển xã hội. Nhân cách đó quá tù hãm con người, không đáp ứng được các biến động đổi thay của con người và xã hội, cũng không đáp ứng được các yêu cầu mới của lịch sử.

Như đã đề cập, Ngô Thừa Ân muốn đi tìm một mẫu nhân cách mới, rất người và đáp ứng được mơ ước của con người của nhiều thời đại. Ngô Thừa Ân đã kín đáo giới thiệu nhân cách ấy qua tiểu thuyết Tây Du Ký. Mở ra cuộc Tây Du vĩ đại là tác giả muốn mở bung ra cái xã hội bị khép kín của Trung Hoa, khép kín cả nhận thức và thái độ sống, điều mà ngày nay cả thế giới đều thấy: vì sự khép kín ấy mà nền văn minh cổ và có giá trị của Trung Hoa chậm phát triển.

Dựng nên ba hồi 24, 25 và 26 này là Ngô Thừa Ân muốn nói rõ ra một nhân cách mới cần tạo dựng cho văn hóa Trung Hoa. Tác giả đã để cho Tôn Ngộ Không (tiếng nói của nền văn hóa Phật Giáo) đánh bật gốc rễ cây Nhân Sâm quý giá (có mặt từ lúc trời đất mới khai tích) tại Ngũ Trang Quán --- biểu tượng của Âm Dương và Ngũ Hành của Nho Giáo --- và phải nhờ đến nước Cam Lồ (của trí tuệ Phật Giáo) tưới vào mới làm sống lại cây Nhân Sâm.

Tác giả, qua hình ảnh đó, muốn cải thiện, bổ sung vào nhân cách của Nho Giáo, trí tuệ Vô Ngã của Phật Giáo, chỉ có trí tuệ đó mới làm sống lại nhân cách lý tưởng của nền văn hóa truyền thống. Chỉ là công việc chuyển đổi, cải thiện, mà không phải là phế bỏ hay thay thế hoàn toàn. Nhân cách Vô Ngã ấy sẽ được phát triển tốt và mạnh trong một xã hội dân chủ, nhân ái và công bằng.
% section quan_niệm_về_con_người (end)

\section{Quan niệm về xã hội} % (fold)
\label{sec:24_25_26_xa_hoi}

-- Lại nói về chuyện cây Nhân Sâm và Ngũ Trang Quán, như được đề cập ở phần {\bf Tổng Luận}, Tôn Ngộ Không đã đánh bật gốc rễ cây Nhân Sâm, cây nguồn gốc của các pháp hữu vi của hiện tượng giới --- cũng là biểu tượng của Nho và Lão, trí tuệ Vô Ngã của Tôn Ngộ Không chỉ đánh bật gốc rễ của thế giới nhị nguyên, hữu ngã, nhưng chưa có khả năng làm hồi sinh cây, nghĩa là giải quyết sinh tử. Chỉ có trí tuệ Ba La Mật của Bồ Tát Quán Thế Âm là giải quyết được sinh tử, làm hồi sinh cây Nhân Sâm.

-- Bồ Tát đã lấy nước từ trong cây ấy bằng trí tuệ Vô Ngã rồi đem tưới vào cây ấy thì cây liền sống lại. Cũng thế, chân tướng có mặt ở nơi các ngã tướng; vô sinh có mặt nơi sinh diệt; giải thoát có mặt tại cuộc đời, tại phiền não.

Trí tuệ Vô Ngã của Phật Giáo mở ra con mắt tuệ cho con người thấy sự thật đó. Chuyện hồi sinh cây Nhân Sâm là bài thuyết pháp thâm diệu mà Bồ Tát Quán Thế Âm đã giảng bày cho Trấn Nguyên Tử và các Thiên Tiên, Địa Tiên, và phái đoàn Tây Du. Điều này được làm rõ nghĩa thêm ở hình ảnh một quả Nhân Sâm rơi xuống đất liền biến mất vào đất, đến lúc hồi sinh thì quả Nhân Sâm liền trở về lại cây. Với con mắt hữu ngã (sinh --- diệt) thì bảo là quả Nhân Sâm đã mất, rồi phiền não, bực bội vì nó, và nghi oan cho Tôn Ngộ Không lấy trộm.

Thực sự, Bồ Tát Quán Thế Âm đã cắt nghĩa cho thấy quả Nhân Sâm chỉ có hay mất dưới cái nhìn sinh diệt của ngã tướng; nhưng, nó là bất sinh bất diệt đối với thực tại như thật, và đối với cái nhìn Vô Ngã; Quả Nhân Sâm từ trên không rơi chạm đất thì liền biến mất là ý nghĩa khi có mặt đủ Âm, Dương (Đất Trời) thì thật tướng biến mất (biểu tượng bằng hình ảnh biến mất của quả Nhân Sâm), nhưng khi có nước Cam Lồ của trí tuệ Vô Ngã rưới vào thì thực tướng xuất hiện (như hình ảnh quả Nhân Sâm sau khi biến mất lại tái hiện trên cành).

-- Tác giả Ngô Thừa Ân muốn giới thiệu cái nhìn Vô Ngã này với xã hội Trung Hoa, muốn đưa cái nhìn ấy vào nền văn hóa, giáo dục Trung Hoa để hồi sinh cho nền văn hóa này vốn thường ở trong điều kiện hầu như khủng hoảng. Cái nhìn Vô Ngã đi vào một nền văn hóa nào thì liền chuyển nền văn hóa ấy thành nền văn hóa Vô Ngã. Nền văn hóa mới này đến không xáo trộn nền văn hóa cũ, và không phải đến từ bên ngoài nền văn hóa cũ, mà đến từ nền của văn hóa ấy. Nó phát sinh từ trong lòng cuộc sống đang trôi chảy và chuyển hóa cái nhìn Nhị Nguyên thành cái nhìn Vô Ngã về dòng sống đang trôi chảy. Việc xiển dương nền văn hóa, giáo dục này nếu được gọi là cách mạng văn hóa giáo dục, thì cũng không hẳn đúng; nhưng nếu không được gọi là cuộc cách mạng văn hóa giáo dục, thì cũng không đúng. Ở đây, chúng ta tạm gọi là một cuộc cải tổ văn hóa, giáo dục theo tinh thần Phật Giáo dưới cái nhìn của Ngô Thừa Ân.

Sau khi cây Nhân Sâm được hồi sinh, Trấn Nguyên Tử, các Phước, Lộc, Thọ Tinh, Bồ Tát và phái đoàn Tây Du đã hoan hỷ đề huề trong bữa tiệc gọi là hội ``Nhân Sâm quả''. Trấn Nguyên Tử lại mở thêm một bữa tiệc riêng để làm lễ kết nghĩa huynh đệ với Tôn Ngộ Không. Kết nghĩa huynh đệ với Tôn Ngộ Không là tế nhị nói lên rằng: Trấn Nguyên Tử được xếp vào hàng đệ tử Bồ Tát Quán Thế Âm (vốn Trấn Nguyên Tử là tổ của Địa Tiên). Sự kiện này nói lên rằng nền văn hóa, giáo dục Vô Ngã của Phật Giáo sẽ hòa nhập, hay đi song hành, với nền văn hóa giáo dục tại bản xứ, nơi Phật Giáo đặt chân đến. Nền văn hóa giáo dục Phật Giáo có khả năng sẽ làm thăng hoa, hồi sinh nền văn hóa giáo dục cũ.

Tưởng chúng ta cũng cần để tâm suy ngẫm đến câu chuyện cây Nhân Sâm này. Chúng ta có cảm tưởng rằng cảm hứng của Ngô Thừa Ân đã bật khởi mạnh mẽ như thế nào về sự nghiệp Tây Du và hóa đạo của Đường Tăng, thì cảm hứng của ông cũng bật khởi mạnh mẽ như thế ấy khi dựng nên hình ảnh cây Nhân Sâm bật gốc và cây Nhân Sâm hồi sinh.

% section quan_niệm_về_xã_hội (end)
% chapter Hồi 24_25_26 (end)