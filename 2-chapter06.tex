\chapter{Hồi 9, 10, 11 và 12} % (fold)
\label{cha:hoi_9_10_11_và_12}
Hồi 9:

\begin{itshape}
``Trần Quang Nhị nhậm chức gặp nạn.,

Sư Giang Lưu trả thù báo ơn.''
\end{itshape}

Hồi 10:

\begin{itshape}
``Lão Long Vương vụng tính phạm pháp thiên đình.

Ngụy thừa tướng gửi thư nhờ âm phủ.''
\end{itshape}

Hồi 11:

\begin{itshape}
``Qua địa phủ Tái Tôn về dương.

Dâng quả bí Lưu Toàn được vợ.''
\end{itshape}

Hồi 12:

\begin{itshape}
``Vua Đường lòng thành mở đại hội.

Quan Âm hiển Thánh hóa Kim thuyền.''
\end{itshape}

\section{Tư tưởng Phật Học} % (fold)
\label{sec:9_phat_hoc}

-- Về con đường đến trí tuệ Ba La Mật, theo Phật Giáo phát triển (Đại Thừa), hành giả sau khi ngộ pháp tánh (còn gọi là \emph{minh tâm, kiến tánh}) cần phát tâm cầu Phật quả và hành sâu giải thoát (Giới, Định, Tuệ) và tâm đại bi cứu độ chúng sinh thì công đức giải thoát mới sớm thành tựu viên mãn.

Bốn hồi này là phần Bồ Tát Quán Thế Âm sắp đặt cho nhân vật Đường Huyền Trang xuất hiện với lòng đại bi phát đại tâm đi Tây Trúc thỉnh kinh để cứu độ Nam Thiệm Bộ Châu. Đây là nhân vật biểu hiện tâm nguyện Bồ Tát cầu giải thoát Vô Thượng Bồ Đề. Việc thỉnh được Tam Tạng Kinh từ Lôi Âm Tự về đến Tràng An (Trung Hoa) là biểu tượng cho mục tiêu đắc trí tuệ giải thoát Ba La Mật vậy. Về nhân vật Đường Tăng đã được đề cập trong phần {\bf Tổng Luận}, ở đây chỉ bàn thêm một vài điểm.

-- Đường Tăng, nhân vật được Bồ Tát Quán Thế Âm chọn (Tây Du Ký) ủy thác trách nhiệm thỉnh kinh là tiêu biểu cho quyết tâm giải thoát và độ sinh của mỗi hành giả. Đường Tăng tự thân thấy rõ cảnh đời vô thường, khổ đau và ác trược (chứng kiến các oan khổ đến với tự thân, gia đình và xã hội) nên phát khởi đại bi tâm độ sinh.

Đây là điểm giáo lý rất truyền thống của Phật Giáo. Chỉ có những tâm thức nào thấy rõ ba pháp ấn, hay bốn pháp ấn là thấy đạo; chỉ những ai thấy đạo mới phát tâm cầu đạo. Sự kiện phát khởi đại bi tâm chỉ thực sự xẩy đến với điều kiện tâm thức như thế, nghĩa là \emph{đại bi tâm phát khởi từ trí tuệ giải thoát}.

Không thể có sự kiện phát khởi đại bi tâm ngoài trí tuệ. Vì thế hình ảnh Đường Tăng gắn liền với Tôn Ngộ Không trên suốt cuộc hành trình Tây Du. Lúc nào hai hình ảnh ấy, hai tâm thức ấy tách rời khỏi nhau thì tà pháp xuất hiện (ma nạn đến), \ldots. Đây là Đạo.

-- Qua việc giới thiệu thân thế Đường Tăng, bốn hồi truyện này (hồi 9 --- 12) còn giới thiệu một số điểm khác về giáo lý Phật Giáo như:

\begin{enumerate}[label=\itshape\alph*\upshape/]

    \item Trạng nguyên Quang Nhị đã phóng sinh con cá vốn được mua để dâng mẹ, và được mẹ khen ngợi. Tại đây tác giả giới thiệu các giá trị tu tập giải thoát: hiếu hạnh là tốt, nhưng tu tập bố thí thì tốt hơn, phước đức nhiều hơn. Phóng sinh cũng thế. Tương tự, công đức trì giới sẽ nhiều hơn nữa; hành thiền định công đức sẽ hơn trì giới; tu tập trí tuệ là tốt hơn cả.

    \item Tác giả nhân đây giới thiệu giáo lý Nhân Quả Phật Giáo rất phân minh: Quang Nhị phóng sinh cá nên về sau (vốn là con cá được phóng sinh) Long Vương cứu Quang Nhị khỏi chết sau khi bị Lưu Hồng ám hại để đoạt vợ và chức Tri phủ.

    \item Việc Đường Thế Dân du Địa phủ chứng kiến sự thật Nhân Quả: ác nhân thì ác báo, thiện nhân thì thiện báo. Tác giả đã tóm tắt khái quát rằng: \emph{``Người làm thiện thì sinh Tiên; người tận trung thì sinh và đường quý; người hiếu hạnh thì sinh vào đường phúc; người công bằng thì sinh vào kiếp Người; người tích đức thì sinh vào đường giàu có; người độc ác thì sinh vào đường quỷ''}.

    Giáo lý này hướng dẫn người đời sống ngay thẳng, công bằng và thiện lương đem lại an vui cho bản thân, gia đình và xã hội.
\end{enumerate}

-- Chiếc áo Cà Sa mà Phật Tổ ban cho Đường Tăng (Bồ Tát Quán Thế Âm trao) là biểu tượng lý tưởng giải thoát và độ sinh, nghĩa là lý tưởng giải thoát khổ cho tự thân và giúp cho mọi chúng sinh thoát khỏi hết thảy các khổ ách trong sinh tử.

Lý tưởng này là sự nghiệp cao đẹp nên được dệt bằng tơ quý (tiên tơ), bởi bàn tay quý (tiên nga dệt), và được trang trí bằng các hạt trân châu: hạt ngọc ``ma ni'' là biểu tượng cho hạnh sống thanh bạch, trong sáng, không tỳ vết; hạt ``tị trần châu'' là biểu tượng cho hạnh sống xa lìa hết thảy các trần cấu (các cấu uế của tâm); hạt ``định phong châu'' là biểu tượng cho hạnh sống nhiếp tịnh tâm thức thoát ly khỏi các ảnh hưởng khen, chê, thị phi, được, mất, danh vọng, lợi dưỡng (\emph{bát phong}); hạt ``dạ minh châu'' là biểu tượng cho trí tuệ.

Chiếc áo mà Bồ Tát Quán Thế Âm tặng không cho Đường Tăng là biểu tượng cho công phu hành giải thoát được chứa đựng trong Tam Tạng kinh điển Phật Giáo.

% section tư_tưởng_phật_học (end)

\section{Quan niệm về con Người} % (fold)
\label{sec:9_con_nguoi}

Cho đến hồi thứ 12 này, Ngô Thừa Ân đã phác họa đủ đề cương của một nội dung giáo dục con người tốt phục vụ cho xã hội của nền văn hóa giáo dục Vô Ngã.

-- {\bf Đề cương ấy bao gồm các điểm:}

\begin{enumerate}[label=\itshape\arabic*\upshape/]

    \item Giáo dục con người sống phải có mục tiêu, lý tưởng hướng thượng. Lý tưởng đó phục vụ cho an lạc, hạnh phúc của bản thân, tập thể, xã hội, và được biểu tượng qua nhân vật Đường Tăng.

    \item Giáo dục con người có khả năng nhận thức, phân biệt rõ phải, trái, thiện, ác, tốt, xấu, đúng, sai, hư, thật, được biểu trưng qua nhân vật Tôn Ngộ Không.

    \item Giáo dục con người biết hướng thiện, sống đạo đức, biết khép mình vào khuôn phép, kỷ luật tự giác, được biểu tượng qua Trư Bát Giới.

    \item Giáo dục tâm lý con người, giáo dục ý chí và quyết tâm theo đuổi con đường lý tưởng của mình, biểu tượng qua nhân vật Sa Ngộ Tịnh.

    \item Giáo dục đầy tình người (\emph{nhân, hiếu, trung, tình, nghĩa}), biểu tượng qua nhân vật Tiểu Long Mã.

    \item Giáo dục con người thấy rõ tính độc đoán, uy quyền, hưởng thụ, bất công (tệ trạng tại Thiên cung) của những người có quyền thế, thấy rõ sự thật khổ đau ở Âm phủ và cuộc ``hỗn thế'' ở trần gian để xây dựng một xã hội công bằng, nhân ái, an lạc và hạnh phúc.
\end{enumerate}

-- {\bf Xây dựng một nhân cách toàn diện qua các mặt:}

\begin{enumerate}[label=\itshape\arabic*\upshape/]

    \item Lý trí và hành động (Tôn Ngộ Không).

    \item Về tình cảm, đạo đức, thẩm mỹ (vai trò Đường Tăng).

    \item Về thể lực, bản năng, dục vọng (vai trò Bát Giới).

    \item Ý chí kiên định, tâm tư ổn định (vai trò Sa Ngộ Tịnh).

    \item Về tương giao xã hội (vai trò Tiểu Long Mã).
\end{enumerate}

Đó là nội dung giáo dục về \emph{trí dục, đức dục, thể dục, tâm lý và tình cảm, lao động (hành động)}, và giáo dục một khả năng thực hiện sự hòa điệu các mối tương giao của cuộc sống.

-- Nền giáo dục con người toàn diện ấy được xây dựng trên cơ sở sự thật Vô Ngã tính của vạn hữu, và trên căn bản niềm tin rằng tâm thức có khả năng phát triển không giới hạn, và khả năng chuyển hóa không cùng.

% section quan_niệm_về_con_người (end)

\section{Quan niệm về xã hội} % (fold)
\label{sec:9_xa_hoi}

-- Tác giả lần lượt nêu lên những lý do của một xã hội cần được thay đổi. Sự kiện Trần Quang Nhị, rể của quan thừa tướng Ngụy Trưng đi nhậm chức bị kẻ ác giết chết, chiếm vợ và chiếm cả chức vị Tri phủ kéo dài mười tám năm mà triều đình không hay biết, dư luận trong nước cũng không biết, chứng tỏ xã hội Trung Hoa bấy giờ tổ chức không hợp lý, không thể bảo đảm an ninh xã hội cho dân chúng, đưa xã hội vào cảnh hỗn loạn, nhân dân không có quyền làm chủ xã hội và cuộc sống.

-- Sự kiện Long Vương không tôn trọng mệnh lệnh của Thiên đình, tự do hành xử theo tư ý là sự kiện không thượng tôn pháp luật.

-- Sự kiện Lý thị vợ Lưu Toàn, phải tự vẫn khi làm việc thiện là hậu quả của các quan niệm đạo đức luân lý lỗi thời của xã hội Trung Hoa.

-- Sự kiện Đức Phật ủy thác Bồ Tát Quán Thế Âm Phật sự thỉnh Tam Tạng Kinh Phật đến Đông Độ để cứu vớt quần chúng nói lên quá rõ ràng nhân tâm ở đây đã loạn, đạo đức đã suy đồi.

-- Tại giới quan lại, chính trường thì vì danh lợi đã đẩy đưa một số người có tài giúp nước như Lý Định, Trương Tiêu đến chỗ chán nản rút lui về sống an phận bằng nghề đốn củi, bắt cá, tiêu dao ở vùng non xanh nước biếc.

-- Trước cảnh một xã hội, dù không có chiến tranh, đạo đức đi dần vào băng hoại, Ngô Thừa Ân muốn đánh thức nhân dân tỉnh dậy để cùng nhau sửa soạn cho một xã hội dân chủ, công bằng và nhân ái. Xã hội mới sẽ bắt đầu từ ngày Đường Tăng nhận tấm áo Cà sa và chiếc Tích trượng từ tay Bồ Tát Quán Thế Âm để Tây Du. Cần có một cơn gió lớn đưa hương Vô Ngã, trí tuệ và từ bi từ Tây Trúc về xã hội Trung Hoa đương thời.

Đây là hoài bão của Ngô Thừa Ân về việc xây dựng một xã hội con người rất người và tốt đẹp qua tiểu thuyết Tây Du Ký.

-- Công cuộc cải tổ văn hóa, giáo dục mà Tôn Ngộ Không khởi xướng cần được tiến hành nhiều bước hợp lý.

Bước đầu tiên là giáo dục, tuyên truyền ý thức nơi quần chúng nhân đân về những hư hỏng của xã hội cũ, và chứng tỏ các ưu thế của nền văn hóa mới. Cánh đổi mới cần có một lực lượng hậu thuẫn mạnh mẽ để có thể đứng vững trước nhiều sức mạnh phản kháng đến từ nhiều phía, đại để có ba sức mạnh phản kháng chính:

\begin{enumerate}[label=\itshape\arabic*\upshape/]
    \item Sức mạnh ``sấm sét'' của triều đình, quan lại phong kiến và các nhóm ăn theo.

    \item Sức mạnh ``âm hỏa'' là sức mạnh tâm lý phản kháng, phản tuyên truyền tác động vào quần chúng và vào phái canh tân.

    \item Sức mạnh ``bi phong'' là sức mạnh gây ra của sự thiếu lòng tin, nhu nhược, tình cảm thiếu sáng suốt của quần chúng và của phái canh tân.
\end{enumerate}

Đây là bước chuẩn bị cần thiết để xây dựng một nền văn hóa giáo dục mới. Từ đây, Tôn Ngộ Không còn phải giác tỉnh để đối đầu trong từng bước đi cải tổ của mình, mở đầu từ hồi thứ 13, nếu không thì sẽ bị tan xác trước ``sấm sét'', bị thiêu cháy trong ``âm hỏa'', hay bị tan rã trước ``bi phong''.

% section quan_niệm_về_xã_hội (end)
% chapter hồi_9_10_11_và_12 (end)