\chapter{Hồi 59, 60, và 61} % (fold)
\label{cha:hoi_59_60}

Hồi 59:

\begin{itshape}
``Tam Tạng bị nghẽn tại Núi Hỏa Diệm.

Hành Giả lần đầu lấy quạt Ba Tiêu.''
\end{itshape}

Hồi 60:

\begin{itshape}
``Ma Vương ngừng đánh đi dự hao duyên.

Hành giả hai lần lấy Ba Tiêu phiến.''
\end{itshape}

Hồi 61:

\begin{itshape}
``Bát Giới giúp sức bại ma vương.

Hành Giả lần thứ ba lấy quạt.''
\end{itshape}

\section{Tư tưởng Phật Học} % (fold)
\label{sec:59_60_phat_hoc}

— Những sai lầm của phái đoàn Tây Du từ trước cho đến hồi 58 đều do công phu giải thoát chưa nhiếp được ``bát phong'' (\emph{thị phi; được mất; khen, chê; danh vọng và lợi dưỡng}), cả đến Tôn Ngộ Không cũng bị quạt ``ba tiêu'' của bà La-Sát quạt bay xa hơn hai vạn dặm (trong khi đó, với người khác có thể bị bay xa đến tám vạn bốn ngàn dặm). Nhờ ``Định Phong Đan'' của Như Lai, Tôn Ngộ Không mới đứng vững trước quạt ``ba tiêu''. Tại hồi này, phái đoàn Tây Du mới hàng phục được ``bát phong''.

— Ba lần đoạt quạt ``ba tiêu'' mới ổn, điều này tựa như ba lần mới đánh chết được ``Bạch cốt phu nhân'' đã được trình bày về ý nghĩa biểu tượng.

— Hỏa Diệm sơn là biểu tượng của \emph{dục và sân} (kiết sử thứ tư và thứ năm thuộc năm hạ phần kiết sử), Chỉ khi đoạt được ``ba tiêu phiến'', nghĩa là khi đủ định và tuệ để chế ngự ``bát phong'' thì phái đoàn Tây Du mới có khả năng để dập tắt dục và sân. Quả vị chứng đắc tại đây có thể là quả Thánh thứ ba (A—na—hàm), quả vị bất lai.
% section tư_tưởng_phật_học (end)

\section{Quan niệm về con Người} % (fold)
\label{sec:59_60_con_nguoi}

— Có một số vấn đề vấn đề rất người, rất xã hội và rất trí tuệ được đặt ra về nhân cách con người và giá trị của nhân cách con người trong nền văn hóa mới:

\begin{enumerate}[label=\itshape\alph*\upshape/]

    \item Mọi người (nam hay nữ) đều có tự do quyết định đời sống của mình, nhưng tự do ấy cần được giới hạn trong tương quan với tha nhân, tập thể và xã hội. Trường hợp Ngưu Ma Vương có vợ chính là bà La Sát vừa nhan sắc vừa đức hạnh, lại mê sống với nữ hồ ly khác giàu có mà thiếu đức hạnh nên đã được Ngô Thừa Ân quan tâm để cho Trư Bát Giới đánh chết nữ hồ ly và đốt sạch hang động của mụ ta. Ngô Thừa Ân muốn đề cao chế độ một vợ chồng để bảo vệ phẩm chất con người, nhất là người phụ nữ. Đây cũng là một vấn đề tác giả đang đối thoại với nền văn hóa cũ của Trung Hoa chấp nhận đa thê và trọng nam khinh nữ.

    \item Nhân cách con người sẽ cao quý hơn và tâm hồn sẽ an lạc, hạnh phúc hơn nếu con người sống tự chủ làm chủ được ``bát phong''. Ở nghĩa tương đối của đối với đời sống xã hội, chỉ đòi hỏi con người đừng để bị lún sâu vào ``tám thứ gió'' ấy.

    \item Ham muốn nhiều cũng làm giảm giá trị nhân cách và đánh mất đi nhiều hạnh phúc. Ham muốn nhiều thì thất vọng cũng nhiều và khổ đau sẽ nhiều thêm. Càng ham muốn con người càng bị nô lệ bởi ham muốn, tự do tâm lý sẽ bị đánh mất dần, điều mà con người thường vẫn rất mực khát khao.

    \item Sân hận sẽ là nhân tố đầu độc các mối tương giao con người và xã hội. Tương giao xấu sẽ đánh mất tình người và sẽ gieo rắc sầu muộn. Sân hận dễ gây gãy đổ công việc, làm giảm tuổi thọ và dung sắc, và có thể gây ô nhiễm môi sinh. Quạt Ba Tiêu và Hỏa Diệm Sơn nói lên với độc giả các ý nghĩa vừa bàn. Trong nền văn hóa giáo dục mới, dục sân và ``bát phong'' là đối tượng cần được chế ngự, như Tôn Ngộ Không phải dập tắt Hỏa Diệm Sơn đang chắn lối đi Tây Trúc để hoàn thành sứ mệnh độ khổ dân Đông Độ.
\end{enumerate}

— Trong nền văn hóa mới, các nhân tài, các nhân tố cực xây dựng an lạc hạnh phúc cho xã hội, như Ngưu Ma Vương và bà La Sát, được tạo điều kiện thuận lợi để đóng góp, để hiểu rõ trí tuệ Vô Ngã, vị tha và hạnh phúc của con đường sống này.
% section quan_niệm_về_con_người (end)

\section{Quan niệm về xã hội} % (fold)
\label{sec:59_60_xa_hoi}

— Nền văn hóa nhân bản và trí tuệ là nền văn hóa của tình người, hiểu biết và cảm thông, chứ không phải là nền văn hóa xây dựng bằng sức mạnh của sân hận. Ngô Thừa Ân vì thế giới thiệu Hồng Hài Nhi từ bỏ ``tam muội hỏa'' theo hầu Bồ Tát Quán Thế Âm ở Nam Hải. Cả Ngưu Ma Vương và bà La Sát đều quay hướng sống từ bị độ đời của Bồ Tát, đóng góp đáng kể và việc dập tắt Hỏa Diệm sơn, và tạo dựng một đời sống hưng vượng cho nhân dân quanh vùng.

— Các chủ nghĩa hưởng thụ, vị kỷ, cá nhân quá khích đều bị quét sạch trong xã hội mới, như nữ hồ ly và cơ sở vật chất của mụ ta bị hủy diệt. Hướng sống vị kỷ sẽ được giáo dục chuyển hóa thành vị tha như hình ảnh nữ hồ ly chết dưới đinh ba của Trư Bát Giới (người đã giác ngộ rời xa dục vọng thấp kém).
% section quan_niệm_về_xã_hội (end)
% chapter Hồi 59_60 (end)