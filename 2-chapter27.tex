\chapter{Hồi 67} % (fold)
\label{cha:hoi_67}

Hồi 67:

\begin{itshape}
``Cứu xóm Đà — La, lòng vững đạo

Thoát đường ô — uế, tính càng thanh''
\end{itshape}

\section{Tư tưởng Phật Học} % (fold)
\label{sec:67_phat_hoc}

 —  Con đường tâm thức của phái đoàn Tây Du đến hồi 67 là khá ổn định. Thành viên thiếu ổn định là Trư Ngộ Năng bây giờ đã rất tinh tấn, hòa hợp và sống rất tri túc, đã có công lớn phát dọn cây gai mở 800 dặm đường đến am Mộc Tiên. Ở hồi 67 này, Ngộ Năng hết lòng dọn sạch con đường qua núi Thất Tuyệt đầy thối tha, ô uế cho Đường Tăng đi qua, vui vẻ mà không than thở. Thế là phái đoàn đủ điều kiện để vượt qua các trở ngại tâm thức đến tâm và tuệ giải thoát.

 —  Con đường trái cây thối rụng xuống quá nhiều năm thắng làm nghẽn lối đi là biểu tượng của tập khí nghiệp do các đời sinh tử quá khứ để lại. Dọn sạch con đường này là dọn sạch các lậu hoặc (dục lậu, hữu lậu và vô minh lậu). Công phu đoạn trừ các lậu hoặc sẽ kéo dài cho đến khi tuệ giải thoát xuất hiện. Đoạn đường lầy lội quả thối này là mở đầu của công phu đó vậy.

Các yếu tố tác động sinh thành các lậu hoặc chính là con quỷ đại mãng xà đã nhiều năm gieo rắc khổ đau tại xóm Đà — La, Mãng Xà Vương ấy là biểu tượng của dục vọng cám dỗ con người (nói đủ là tham lậu hoặc, sân lậu hoặc và si lậu hoặc). Ngộ Không cùng với Ngộ Năng giết được mãng xà là biểu tượng đã dứt được gốc tham, sân, si (nhờ Giới, Định, Tuệ). Các thượng phần kiết sử còn lại (hữu ái, vô hữu ái, mạn, trạo cử và vô minh) chỉ là tập khí sẽ dần dần tự tiêu. Nếu khéo tu thì sẽ đốt cháy chúng tức thì trong hiện tại.

 —  Sau cái chết của quỷ vương mãng xà, nhân dân địa phương vô cùng an vui, hạnh phúc, tiêu hết các sợ hãi, âu lo. Đây là tâm ảnh an lạc, giải thoát của công phu tu tập Giới, Định, Tuệ cho đến mức độ giải thoát này.

 —  Cây thị có bảy điểm gọi là thất tuyệt:

1/ Sống lâu.

2/ Có bóng rợp mát.

3/ Chim không làm tổ.

4/ Không có sâu bọ trên cành, lá.

5/ Lá vui mắt.

6/ Quả ngon.

7/ Cành, lá to mập.



— Cũng vậy, người tu hành đến đây có 7 sự thành tựu:

1/ Đoạn trừ các ngã tưởng, thấy thời gian Vô Ngã. Đây là ý nghĩa sống lâu.

/ Có khả năng chỉ đường tu cho người khác ($\ll$ có bóng rợp khác)

3/ Tâm vô chấp, không tham đắm các ngã tướng ($\ll$ chim không làm tổ)

4/ Tâm bất thoái, duy tác, không tạo ra lậu hoặc nữa ($\ll$ không có sâu bọ trên cành lá)

5/ Thân và tâm khinh an ($\ll$ lá vui mắt)

6/ Đi vào quả Thánh ($\ll$ Quả ngon)

7/ Trí tuệ giải thoát dần dần hình thành ($\ll$ cành lá to mập)
% section tư_tưởng_phật_học (end)

\section{Quan niệm về con Người} % (fold)
\label{sec:67_con_nguoi}

 —  Nhân cách con người cần được giáo dục qua 2 mặt:

$\star$ Mặt tự độ: loại bỏ tính ích kỷ, các tâm lý tham, sân, si và sợ hãi do chấp ngã sinh.

$\star$ Mặt độ tha: Quan tâm đến hạnh phúc và khổ đau của tha nhân và xã hội, giúp đỡ tha nhân.

Thể hiện hai mặt sống ấy là xây dựng kiện toàn nhân cách. Ngô Thừa Ân đã dựng nên cảnh núi cây thị và cứu xóm Đà — La là nói đến hai mặt giáo dục ấy.

Xây dựng con người cũng cần dựa vào căn bản của Giới, Định, Tuệ (kỷ luật, thiền định, hiểu biết, trí tuệ) như cả ba Ngộ Không, Ngộ Năng và Ngộ Tịnh kết hợp làm nên ngôi nhà trú ẩn.
% section quan_niệm_về_con_người (end)

\section{Quan niệm về xã hội} % (fold)
\label{sec:67_xa_hoi}

 —  Một Nước tốt đẹp (ý nghĩa Tiểu Tây Thiên) nếu chưa là một Nhà nước lý tưởng (Đại Tây Thiên) thì cũng cần có bảy đặc điểm gọi là ``thất tuyệt'':

1. Được lòng dân nên tồn tại lâu dài ($\ll$ sống lâu).

2. Nhân dân tin tưởng, công bằng và vì dân ($\ll$ bóng mát).

3. Tổ chức tốt, không phát sinh các nhóm tà mi ($\ll$ chim không làm tổ).

4. Không dung dưỡng hạng sâu dân mọt nước ($\ll$ cành lá không có sâu bọ).

5. Các tổ chức và nhân sự tốt ($\ll$ cành lá vui mắt).

6. Tổ chức thành công về văn hóa, giáo dục, kinh tế, an ninh, \ldots, ($\ll$ hoa quả ngon).

7. Phát triển xứ sở hưng vượng ($\ll$ cành lá mập to).

Để được các thành quả trên, Nhà nước cần giải quyết vấn đề an cư lạc nghiệp cho dân, như phái đoàn Tây Du cứu an xóm Đà — la; trừ các nhân tố gieo rắc khổ đau cho nhân dân; như Ngộ Không đánh chế mãng xà; giáo dục theo tinh thần giáo lý nhà Phật như phái đoàn Đường Tăng đem ai vui đến cho vùng núi Thất Tuyệt.

Hẳn nhiên là ở mỗi hồi truyện Ngô Thừa Ân đã chuyển vào các thao thức của ông ta về việc xây dựng con người và xã hội Trung Hoa trên nền tảng văn hóa Phật Giáo. Hy vọng người viết đã không có quá nhiều tưởng tượng đi xa các hàm ý của tác giả.
% section quan_niệm_về_xã_hội (end)
% chapter Hồi 67 (end)