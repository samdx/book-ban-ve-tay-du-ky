\chapter{Hồi 1} % (fold)
\label{cha:hoi_thu_1}

\begin{itshape}
``Gốc thiêng nẩy nở, nguồn rộng mở.

Tâm tánh tu trì, đạo lớn sinh.''
\end{itshape}

\section{Tư tưởng Phật Học} % (fold)
\label{sec:hoi_1_tu_tuong_phat_hoc}

— Theo Phật Giáo (cả Nam và Bắc Truyền), thế giới hệ này gồm có bốn châu thiên hạ:

\begin{multicols}{2}
\begin{enumerate}[label=\itshape\arabic*\upshape/]
    \item Nam Thiện Bộ Châu

    \item Bắc Câu Lôi Châu

    \item Đông Thắng Thần Châu

    \item Tây Ngưu Hóa Châu
\end{enumerate}
\end{multicols}

Bốn châu ấy nằm chung quanh núi Tu-Di (\emph{Sumeru}) rất lớn, sừng sững giữa đại dương. Tác giả Tây Du Ký đã mở đầu truyện giới thiệu nơi sinh của Mỹ Hầu Vương tại núi Hoa Quả tựa như là vị trí núi Tu-Di kia, và xem là nơi ``mạch Tổ của mười châu''. Từ ngữ Hoa Quả là từ biểu tượng quy luật vận hành Nhân Quả của hiện tượng giới, thế giới sinh diệt của nhị nguyên tính (\emph{Dualism}), của sự phân biệt rõ ràng giữa đục và trong, chủ thể và đối tượng. Đây là quy luật vận hành mà giáo lý Phật Giáo giới thiệu. Nó thuộc giáo lý trí tuệ, giúp con người phát triển trí tuệ.

Mỹ Hầu Vương sinh ra từ trứng đá như đã được đề cập ấy, bởi vì theo quy luật này thì Nhân Quả phải cùng loại (đá thì phải sinh ra đá, khỉ thì sinh ra khỉ). Nói khác đi, Mỹ Hầu Vương là biểu tượng của trí tuệ (Chánh Kiến và Chánh Tư Duy) hay Thánh Tuệ Uẩn trong giáo lý Phật Giáo, mà tiêu biểu là đôi mắt rực sáng làm kinh động đến Thiên Đình của Mỹ Hầu Vương. Đây là đôi mắt thấy rõ sự thật Vô Ngã, Vô Thường, Khổ và Không của hiện tượng giới. Sự thật đó làm kinh động, đảo lộn cái thấy thường hằng, hữu ngã của chúng sinh ở trên Trời và dưới Thế.

— Trí tuệ Vô Ngã đó sẵn có trong tâm của mỗi người, theo Ngô Thừa Ân, như một giang sơn đã soạn sẵn trong động Thủy Liêm dành cho Mỹ Hầu Vương. Sự thật Vô Ngã, Vô Thường, Khổ và Không cũng thế, là sự thật của sinh diệt, nằm ngay nơi hiện tượng giới như động Thủy Liêm nằm ngay núi Hoa Quả. Ngô Thừa Ân đã khéo léo giới thiệu động Thủy Liêm như là đang giới thiệu bốn pháp ấn của giáo lý Phật Giáo.

— Tại sao con người không thấy được sự thật về bốn pháp ấn ấy nơi thế giới chung quanh mình? Tại sao loài khỉ kia không khám phá ra động Thủy Liêm giữa khi động Thủy Liêm vẫn thường hiện diện tại đó, trước mắt chúng?

— Hỏi tức là trả lời vậy. Chính thác nước đổ mạnh đã che khuất khỏi cái nhìn hời hợt của đàn khỉ, hệt như i\emph{lòng tham ái, chấp thủ ngã} đã che khuất tâm con người khỏi sự thật Vô Ngã, Vô Thường, Khổ và Không. Ở giáo lý Phật Giáo, dòng thác đổ mạnh gọi là dòng bộc lưu, biểu tượng cho dòng nước Ái, dòng thủ, dòng sinh tử. Tại đây, Ngô Thừa Ân đang thật sự kiến trúc động Thủy Liêm theo giáo lý nhà Phật.

Chỉ có Mỹ Hầu Vương sinh ra từ đá, từ tú khí của trời đất, mà không sinh từ dục vọng (hay từ nghiệp), mới có trí tuệ Vô Ngã phát hiện ra sự thật vô thường, khổ đau. Mỹ Hầu Vương đã dàn dụa nước mắt, sầu não thấy lửa vô thường đang bốc cháy chung quanh cuộc sống, lo lắng tìm lối thoát ra khỏi khổ đau. Tất cả đàn khỉ còn lại thì vẫn nhởn nhơ với niềm hạnh phúc nhỏ trước mắt.

Thấy rõ sự thật Vô Thường là dấu hiệu của đạo tâm phát khởi, và là dấu hiệu của sự thấy rõ con đường ra khỏi sự trói buộc của vô thường, như một chú khỉ đã phát biểu muộn màng rằng: \emph{``Đại vương lo xa như thế là đạo tâm ngài đã phát khởi rồi đấy''} và đã giới thiệu với Mỹ Hầu Vương hiện nay tại cõi đời này, ở nơi động cổ, núi Tiên có ba đấng Phật, Tiên, Thần Thánh là thoát khỏi sinh diệt, luân hồi (thật ra là Phật, Bồ Tát, Duyên Giác và A-La-Hán); từ đó Mỹ Hầu Vương ra đi, tìm đến đại Tôn giả Tu Bồ Đề, một đại đệ tử của Đức Phật, để học đạo giải thoát của {\bf Tam Thừa Giáo} (Bồ Tát Thừa, Duyên Giác Thừa, Thinh Văn Thừa) và {\bf Nhất Thừa Giáo} (Phật Thừa). Đây là điểm giáo lý rất {\bf Đại Thừa} (Phật Giáo phát triển).

— Có một nếp sống thanh đạm, rời khỏi phiền não hằng ngày, và hiếu hạnh của một bác tiều phu ở trước cổng vào trú xứ của Tôn giả Tu Bồ Đề mà Mỹ Hầu Vương ca ngợi, vui mừng khi thấy, như là thấy dấu hiệu của quê hương giải thoát có mặt gần kề. Chính bác tiều phu chỉ đường cho Mỹ Hầu Vương đến yết kiến bậc đạo sư. Tại đây Ngô Thừa Ân gián tiếp giới thiệu hiếu đạo cũng là một phần thể hiện của Đạo Giải Thoát, tương ứng với Phật Giáo.

— Sự thật mà Mỹ Hầu Vương sẽ học từ Tôn giả Tu Bồ Đề và sẽ chứng ngộ là sự thật Vô Ngã tướng của các pháp (còn được gọi là Vô Ngã tính hay Vô Tự tính hoặc Không tính). Sự thật ấy được hé mở từ mẫu đối thoại đầu tiên giữa Tôn giả Tu Bồ Đề và Mỹ Hầu Vương mang đầy hương vị của một cảnh khai tâm của một thiền sư:

Tôn giả Tu Bồ Đề hỏi:

— Tính danh ngươi là gì?

— Thưa, con không có tính gì cả. Nhất sinh không có tính.

— Không phải là tính tình. Tính danh bố mẹ nhà người kìa?

(Đây là câu hỏi về bản lai diện mục) \ldots

Thế rồi, Tôn giả Tu Bồ Đề cho pháp danh là Tôn Ngộ Không (đã đề cập ở phần {\bf Tổng Luận:} \nameref{sub:ve_ton_ngo_khong}).

Đạo giải thoát khỏi cội nguồn của mọi khổ đau mở nguồn từ Ngộ Không này, và đạo ấy có mặt trong tâm của mỗi người, như Tôn giả Tu Bồ Đề thường có mặt ở ``Tà Nguyệt Tam Tinh'' động (Tà Nguyệt Tam Tinh là chữ tâm gồm có một nét như vành trăng và ba chấm -- \begin{CJK*}{UTF8}{bkai}心\end{CJK*}). Tác giả Tây Du Ký đã giới thiệu đúng con đường tu tập của Phật Giáo là con đường trở về chính mình để \emph{tu tập tâm, huấn luyện tâm và giải thoát tâm}. Con đường này rất là nhân bản và rất là hiện thực.
% section tư_tưởng_phật_học (end)

\section{Quan niệm của Ngô Thừa Ân về con người} % (fold)
\label{sec:1_quan_niem_ve_con_nguoi}

— Nếu về mặt Phật Học, phái đoàn Tây Du biểu thị các phần tố tâm lý của một tâm thức đi về giải thoát khỏi sinh tử, thì về mặt tương đối của đời sống xã hội phái đoàn Tây Du là biểu hiện sự cấu trúc của nhân tính (\emph{personality}) thể hiện sự hòa điệu giữa tim và óc, giữa tiềm thức, siêu thức và ý thức, giữa cá nhân và xã hội.

Tại đây, Mỹ Hầu Vương là biểu tượng của khối óc (hay lý trí). Lý trí này vốn có mặt trong tâm mỗi người và có khả năng có thể mở rộng ra vô hạn như ý nghĩa Mỹ Hầu Vương tìm đường đi ra khỏi thế giới hữu hạn của sinh tử. Khả năng vô hạn ấy đã được nền giáo dục hiện đại chứng tỏ.

— Khối óc, hay nguồn hiểu biết của con người, như đạo Nho quan niệm (nói đúng là quan niệm của Khổng Tử) có ba dạng hiểu biết gọi, là \emph{``sinh nhi tri, học nhi tri và khốn nhi tri''} biểu hiện ra khác nhau ở mỗi con người. Ở Mỹ Hầu Vương có đủ ba dạng hiểu biết ấy. Mỹ Hầu Vương sinh ra là đã tự biết (\emph{sinh nhi tri}); rồi tìm thầy học đạo (\emph{học nhi tri}); sau đó là lên đường hành đạo, chiến thắng ác ma (\emph{khốn nhi tri})

— Tiếng nói của lý trí là tiếng nói của sự ngay chính, của tình đồng loại, của óc tổ chức, của sự dự phòng tốt đẹp cho tương lai, của sự công bằng, và của sự phân biệt minh bạch phải, trái, tốt, xấu, như sự hiểu biết của chính cá nhân Mỹ Hầu Vương. Lý trí giữ vai trò hướng dẫn mọi hành động của con người, như đôi mắt, hệt như Tôn Hành Giả hướng dẫn phái đoàn Tây Du. Chúng ta sẽ tiếp tục khảo sát vai trò của lý trí trong những hồi kế tiếp, theo quan niệm của tác giả Ngô Thừa Ân.

% section quan_niệm_của_ngô_thừa_ân_về_con_người (end)

\section{Quan niệm của tác giả về xã hội} % (fold)
\label{sec:1_quan_ve_xa_hoi}

— Một xã hội tốt đẹp, an lạc, hạnh phúc, theo tác giả phải là một xã hội thái bình (không chiến tranh), độc lập với các sức mạnh thống trị như xã hội khỉ ở động Thủy Liêm.

— Xã hội ấy cần được tổ chức tốt về an ninh và kinh tế, như xã hội ở núi Hoa Quả ở trong vị thế an toàn và phong phú thực phẩm.

— Xã hội ấy phải được lãnh đạo bởi người có đủ tài, đức do dân chọn lựa như Mỹ Hầu Vương, mà không phải do tuổi tác hay dòng dõi. Quan niệm này của Ngô Thừa Ân được viết ra từ thế kỷ thứ XVI của xã hội phong kiến Trung Hoa quả là một quan niệm đầy dân chủ và tiến bộ. Đây là mô hình xã hội mà nhân dân làm chủ, bảo vệ được quyền sống của nhân dân.
% section quan_của_tác_giả_về_xã_hội (end)
% chapter hồi_thứ_1 (end)